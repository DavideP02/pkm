\subsection{Proprietà delle funzioni lineari}\marginnote{8 nov 2021}

\proposizione{comp}{
    La composizione di funzioni lineari è sempre una funzione lineare
}
\dimostrazioneprop{comp}{
    Siano $ V, W, Z $ spazi vettoriali su un campo $ \K $, $
    F:V\to W $ $G:W\to Z$ funzioni lineari, e prendiamo \[
        V \xrightarrow[]{F} W \xrightarrow[]{G} Z
    \]
    ovvero $ G \circ F $, quindi $ G \circ F (v)=G(F(v)) $

    Siano $ v, w \in V$, $ \lambda \mu \in \K $ \[
        G \circ F (\lambda v + \mu w)= 
    \]
    dato che $ F $ è lineare \[
        =G(F(\lambda v + \mu w))=G(\lambda F(v)+\mu F(w))=
    \]
    dato che $ G $ è lineare \[
        =\lambda G(F(v))+ \mu G(F(w))) =\lambda (G\circ F)(v)+ \mu(G\circ F)(w)\qedhere
    \]
}

\proposizione{sdfsdf}{
    Siuano $ V, W$ spazi vettoriali su un campo $ \K $, sia $ F: V\to W $ lineare biettiva ($ F $ è un isomorfismo), \[
        F^{-1}:W\to V \text{ è lineare}
    \]
}
Questa proprietà ci mostra quanto sia rigida la linearità di una funzione
\dimostrazioneprop{sdfsdf}{
    $ F^{-1}(a) $ è l'unico $ x \in V $ tale che $ F(x)=a $

    Siano $ a, b \in W $, $ \lambda \mu \in \K $, dimostro che \[ F^{-1}(\lambda a + \mu b)= \lambda F^{-1}(a)+ \mu F^{-1}(b)\]
    
    Denoto $ x = F^{-1}(a) $ e $ y=F^{-1}(b) $: ciò significa $ F(x)=a $ e $ F(y)=b $
    \[
        F(\lambda x + \mu y)=\lambda F(x)+ \lambda F(y)=\lambda a + \mu b
    \] 
    
    $\implies$ per come è definita $ F^{-1} $ questo implica \[
        F^{-1}(\lambda a + \mu b)=\lambda x + \mu y = \lambda F^{-1}(a)+ \mu F^{-1}(b)
    \] 
    
    $\implies$ $ F^{-1} $ è lineare\qed
}

\esempi{
    \begin{itemize}
        \item $ V= \K^n $, $ A \in \K^{n,n} $ invertibile, 
        \begin{align*}
        F_A: \K^n & \to \K^n\\
        \begin{pmatrix}
            x_1 \\ \vdots \\ x_{n} 
        \end{pmatrix} & \mapsto A\begin{pmatrix}
            x_1 \\ \vdots \\ x_{n} 
        \end{pmatrix}
        \end{align*}

        Se $ A $ è invertibile, esiste $ A^{-1} \in \K^{n,n} $ tale che $ A^{-1} A = A A^{-1} = \Id $ dove $ \Id \in \K^{n,n} $ è la matrice identita.

        Posso considerare \begin{align*}
            F_A^{-1}: \K^n & \to \K^n\\
            \begin{pmatrix}
                x_1 \\ \vdots \\ x_{n} 
            \end{pmatrix} & \mapsto A^{-1}\begin{pmatrix}
                x_1 \\ \vdots \\ x_{n} 
            \end{pmatrix}
            \end{align*}

        \[
            F_{A^{-1}}\circ F_{A} (x)= F_{A^{-1}} (F_{A}(x) ) = A^{-1}(A x)= \Id \cdot x = x
        \] 
        
        $\implies$ $ F_{A^{-1}}\circ F_{A}   $ è la funzione identità $ I: \K^{n} \to \K^n$ 
        
        $\implies$ $F_A$ è invertibile e la sua inversa è $ F_{A^{-1}}  $
        \item Sia $ V $ uno spazio vettoriale su un campo $ \K $, finitamente generato.
    
        Fisso $ \mathscr{B} =\{v_1,\cdots,v_{n}\}$ una base di $ V $, \begin{align*}
        F:V & \to \K^n\\
        v=\sum_{i=1}^n x_{i}v_{i} & \mapsto \begin{pmatrix}
            x_1 \\ \vdots \\ x_{n} 
        \end{pmatrix}
        \end{align*}
        funzione lineare iniettiva, poiché il suo nucleo è banale; poiché $ V$ e $ \K^n $ hanno la stessa dimensione, la funzione è un isomorfismo

        Si noti che \[F(v_{1}) = \begin{pmatrix}
            0\\ \vdots \\ 1 \\ \vdots \\ 0
        \end{pmatrix}\]
        con un unico 1 nella posizione $ i $-esima, quindi la base di $ V$ viene portata tramite $ F $ nella base canonica di $ \K^n $
    \end{itemize}
}

\definizione{}{Due spazi vettoriali $ V, W $ sullo stesso campo $ \K $ sono isomorfi se esiste $ F:V\to W $ isomorfismo}

\proposizione{ccdcdc}{
    Supponiamo che $ V $ e $ W $ siano due spazi vettoriali su un campo $ \K $, entrambi finitamente generati. 
    
    $ V $ è isomorfo a $ W $ $ \iff $ $ V $ e $ W $ hanno la stessa dimensione
}
\dimostrazioneprop{ccdcdc}{
    \begin{itemize}
        \item ["$ \implies $"] Supponiamo che esiste $ F:V\to W $ isomorfismo, 
        
        $ F $ iniettiva $ \implies $ $ \dim \text{Im}(F)=\dim V $

        $ F $ suriettiva $ \implies $ $ \dim F(V)=\dim W $ 
        
        $\implies$ $ \dim V = \dim W $

        \item ["$\impliedby$"] Supponiamo $ \dim V = \dim W = n $. Sia $ \mathscr{B} $ base di $ V $ e $ \mathscr{C} $ base di $ W $ \begin{align*}
        F:V & \to \K^n \\
        v & \mapsto (v)_{\mathscr{B}}
        \end{align*}
        un isomorfismo, \begin{align*}
        G:W & \to \K^n \\
        w & \mapsto (w)_{ \mathscr{B}}
        \end{align*}
        un isomorfismo
        \[
          V \xrightarrow[]{F} \K^n \xleftarrow[]{G} W \quad\implies\quad
          V \xrightarrow[]{F} \K^n \xrightarrow[]{G^{-1}} W
        \]

        Considero $ G^{-1}\circ F $, biettiva 
        
        $\implies$ $ G^{-1} \circ F $ è un isomorfismo 
        
        $\implies$ $ V, W $ sono isomorfi \qed
    \end{itemize}
}

\subsection{Funzioni lineari e cambiamenti di base}

Siano $ V, W $ spazi vettoriali su un campo $ K$, entrambi finitamente generati \[
    \dim  V = n, \dim W =m
\]

Considero $ F:V\to W $ lineare, e fisso $ \mathscr{B} $ base di $ V $ e $ \mathscr{C} $ base di $ W $. 

$ F $ è rappresentata da una matrice $A = M^{ \mathscr{B}, \mathscr{C}}(F) \in \K^{m,n}$ tramite la relazione \[(F(v))_{ \mathscr{C}}=M^{ \mathscr{B}, \mathscr{C}}(F)\cdot(v)_{\mathscr{B}}\]

\begin{center}
\begin{tikzpicture}
    \matrix (m) [matrix of math nodes,row sep=3em,column sep=4em,minimum width=2em]
    {
       V & W \\
       \K^n & \K^m \\};
    \path[-stealth]
      (m-1-1) edge node [left] {iso} (m-2-1)
              edge node [above] {$F$} (m-1-2)
      (m-2-1) edge node [below] {$F_{A}$} (m-2-2)
      (m-1-2) edge node [right] {iso} (m-2-2);
\end{tikzpicture}
\end{center}

Questo è un diagramma commutativo

Considero altre due base $ \mathscr{B}' $ di $ V$ e $ \mathscr{C}' $ di $ W $. 

Rispetto a queste basi, ad $ F $ corrisponde un'altra matrice $ M^{ \mathscr{B}', \mathscr{C}'}(F) $, voglio campire come $ M^{ \mathscr{B}, \mathscr{C}}(F) $ e $ M^{ \mathscr{B}', \mathscr{C}'}(F) $ sono relazionate.

Indico $ A=M^{ \mathscr{B}, \mathscr{C}}(F) $ e $ A'=M^{ \mathscr{B}', \mathscr{C}'}(F) $

Sia $ v \in V $ quindi \[
    (v)_{ \mathscr{B}}=\begin{pmatrix}
        x_1\\ \vdots \\ x_{n} 
    \end{pmatrix}= x \in \K^n
\] e \[
    (v)_{ \mathscr{B}'}=\begin{pmatrix}
        x_1'\\ \vdots \\ x_{n}'
    \end{pmatrix}= x' \in \K^n
\]

So che $ x=P x' $ con $ P \in \K^{n,n} $ invertibile del cambiamento di base da $ \mathscr{B} $ a $ \mathscr{B}' $. Considero $ F(v) \in W $
\[
    (F(v))_{ \mathscr{C}}=\begin{pmatrix}
        y_1\\ \vdots \\ y_{n} 
    \end{pmatrix}= y \in \K^m
\]
\[
    (F(v))_{ \mathscr{C}'}=\begin{pmatrix}
        y_1'\\ \vdots \\ y_{n}' 
    \end{pmatrix}= y' \in \K^m
\]

So che $ y=Qy' $, con $ Q $ matrice del cambiamento di base da $ \mathscr{C} $ a $ \mathscr{C}' $, dove $ Q \in \K^{m,m} $ è invertibile

\[y=Ax,\, y'=A'x,\,x=Px',\,y=Qy'\]

$ Qy'=Ax $ $ \implies $ $ Qy'=APx' $ 

$\implies$ $ y'=Q^{-1}APx' $ 

$\implies$ $ A' x'=Q'APx' $ $ \forall x' \in \K^n $ 

$\implies$ $ A'=Q^{-1}AP $

\subsubsection{Caso particolare} $ W=V $, quindi $ F:V\to V $ e considero $ \mathscr{C}= \mathscr{B}'$ e $ \mathscr{C}' = \mathscr{B}'$ ($\implies\, Q=P$). 

In questo caso la formula implica $ A'=P^{-1}AP $ dove $ P $ è la matrice del cambiamento di base da $ \mathscr{B} $ a $ \mathscr{B}' $

\definizione{}{Due matrici $ A, B \in \K^{n,n}$ sono simili se esiste $ P \in \K^{n,n} $ matrice invertibile tale che $ B=P^{-1}AP $}

\esercizio{
    Siano $ A, B \in \K^{n,n}$ matrici simili 
    
    $\implies$ $ \det A =\det B $, $ \tr A =\tr B $, $\rank A = \rank B$ 
}{
    Supponiamo $ A, B $ simili, allora esiste $ P \in \K^{n,n} $ invertibile tale che $ B=P^{-1}AP $

    Per il teorema di Binet: \[
        \det B = \det(P^{-1})\det A \det P = \det A
    \]

    Poi \[
        \tr B = \tr (P^{-1} A P) = \tr (P^{-1}P A)=\tr A
    \]

    Poiché $ P $ e $ P^{-1} $ hanno rango $ n $, risulta \[\rank(P^{-1}AP)=\rank A\qedhere\]
}

\esercizio{Si verifichi che la similitudine (la proprietà di due matrici di essere simili) in $ \K^{n,n} $ è una relazione di equivalenza}{
    Indico con $ \sim $ la relazione \[
        A\sim B\text{ se esiste } P \in \K^{n,n}\text{ invertibile }|\, B=P^{-1}AP
    \]
    \begin{itemize}
        \item $ \sim $ è riflessiva, $ A=(\Id)^{-1} A \cdot \Id $ $\implies$ $ A\sim A $
        \item $ \sim $ è simmetrica, infatti, se $ A\sim B $ 
        
        $ \implies $ $ B=P^{-1}AP $ 
        
        $\implies$ $A= PBP^{-1}$ 
        
        $\implies$ $ B\sim A $
        \item Supponiamo $ A\sim B $ e $ B\sim C $ e dimostro $ A\sim C $
        \[
            A\sim B \,\implies\, B=P^{-1}AP\quad B\sim C \,\implies\, C=Q^{-1} BQ
        \] 
        
        $\implies$ $ C=Q^{-1}P^{-1} APQ=(PQ)^{-1}A (PQ)$ 
        
        $\implies$ $ A\sim C $\qed
    \end{itemize}
}

\esercizio{
    In $ \R^3 $ considero la base canonica $ \mathscr{B}={e_1,e_2, e_3} $ e la base data dai tre vettori \[
        v_1=(1,2,0), v_2=(1,0,1), v_3=(-1,0,-2)
    \]
    \begin{enumerate}
        \item Si verifichi che $ \mathscr{C}=\{v_1,v_2,v_3\}$ sia una base di $ \R^3 $
        \item Sia $ F $ la funzione lineare $ F:\R^3\to \R^3 $ determinata dalle relazioni \begin{gather*}
            F(v_1)=v_1+v_2\\
            F(v_2)=2v_1-v_2\\
            F(v_3)=-v_2+v_3
        \end{gather*}

        Si trovi la matrice che rappresenta $F$ rispetto alla base $ \{v_1, v_2,v_3\} $ e la matrice che rappresenta $ F $ rispetto alla base canonica $ \mathscr{B} $
    \end{enumerate}
}{
    \begin{enumerate}
    \item Sia \[
        A=\begin{pmatrix}
            1 & 2 & 0\\
            1 & 0 & 1\\
            -1 & 0 & -2
        \end{pmatrix}
    \]  
    si noti che $ \det A \neq 0 $, quindi i tre vettori sono linearmente indipendenti, ovvero sono una base
    \item \[M^{\mathscr{C}, \mathscr{C}}(F)=\begin{pmatrix}
        1 & 2 & 0 \\
        1 & -1 & -1 \\
        0 & 0 & 1
    \end{pmatrix}\]
    $M^{ \mathscr{B}, \mathscr{B}}(F)$: per quanto visto oggi $ M^{\mathscr{C}, \mathscr{C}}(F)=P^{-1}M^{ \mathscr{B}, \mathscr{B}}(F)P $ 
    
    $\implies$ $ M^{ \mathscr{B}, \mathscr{B}}(F) = P M^{ \mathscr{C}, \mathscr{C}}(F) P^{-1}$
    %finirlo per esercizio
    \end{enumerate}
}{}

\definizione{}{
    Sia $ V $ uno spazio vettoriale di dimensione finita, e sia $ F: V\to V $ lineare. Se $ \mathscr{B} $ è la base fissata di $ V $, allora $ M^{ \mathscr{B}, \mathscr{B}}(F) $, si definisce
    \[
        \det F = \det (M^{ \mathscr{B}, \mathscr{B}}(F))
    \] e \[
        \tr F = \tr (M^{ \mathscr{B}, \mathscr{B}}(F))
    \]

    Per un risultato precedente, $ \tr F $ e $ \det F $ sno ben definiti, ovvero non dipendono dalla base fissata, mentre $ M^{ \mathscr{B}, \mathscr{B}}(F) $ sì
}

\attenzione{Esistono matrici $ A, B \in \K^{n,n}$ tali che $ \tr A =\tr B $, $ \det A = \det B $, $ \rank A = \rank B $ ma non simili}

\esempio{}{
    \[A=\begin{pmatrix}
        1 & 0 \\
        0 & 1
    \end{pmatrix} \in \R^{2,2}\quad B=\begin{pmatrix}
        1 & 1 \\
        0 & 1
    \end{pmatrix} \in \R^{2,2}\]

    Notiamo che $ \det A =\det B $, $ \tr A = \tr B $, $\rank A =\rank B$, ma $ A $ e $ B $ non sono simili, infatti
    \[
        P^{-1}AP=\Id \forall P \in \text{GL}(2, \R), B\neq \Id
    \]
}

% TODO riguardare questa parte
% TODO cambiare tutte le matrici identità

\subsection{Spazio delle funzioni lineari}
\subsubsection{Somma di funzioni lineari}

Siano $ V, W $ spazi vettoriali sullo stesso campo $ \K $. Siano $ F, G: V\to W $ lineari. Si introduce \begin{align*}
F+G: V & \to  W \\
 v & \mapsto F(v)+G(v)
\end{align*}
funzione da $ V $ in $ W $

\esercizio{Si dimostri che $ F+G $ è funzione lineare}{
    Da fare% TODO risolvere l'esercizio
}

\subsubsection{Prodotto per scalari di funzioni lineari}

Si introduce inoltre, se $ \lambda \in \K $, la funzione \begin{align*}
\lambda F:V & \to W \\
v & \mapsto \lambda F
\end{align*}

\esercizio{Si dimostri che $ \lambda F $ è funzione lineare}{
    Da fare% TODO risolvere l'esercizio
}

\rule{7em}{.4pt}

Indico con \[
    L(V, W)=\{F:V\to W | F \text{ lineare}\}
\]
$ L(V,W) $ eredita una struttura di spazio vettoriale su $ \K $, dove il vettore nullo di $ L(V,W) $ è la funzione costante \begin{align*}
0_{L(V,W)}: V & \to W \\
v & \mapsto \underline{0}_W
\end{align*}