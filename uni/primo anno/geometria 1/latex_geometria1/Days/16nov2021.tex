In generale $ V $ non ha un'orientazione canonica, ma nei casi in cui $ V $ abbia una base canonica $ \mathscr{B}_0 $ 

$\implies$ $ V $ ha l'orientazione canonica $ [ \mathscr{B}_0] $; ad esempio in $ \R^{n} $ si considera la base canonica $ \mathscr{B}_0=\{e_1,\cdots,e_{n} \} $, e ci si riferisce a $ [B_{0} ] $ come all'orientazione di $ \R^{n} $

\section{Complementi ortogonali}

Sia $ (V, \cdot ) $ uno spazio vettoriale euclideo, e sia $ W \subseteq V $ sottospazio vettoriale. Si definisce \[ W^{\bot }=\{v \in V\,|\, v \cdot w = 0 \, \forall\, w \in W\}\]

Questi sono i vettori di $ V $ ortogonali a tutti i vettori di $ W$, $ W^{\bot} \subseteq V $

\esempi{}{
    \begin{enumerate}
        \item $ W=\{\underline{0}\} $ 
        
        $\implies$ $ W^{\bot}=V $
        \item $ W=V $ 
        
        $\implies$ $ W^{\bot}=\{\underline{0}\}$
        \item $ W \subseteq V $ sottospazio vettoriale, $ \dim W = k $. Fisso $ \mathscr{B}=\{w_1, \cdots, w_{k} \} $ base di $ W $.
        
        Si osserva che \[W^{\bot}=\{v \in V \,|\, v \cdot w_{r}\, \forall\, r = 1, \cdots, k \}.\] Lo dimostro con la doppia inclusione. L'inclusione $ \supseteq  $ è ovvia, dimostro $ \subseteq $.

        Sia $ v \in V $ tale che $ v \cdot w_{r}= 0$ $ \forall\,r = 1, \cdots, k $. Sia $ w \in W $ generico. Posso scrivere $ w = a_1w_1 + \cdots + a_{k}w_{k}$ per qualche $ a_1, \cdots, a_{k}  \in \R $ \begin{multline*}
            v \cdot w =v ( a_1w_1 + \cdots + a_{k}w_{k})= \\
            = a_1v \cdot w_1 + a_2v \cdot w_2 + \cdots+ a_{k}v \cdot w_{k}=0  
        \end{multline*} 
        
        $\implies$ $ v \cdot w =0 $ $ \forall\, w \in W $ 
        
        $\implies$ $ v \in W^{\bot} $
    \end{enumerate}
}

\esercizio{
    Sia $ W \subseteq \R^{3} $ il sottospazio vettoriale di equazioni \[
      \begin{cases}
          x+y=0 \\
          x+z=0
      \end{cases}  
    \]

    Determinare $ W^{\bot} $ rispetto al prodotto scalare canonico
}{
    $ \dim W=1 $. Una base di $ W  $ è data da $ \mathscr{B}=\{(1, -1, -,1)\} $. \[
        W^{\bot}=\{(x,y,z) \in \R^{3} \,|\, (x,y,z) \cdot (1, -1, -,1)=0\}
    \]

    \[
        (x,y,z) \cdot (1, -1, -,1)=x-y-z
    \]

    \[
        W^{\bot}=\{(x,y,z) \in \R^{3} \,|\, x-y-z=0\}\quad\text{piano in }\R^{3}
    \]
}{}   

\teorema{kjkjhjjdjdj}{
    Sia $ (V, \cdot ) $ uno spazio vettoriale euclideo, e sia $ W \subseteq V $ un sottospazio vettoriale. Allora \begin{enumerate}
        \item $ W^{\bot} $ è un sottospazio vettoriale di $ V $
        \item $ W\oplus W^{\bot}=V $
        \item $ (W^{\bot})^{\bot}=W $
    \end{enumerate}
}
\dimostrazione{kjkjhjjdjdj}{
    \begin{enumerate}
        \item Siano $ v_1 $, $ v_2 \in W^{\bot}$, e  $\lambda$, $\mu \in \R$, dimostriamo che $ \lambda v_1+ \mu v_2 \in W^{\bot} $
        
        \begin{align*}
            v_1 \in W^{\bot} \,&\implies\, v_1 \cdot w =0\, \forall\,w \in W\\
            v_2 \in W^{\bot} \,&\implies\, v_2 \cdot w =0\, \forall\,w \in W
        \end{align*}

        Sia $ w \in W $ \[
            (\lambda v_1 \cdot \mu v_2) \cdot w = \lambda \overset{0}{\overbrace{v_1 \cdot w }} + \mu\overset{0}{\overbrace{v_2 \cdot w}}=0
        \] 
        
        $\implies$ $ \lambda v_1 \cdot \mu v_2 \in W^{\bot}$ 
        
        $\implies$ $ W^{\bot} $ sottospazio vettoriale.
        \item Supponiamo $ \dim  W =k$, sia $ \{v_1, \cdots, v_{k} \} $ una base di $ W $. Possiamo completare $ \{v_1, \cdots, v_{k} \} $ ad una base \[\{v_1, \cdots, v_{k}, v_{k+1}, \cdots, v_{n}\}\] base di tutto $ V $ 
        
        Utilizziamo l'algoritmo di Gram-Schmidt e troviamo una base ortonormale $ \{e_1, \cdots, e_{n} \} $ di $ V $ tale che \[
            \mathscr{L}(e_1, \cdots, e_{r} )= \mathscr{L}(v_1, \cdots, v_{r} )\quad \forall\, r = 1, \cdots, n.
        \]
        Quindi $ \{e_1, \cdots, e_{k} \} $ è una base di $ W $.

        Sia $ v \in V $. Possiamo scrivere \[
            v=\sum_{r=1}^{n} \lambda_{r}e_{r}   
        \]
        $ v \in W^{\bot} $ $\iff$ $ v \cdot e_{i} =0  $ $ \forall\,i=1, \cdots,k $

        $ \iff $ $ \underset{\lambda_{i}}{\underbrace{\sum_{r=1}^{n}(\lambda_{r} e_{r}  ) \cdot e_1}} =0  $ $ \forall\, i =1, \cdots, k $

        $ \iff $ $ \lambda_{i}=0 $ $ \forall\, i = 1, \cdots, k $ 

        $ \iff $ $ v \in \mathscr{L}(e_{k+1}, \cdots, e_{n}  ) $

        Cioè $ W^{\bot}= \mathscr{L}(e_{k+1}, \cdots, e_{n}  ) $ 
        
        $\implies$ $ V= \mathscr{L}(e_{1}, \cdots, e_{k}  )+ \mathscr{L}(e_{k+1}, \cdots, e_{n}  )= W \oplus W^{\bot}$

        In particolare $ W\cap W^{\bot}=\{\underline{0}\} $, infatti se $ \overline{v} \in W\cap W^{\bot} $ 
        
        $\implies$ $ \overline{v} \cdot \overline{v}=0 $ 
        
        $\implies$ $ \overline{v}=\underline{0} $
        \item $ (W^{\bot})^{\bot} $: \[
            v \in (W^{\bot})^{\bot} \,\iff\, v \cdot z = 0 \,\forall\, z \in W^{\bot}\,\iff\, v \in W\qedhere
        \]
    \end{enumerate}
}

\proposizione{jjdijijijiij}{
    Sia $ (V, \cdot) $ uno spazio vettoriale Euclideo, siano $ W_1 $ e $ W_2 \in V $ sottospazi vettoriali
    \begin{enumerate}
        \item Se $ W_1 \subseteq W_2 $ $\implies$ $ W_2^{\bot} \subseteq W_1^{\bot} $
        \item $ (W_1+W_2)^{\bot}=W_1^{\bot}\cap W_2^{\bot} $
        \item $ (W_1\cap W_2)^{\bot}=W_1^{\bot}+ W_2^{\bot} $
    \end{enumerate}
}
\dimostrazioneprop{jjdijijijiij}{
    Dimostrazione per esercizio %TODO fare dimostrazione
}

\esercizio{
    Prendiamo $ \R^{2,2} $ con il prodotto scalare canonico, considero \[ A=\begin{pmatrix}
        1 & 3 \\
        0 & -1
    \end{pmatrix},\] e prendo $ W=\{X \in \R^{2,2}\,|\, AX=XA\} $ sottospazio vettoriale di $ \R^{2,2} $. Trovare $ W^{\bot} $
}{
    Trovo una base di $ W $. Sia \[
        X=\begin{pmatrix}
            x_1 &x_2\\
            x_3 & x_4
        \end{pmatrix}
    \]
    matrice generica in $ \R^{2,2} $ e impongo $ AX=XA $

    \[
        AX=\begin{pmatrix}
            1 & 3 \\
            0 & -1
        \end{pmatrix} \cdot \begin{pmatrix}
            x_1 & x_2 \\
            x_3 & x_4
        \end{pmatrix}=\begin{pmatrix}
            x_1+3x_3 & x_2+3x_4 \\
            -x_3 & -x_4
        \end{pmatrix}
    \]
    \[
        XA=\begin{pmatrix}
            x_1 & x_2 \\
            x_3 & x_4
        \end{pmatrix}\cdot\begin{pmatrix}
            1 & 3 \\
            0 & -1
        \end{pmatrix}  =\begin{pmatrix}
            x_1 & 3x_1-x_2 \\
            x_3 & 3x_3-x_4
        \end{pmatrix}
    \]

    Otteniamo il sistema \[
      \begin{cases}
          x_1+3x_3=x_1 \\
          x_2+3x_4=3x_1-x_2\\
          x_3=-x_3\\
          -x_4=3x_3-x_4
      \end{cases} \,\implies\, \begin{cases}
        x_1=x_1\\
        x_2=\frac{3}{2}(x_1-x_4)\\
        x_3=0 \\
        x_3=0
      \end{cases} \,\implies\, \dim  W =2
    \]
      Una base di $ W $ è data da \[
          A_1=\begin{pmatrix}
              2 & 3\\
              0 & 0
          \end{pmatrix}\qquad A_2=\begin{pmatrix}
              0 & -3 \\ 0 & 2
          \end{pmatrix}
      \]

      Impongo per $ Y \in \R^{2,2} $, $ Y \cdot A_1= Y \cdot A_2 = \underline{0}$
      \[
          Y=\begin{pmatrix}
              y_1 & y_2 \\
              y_3 & y_4
          \end{pmatrix}
      \] \begin{multline*}
          Y \cdot A_1=\tr (\null^{t}Y A_1)= \\=\tr\Biggl(\begin{pmatrix}
            y_1 & y_2 \\
            y_3 & y_4
        \end{pmatrix} \begin{pmatrix}
            2 & 3\\
            0 & 0
        \end{pmatrix}\Biggr)=\\
        =\tr \begin{pmatrix}
            2y_1 & 3y_1 \\ 2y_2 & 3y_2
        \end{pmatrix}= 2y_1+3y_2
    \end{multline*}
    \begin{multline*}
        Y \cdot A_2=\tr (\null^{t}Y A_2)= \\=\tr\Biggl(\begin{pmatrix}
          y_1 & y_2 \\
          y_3 & y_4
      \end{pmatrix} \begin{pmatrix}
        0 & -3 \\ 0 & 2
    \end{pmatrix}\Biggr)=\\
      =\tr \begin{pmatrix}
          0 & -3y_1+2y_3 \\ 0 & -3y_2+2y_4
      \end{pmatrix}= -3y_2+2y_4
  \end{multline*}

  Quindi \begin{gather*}Y \in W^{\bot}\,\iff\, 2y_1+3y_2=0\,\land\, -3y_2+2y_4=0\\
y_1=-3/2 y_2, \quad y_4 = 3/2 y_2,\quad y_3 = y_3 \end{gather*}

Una base di $ W^{\bot} $ è data da \[
    A_{3}=\begin{pmatrix}
        -3/2 & 1 \\0 & 3/2
    \end{pmatrix}, A_{4}=\begin{pmatrix}
        0 &0 \\0 &1
    \end{pmatrix} .
\]
$ W= \mathscr{L}(A_3,A_4) $
}{}

\esercizio{
    $ \R^{n,n} $ con il prodotto scalare canonico, \[
        \mathcal{S}(\R^{n,n})=\{A \in \R^{n,n}\,|\, \null^{t}\!A=A\}\qquad \mathcal{A}(\R^{n,n})=\{A \in \R^{n,n}\,|\, \null^{t}\!A=-A\}
    \]
    Dimostrare $ \mathcal{S}(\R^{n,n})^{\bot}= \mathcal{A}(\R^{n,n})$
}{
    \[
    \mathcal{S}(\R^{n,n})\cap\mathcal{A}(\R^{n,n})=\{\underline{0}\}, 
    \] infatti una matrice è sia simmetrica che antisimmetrica $ \iff $ è la matrice nulla.

    \[
        \dim \mathcal{A}(\R^{n,n}) + \dim \mathcal{S}(\R^{n,n}) = n^{2}%TODO calcolarlo per esercizio
        = \dim  \R^{n,n}
    \] 
    
    $\implies$ $ \R^{n,n}= \mathcal{S}(\R^{n,n}) \oplus\mathcal{A}(\R^{n,n})$

    Mi basta dimostrare che date $ A \in \mathcal{A}(\R^{n,n}) $ e $ S \in \mathcal{S}(\R^{n,n})  $ risulta $ A \cdot S =0 $
    \[
        A \cdot S=\tr (\null^{t}\!A \cdot S) = -\tr (A \cdot S)
    \] ma \[
        A \cdot S = S \cdot A =\tr( \null^{t}\!S A) = \tr (S A) = \tr (A S).
    \]

    Quindi $ A \cdot S = - A \cdot S$ 
    
    $\implies$ $ A \cdot S = 0 $. Segue l'esercizio.
}{}

\subsection{Proiezioni ortogonali} 

$ (V, \cdot ) $ spazio vettoriale Euclideo, $ W \subseteq V $ sottospazio. $ V=W \oplus W^{\bot} $. Ogni $ v \in V $ si scrive in modo unico come $ v=v'+v'' $ con $ v' \in W $ e $ v'' \in W^{\bot} $.

Si definisce \begin{align*}
    pr_{w}: V& \to W \\
 v & \mapsto v' 
\end{align*}
$ pr_{w} $ si dice la \textit{proiezione ortogonale} su $ W $. $ pr_{w} $ è lineare, suriettiva, e $ pr_{w} = \I_{W}  $%TODO rivedere appunti
, $ \ker(pr_{w})= W^{\bot} $

\section{Isometrie}

Sia $ (V, \cdot ) $ spazio vettoriale, e sia $ F \in End(V) $\footnote{è un endomorfismo}, $ F:V \to V $ lineare. $ F $ è un'\textit{isometria} se $ F(v) \cdot  F(w) = v \cdot w $ $ \forall\, v,w \in V $.

\teorema{jjjjkdpojkdkks}{
    Se $ F $ è una isometria, 
    
    $\implies$ $ F $ è un automorfismo di $ V $ (cioè $ F $ è un isomorfismo $ V \to V$)
}
\dimostrazione{jjjjkdpojkdkks}{
    Basta verificare che $ F $ iniettiva, ovvero che $ \ker F =\{\underline{0}\} $. Sia $ v \in \ker F $, cioè $ v $ è tale che $ F(v)=0 $ 
    
    $\implies$ $ F(v) \cdot F(v)= \underline{0} \cdot \underline{0} = \underline{0} $, ma poiché $ F $ isometria risulta che $ F(v) \cdot F(v) = v \cdot v $ 
    
    $\implies$ $ v \cdot v = 0 $ 
    
    $\implies$ $ v=\underline{0} $ 
    
    Quindi $ \ker F = \{\underline{0}\} $ e $ F $ è iniettiva. \qed
}

\esempio{
    $ \I $ e $ -\I $ sono isometrie su $ V $ rispetto a tutti i prodotti scalari.
}

\esercizio{
    Sia $ F \in End(V) $ tale che $ ||F(v)||= ||v|| $ $ \forall\, v \in V $. Si dimostri che $ F $ è un'isometria
}{
    Supponiamo che $ F $ soddisfi $ ||F(v)||=||v|| $ $ \forall\, v \in V $, e siano $ v,w \in V $. Per ipotesi $ ||F(v+w)||^{2}=||v+w|| $ 
            
    $\implies$ $ ||F(v)+F(w)||^{2}= ||v+w||^{2}$ \[
        ||F(v)||^{2}+||F(w)||^{2}+ 2 F(v) \cdot F(w)= ||v||^{2}+||w||^{2}+2v \cdot w
    \] uso $ ||F(v)||=||v|| $ e $ ||F(w)||=||w|| $ 
    $\implies$ \[
        \cancel{||F(v)||^{2}}+\cancel{||F(w)||^{2}}+ 2 F(v) \cdot F(w)= \cancel{||v||^{2}}+\cancel{||w||^{2}}+2v \cdot w
    \] 
    
    $\implies$ $ F(v) \cdot F(w) = v \cdot w $ 
    
    $\implies$ $ F $ isometria
}{}
