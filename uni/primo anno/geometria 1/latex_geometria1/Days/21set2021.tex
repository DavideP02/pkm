\subsection{Moltiplicazione}

Si\marginnote{21 set 2021} può moltiplicare $\lambda \in \R$ con matrici $A\in\rmn$
\[
\lambda A=\lambda\qmatrice{a}=\qmatrice{\lambda a}
\]
\[
-1\cdot A = -A\qquad\text{coerente con la definizione di }-A
\]

\esempio{
\[
2\begin{pmatrix}
3 & 1 & 0\\
-1 & 4 & 1
\end{pmatrix}=\begin{pmatrix}
6 & 2 & 0\\
-2 & 8 & 2
\end{pmatrix}
\]
}

\osservazione{$0\cdot A$ è la matrice nulla $\forall A\in \rmn$}

\proprieta[del prodotto per scalari]{
\begin{itemize}
\item [(\textit{i})] $\displaystyle \lambda(A+B)=\lambda A + \lambda B \qquad\forall\lambda\in\R,\:A,B\in\rmn$
\item [(\textit{ii})] $\displaystyle (\lambda+\mu)\cdot A=\lambda\cdot A+\mu\cdot A\qquad\forall\lambda\mu\in\R,\:A\in\rmn$
\item [(\textit{iii})] $\displaystyle(\lambda\mu)A=\lambda(\mu A)\qquad\forall\lambda\mu\in\R,\:A\in\rmn$
\item [(\textit{iv})] $\displaystyle 1\cdot A = A\qquad\forall A\in\rmn$
\end{itemize}
}

$(\rmn, +)$ è un \textbf{gruppo abeliano} in cui è definita una moltiplicazione per scalari in cui valgono le proprietà \textit{i}-\textit{iv} (prototipo per gli spazi vettoriali).

\subsection{Prodotto tra matrici}
\[
A,B\:\tc\: A\in\R^{m,\textcolor{red}{n}}, B\in\R^{\textcolor{red}{n}, k}\implies AB\in \R^{m, k}
\]
Questo è definito come il prodotto \textbf{righe per colonne}. Il numero di colonne della prima matrice deve corrispondere con il numero di righe della seconda matrice.

\definizione{
	Siano $ A \in \R^{m,n} $ e $ B \in \R^{n,k} $ due matrici, siano $ a_{ij}  $ gli elementi di $ A $ e $ b_{rs}  $ gli elementi di $ B $ [Notazione: $ A=(a_{ij})$, $ B=(b_{rs}) $]

	La matrice $ A\cdot B $ è la matrice in $ R^{m,k} $ il cui $ ij $-esimo elemento è
	\[
		a_{i1} \cdot b_{1j} + a_{i 2} \cdot b_{2j} + \cdots+ a_{in} \cdot b_{ni} =\sum_{r=1}^n a_{ir} \cdot b_{rj}  
	\]
}

\subsubsection{Prodotto tra matrici quadrate}

Siano $ A, B \in \R^{m,m} $, $ AB \in \R^{m,m} $; in questo caso il prodotto tra matrici definisce una operazione in $ \R^{m,m} $.
\begin{itemize}
	\item [(\textit{i})] il prodotto è associativo: $ (A\cdot B) \cdot C = A \cdot (B\cdot C) $, $ \forall A, B, C \in \R^{m,m} $
	\item [(\textit{ii})] esiste un elemento neutro
\end{itemize}

\proposizione{matrid}{
	Sia $ I \in \R^{m,m} $ la matrice identità, $ A \in \R^{m,m} $ 
	
	$\implies$ $ A\cdot I = I\cdot A = A $ $ \forall A \in \R^{m,m} $
}
\dimostrazioneprop{matrid}{
	Sia $ (r_{ij} ) $ l'$ij$-esimo elemento della matrice $ A\cdot I $ con $ A=(a_{ij} ) $ e $ I = (b_{ij} ) $ \[
		r_{ij} = \sum_{n=1}^m a_{in}\cdot b_{ni}  
	\]
	Si noti che $ b_{kh} =0 $ $ \forall k,h | k \neq h $ $\implies$ 
	\begin{multline*}
		r_{ij} = \sum_{n=1}^m a_{in}\cdot b_{ni}  =\\
		= \cancel{a_{i1}b_{1j}}  + \cancel{\cdots} + a_{ij}b_{jj}+ \cancel{\cdots}+\cancel{a_{in}b_{nj}}=\\
		=a_{ij} \cdot b_{jj} = a_{ij} \cdot 1     
	\end{multline*} 
	
	$\implies$ $ r_{ij}=a_{ij}$\qed
}

In generale se $ A \in \R^{m,m} $ $ \nexists $ un inverso per $ A $, cioè non esiste $ B \in \R^{m,m} $ tale che $ A \cdot B = B \cdot A = I $

\esempio{
	\begin{itemize}
		\item Se $ A $ è la matrice nulla 
		
		$\implies$ $ A \cdot B = $ matrice nulla $ \neq I $
		\item Se $ A $ ha una riga o una colonna nulla (ovvero fatta tutta di zeri)
		
		$\implies$ non è invertibile
	\end{itemize}
}

Data $ A \in \R^{m,m} $, è necessario capire se è invertibile, ed eventualmente calcolarne l'inversa. Si definisce \begin{equation}
	\gl (n, \R)=\{A, \R^{n,n}\,\tc\: A\text{ è invertibile}\}
\end{equation}
(con $ n\neq 1 $) ed è u gruppo non abeliano con la moltiplicazione tra matrici. 
\[
	\exists\, A, B \in \gl(n, \R)\,\tc\: AB\neq BA
\]

\teorema{proprioeoitaohiuhkjhdafoiughdafjhgoviualdbf}{
	Valgono le seguenti proprietà
	\begin{enumerate}
		\item Siano $ A, B \in \R^{n,n} $ matrici invertibili: $ AB $ è invertibile, e \[
			(AB)^{-1}=B^{-1}A^{-1}
		\]
		\item Se $ A \in \R^{n,n} $ è inveribile, allora $ A^{-1} $ è invertibile, e \[
			(A^{-1})^{-1}=A
		\]
	\end{enumerate}
}
Si adotta come notazione $ A^{-1} $ matrice inversa di $ A $
\dimostrazione{proprioeoitaohiuhkjhdafoiughdafjhgoviualdbf}{
	\begin{enumerate}
		\item Si dimostra che il prodotto tra $ (AB) $ e $ (B^{-1}A^{-1}) $ è l'identità, e che la cosa è commutativa
		\begin{multline*}
			(AB)(B^{-1}A^{-1})=A(BB^{-1})A^{-1}=\\=A\I A^{-1}= AA^{-1}=\I \,\implies\\
			\implies\, (AB)(B^{-1}A^{-1})=\I
		\end{multline*}
		\begin{multline*}
			(B^{-1}A^{-1})(AB)=B^{-1}(A^{-1}A)B=\\
			= B^{-1}\I B=B^{-1}B=\I \,\implies\\
			\implies\, (B^{-1}A^{-1})(AB)=\I
		\end{multline*}
		Ne risulta quindi che \[
			(AB)(B^{-1}A^{-1})=\I=(B^{-1}A^{-1})(AB)
		\]
		quindi $ AB $ è invertibile e la sua inversa è $ B^{-1}A^{-1} $.
		\item $ A $ è invertibile, quindi $ AA^{-1}=A^{-1}A=\I $ 
		
		$\implies$ sia $ A $ che $ A^{-1} $ sono invertibili: l'inversa di $ A^{-1} $ è $ A $, quindi \[
			(A^{-1})^{-1}=A\qedd
		\]
	\end{enumerate}
}

\teorema[(di unicità degli inversi)]{unicinoinveoiriversoaisaoijlkuglkjughl}{
	Sia $ A \in \R^{n,n} $ invertibile 
	
	$\implies$ esiste un'unica inversa di $ A $, cioè \begin{multline}
		X, X' \in \R^{n,n}\,\tc\\ AX=XA=\I\,\land\, AX'=X'A=\I\\
		\implies\,X=X'
	\end{multline}
}
\dimostrazione{unicinoinveoiriversoaisaoijlkuglkjughl}{
	$ X'=\I X=XAX'=X\I=X $\qed
}

\esempi[di inverse di matrici]{
	\begin{itemize}
		\item [(\textit{i})] Dato $ \R^{1,1}=\R $, $ x \in \R $ è invertibile $ \iff $ $ x\neq 0 $, e la sua inversa è $ 1/x $.
		\item [(\textit{ii})] In $ \R^{2,2} $ si prendano due matrici \[
			A=\begin{pmatrix}
				a & b\\
				c & d
			\end{pmatrix}\qquad A'=\begin{pmatrix}
				d & -b\\
				-c & a
			\end{pmatrix}
		\]
		\begin{gather*}
			AA'= \begin{pmatrix}
				a & b\\
				c & d
			\end{pmatrix}\begin{pmatrix}
				d & -b\\
				-c & a
			\end{pmatrix} = \begin{pmatrix}
				ad-cb & 0\\
				0 & ad-cb
			\end{pmatrix}=(ad-cb)\,\I\\
			A'A= \begin{pmatrix}
				d & -b\\
				-c & a
			\end{pmatrix}\begin{pmatrix}
				a & b\\
				c & d
			\end{pmatrix} = \begin{pmatrix}
				ad-cb & 0\\
				0 & ad-cb
			\end{pmatrix}=(ad-cb)\,\I
		\end{gather*}

		Se $ ad-cb\neq 0 $ possiamo definire \[
			X=\frac{1}{ad-cb}\begin{pmatrix}
				d & -b\\
				-c & a
			\end{pmatrix}
		\]
		per cui vale $ AX=XA=\I \,\implies\, X=A^{-1} $.

		Se $ A \in \R^{2,2} $, $ A=\left(\begin{smallmatrix}
			a & b\\ c & d
		\end{smallmatrix}\right) $ con $ ad-cb \neq_0$ 
		
		$\implies$ $ A $ è invertibile, con inversa \[
			A^{-1}=\frac{1}{ad-cb}\begin{pmatrix}
				d & -b\\
				-c & a
			\end{pmatrix}
		\]
		$ ad-cb $ si dice il determinante di $ A$, e si indica con $ \det A $\begin{equation}
			A^{-1}=\frac{1}{\det A}\begin{pmatrix}
				d & -b\\
				-c & a
			\end{pmatrix}
		\end{equation}
		\item [(\textit{iii})] Sia $ A $ così costruita:
		\[
			A=\begin{pmatrix}
				\lambda_1 & 0 & \cdots & 0\\
				0 & \lambda_2 & & \vdots\\
				\vdots & & \ddots & \vdots\\
				0 & \hdotsfor{2} & \lambda_{n} 
			\end{pmatrix}
		\]
		$ A $ ha $ (\lambda_1, \lambda_2, \cdots, \lambda_{n} ) $ sulla diagonale principale, e $ 0$ altrove.
		
		Se qualcuno dei $ \lambda_{i}=0  $ allora la matrice non è invertibile, mentre se $ \lambda_{i}\neq 0  $ $ \forall\, i = 1,\cdots, n $ allora $ A $ è invertibile, e vale 
		\[
			A^{-1}=\begin{pmatrix}
				1/\lambda_1 & 0 & \cdots & 0\\
				0 & 1/\lambda_2 & & \vdots\\
				\vdots & & \ddots & \vdots\\
				0 & \hdotsfor{2} & 1/\lambda_{n} 
			\end{pmatrix}
		\]
	\end{itemize}
}
\subsection{Trasposta di una matrice}
Sia $ A \in \R^{m,n} $, si definisce \textit{trasposta} di $ A $, $ \null^{t}\!A $, $ \null^{t}\!A \in \R^{n,m} $, ottenuta scambiando le righe di $ A $ con le colonne di $ A $.

Se $ a_{ij}  $ è l'elemento $ ij $-esimo di $ A $, $ a_{ij}  $ è l'elemento $ ji $-esimo di $ \null^{t}\!A $ 

\esempio{
	\[
		A=\begin{pmatrix}
			1 & 2 & 0 & 3\\
			-1 & 1 & 1 & 2
		\end{pmatrix} \in \R^{2,4}\qquad \null^{t}\!A =\begin{pmatrix}
			1 & -1\\
			2 & 1\\
			0 & 1\\
			3 & 2
		\end{pmatrix}\in \R^{4,2}
	\]	
}

\definizione{}{
	Una matrice $ A \in \R^{n,n} $ è \textit{simmetrica} se $ A= \null^{t}\!A $ 
\[
	\left(\iff\, a_{ij}=a_{ji}\: \forall\,i,j = 1,\cdots,n  \right)
\]}

\esempio{
	$ A $ è simmetrica \[
		A=\begin{pmatrix}
			1 & 2 & 3\\
			2 & 4 & 1\\
			3 & 1 & 5
		\end{pmatrix}
	\]
}

\osservazione{
	Ogni matrice diagonale è simmetrica.
}

\proprieta{}{
	%ESERCIZIO dimostrare queste proprietà
	\begin{enumerate}
		\item $ \null^{t}\!(A+B)= \null^{t}\!A+\null^{t}\!B $ $ \forall\,A, B \in \R^{m,n} $;
		\item $ \null^{t}\!(\lambda\,A)= \lambda \null^{t}\!A $ $ \forall\, \lambda \in \R, A \in \R^{m,n} $;
		\item $ \null^{t}\!(AB)=\null^{t}\!B\null^{t}\!A $ $ \forall\,A, B \in \R^{m,n} $;
		\item se $ A $ è invertibile, allora $ \null^{t}\!A $ è invertibile e $ (\null^{t}\!A)^{-1}=\null^{t}\!(A^{-1}) $.
	\end{enumerate}
}