Suppongo\marginnote{9 nov 2021} che $ V $ e $ W $ abbiano dimensione finita, $ \dim V=n, \dim W=m $. Fisso $ \mathscr{B} $ base di $ V $ e $ \mathscr{C} $ base di $ W $, ogni $ F \in L(V,W) $ induce la matrice $ M^{ \mathscr{B}, \mathscr{C}}(F) \in \K^{m,m} $

Abbiamo quindi una funzione \[
    M^{ \mathscr{B}, \mathscr{C}}:L(V,W)\to \K^{m,m}
\]
\esercizio{La funzione $ M^{ \mathscr{B}, \mathscr{C}} $ è un isomorfismo di spazi vettoriali:
\begin{itemize}
    \item $ M^{ \mathscr{B}, \mathscr{C}} $ è lineare cioè \begin{gather*} M^{ \mathscr{B}, \mathscr{C}}(F+G)=M^{ \mathscr{B}, \mathscr{C}}(F)+M^{ \mathscr{B}, \mathscr{C}}(G)\\ M^{ \mathscr{B}, \mathscr{C}}(\lambda F) = \lambda M^{ \mathscr{B}, \mathscr{C}}(F)\end{gather*}
    \item $ \ker (M^{ \mathscr{B}, \mathscr{C}}) = \underline{0}_{L(V,W)} $ $\implies$ $ M^{ \mathscr{B}, \mathscr{C}} $ è iniettiva
    \item $ M^{ \mathscr{B}, \mathscr{C}} $ è suriettiva, cioè \[
        M^{ \mathscr{B}, \mathscr{C}}\biggl(L(V,W)\biggr)= \K^{m,n}
    \]
\end{itemize}
}{
    Da fare %TODO fare esercizio
}{}

\subsection{Composizione di funzioni lineari}

Siano $ V, W, Z $ spazi vettoriali sullo stesso campo $ \K $.
\[
    V \xrightarrow{F} W \xrightarrow{G} Z \,\implies\, G\circ F:V\to Z \text{ è lineare}
\]

Supponiamo che $ V, W, Z $ abbiano dimensione finita. Siano $ \mathscr{B} $ una base di $ V $, $ \mathscr{C} $ una base di $ W $ e $ \mathscr{D} $ una base di $ Z $. Abbiamo $ M^{ \mathscr{B}, \mathscr{C}}(F) $ e $ M^{ \mathscr{C}, \mathscr{D}}(G) $ matrici, con \[M^{ \mathscr{B}, \mathscr{C}}(F) \cdot  (v)_{ \mathscr{B}}=(F(v))_{ \mathscr{C}}\text{ e }M^{ \mathscr{C}, \mathscr{D}}(G) \cdot  (w)_{ \mathscr{C}}=(G(v))_{ \mathscr{D}}\]

Considero \begin{multline*}
    \bigl((G\circ F)(v)\bigr)_{ \mathscr{D}}=\bigl(G(F(v))\bigr)_{ \mathscr{D}}=\\
    =M^{ \mathscr{C}, \mathscr{D}}(G) \cdot (F(v))_{ \mathscr{C}}=\\=M^{ \mathscr{C}, \mathscr{D}}(G) \cdot M^{ \mathscr{B}, \mathscr{C}}(F) \cdot (v)_{ \mathscr{B}}
\end{multline*} 
cioè la matrice che rappresenta $ G\circ F $ rispetto alle basi $  \mathscr{B} $ e $ \mathscr{D} $ è il prodotto della matrice che rappresenta $ G $ per la matrice che rappresenta $ F $

\esercizio{
    Siano $ F: \R^{4}\to \R^{3}$, $ H:\R^{4}\to \R^{2} $ le funzioni lineari definite come \[
        F\begin{pmatrix}
            x_1\\x_2\\x_3\\x_4
        \end{pmatrix}=\begin{pmatrix}
            x_1+x_2+x_3+x_4\\
            x_2-x_3+3x_4\\
            2x_1+2x_2-x_3-x_4
        \end{pmatrix}
    \]
    \[
        H\begin{pmatrix}
            x_1\\x_2\\x_3\\x_4
        \end{pmatrix}=\begin{pmatrix}
            x_1+2x_2-3x_4\\
            x_1+x_2+x_3-2x_4
        \end{pmatrix}
    \]

    Si determini (se esiste) una funzione lineare $ G:\R^{3}\to \R^{2} $ tale che $ H=G\circ F $
}{
    $ F $ è rappresentata dalla matrice \[
        A=\begin{pmatrix}
            1 & 1& 1& 1\\
            0&1&-1&3\\
            2&2&-1&-1
        \end{pmatrix} \in \R^{3,4}
    \] $ H $ è rappresentata dalla matrice \[
        B=\begin{pmatrix}
            1 & 2 & 0 & -3\\
            1 & 1 & 1 & -2
        \end{pmatrix}
    \]  

    Sia $ C $ la matrice che rappresenta $ G $

    $\implies$ $ C $ soddisfa $ B = C \cdot A $ con $ B $ e $ A $ note e $ C $ matrice incognita. Studio il sistema matriciale, che posso scrivere nella forma $ \null^{t}B=\null^{t}A X $, con $ X=\null^{t}C $
}{}

\osservazione{}{
    Siano $ A \in \K^{m,n} $, $ B \in \K^{p,m} $, con $ BA \in \K^{p,n} $. Si noti che \[\rank BA \le \min\{\rank A, \rank B\} \]

    Motivazione geometrica: $ A $ induce \begin{align*}
    F_{A} : \K^{n} & \to \K^{m}  \\
    x & \mapsto Ax
    \end{align*} e $ \rank A = \dim \text{Im}(F_{A} ) $
    
    $ B $ induce \begin{align*}
        F_{B} : \K^{m} & \to \K^{p}  \\
        x & \mapsto Bx
        \end{align*} e $ \rank B = \dim \text{Im}(F_{B} ) $

    $ BA $ induce \begin{align*}
        F_{BA} : \K^{n} & \to \K^{p}  \\
        x & \mapsto BAx
        \end{align*} e $ \rank BA = \dim \text{Im}(F_{BA} ) $

        Ma $ \text{Im}(F_{BA} ) \subseteq \text{Im}(F_{B}) $, perché $ F_{BA}=F_{B} \circ F_{A}$: \[
            \K^{n}\xrightarrow[]{F_{A}} \K^{m} \xrightarrow[]{F_{B}} \K^{p}
        \] 

        $\implies$ $ \dim \text{Im}(F_{BA} ) \subseteq \dim \text{Im}(F_{B} )  $ $ \,\implies\,$ \[
            \rank BA \le \rank B
        \]

        Si noti che $ \ker F_{A} \subseteq \ker F_{BA}$; per il teorema del rango \[n-\rank A \le n-\rank BA\] 
        
        $\implies$ $ \rank BA \le \rank A $

        Ottengo quindi che \[\rank BA \le \min\{\rank A, \rank B\}\]
}

\definizione{}{
    Sia $ V $ uno spazio vettoriale su un campo $ \K $ e sia $ F: V\to V$ lineare (un automorfismo). $ F $ è \textit{nilpotente} se $ \exists k \in \N $ tale che \[
        \underset{k-\text{volte}}{\underbrace{F\circ F \circ \cdots \circ F}}=0 
    \]
    con $ 0: V\to V $ funzione identicamente nulla
}

\esempio{
    \begin{align*}
    F:\R^{2} & \to \R^{2} \\
    \begin{pmatrix}
        x\\y
    \end{pmatrix} & \mapsto \begin{pmatrix}
        y\\0
    \end{pmatrix}
    \end{align*}

    $ F $ è lineare e $ F\circ F \begin{pmatrix}
        x\\y
    \end{pmatrix} = \begin{pmatrix}
        0\\0
    \end{pmatrix}$, $ F $ è nilpotente.

    $ F $ è rappresentata dalla matrice $ \begin{pmatrix}
        0 & 1 \\0 & 0
    \end{pmatrix}= A $, infatti $ A^{2}= \begin{pmatrix}
        0 & 0\\0 & 0
    \end{pmatrix} $
}
\definizione{}{
    Una matrice $ A \in \K^{n,n} $ tale che $ A^{k} $ è la amtrice nulla per qualche $ k \in \N $ si dice nilpotente
}
\definizione{}{
    Data $ F: V\to V $ lineare nilpotente, $ F $ ha grado di nilpotenza $ k$ se \[
        \underset{k-\text{volte}}{\underbrace{F\circ \cdots\circ F}}=0 \,\land\,\underset{k-1-\text{volte}}{\underbrace{F\circ \cdots\circ F}}\neq0
    \]
}

\esercizio{Si trovi una funzione $ F:\R^{4}\to \R^{4} $ nilpotente con grado di nilpotenza 2}{Da fare %TODO risolvere l'esercizio
}{}

\rule{7em}{.4pt}

$ F: V\to V $ lineare con $ V $ di dimensione finita. Possiamo usare il teorema di nullità più rango:
\[
    \dim  V = \dim \ker (F)+ \dim (\text{Im}(F))
\]
ma in generale \[
    V \neq \ker (F) + \text{Im}(F)
\]

\section{Spazi vettoriali Euclidei}

Si consideri $ V $ spazio vettoriale su $ \R $

\definizione{}{
    Un prodotto scalare su $ V $ è una funzione $ \cdot:V\times V\to \R $ tale che:
    \begin{enumerate}
        \item $ v\cdot w = w \cdot v $ (simmetria)
        \item $ (v_1+v_2)\cdot w = v_1\cdot w + v_2\cdot w $
        \item $ (\lambda v)\cdot w = \lambda(v \cdot w) $
        \item $ v\cdot v \ge 0 $ e $ v\cdot v = 0 $ se e solo se $ v=\underline{0} $ ($ \cdot $ è definito positivo)
    \end{enumerate}
    Si noti che $ \cdot $ è lineare in ogni componente
}

\esempio{}{
    In $ V_{3}  $ è definito il prodotto scalare $ v\cdot w=||v||\,||w||\, \cos \hat{vw}  $. "$\cdot$" definisce un prodotto scalare in $ V_{3}  $
}

% TODO manca il prodotto scalare canonico in \R^{n}

\conseguenza{
    Su ogni spazio vettoriale di dimensione finita esiste un prodotto scalare, infatti se $ V $ è uno spazio vettoriale di dimensione finita, si fissa una base $ \mathscr{B}=\{v_1, \cdots,v_{n} \}$ si definisce per $ x,y \in V $ \[
        x\cdot y = \biggl(\sum_{i=1}^{n}x_{i}v_{i}\biggr)\cdot\biggl(\sum_{j=1}^{n}y_{j}v_{j}\biggr):=\sum_{i=1}^{n}x_{i}y_{i}
    \]
}

\esempio{
Anche $ \R^{m,n} $ ha un prodotto scalare canonico, dato da \[
    A\cdot B := \tr(\null^{t}B \cdot A)
\] 
\begin{enumerate}
\item Per le proprietà della traccia $ \tr(\null^{t}B \cdot A) = \tr(\null^{t}(\null^{t}B \cdot A))= \tr(\null^{t}A B)= B \cdot A$ 

$\implies$ $ \cdot  $ è simmetrico

\item \[(A+C) \cdot B=\tr(\null^{t}B(A+C))=\tr(\null^{t}BA)+\tr(\null^{t}BC) = A \cdot B+C \cdot B \] 

$\implies$ la proprietà 2 è verificata

\item \[
    (\lambda A) \cdot B = \tr(\null^{t}B(\lambda A))= \lambda \tr (\null^{t}BA)=\lambda A \cdot B 
\] 

$\implies$ la proprietà 3 è verificata

\item $ A \cdot A =\tr(\null^{t}A A) $; % TODO manca un pezzo, non ho capito

\end{enumerate}
}

\esempio{}{ 
    Si consideri lo spazio vettoriale \[
        \R_{n} [x]=\{p \in \R[x] | \deg p \le n\}
    \] spazio vettoriale su $ \R $ di dimensione $ n+1 $

    Se $ p,q \in \R_{n} [x] $ possiamo scrivere $ p(x)=\sum_{k=0}^{n}a_{k}x^{k}$, $ q(x)=\sum_{k=0}^{n}b_{k}x^{k}$

    Si definisce $ p \cdot  q $ come \[
        p \cdot q := \sum_{k=0}^{n}a_{k}b_{k}   
    \]
}

\definizione{}{
    Uno \textit{spazio vettoriale euclideo} è uno spazio vettoriale su $\R$ di dimensione finita in cui è stato fissato un prodotto scalare, indicato generalmente con $ (V, \cdot) $
}

In uno spazio vettoriale euclideo è definita la norma di un vettore come \[
    ||v||=\sqrt{v \cdot v}
\]
$ ||v|| $ definisce una funzione $ ||v||:V \to \R_{+} $ dove $ \R_{+}=\{t \in \R | t\ge 0\}  $. Si noti che \[
    ||\lambda v || = \sqrt{(\lambda v) \cdot (\lambda v)}= \sqrt{\lambda^{2} v \cdot  v}=|\lambda|\sqrt{v \cdot v}=|\lambda|\,||v||
\]

\esempio{
    Su $ \R^{3} $ si consideri il prodotto scalare \[x \cdot y = 2 x_1 y_1 + 3x_2 y_2 + 5x_3y_3.\] Rispetto a questo prodotto scalare risulta \[
        ||x||=\sqrt{2x_1^{2}+3x_2^{2}+5x_3^{2}}.
    \]
} 

\definizione{}{
    Due vettori $ v, w $ in uno spazio vettoriale euclideo sono \textit{ortogonali} se $ v \cdot w =0$     
}

\esercizio{
    In $ \R^{3} $ con il prodotto scalare \[x \cdot y = 2 x_1 y_1 + 3x_2 y_2 + 5x_3y_3\] si trovino tutti i vettori ortogonali a $ (1, 1, 0) $
}{Da fare % TODO risolvere l'esercizio
}{}
