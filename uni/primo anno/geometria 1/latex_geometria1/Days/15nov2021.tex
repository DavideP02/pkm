\esercizio{
    In \marginnote{15 nov 2021} $ \R^{3} $ si consideri la base $ \mathscr{B}={v_1,v_2,v_3} $, \[v_1=(1,0,1), v_2=(0,1,1), v_{3}=(2,1,2).\]

    Si applichi l'algoritmo per ottenere una base ortonormale di $ \R^{3} $ rispetto al prodotto scalare \[
        x \cdot y =\sum_{k=1}^{3}x_{k} y_{k}   
    \]
}{
    $ e_1=\frac{v_1}{||v_1||} $, con $ ||v_1||=\sqrt{2} $: quindi \[
        e_1=\biggl(\frac{1}{\sqrt{2}}, 0, \frac{1}{\sqrt{2}}\biggr)
    \]

    $ e_2=\frac{v_2-(v_2 \cdot e_1)e_1}{||v_2-(v_2 \cdot e_1)e_1||} $; $ v_2 \cdot e_1=\frac{1}{\sqrt{2}} $; \begin{multline*} v_2-(v_2 \cdot e_1)e_1 = \\=(0,1,1)-\frac{1}{\sqrt{2}}\biggl(\frac{1}{\sqrt{2}}, 0, \frac{1}{\sqrt{2}}\biggr)=\\=\biggl(-\frac{1}{2}, 1, \frac{1}{2}\biggr)=\frac{1}{2}(-1, 2, 1)\end{multline*}
    \[
        ||v_2-(v_2 \cdot e_1)e_1||=\frac{1}{2}||(-1,2,1)||=\frac{1}{2}\sqrt{6}
    \] 
    \[
        \,\implies\, e_2=\frac{1}{\sqrt{6}}(-1, 2, 1)
    \]

    $ e_3=\frac{v_3-(v_3 \cdot e_2)e_2 - (v_3 \cdot e_1)e_1}{||v_3-(v_3 \cdot e_2)e_2 - (v_3 \cdot e_1)e_1||} $; $ v_3 \cdot e_1 = 4/\sqrt{2}$; $ v_3 \cdot e_2 = 2/\sqrt{6} $
    \begin{multline*}
        v_3-(v_3 \cdot e_2)e_2 - (v_3 \cdot e_1)e_1 =\\
        =(2,1,2)-\frac{2}{\sqrt{6}}\frac{1}{\sqrt{6}}(1, -2,1)-\frac{4}{\sqrt{2}}\frac{1}{\sqrt{2}}(1,0,1)=\\
        =(2,1,2)-\frac{1}{3}(-1,2,1)-2(1,0,1)=\biggl(\frac{1}{3}, \frac{1}{3}, -\frac{1}{3}\biggr)= \\
        =\frac{1}{3}(1,1,-1)
    \end{multline*}
    \[
        e_3=\frac{1}{\sqrt{3}}(1,1,-1)
    \] 
    
    $\implies$ $ \{e_1,e_2,e_3\} $ base ortonormale.
}{}

\subsection{Matrici ortogonali}

\definizione{}{
    Sia $ A \in \R^{n,n} $, $ A $ si dice ortogonale se $ \null^{t}\!A = A^{-1} $ (ortogonale $ \implies $ invertibile)
}

\proposizione{ortomatrx}{
    Valgono le seguenti proprietà:
    \begin{enumerate}
        \item se $ A $ è ortogonale $ \implies $ $ A^{-1} $ è ortogonale
        \item se $ A $ è ortogonale $ \implies $ $ \null^{t}\!A $ è ortogonale
        \item se $ A, B $ sono ortogonali $ \implies $ $ AB $ è ortogonale
        \item se $ A \in \text{O}(n) $ $\implies$ $ \det(A) \in\{-1,1\} $
    \end{enumerate}
}
Indico con \[ \text{O}(n)=\{A \in \R^{n,n}| A\text{ è ortogonale}\} \in \text{GL}(n, \R)\]

\definizione{}{Sia $ (G, \cdot ) $ un gruppo. Un sottoinsieme $ H $ di $ G $ è un sottogruppo se \[
   h^{-1} \in H \, \forall\, h \in H\quad \text{e}\quad h_1 \cdot h_2 \in H \, \forall\,h_1,h_2 \in H 
\]}

Le prime due proprietà implicano che $ \text{O}(n) $ è un sottogruppo di $ \text{GL}(n,\R) $
\dimostrazioneprop{ortomatrx}{
    \begin{enumerate}
        \item $ A \in \text{O}(n) $ 
        
        $\implies$ $ \null^{t}\!A=A^{-1} $ 
        
        $\implies$ $ A=\null^{t}\!(A^{-1})=(\null^{t}\!A)^{-1} $ 
        
        $\implies$ $ A^{-1} \in \text{O}(n) $
        \item $ A \in \text{O}(n) $ 
        
        $\implies$ $ A\null^{t}\!A = \id $ 
        
        $\implies$ $ \null^{t}\!A $ è invertibile e $ A $ è la sua inversa
        \item $ A, B $ ortogonali 
        
        $\implies$ $ A\null^{t}\!A =\id $ e $ B\null^{t}\!B =\id $ 
        
        $\implies$ $ AB\null^{t}\!(AB) = AB\null^{t}\!B\null^{t}\!A=A\null^{t}\!A=\id$ 
        
        $\implies$ $AB\null^{t}\!(AB)=\id$ 
        
        $\implies$ $ AB \in \text{O}(n) $
        \item Se $ A \in \text{O}(n) $ 
        
        $\implies$ $ A\null^{t}\!A=\id $ 
        
        $\implies$ $ \det(\null^{t}\!AA)=1 $ 
        
        $\overset{\text{Binet}}{\implies}$ $ \det (\null^{t}\!A)\det(A)=1 $ 
        
        $\implies$ $ (\det(A))^{2}=1 $ 
        
        $\implies$ $ \det (A) \in \{-1, 1\} $
    \end{enumerate}
}

Ne risulta che \[
    \text{O}(n)=\{A \in \text{O}(n)\,|\, \det (A)=1\}\amalg \{A \in \text{O}(n)\,|\, \det (A)=-11\}
\]
con $ \amalg $ ``unione disgiunta''

Si indica con $ \text{SO}(n)=\{A \in \text{O}(n)\,|\, \det (A)=1\} $ sottogruppo di $ \text{O}(n) $, detto \textit{delle matrici ortogonali speciali}, infatti se $ A, B \in \text{SO}(n) $ 
\begin{itemize}
    \item $ \det (A^{-1}) = \frac{1}{\det A}=1 $ 

    $\implies$ $ A \in \text{SO}(n) $;
    \item $ \det(AB)=\det A \det B = 1 $ 

    $\implies$ $ A,B \in \text{SO}(n) $
\end{itemize}
\teorema{strutturaaaa}{
    Sia $ A \in \R^{n,n} $. Sono fatti equivalenti:
    \begin{enumerate}
        \item $ A \in \text{O}(n) $
        \item Le righe di $ A $, $ R_1,\cdots,R_{n}  $ formano una base ortonormale di $ \R^{n} $ con il prodotto scalare canonico
        \item Le colonne di $ A $, $ C_1,\cdots,C_{n}  $ formano una base ortonormale di $ \R^{n} $ con il prodotto scalare canonico
    \end{enumerate}
}
\dimostrazione{strutturaaaa}{
    \begin{itemize}
        \item [$2\iff 3$] poiché $ A \in \text{O}(n)\,\iff\,\null^{t}\!A \in \text{O}(n) $
        \item [$2\iff 1$] Infatti $ A \in \text{O}(n) $ $ \iff $ $ A \null^{t}\!A =\id $, $ A=\begin{pmatrix}
            R_1\\\vdots\\R_{n} 
        \end{pmatrix} $, $ \null^{t}\!A=\begin{pmatrix}
            R_1 & \cdots & R_{n} 
        \end{pmatrix} $

        $ (A \null^{t}\!A)_{ij}=R_{i} \cdot R_{j}$ dove $ \cdot  $ è il prodotto scalare canonico in $ \R^{n} $.

        Quindi 
        
        $
            A \null^{t}\!A=\id\,\iff\, R_{i} \cdot R_{j} =\delta_{ij} 
        $

        $\iff\,\{R_1, \cdots,R_{n} \}$ base ortonormale di $(\R^{n}, \cdot )$\qed
    \end{itemize}
}

Descriviamo $ \text{O}(2) $ e $ \text{SO}(2) $

Descrivo tutte le basi ortonormali di $ (\R^{2}, \cdot ) $, una base ortonormale è della forma $ \mathscr{B}=\{e_1,e_2\} $ con $ ||e_1||=||e_2||=1 $, e $ e_1 \cdot e_2 = 0$

Fissiamo $ e_1 $. Sia $ \alpha $ l'angolo tra $ e_1 $ e l'asse $ x $ 

$\implies$ $ e_1=(\cos\alpha, \sin\alpha) $

Per $ e_2 $ ho solo le due possibilità:
\begin{itemize}
    \item $ e_2=(-\sin\alpha, \cos\alpha) $
    \item $ e_2=(\sin\alpha, -\cos\alpha) $
\end{itemize}

Ogni $ A \in \text{O}(2) $ è della forma \[
    A=\begin{pmatrix}
        \cos\alpha & \sin\alpha \\
        \sin\alpha & -\cos\alpha
    \end{pmatrix}
\]
oppure
\[
    A=\begin{pmatrix}
        \cos\alpha & \sin\alpha \\
        -\sin\alpha & \cos\alpha
    \end{pmatrix} \in \text{SO}(2)
\]

\teorema{dkkkdkkdkkkk}{
    Considero $ (V, \cdot ) $ spazio vettoriale Euclideo, e $ \mathscr{B}=\{e_1,\cdots, e_{n} \} $ base ortonormale. Sia $ \mathscr{B}' $ una seconda base

    $ \mathscr{B}'  $ è ortonormale $ \iff $ la matrice del cambiamento di base da $ \mathscr{B} $ a $ \mathscr{B}' $ è ortogonale
}
\dimostrazione{dkkkdkkdkkkk}{
    $ \mathscr{B}=\{e_1, \cdots, e_{n} \} $ ortonormale, $ \mathscr{B}'=\{v_1,\cdots,v_{n} \} $
    \[
        v_{i} =\sum_{j=1}^n a_{ij} e_j \,\forall\, i =1,\cdots,n
    \]
    $ \mathscr{B}' $ è ortonormale se e solo se \[
        v_{i} \cdot v_{j} =\delta_{ij}   \,\forall\, i,j =1,\cdots,n
    \]

    Risulta \begin{multline*}
        v_{i} \cdot v_{j}=\biggl(\sum_{k=1}^{n}a_{ik} e_{k}\biggr) \cdot \biggl(\sum_{s=1}^{n}a_{js} e_{s}\biggr)=\\   
        =\sum_{k=1}^{n}\sum_{s=1}^{n}a_{ik}a_{js}\underset{\delta_{ks} }{\underbrace{e_{k}e_{s}}} = \sum_{k=1}^{n}a_{ik}a_{jk}   
    \end{multline*}

    Quindi $ \mathscr{B}' $ è ortonormale 
    
    $ \iff $ $\sum_{k=1}^{n}a_{ik}a_{jk} =\delta_{ij}$

    $\iff$ le righe della matrice $ A=(a_{ij} ) $ formano una base ortonormale di $ \R^{n} $ con il prodotto scalare canonico

    $ \underset{\text{per il teorema}}{\iff} $ $ A $ è ortogonale

    $ \iff $ la matrice del cambiamento di base da $ \mathscr{B} $ a $ \mathscr{B}' \in \text{O}(n)$\qed
}

\esercizio{
    Si trovi in $ \R^{3}$ con il prodotto scalare canonico una base ortonormale il cui primo vettore sia \[u=\biggl(\frac{\sqrt{2}}{2}, 0,\frac{\sqrt{2}}{2}\biggr)\]
}{
    Approccio risolutivo: trovo $ v $ in $ \R^{3} $ ortogonale a $ u $, con $ ||v||=1 $ e quindi si considera $ z $ come $ z=u\wedge v $ 
    
    $\implies$ $ \{u,v,z\} $ base ortonormale di $ \R^{3} $
}{}

\section{Orientazione di uno spazio vettoriale (reale)}

Sia $ V $ spazio vettoriale su $ \R $ finitamente generato, siano $ \mathscr{B} $ e $ \mathscr{B}' $ due basi e sia $ M^{ \mathscr{B}, \mathscr{B}'} $ la matrice del cambiamento di base $  M^{ \mathscr{B}, \mathscr{B}'} \in \text{GL}(n,\R) $ 

$\implies$ $ \det ( M^{ \mathscr{B}, \mathscr{B}'})\neq 0 $ 

$\implies$ ci sono due possibilità: \begin{enumerate}
    \item $ \det ( M^{ \mathscr{B}, \mathscr{B}'})>0 $: si dice che $ \mathscr{B} $ e $ \mathscr{B}' $ hanno la stessa orientazione
    \item $ \det ( M^{ \mathscr{B}, \mathscr{B}'})<0 $: si dice che $ \mathscr{B} $ e $ \mathscr{B}' $ hanno orientazione opposta
\end{enumerate}

\esempio{}{
    Consideriamo $ \R $ come spazio vettoriale su $ \R $. Le basi di $ \R $ sono del tipo $ \mathscr{B}=\{t_0\} $, con $ t_{0} \neq 0 $ 
    
    $\implies$ $ \mathscr{B}=\{t_{0} \} $ e $ \mathscr{B}' =\{t_{0}' \}$ hanno la stessa orientazione 

    $ \iff $ $ t_0 $ e $ t_0' $ hanno lo stesso segno
}

\rule{7em}{.4pt}

Sia $ V $ spazio vettoriale su $ \R $ finitamente generato. $ \text{Basi}(V)$ insieme di tutte le basi di $ V $.
In $ \text{Basi}(V) $ si considera la relazione $ \sim $ dove due basi $ \mathscr{B} $ e $ \mathscr{B}' $ sono in relazione 

$ \iff $ hanno la stessa orientazione
\[
    \mathscr{B}\sim \mathscr{B}'\,\iff\, \det ( M^{ \mathscr{B}, \mathscr{B}'})>0
\]

\proposizione{classibosldo}{
    $ \sim $ è una relazione di equivalenza e nel quoziente $ \text{Basi}(B)/\sim $ ci sono solo due classi
}
\esempio{}{
    $ \text{Basi}(\R)=\big\{\{t_0\}\,|\, t_{0}\neq 0 \big\} $, $ \{t_0\}\sim\{t_0'\} $ $ \iff $ $ t_0 $ e $ t_0' $ hanno lo stesso segno.
    
    $ \sim $ è una relazione di equivalenza, e $ \text{Basi}(R)/\sim$ consta di sole due classi, infatti, prendendo le classi \[
        [\{1\}], [\{-1\}],
    \]
    una qualsiasi base $ \{t_0\} $ o è in relazione con $ [\{1\}] $ o con $ [\{-1\}] $
}
\dimostrazioneprop{classibosldo}{
    Idea della dimostrazione: dimostro che $ \sim $ è riflessiva, simmetrica e transitiva.
    \begin{itemize}
        \item $ \sim $ riflessiva, infatti se $ \mathscr{B} \in\text{Basi}(V) $ si ha che $ M^{ \mathscr{B}, \mathscr{B}}=\id $ 
    
        $\implies$ $ \det M^{ \mathscr{B}, \mathscr{B}}=1 $ 
        
        $\implies$ $ \mathscr{B}\sim\mathscr{B} $

        \item $ \sim $ simmetrica, infatti siano $ \mathscr{B} $ e $ \mathscr{B}' \in \text{Basi}(V)$, supponiamo $ \mathscr{B}\sim \mathscr{B}' $, cioè $ \det M^{ \mathscr{B}, \mathscr{B} }> 0 $ 
        
        $\implies$ $ (v)_{ \mathscr{B}}= M^{ \mathscr{B}, \mathscr{B}'} (v)_{ \mathscr{B}'} $ $ \forall\,v \in V $

        $ \iff $ $(v)_{ \mathscr{B}'}= (M^{ \mathscr{B}, \mathscr{B}'})^{-1} (v)_{ \mathscr{B}} $ $ \forall\,v \in V $, cioè $ M^{ \mathscr{B}', \mathscr{B}} = (M^{ \mathscr{B}, \mathscr{B}})^{-1} $ 
        
        $\implies$ $ \det (M^{ \mathscr{B}', \mathscr{B}}) =\frac{1}{\det M^{ \mathscr{B}, \mathscr{B}'}}>0$ 
        
        $\implies$ $ \mathscr{B}'\sim \mathscr{B} $

        \item $ \sim $ transitiva, infatti siano $ \mathscr{B}, \mathscr{B}', \mathscr{B}'' \in \text{Basi}(V) $ con 
        \begin{itemize}
            \item $ \mathscr{B}\sim \mathscr{B}' $ $ \iff $ $ \det (M^{ \mathscr{B}, \mathscr{B}'})>0 $
            \item $ \mathscr{B}'\sim \mathscr{B}'' $ $ \iff $ $ \det (M^{ \mathscr{B}', \mathscr{B}''})>0 $
        \end{itemize}
        \begin{multline*}
            \det (M^{ \mathscr{B}, \mathscr{B}''})=\det (M^{ \mathscr{B}, \mathscr{B}'} \cdot M^{ \mathscr{B}', \mathscr{B}''})=\\
            \overset{\text{Binet}}{=}\det (M^{ \mathscr{B}, \mathscr{B}'}) \cdot \det (M^{ \mathscr{B}', \mathscr{B}''})>0\qed
        \end{multline*}
    \end{itemize}
    
    %TODO dimostrare per esercizio che le classi siano solo due
}