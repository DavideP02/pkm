\fancypagestyle{firststyle}
{
	\renewcommand{\headrulewidth}{0pt}
    \fancyhead[LE, RO]{\underline{20 settembre 2021}}
}

\thispagestyle{firststyle}

\section{Matrici}

Una matrice è una tabella rettangolare di numeri reali ($\in\R$)

\[
A=\begin{pmatrix}
a_{1 1} & a_{1 2} & \cdots & a_{1 n} \\
a_{2 1} & a_{2 2} & \cdots & a_{2 n}\\
\vdots & \vdots & \vdots & \vdots \\
a_{m 1} & \cdots & \cdots & a_{m n}
\end{pmatrix}\qquad \begin{aligned}
\text{contiene } &m\cdot n \text{ numeri}\\
\text{contiene } &m \text{ righe}\\
\text{contiene } &n \text{ colonne}
\end{aligned}
\]

$a_{ij}$ è l'elemento della matrice nella $i$-esima riga e nella $j$-esima colonna. $a_{ij}\in\R$.

$A$ è una matrice $m\cdot n$. Se $m=n$ allora $A$ è una \textbf{matrice quadrata}.

Le matrici servono per:
\begin{itemize}
\item risolvere sistemi lineari
\item studiare spazi vettoriali
\item classificarre strutture geometrice (es. coniche)
\item presentare funzioni (semplificandone lo studio)
\end{itemize}

$\R^{m,n}$ è l'insieme delle matrici $m\cdot n$:
\begin{itemize}
\item $\Q^{m, n}$ è l'insieme delle matrici $m\cdot n$ le cui entrate sono elementi di $\Q$.
\end{itemize}

\esempi{
\begin{itemize}
\item $\R^{2,2}$: matrici $2\cdot 2$
	\[
	\begin{pmatrix}
		a_{11} & a_{12} \\
		a_{21} & a_{22}
	\end{pmatrix},\quad
	\begin{pmatrix}
		1 & 2 \\
		0 & 3
	\end{pmatrix},\quad
	\begin{pmatrix}
		5 & 6 \\
		-1 & \frac{1}{2}
	\end{pmatrix}\cdots\in\R^{2,2}
	\]
\item $\R^{1,1}=\R$
\item $\R^{m,1}$:
	\[
	A=\begin{pmatrix}
	a_{11} \\
	a_{21} \\
	\vdots \\
	\vdots \\
	a_{m1}
	\end{pmatrix}\in\R^{m,1}\qquad\text{anche \textbf{vettori colonna}}
	\]
\item $\R^{1,n}$:
	\[
	A=\begin{pmatrix}
	a_{11} &
	a_{12} &
	\cdots&
	a_{1n}
	\end{pmatrix}\in\R^{1,n}\qquad\text{anche \textbf{vettori riga}}
	\]
\end{itemize}
}

In  $\R^{m,n}$ è sempre definita la \textbf{matrice nulla}, in cui tutte le entrate sono nulle. In $\R^{n,n}$ è sempre definita la \textbf{matrice identità}:
\[
I=\begin{pmatrix}
1 & 0 & \cdots & \cdots & \cdots & 0 \\
0 & 1 & 0 & \cdots & \cdots & 0 \\
0 & 0 & 1 & 0 &\cdots & 0 \\
\vdots & & & \ddots & & \vdots \\
\vdots & & & & \ddots & \vdots \\
0 & \cdots & \cdots & \cdots & 0 & 1
\end{pmatrix}
\]
\begin{itemize}
\item In $\R^{1,1}$, $I=1$
\item In $\R^{2,2}$
\[
I=\begin{pmatrix}
1 & 0 \\
0 & 1
\end{pmatrix}
\]
\item In $\R^{3,3}$
\[
I=\begin{pmatrix}
1 & 0 & 0 \\
0 & 1 & 0 \\
0 & 0 & 1
\end{pmatrix}
\]
\end{itemize}

La diagonale composta unicamente da $1$ nella matrice identità è ila \textbf{diagonale principale} della matrice.

\subsection{Somma}

Siano $A, B \in \R^{m,n}$
\[
A=\qmatrice{a}\qquad B=\qmatrice{b}
\]
\[
A+B=\begin{pmatrix}
a_{11}+b_{11} & a_{12}+b_{12} & \cdots & a_{1n}+b_{1n} \\
\vdots & & & \vdots \\
a_{m1}+b_{m1} & \hdotsfor{2} & a_{mn}+b_{mn}
\end{pmatrix}
\]

\esempi{
\begin{itemize}
\item In $\R^{1,1}$ la somma tra matrici coincide con la somma usuale di numeri reali.
\item $\displaystyle \begin{pmatrix}
1 & 2 & 3 \\
0 & -1 & 4
\end{pmatrix}+\begin{pmatrix}
0 & -2 & 1 \\
3 & -1 & 4
\end{pmatrix}=\begin{pmatrix}
1 & 0 & 4 \\
3 & -2 & 8
\end{pmatrix}$
\end{itemize}
}

\proprieta[della somma]{
\begin{itemize}
\item [(\textit{i})] La somma è \textbf{associativa}:\[\forall A,B,C\in\R^{m,n} \qquad (A+B)+C=A+(B+C)\] e posso scrivere $A+B+C$ senza ambiguità.
\item [(\textit{ii})] La somma è \textbf{commutativa} (o abeliana):
\[
\forall A,B\in\R^{m,n}\qquad A+B=B+A
\]
\item [(\textit{iii})] Se $A\in\R^{m,n}$ e $B\in\R^{m,n}$ è la matrice nulla ($B=\underline{0}$), allora $A+B=B+A=A$
\item [(\textit{iv})] $A-A=\underline{0}$: 
\[
\forall A \in \R^{m,n} \exists -A\in \R^{m,n} \:\tc\: A-A=0
\]
\definizione{
Data $A\in\R^{m,n}$,
\[
\text{con }A=\qmatrice{a}
\]
si definisce $-A$,
\[
\text{con }-A=\qmatrice{-a}
\]
}
\notazione{
In genere si scrive $A-B$ in luogo di $A+(-B)$, e si considera come una sottrazione di matrici
}
\end{itemize}}

\definizione{
Due matrici $A,B\in\R^{m,n}$ sono uguali se hanno le stesse entrate ($A=B$)
}
\proprieta{$A=B\iff B-A=0$}

\section{Gruppo}

\definizione{Siano $A, B$ due insiemi, si definisce \textbf{prodotto cartesiano}:
\[
A\times B=\{(a,b)\:\tc\: a\in A, b\in B\}
\]
in cui conta l'ordine: $(a,b)\neq_(b,a)$

\[
A\times A = \{(a_1, a_2)\:\tc\: a_1, a_2\in A\}
\]}

\definizione{Sia $G$ un insieme. Una \textbf{operazione} in $G$ è una funzione
\begin{align*}
\star : G\times G &\to G\\
(g, h) &\mapsto g\star h
\end{align*}}
\proprieta{
\begin{itemize}
\item[(\textit{i})] L'operazione è \textbf{associativa} se $(g\star h)\star k=g\star (h\star k)$
\item[(\textit{ii})] L'operazione ha un \textbf{elemento neutro} se \[\exists\, e\in G \:\tc\: g\star e=e\star g = g, \:\forall g \in G\]
\item [(\textit{iii})] Se $g\in G$ chiamiamo \textbf{inverso di $g$} un elemento
\[
k\in G \:\tc\: g\star k=k\star g = e
\]
\end{itemize}
}
\definizione{
Un \textbf{gruppo} è un insieme $G$ con un'operazione~$\star\:\tc$
\begin{enumerate}
\item $\star$ è associativa
\item esiste un elemento neutro
\item ogni elemento ha un inverso
\end{enumerate}
}

\esempi{Sono gruppi
\[
(\R, +),\: (\Z, +),\: (\Q, +),
\]
\[
\cancel{(\R, \cdot)}\text{: lo zero non ha un inverso},\]
\[
(\R\setminus\{0\}, \cdot),\: (\R^{m,n}, +)
\]
}
\definizione{Un gruppo $(G, \star)$ è \textbf{abeliano} se
\[
g\star h=h\star g \:\forall\: g,h\in G
\]
Nel caso di un gruppo abeliano l'operazione è indicata con $+$ e l'elemento neutro con $0$.
\[
(\R^{m,n}, +)\text{ è un gruppo abeliano}
\]}