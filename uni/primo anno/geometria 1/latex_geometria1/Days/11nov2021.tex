\proposizione{norma}{
    La\marginnote{11 nov 2021} norma associata ad un prodotto scalare ha le seguenti proprietà:
    \begin{enumerate}
        \item $ ||v||\ge 0 $ e $ ||v||=0 $ $\iff$ $v=\underline{0}$
        \item $ ||\lambda v|| = |\lambda| ||v|| $
        \item Teorema di Pitagora:\\ Siano $ v, w \in V $. $ vw=0 $ $\iff$ $ ||v+w||^{2}=||v||^{2}+||w||^{2} $
        \item Disuguaglianza di Cauchy-Swartz: $ |v \cdot w|\le ||v||\,||w|| $
        
        L'uguaglianza vale $ \iff $ $ v $ e $ w $ sono linearmente dipendenti
        \item Disuguaglianza triangolare: $ ||v+w||\le ||v||+||w|| $
    \end{enumerate} 
}
\osservazione{
    $ \R $ è uno spazio vettoriale su $ \R $ di dimensione 1. 
    
    $ \cdot: \R\times \R \to \R $, $ x\cdot y = xy $, dove a destra vi è la moltiplicazione in $ \R $. $ \cdot  $ è un prodotto scalare.

    Si noti che \[
        ||x||=\sqrt{x*x}=\sqrt{x^{2}}=|x|
    \]
    La 5. è coerente con la disuguaglianza triangolare soddisfatta dal valore assoluto in $ \R $
}
\dimostrazioneprop{norma}{
    \begin{enumerate}
        \item [1. 2.] già viste
        \item [3.] si considera $ ||v+w||^{2} $ \begin{multline*}
            ||v+w||^{2}=(v+w) \cdot (v+w) = \\ =v \cdot v+v \cdot w+ w \cdot v +W \cdot w =\\
            = ||v||^{2} + 2 v \cdot w+||w||^{2}
        \end{multline*}
        Segue la proprietà

        \item [4.] Sicuramente la formula vale se $ v=\underline{0} $ o $ w=\underline{0} $. Supponiamo $ v,w \neq \underline{0} $. Per $ \lambda \in \R  $ sia $ p(\lambda)=||\lambda v+w||^{2} $
        \[
            p(\lambda)=(\lambda v + w) \cdot  (\lambda v+w)= \lambda^{2}||v||^{2}+2\lambda v \cdot w + ||w||^{2}
        \]
        
        $\implies$ $ p(\lambda) \in \R_{2}[\lambda]  $

        Sappiamo che $ p(\lambda)\ge 0 $ $ \forall\, \lambda \in \R $ 
        
        $\implies$ il suo $ \Delta $ soddisfa $ \Delta\le 0 $. \[
            \Delta=4(v \cdot w)^{2}-4 ||v||^{2}||w||^{2}
        \]

        Si ottiene che $ (v \cdot w)^{2}\le ||v||^{2}||w||^{2} $ 
        
        $\implies$ $ |v \cdot w|\le ||v||\,||w|| $

        Vale l'uguaglianza $ \iff $ $ \Delta=0 $, quindi se $ \exists \lambda \in \R $ per cui $ p(\lambda)=0 $

        $ p(\lambda)=0 $ $ \iff $ $ ||\lambda v+w|| = 0 $ $ \iff $ $ \lambda v+w = \underline{0} $ 

        $\iff$ $ v $ e $ w $ sono linearmente dipendenti
        \item [5.] $ ||v+w||^{2}=||v||^{2}+||w||^{2}+2\, v \cdot w $. 
        
        Per Cauchy-Swartz $ |v \cdot w|\le ||v||\,||w|| $ 
        
        $\implies$ $ -||v||\,||w||\le|v \cdot w|\le ||v||\,||w|| $ 
        
        $\implies$ $||v+w||^{2}\le ||v||^{2}+||w||^{2}+2\,||v||\,||w||=(||v||\,||w||)^{2}$ 
        
        $\implies$ $ ||v+w||\le ||v||\,||w|| $\qed
    \end{enumerate}
}

\paragraph{Applicazione di Cauchy-Swartz}
Siano $ v, w \in V $, $ v, w \neq 0$: so che $ |v \cdot w| \le ||v||\,||w|| $ 
\[
\implies\, -1\le\frac{v \cdot w}{||v||\,||w||}<1  
\] 

$\implies$ $ \exists\, \theta \in [0, \pi] $ tale che $ \frac{v \cdot w}{||v||\,||w||}=\cos \theta\displaystyle $ 
\[
  \theta=\arccos \biggl(\frac{v \cdot w}{||v|| \cdot ||w||} \biggr)
\]

$ \theta $ è per definizione l'angolo tra $ v $ e $ w $, e dipende dal prodotto scalare considerato

\osservazione{}{
    \begin{itemize}
        \item Se considero $ V_3 $, il prodotto scalare è stato definito come $ v \cdot w = ||v|| \cdot ||w|| \cdot \cos \hat{vw} $. Anche in questo caso l'angolo che formano i due vettori è 
        \[
          \hat{vw}=\arccos \biggl(\frac{v \cdot w}{||v|| \cdot ||w||} \biggr)
        \]
        \item Se $ v,w \in V $ e $ v, w \neq 0 $ si dicono ortogonali se $ v \cdot w = 0 $ $ \iff $ l'angolo formato dai due vettori sia $ \frac{\pi}{2} $
    \end{itemize}
}

\definizione{}{
    Se $ A $ è un insieme si definisce una \textit{distanza} su $ A $ come una funzione  $ d:A\times A \to \R_{+}  $ che soddisfa le seguenti proprietà
    \begin{enumerate}
        \item $ d(a,b)=0 $ $ \iff $ $ a=b $
        \item $ d(a,b)=d(b,a) $
        \item $ d(a,c)\le d(a,b)+d(b,c) $\hspace{1em} (Disuguaglianza triangolare)
    \end{enumerate}

    $(A, d)$ si dice uno \textit{spazio metrico}.
}

\esempio{
    $ \R $ con la distanza $ d(x,y)=|x-y| $
}

Se $ (V, \cdot ) $ è uno spazio vettoriale Euclideo si definisce $ d:V\times V\to \R_{+} $ \[
    d(v, w)=||v-w||.
\]
Per la proposizione precedente $ d $ definisce una distanza su $ V $

\definizione{}{
    Se $ v \in V $ con $ v \neq \underline{0}$, il versore di $ v $ è il vettore \[
        \text{vers}(v):=\frac{v}{||v||}
    \] 
}
\osservazione{
    $ ||\text{vers}(v)|| = \big|\big|\frac{v}{||v||}\big|\big|=\frac{||v||}{||v||}=1$

    $ v $ ha la stessa direzione, stesso verso di $ v $ ma norma 1
}

\subsection{Basi ortogonali e Basi ortonormali}

$ (V, \cdot ) $ uno spazio vettoriale Euclideo, $ \mathscr{B}=\{v_1, \cdots, v_{n}\}  $ una base.
\begin{itemize}
    \item $ \mathscr{B} $ è ortogonale se $ v_{i} \cdot v_{j} = 0 $ $ \forall\, i \neq j $
    \item $ \mathscr{B} $ è ortonormale se è ortogonale e tutti i vettori della base hanno norma 1 \[
        v_{i} \cdot v_{j}= \begin{cases}
            1 & i=j \\
            0 & i \neq j
        \end{cases}  
    \]
\end{itemize}

In generale si scrive $ \delta_{ij} $ per indicare
\[
        \delta_{ij}= \begin{cases}
            1 & i=j \\
            0 & i \neq j
        \end{cases}  
    \]
    e prende il nome di ``Delta di Knonecker''

\esempi{}{
    \begin{itemize}
        \item La base canonica in $ \R^{n} $ è ortonormale rispetto al prodotto scalare standard ($x \cdot y =\sum_{k=1}^{n}x_{k} y_{k}$)
        \item In $ \R^{m,n} $ la base canonica $ {E_{ij} } $ è ortonormale rispetto al prodotto scalare: \[
            A \cdot B=\tr (\null^{t}B A)
        \]
    \end{itemize}
}

\esercizio{
    In $ \R^{3} $ con il prodotto scalare \[x\cdot y = 5x_1y_1+3x_2y_2+4x_3y_3\] si trovi una base ortonormale
}{
    $ \mathscr{B}=\{e_1, e_2, e_3\} $ base canonica di $ \R^{3} $
    \begin{align*}
        e_1 \cdot e_1 & = 5\\
        e_1 \cdot e_2 & = 0\\
        e_1 \cdot e_3 & = 0\\
        e_2 \cdot e_2 & = 3\\
        e_2 \cdot e_3 & = 0\\
        e_3 \cdot e_3 & =4
    \end{align*}

    $ \mathscr{B} $ è una base ortogonale, ma non ortonormale. \[
        \mathscr{B}'=\bigg\{\frac{1}{\sqrt{5}}e_1,\frac{1}{\sqrt{3}}e_2,\frac{1}{2}e_3 \bigg\}
    \]
    è una base ortonormale
}{}

\esercizio{
    Sia $ (V, \cdot ) $ uno spazio vettoriale Euclideo, sia $ \{v_1, \cdots, v_{l} \} \subseteq V $ tale che $ v_{i} \cdot v_{j} =\delta_{ij}$ $ \forall\,i,j=1, \cdots, l $

    Si dimostri che $ \{v_1, \cdots, v_{l} \} $ sia sempre libero.

    ($\implies$ $l\le \dim  V$, $ l=\dim V \iff \{v_1, \cdots, v_{l} \} $ è una base)
}{
    Suppongo \[\lambda_1 v_1 + \lambda_2 v_2+ \cdots+\lambda_{l} v_{l}=0\] e dimostro $ \lambda_{i} =0 $ $ \forall \, i=1,\cdots,l $

    \begin{gather*}
        (\lambda_1 v_1 + \lambda_2 v_2+ \cdots+\lambda_{l} v_{l}) \cdot v_{i} =0\\
        \lambda_1v_1v_{i}+ \lambda_2v_2v_{i}  + \cdots+\lambda_{l}v_{l}v_{i}=0   
    \end{gather*} 
    
    $\implies$ $ \lambda_{i}=0  $ 
    
    $\implies$ $ \lambda_1=\lambda_2=\cdots=\lambda_3=0 $ \qed
}{}

Sia $ (V, \cdot ) $ spazio vettoriale Euclideo e $ \mathscr{B}=\{e_1, \cdots, e_{n}\}  $ base ortonormale di $ V $ rispetto a $ \cdot  $

Sia $ v \in V $, \[
    v=\sum_{k=1}^{n}x_{k}e_{k}  
\]
quindi
\[
    v \cdot e_{r} =\sum_{k=1}^{n}x_{k}(e_{k} e_{r}) 
\] 

$\implies$ $ x_{r}=v \cdot e_{r}$

Rispetto ad una base ortonormale ogni $ v $ si scrive come
\[
    v=\sum_{k=1}^{n}(v \cdot e_{k} )e_{k}
\]

\teorema[(Gram-Schmidt)]{algoritmograamsmith}{
    Sia $ (V, \cdot ) $ uno spazio vettoriale Euclideo e sia $ \mathscr{B}=\{v_1, \cdots,v_{n}\}$ una base. 

    Esiste $ \mathscr{B}'=\{e_1, \cdots, e_{n}\}  $ base ortonormale di $ (V, \cdot ) $ tale che \[ \mathscr{L}(v_1, \cdots, v_{k} )= \mathscr{L}(e_1, \cdots, e_{k})\quad \forall\, k=1, \cdots, n \]
}
\dimostrazione{algoritmograamsmith}{
    La dimostrazione corrisponde all'\textit{algoritmo di Gram-Schmidt}
    $\mathscr{B}=\{v_1, \cdots,v_{n}\}$. Per $ e_1 $ non ho facoltà di scelta: \[
        e_1=\frac{v_1}{||v_1||}
    \]

    Sia $ e_2'=v_2-(v_2 \cdot e_1)e_1 $, si noti che $ e_2' \cdot e_1 = v_2 \cdot e_1 - v_2 \cdot 1 = 0 $
    
    $ e_2' $ è ortogonale a $ e_1 $; $ \mathscr{L}(e_1, e_2')= \mathscr{L}(v_1, v_2) $. %TODO capire e chiarire meglio il perché
    Ad $ e_2' $ manca solo la proprietà di avere norma 1
 
    A questo punto si può definire $ e_2 $ come \[
        e_2=\frac{v_2-(v_2 \cdot e_1)e_1}{||v_2-(v_2 \cdot e_1)e_1||}
    \]

    Itero fino ad ottenere
    \[
        e_{j}=\frac{v_{j} -\sum_{i=1}^{j-1}(v_{j} \cdot e_i ) e_i}{||v_{j} -\sum_{i=1}^{j-1}(v_{j} \cdot e_i ) e_i||} \qedd
    \]
}

