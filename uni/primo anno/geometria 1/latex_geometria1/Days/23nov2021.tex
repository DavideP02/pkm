\proposizione{lkmlkmlkml}{\marginnote{23 nov 2021}
    \begin{enumerate}
        \item Se $ A \in U(n)$ $\implies$ $ \null^{t}\!A \in U(n) $.
        \item Data $ A \in \C^{n,n} $, $ A \in U(n) $ 
        
        $ \iff $ le colonne di $ A $ formano una base unitario di $ C^{n} $ con il prodotto Hermitiano canonico.
    \end{enumerate}
}

\definizione{}{
Dato $ (V, \cdot ) $ spazio vettoriale Hermitiano, $ W \subseteq V $ sottospazio vettoriale, si definisce \[
    W^{\bot}=\{v \in V\,|\, v \cdot w =0\, \forall\, w \in W\}
\] chiamato complemento ortogonale di $ W $}

\proposizione{ddddd}{
    \[
        V=W \oplus W^{\bot}
    \]
}

\definizione{}{
$ F:V\to V $ funzione lineare su $ (V, \cdot) $ è una \textit{isometria} se \[
    F(v) \cdot F(w) = v \cdot w\, v,w \in V
\]
}

$ F $ isometria $ \implies $ $ F \in \text{Aut}(V) $, definendo \[
    \text{Iso}(V, \cdot)=\{F:V \to V\,|\, F\text{ isometria}\}
\] si ha che $ \text{Iso}(V, \cdot) $ è sottogruppo di $ \text{Aut}(V) $ con l'operazione di composizione tra funzioni.

Se si fissa $ \mathscr{B} $ base unitario di $ (V, \cdot ) $, $ M^{ \mathscr{B}, \mathscr{B}}(F) $, 

$ F $ isometria $ \iff $ $ M^{ \mathscr{B}, \mathscr{B}}(F) \in U(n) $

Si definisce quindi un isomorfismo \begin{align*}
\Phi: \text{Iso}(V, \cdot) & \to  U(n)\\
F & \mapsto M^{ \mathscr{B}, \mathscr{B}}(F)
\end{align*}

\rule{7em}{.4pt}

Siano $ (V, \cdot ) $ e $ (W, \cdot ) $ spazi vettoriale Hermitiani, $ F: V \to W $ lineare, allora esiste un'unica funzione lineare $ F^{*}: W \to V$ lineare, tale che \[
    F(v) \cdot w = v \cdot F^{*}(w)\quad \forall\,v \in V, w \in W
\] 
$ F^{*} $ si dice l'aggiunta di $ F $.

Se $ \mathscr{B} $ base unitaria di $ (V, \cdot ) $ e $ \mathscr{C} $ base unitaria di $ (W, \cdot ) $ 

$\implies$ $ F^{*} $ soddisfa $ M^{ \mathscr{C}}(F^{*})=\overline{\null^{t}\!M^{ \mathscr{B}, \mathscr{C}}(F)} $

\definizione{}{
    $ F \in \text{End}(V) $ è \textit{autoaggiunta} se $ F=F^{*} $
}

Se $ \mathscr{B} $ è una base unitaria di $ (V, \cdot ) $, $ F \in \text{End}(V) $ è autoaggiunta 
    
    $ \iff $ $\overline{\null^{t}\!M^{ \mathscr{B}, \mathscr{B}}(F)}=M^{ \mathscr{B}, \mathscr{B}}(F)$

\definizione{}{Una matrice $ A \in \C^{n,n} $ è \textit{Hermitiana} se \[
    \null^{t}\!\overline{A}=A
\]
\[
    \text{Herm}(n)=\{A \in \C^{n,n}\,|\, \null^{t}\!\overline{A}=A\}
\]
$ \text{Herm}(n) $ è un sottospazio vettoriale di $ \C^{n,n} $ su campo $ \R $. \todo{Capire se è sottospazio vettoriale di $ \C^{n,n} $ o di $ \C^{n} $}
}

\esercizio{
    Calcolare $ \dim \text{Herm}(n) $
}{
    Da fare %ESERCIZIO fare esercizio
}{}

\section{Autovalori e autovettori}

Sia $ V $ uno spazio vettoriale su un campo $ \K $, $ F \in \text{End}(V) $ i.e. $ F:V\to V $ lineare.

\definizione{}{
    $ v \in V $ è un autovettore di $ F $ se $ v \neq \underline{0} $ e \[
        \exists\, \lambda \in \K\,\tc\, F(v)= \lambda v
    \]
}

\osservazione{
    Se $ F(v)=\lambda v $ e $ F(v)= \mu v $ 
    
    $\implies$ $ \lambda  v= \mu v $ 
    
    $\implies$ $ (\lambda - \mu)v = \underline{0} $, quindi se $ v\neq \underline{0} $ 
    
    $\implies$ $ \lambda=\mu $.

    Quindi $ v \in V $ con $ v \neq 0 $ può essere autovettore di $ F $ per al più un $ \lambda $.
}

\definizione{}{
Se $ v $ autovettore di $ F $ e $ F(v)= \lambda v $, $ \lambda $ si dice un autovalore di $ F $. \[
    \text{Spettro}(F)=\{\text{autovalori di }F\} \subseteq \K
\]
e si dice spettro di $ F $.

Se $ \lambda \in \text{Spettro}(F) $, \[
    V_{\lambda}=\{v \in V\,|\, F(v)=\lambda v\} 
\]
e $ V_{\lambda}  $ si dice l'autospazio di $ F $ relativo a $ \lambda $}

\proposizione{diiidokjlikjlkjladkskks}{
    $ V_{\lambda}  $ è sempre un sottospazio vettoriale di $ V $
}
\dimostrazioneprop{diiidokjlikjlkjladkskks}{
    Siano $ v_1 $ e $ v_2 \in V_{\lambda} $, e $\mu_1$ e $ \mu_2 \in \K $ e dimostriamo che $ \mu_1v_1+\mu_2v_2 \in V_{\lambda}  $
    \begin{multline*}
        F(\mu_1v_1+\mu_2v_2)\underset{F\text{ lin.}}{=} \mu_1 F(v_1)+ \mu_2 F(v_2)=\\
        \underset{\footnotemark}{=} \mu_1 \lambda v_1+\mu_2 \lambda v_2= \lambda(\mu_1v_1+\mu_2v_2)
    \end{multline*} 
    
    $\implies$ $ \mu_1v_1+\mu_2v_2 \in V_{\lambda}  $ \qed
}
\footnotetext{$v_1, v_2 \in V_{\lambda} $}

\proposizione{vlambd}{
    $ F(V_{\lambda}) \subseteq V_{\lambda}  $
}
\dimostrazioneprop{vlambd}{
    Sia $ v \in F(V_{\lambda} ) $ 
    
    $\implies$ $ v=F(v_1) $ con $ v_1 \in V_{\lambda}  $ \[
        F(v)=F(F(v_1))=F(\lambda v_1) = \lambda F(v_1)=\lambda v
    \] 
    
    $\implies$ $ v \in V_{\lambda}  $ 
    
    $\implies$ $ F(V_{\lambda} ) \subseteq V_{\lambda} $\qed
}

\esempi{}{
    \begin{enumerate}
        \item $ \I:V\to V $, $ \I (v)=v $ $ \forall\,v \in V $ \[
            \text{Spettro}(I)=\{1\}\qquad V_{1}=V.
        \]
        \item $ F:V\to V $, $ F (v)=\lambda_0 v $ $ \forall\,v \in V, \lambda_0 \in \K $ fissato \[
            \text{Spettro}(F)=\{\lambda_0\}\qquad V_{1}=V.
        \]
        \item $ F:V\to V $ non iniettiva, ovvero $ \ker (F)\neq \{\underline{0}\} $, $ \underline{0} \in \text{Spettro}(F) $ e $ \ker(F)=V_{0}  $.
        \item Sia $ F:\R^{2}\to \R^{2} $ tale che $ F(x,y)=(x, x+y) $. Per trovare lo spettro di $ F $ si imposta il sistema $ F(x,y)=\lambda(x,y) $ 
        \[
            \,\implies\,\begin{cases}
                x=\lambda x \\
                x+ y=\lambda y
            \end{cases}
        \]
        \begin{itemize}
            \item Se $ x \neq 0 $ $ \,\implies\, $ l'equazione $ x=\lambda x $ implica $ \lambda=1 $, e la seconda equazione $(x+y)=y$ è soddisfatta se $ x=0 $, quindi mai.
            \item Se $ x =0 $ $ \,\implies\, $ l'equazione $ x=\lambda x $ è soddisfatta per ogni $\lambda$, mentre la seconda si riduce a $ y=\lambda y $, e se $ y \neq 0 $ 
            
            $\implies$ $ \lambda=1 $
        \end{itemize}
        
        \[
            \text{Spettro}(F)=\{1\}\qquad \R^{2}_1 = \mathscr{L}((0,1))
        \]
        L'insieme degli autovettori rispetto a $ 1 $ è dato da \[
            \{(0,y)\,|\, y \neq 0\}
        \]
        e non è un sottospazio vettoriale.
    \end{enumerate}
}

\osservazione{
    $ F \in \text{End}(V) $, $ \lambda, \mu \in \text{Spettro}(F) $, $ \lambda\neq \mu $ 
    
    $\implies$ $ V_{\lambda}\cap V_{\mu} =\{\underline{0}\}  $

    Infatti se $ v \in V_{\lambda}\cap V_{\mu} $ 
    
    $\implies$ $ F(v)= \lambda v  $ e $ F(v)= \mu v $ 
    
    $\implies$ $ (\lambda-\mu)v=\underline{0} $ 
    
    $\implies$ $ v=\underline{0}. $
}

\teorema{jjijjlkjdkdkslkjd}{
    $ V $ spazio vettoriale su $ \K $, $ F \in \text{End}(V) $, \[ \lambda_1, \cdots, \lambda_{l} \in \text{Spettro}(F)  \] con $ \lambda_i \neq \lambda_j $ se $ i \neq j $

    Sia $ v_{i} \in V_{\lambda_{i} }   $ con $ v_{i} \neq \underline{0} $ 
    
    $\implies$ $ \{v_1, \cdots, v_{l} \} $ è libero.
}
\dimostrazione{jjijjlkjdkdkslkjd}{
    Per induzione su $ l $.
    
    Se $ l =1 $ allora $ \lambda_1 \in \text{Spettro}(F) $ 
    
    $\implies$ $ v_1 \in V_{\lambda_1}  $ con $ v_1 \neq 0 $ 
    
    $\implies$ $ \{v_1\} $ è libero.

    Supponiamo l'enunciato vero per $ l-1 $ autovalori, e dimostriamolo per $ l $ autovalori.

    Dati $ \lambda_1, \cdots, \lambda_{l} \in \K  $, $ v_1, \cdots, v_{l}  $ consideriamo $ \mu_1, \cdots, \mu_{l} \in \K $ tali che \[
        \mu_1v_{1}+\mu_2v_2+\cdots+\mu_{l}v_{l}=\underline{0}.  
    \]

    Devo verificare che necessariamente risulta $ \mu_1=\cdots=\mu_{l}=0  $

        \[F(\mu_1v_{1}+\mu_2v_2+\cdots+\mu_{l}v_{l})=F(\underline{0})=\underline{0}\]
    \begin{multline*}
        F(\mu_1v_{1}+\mu_2v_2+\cdots+\mu_{l}v_{l})=\\
        = \mu_1 F(v_1)+\cdots+\mu_{l}F(v_{l} ) = \lambda_1 \mu_1v_1 + \cdots + \lambda_{l} \mu_{l} v_{l}
    \end{multline*} 
    
    $\implies$ \begin{equation}
        \lambda_1 \mu_1v_1 + \cdots + \lambda_{l} \mu_{l} v_{l}  =\underline{0}\label{1ddsfdfsdfsd}
    \end{equation}

    Si considera la relazione $ \mu_1v_{1}+\mu_2v_2+\cdots+\mu_{l}v_{l} = \underline{0}$ e la si moltiplica per $ \lambda_1 $ (qui senza perdere di generalità si può supporre $\lambda_1\neq 0$)
    \begin{equation}
        \lambda_1 \mu_1v_1 + \cdots + \lambda_{1} \mu_{l} v_{l} \label{xv:2}
    \end{equation}

    Facendo ora \eqref{1ddsfdfsdfsd}$-$\eqref{xv:2} \[
        \mu_2 (\lambda_1-\lambda_2)v_2 + \mu_3(\lambda_1-\lambda_3)v_3+\cdots+\mu_l(\lambda_1-\lambda_l)v_l=\underline{0}
    \]

    Per ipotesi induttiva $ \{v_2, \cdots, v_{l} \} $ è libero 
    
    $\implies$ $
        \mu_2 (\lambda_1-\lambda_2) = \mu_3 (\lambda_1-\lambda_3) = \cdots = \mu_l(\lambda_1-\lambda_l)=\underline{0}
    $

    Per ipotesi $ \lambda_{i}-\lambda_{j} \neq 0  $ se $ i\neq j$ 
    
    $\implies$ $ \mu_2=\mu_3=\cdots=\mu_{l}=0  $ 
    
    $\implies$ la relazione $ \mu_1v_{1}+\mu_2v_2+\cdots+\mu_{l}v_{l} = \underline{0}$ si riduce a $ \mu_1 v_1=\underline{0} $, $ v_1 \neq 0$ 
    
    $\implies$ $\mu_1=0$ 
    
    $\implies$ $ \mu_1=\mu_2=\cdots=\mu_{l}=\underline{0}  $ 
    
    $\implies$ $ \{v_1, \cdots, v_{l} \} $ è libero.\qed

}

\conseguenza{
    Se $ \text{Spettro}(F) = \{\lambda_1,\cdots, \lambda_{l} \}$ 
    
    $\implies$ $ V_{\lambda_1} \oplus \cdots \oplus V_{\lambda_{l} } \subseteq V $, e in particolare se $ V $ ha dimensione finita lo spettro è finito e $ \dim  V=n $ 
    
    $\implies$ lo spettro di $ F $ ha al più $ n $ elementi.
}

Sia $ V $ a dimensione finita, $ V=V_{\lambda_1}\oplus\cdots\oplus V_{\lambda_{l} }   $, dove \[\text{Spettro}(F)=\{\lambda_1, \cdots, \lambda_{l} \}\] sia $ \mathscr{B}_i$ una base di $ V_{\lambda_{i} }  $, sia $ \mathscr{B}=\bigcup_{i} \mathscr{B}_i  $ base di $ V $, $ \mathscr{B}=\{v_1, \cdots, v_{n} \} $

\[
    M^{ \mathscr{B}, \mathscr{B}}(F)=\begin{pmatrix}
        \lambda_1 & 0 & 0 & 0 & 0\\
        0 & \star & 0 & \vdots & \vdots\\
        \vdots & 0 & \star & 0 & \vdots\\
        \vdots & \vdots & 0 & \ddots & 0\\
        \vdots & \vdots & \vdots & 0 & \ddots\\
        0 & 0 & 0 & 0 & 0
    \end{pmatrix}
\]

In questa situazione $ M^{ \mathscr{B}, \mathscr{B}}(F) $ è una matrice diagonale.

\subsection{Calcolo degli autovalori}

Sia $ V $ uno spazio vettoriale su un campo $ \K $, con $ V $ finitamente generato, $ F \in \text{End}(V) $ e cerco gli autovalori di $ F $.

Fisso $ \mathscr{B} $ base di $ V  $ e sia $ A = M^{ \mathscr{B}, \mathscr{B}}(F)$, $\lambda \in \K$. 

$\lambda$ autovattore di $ F $ 

$ \iff $ $ \exists\, v \in V $ con $ v \neq 0 $ tale che $ F(v)=\lambda v $ 

$ \iff $ $ \exists\, v \in V $, $ v \neq 0 $ tale che $ A(v)_{ \mathscr{B}}= \lambda (v)_{ \mathscr{B}} $

$ \iff $ $ (A- \lambda \Id)(v)_{ \mathscr{B}}=\underline{0} $

$ \lambda $ autovalore di $ F $ $ \iff $ $ \exists\,X \in \K^{n} $ tale che \[
    (A-\lambda\id)X=\underline{0}
\]

$ \iff $ $ \det(A-\lambda\id)=0 $

Quindi gli autovalori di $ F $ sono gli elementi $\lambda$ di $ \K  $ che soddisfano l'equazione \[
    \det(A-\lambda\id)=0.
\]

Si definisce $ p(\lambda):=\det(A-\lambda\id) $, un polinomio in $ \lambda $ di grado $ n $. ($p \in \K[\lambda]$)

\definizione{}{ 
    $ p(\lambda):=\det(A-\lambda\id) $ si dice il polinomio caratteristico di $ F $.
    \[
        \text{Spettro}(F)=\{\text{radici di }p\}
    \]
}

% ATOD creare pdf individuale e pagina obsidian