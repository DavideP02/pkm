Sia $V$ spazio vettoriale di dimensione finita, $V=W_1+W_2$ con $W_1$, $W_2$ sottospazi vettoriali $\implies$ 
\[\dim V = \dim W_1 + \dim W_2 - \dim (W_1 \cap W_2)\]
In particolare se $ W_1 \cap W_2 = \{\underline{0}\}$ , allora
\[V=W_1 \oplus W_2 \,\implies\, \dim V = \dim W_1 + \dim W_2\]
    
\esercizio{
    Sia $V=R^4$, 
    \begin{align*}
        W_1&=\{(x_1, x_2, x_3, x_4)\,|\,2x_1 - x_2 + x_3 = 0, x_1+x_2-x_4=0\}\\
        W_2&= \mathscr{L}\left((0,0,1,1), (1,1,0,0), (2,2,-2,-2)\right)
    \end{align*}
    \begin{enumerate}
        \item Si trovino una base di $ W_1 $ e una base di $ W_2 $.
        \item Si calcoli $\dim (W_1+W_2)$ e si dica se la somma è diretta.
        \item Si scriva $(2, 1, -1, 1)$ come somma di un vettore in $ W_1 $ con un vettore in $ W_2 $.
    \end{enumerate}
}{
    \begin{enumerate}
        \item $ W_1 $ è formato dai vettori $x=(x_1, x_2, x_3, x_4)$ le cui componenti risolvono questo sistema
        \[
        \begin{cases}
        2x_1-x_2+x_3=0\\
        x_1+x_2-x_4=0
        \end{cases} \,\implies\, 
        \begin{cases}
        x_1=t\\
        x_2=s\\
        x_3=-2+s\\
        x_4=t+s
        \end{cases}
        \]
        \[
            \begin{pmatrix}
                x_1\\ x_2\\ x_3\\ x_4
            \end{pmatrix}= t\,\begin{pmatrix}
                1\\ 0\\ -2\\ 1
            \end{pmatrix}+s\,\begin{pmatrix}
                0\\ 1\\ 1\\ 1
            \end{pmatrix}
        \] 
        
        $\implies$ $W_1= \mathscr{L}\left((1, 0, -2, 1), (0, 1, 1, 1)\right)$, ma i due generatori sono linearmente indipendenti 
        
        $\implies$ $ \mathscr{B}_1=\{(1,0,-2,1),(0,1,1,1)\}$ base di $ W_1 $, con $v_1=(1,0,-2,1)$ e $v_2=(0,1,1,1)$
    
        Per $ W_2 $ si osserva che \[(2, 2, -2, -2)=-2(0,0, 1, 1)+2(1, 1, 0, 0)\] 
        
        $\implies$ $W_2= \mathscr{L}\left((0,0,1,1), (1, 1, 0, 0)\right)$. 
        
        Poiché i due generatori sono linearmente indipendenti 
        
        $\implies$ $ \mathscr{B}_2=\{(0,0,1,1),(1,1,0,0)\}$ base di $ W_2 $, con $w_1=(0,0,1,1)$ e $w_2=(1,1,0,0)$.
        \item $W_1+W_2$ contiene $v_1, v_2, w_1, w_2$, so che $ \mathscr{B}_1=\{v1,v2\}$ è libero. 
        
        Considero $\{v1, v2, w1\}$ e mi chiedo se è libero, quindi studio il sistema $w_1=\lambda\, v_1 + \mu\, v_2$
        \[
            \begin{pmatrix}
                0\\ 0\\ 1\\ 1
            \end{pmatrix} = \lambda \begin{pmatrix}
                1\\ 0\\ -2\\ 1
            \end{pmatrix} + \mu \begin{pmatrix}
                0\\ 1\\ 1\\ 1 
            \end{pmatrix}= \begin{pmatrix}
                \lambda \\ \mu\\ -2\lambda + \mu\\ \lambda+\mu
            \end{pmatrix} \,\implies\, \begin{cases}
                \lambda=0\\
        \mu=0\\
        0=1\\
        0=1
            \end{cases}
        \] 
        
        $\implies$ il sistema non ha soluzione, $\{v_1, v_2, w_1\}$ è libero
        
        Considero $\{v_1, v_2, w_1, w_1\}$ e mi chiedo se è libero, quindi studio il sistema \[
            w_2=\lambda_1\,v_1 + \lambda_2\,v_2 + \lambda_3\,w_1
        \]
        \begin{multline*}
            \begin{pmatrix}
                1\\ 1\\ 0\\ 0
            \end{pmatrix}=\lambda_1 \begin{pmatrix}
                1\\ 0\\ -2\\ 1
            \end{pmatrix}+\lambda_2 \begin{pmatrix}
                0\\1\\1\\1
            \end{pmatrix}+\lambda_3 \begin{pmatrix}
                0\\0\\1\\1
            \end{pmatrix}=\\=\begin{pmatrix}
                \lambda_1\\ \lambda_2\\-2\lambda_1+\lambda_2+\lambda_3\\ \lambda_1+\lambda_2+\lambda_3
            \end{pmatrix} \,\implies\,
            \begin{cases}
                \lambda_1=1\\
                \lambda_2=1\\
                -1+\lambda_3=0 \,\implies\, \lambda_3=1\\
                2+\lambda_3=0 \,\implies\, \lambda_3=-2
            \end{cases}
        \end{multline*}

        Il sistema non ha soluzione e $\{v_1, v_2, w_1, w_2\}$ è libero 
        
        $\implies$ $W_1+W_2=\R^4$, per la formula di Grassmann $W_1 \cap W_2=\{\underline{0}\}$
        
        $\implies$ $R^4=W_1 \oplus W_2$ e $\{v_1, v_2, w_1, w_2\}$ è base di $\R^4$.
        \item Voglio scrivere $v=(2, 1, -1, 1)$ come $v=a+b$ con $a \in W_2$ e $b \in W_2$
        \[
        v=\lambda_1 v_1+ \lambda_2 v_2+ \mu_1 w_1 + \mu_2 w_2\] con $\lambda_1, \lambda_2, \mu_1, \mu_2$ unici, poiché $\{v_1, v_2, w_1, w_2\}$ base di $\R^4$. So che $\lambda_1 v_1 + \lambda_2 v_2 \in W_1$ e $\mu_1 w_1 + \mu_2 w_2 \in W_2$
        \[
            \begin{pmatrix}
                2\\ 1\\ -1 \\ 1
            \end{pmatrix} = \begin{pmatrix}
                \lambda_1\\ 0\\ -2\lambda_1\\ \lambda_1
            \end{pmatrix} + \begin{pmatrix}
                0\\ \lambda_2\\ \lambda_2\\ \lambda_2
            \end{pmatrix} + \begin{pmatrix}
                0\\ 0\\ \mu_1\\ \mu_1
            \end{pmatrix} + \begin{pmatrix}
                \mu_2\\ \mu_2\\ 0\\ 0
            \end{pmatrix}
        \] 
        \[
            \implies\,\begin{cases}
                \lambda_1+\mu_2=2\\
                \lambda_2+\mu_2=1\\
                -2\lambda_1+\lambda_2+\mu_1=-1\\
                \lambda_1+\lambda_2+\mu_1=1
            \end{cases} \,\implies\, \begin{cases}
                \lambda_1=2-\mu_2\\
                \lambda_2=1-\mu_2\\
                -4+2\mu_2+1-\mu_2+\mu_1=-1\\
                2-\mu_2+1-\mu_2+\mu_1=1
            \end{cases}
        \]
        \[
        \begin{cases}
            \lambda_1=2-\mu_2 \,\implies\, \lambda_1=\frac{2}{3}\\
            \lambda_2=1-\mu_2 \,\implies\, \lambda_2 = -\frac{1}{3}\\
            \mu_1+\mu_2=2 \,\implies\, \mu_1=2-\mu_2 \,\implies\, \mu_1=2-\frac{4}{3}=\frac{2}{3}\\
            \mu_1-2\mu_2=-2 \,\implies\, 2-\mu_2-2\mu_2=-2 \,\implies\, \mu_2=\frac{4}{3}
        \end{cases}
        \]
      
        $v=\frac{2}{3} v_1-\frac{1}{3} v_2+\frac{2}{3} w_1+\frac{4}{3} w_2$ 
        
        $\implies$ $a=\frac{2}{3} v_1-\frac{1}{3} v_2 \in W_1$ e $b=\frac{2}{3} w_1+\frac{4}{3} w_2 \in W_2$
    \end{enumerate}
}{}

\esercizio{
    Sia \[W=\{(x_1, x_2, x_3) \in \R^3 \,|\: x_1-x_2+x_3=0\}.\] Si trovino due sottospazi $W_1, W_2 \in \R^3$ con \[W_1 \neq W_2 \,|\: \R^3=W \oplus W_1,\: \R^3=W \oplus W_2\]

    Dal punto di vista geometrico $W$ è un piano nello spazio, mentre $ W_1 $ e $ W_2 $ sono due rette, passanti per l'origine, che non appartengano al piano.
}{
    $\dim W=2$, quindi $\dim W_1 = \dim W_2 = 1$

    Risulta $W_1= \mathscr{L}(w_1), W_2= \mathscr{L}(w_2)$, con $w_2, w_2\neq \underline{0}$, $ w_1 $, $ w_2 $ non paralleli e $w_1, w_2 \notin W$
    \begin{align*}
    w_1=(1, 1, 1) \notin W \,\implies\, W_1= \mathscr{L}(w_1):&\text{ risulta }\R^3=W \oplus W_1\\
    w_2=(3, 1, 1) \notin W \,\implies\, W_2= \mathscr{L}(w_2):&\text{ risulta }\R^3=W \oplus W_2
    \end{align*}
    Inoltre w1 e w2 non sono paralleli implies W1 neq W2
}{}

    
\section{Matrici II}
\subsection{Rango di una matrice e algoritmo di Gauss}
    
Sia $A \in \K^{m,n}$ con $\K$ campo. 
\[A=\begin{pmatrix}
    R_1\\ \vdots\\ R_m
\end{pmatrix}\]
dove $R_1, \cdots, R_m$ sono le righe di $A$.

Ogni riga ha $n$ componenti, e può essere identificata con un elemento in $\K^n$, quindi $R_1, R_2, \cdots, R_m \in \K^n$

$ \mathscr{L}(R_1,\cdots,R_m)$ è sottospazio vettoriale di $\K^n$

Quando modifico la matrice $A$ implementando le tre operazioni dell'algoritmo di Gauss trovo una nuova matrice, 
\[A'=\begin{pmatrix}
    R_1'\\ \vdots\\ R_m'
\end{pmatrix}\] 
ma per la natura delle operazioni \[ \mathscr{L}(R_1',\cdots,R_m')= \mathscr{L}(R_1,\cdots,R_m)\] se $A'$ è ridotta per righe, le righe non nulle di $A'$ sono una base di $ \mathscr{L}(R_1,\cdots,R_m) $.

\proposizione{}{
    Il rango di una matrice è la dimensione del sottospazio vettoriale generato dalle righe. Denoto
    \begin{equation}
        R(A):= \mathscr{L}(R_1,...,R_m)\subseteq \K^n, \dim R(A)=\rank(A)
    \end{equation}
}

Data $A \in \K^{m,n}$, 
\begin{equation*}
    A=\begin{pmatrix}
        R_1\\ \vdots\\ R_m
    \end{pmatrix}
\end{equation*}
dove $R_1, \cdots, R_m \in \K^n$ sono le righe della matrice; \begin{equation*}
    A=\begin{pmatrix}
        C_1 & \cdots & C_n
    \end{pmatrix}
\end{equation*} dove $C_1, \cdots, C_n$ sono le colonne di $A$, $C_1, \cdots,C_n \in \K^m$
\[
    R(A)= \mathscr{L}(R_1,...,R_m), \rank A=\dim R(A)
\]
Sia 
\[C(A):= \mathscr{L}(C_1,\cdots,C_n)\] sottospazio vettoriale di $\K^n$.
$C(A)=R(\null^t\!A)$ poiché per definizione di matrice trasposta le righe di $\null^t\!A$ sono le colonne di $A$.

\teorema[(del rango)]{rankequellleoijdafliygkjygjhgfjhgfjhgfjhgfcjghgffgfgffgfgfgfg}{
    Sia $A \in \K^{m,n}$ allora
    \begin{gather*}
        \rank(A) = \rank(\null^t\!A)\\
        \iff\\ 
        \dim R(A) = \dim C(A)
    \end{gather*}
}
\dimostrazione{rankequellleoijdafliygkjygjhgfjhgfjhgfjhgfcjghgffgfgffgfgfgfg}{
    Data $A \in \K^{m,n}$ e $R(A), C(A)$, siano $l=\dim C(A)$, $h=\dim R(A)$. Si dimostra che $l\le h$ e $h\le l$ 
    
    $\implies$ $l=h$.

    Sia $ \mathscr{B}=\{w_1, \cdots, w_h\}\subseteq \K^n$ una base di $R(A)$. Per ogni $R_i\in \mathscr{L}(w_1, \cdots, w_h)$ possiamo scrivere 
    \[R_i=b_{i1}\,w_1+b_{i2}\,w2+\cdots+b_{ih}\,w_h=\sum_{j=1}^h b_{ij}\,w_j\] per qualche $b_{i1}, \cdots, b_{ih} \in \K$

    Sia $D$ la matrice le cui righe sono i vettori $w_1, \cdots, w_h$, \[
        D=\begin{pmatrix}
            w_1\\ \vdots\\ w_k
        \end{pmatrix} \in \K^{h,n}
    \]
    Sia $B$ la matrice la cui entrata ($ij$)-esima è $b_{ij}$, $B \in \K^{m, h}$.

    Sappiamo che $R_i=\sum_{j=1}^h b_{ij}\,w_j$ 
    
    $\implies$ $A=BD$ $\implies$ $ \null^{t}\!A=\null^{t}\!D\,\null^{t}\!B $

    Le righe di $ \null^{t}\!A $ sono le colonne di $A$ 
    
    $\implies$ ogni colonna di $A$ è combinazione lineare degli stessi $h$-vettori  
    
    $\implies$ $\dim C(A)\le h$ $\implies$ $l\le h$

    Per dimostrare che $h\le l$ si applica lo stesso ragionamento partendo da $ \null^{t}\!A $, e concludendo che ogni riga di $A$ è combinazione lineare degli stessi $l$-vettori da cui segue $h\le l$.\qed
}

Se ne deduce che l'algoritmo di Gauss può essere implementato sulle colonne. Data $A \in \K^{m,n}$, le tre operazioni possono essere operate sulle colonne.
    
\teorema[(di nullità più rango)]{nullitrasoijfdsoirango}{
    Sia $A \in \K^{m,n}$, e si consideri il sistema lineare omogeneo associato $AX=\underline{0}$, con $X \in \K^{n}$.

    Sia $N(A)=\{X \in \K^n\, |\: AX=\underline{0}\}$ (nullspace di $A$), $N(A) \subseteq \K^n$ sottospazio vettoriale.

    Vale
    \begin{equation}
        \dim N(A) = n - \rank(A)
    \end{equation}
}
\dimostrazione{nullitrasoijfdsoirango}{
    $N(A)$ non cambia se sostituisco $A$ con una matrice ridotta per righe tramite l'algoritmo di Gauss. 
    
    Possiamo supporre $A$ ridotta per righe. $\rank(A)$ è il numero di righe di $A$, e sappiamo che le soluzioni del sistema $AX=\underline{0}$ dipendono da $n-r$ parametri.
    
    $\implies$ $\dim N(A)=n-r=n-\rank(A)$\qed
}