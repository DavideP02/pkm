\marginnote{5 ott 2021}

\teorema{primo}{Sia $V$ uno spazio vettoriale su campo $\K$, e $W\subseteq V$ un sottospazio vettoriale:
\begin{enumerate}
\item \label{thm_1:1} se $V$ è finitamente generato $\implies$ $W$ è finitamente generato;
\item se  $V$ è finitamente generato $\implies$ $\dim W\le\dim V$
\item se $V$ è finitamente generato e $\dim W = \dim V$ $\implies$ $W=V$
\end{enumerate}
 }
\dimostrazione{primo}{
	\begin{enumerate}
	\item Supponiamo che $V$ sia finitamente generato, e per assurdo che $W$ non lo sia.
	
	$V$ è finitamente generato $\implies$ $V$ ha una base \[\mathscr{B}=\{v_1,\cdots, v_n\}\]
	
	$W$ non è finitamente generato, e sia $w_1\in W,\:w_1\neq \underline{0}$, considero $\mathscr{L}(w_1)\subseteq W$, ma $W\neq \mathscr{L}(w_1)$, altrimenti $W$ sarebbe generato da $w_1$. $\implies$ $\exists\: w_2\in W \land w_2 \notin \mathscr{L}(w_1)$.
	
	Considero $\mathscr{L}(w_1, w_2)\subseteq W$, ma $W\neq \mathscr{L}(w_1, w_2)$, altrimenti $W$ sarebbe generato da $\{w_1, w_2\}$. $\implies$\\$\implies$ $\exists\: w_3\in W \land w_3 \notin \mathscr{L}(w_1, w_2)$.
	
	Itero il procedimento e trovo \begin{multline*}\{w_1, \cdots, w_{n+1}\}\subseteq W\:\tc\: w_{n+1}\notin \mathscr{L}(w_1, \cdots, w_n)\:\implies \\ \implies\{w_1, \cdots, w_{n+1}\}\text{ è un insieme libero}\end{multline*}e contiene più elementi di una base $\mathscr{B}$. Assurdo per teorema precedente. % TODO aggiungere numero del teorema
	
	\item Supponiamo $V$ finitamente generato, e sia $W\subseteq V$ un sottospazio vettoriale. $W$ è finitamente generato (per \ref{thm_1:1}.) 
	$\implies$ $\exists \:\mathscr{B}=\{w_1, \cdots, w_m \}$ base di $W$ $\implies$ $\mathscr{B}\subseteq V$ è un sottoinsieme libero $\implies$ $m\le \dim V$ $\implies$ $\dim W \le \dim V$
	
	\item Sia $W\subseteq V$ uno spazio vettoriale, con $V$ finitamente generato. $\dim W = \dim V$.
	
	$W$ ha una base $\mathscr{B}$ con $n$ vettori, dove $n=\dim V$ $\implies$ $\mathscr{B}$ è una base di $V$.
	
	Se $\mathscr{B}=\{w_1, \cdots, w_n\}$ $\implies$ $W=\mathscr{L}(w_1, \cdots, w_n)=V$ $\implies$ $W=V$\qed
	\end{enumerate}
}

\osservazione{
Se $V$ è uno spazio vettoriale finitamente generato, e $\dim V=n$ $\implies$ ogni insieme libero con $n$ elementi è una base. Infatti se $\mathscr{B}=\{v_1, \cdots, v_n\}$ è un insieme libero, se per assurdo esistesse $v\in V\land v\notin \mathscr{L}(v_1, \cdots, v_n)$ $\implies$ $\{v_1, \cdots, v_n, v\}\subseteq V$ è un insieme libero di cardinalità $n+1$ (ovvero con $n+1$ elementi). Assurdo.
}

\teorema[(del completamento di una base)]{complbs}{
Sia $V$ uno spazio vettoriale su un campo $\K$ finitamente generato. Sia $\mathscr{B}=\{v_1,\cdots,v_n\}$ una base di $V$ e sia $I=\{a_1, \cdots,a_l\}\subseteq V$ un sottoinsieme libero. Esiste sempre $\mathscr{B}'$ base di $V$ i cui primi $l$-elementi sono $a_1,\cdots,a_l$ e i restanti $n-l$-elementi sono elementi di $\mathscr{B}$.
\[
\mathscr{B}'=\{a_1,\cdots,a_l, w_1, \cdots, w_{n-l}\}\text{ con }w_1, \cdots, w_{n-l}\in\mathscr{B}
\]}

\dimostrazione{complbs}{
Applico il metodo degli scarti successivi
\begin{itemize}
\item [$l=n$] l'enunciato è banale ($I$ è già una base e non va completata);
\item [$l<n$] $\implies$ $\mathscr{L}(a_1,\cdots, a_l)\varsubsetneq V$ \\ $\implies$ $\exists\: w_1 \in \mathscr{B} \:\tc\: w_1\notin \mathscr{L}(a_1,\cdots, a_l)$. Infatti, se tutti i generatori appartenenti a $\mathscr{B}$ fossero combinazioni lineari di $a_1, \cdot, a_l$, non sarebbero più tutti linearmente indipendenti. $\implies$ $I_1=\{a_1,\cdot,a_l,w_1\}$ è libero.

Se $I_1$ è una base, la dimostrazione si conclude, altrimenti $\exists\: w_2 \in \mathscr{B} \:\tc\: w_2\notin \mathscr{L}(a_1,\cdots, a_l, w_2)$ \\ $\implies$ $I_1=\{a_1,\cdot,a_l,w_1, w_2\}$ è libero.

Se $I_2$ è una base la dimostrazione si conclude, altrimenti si itera fino a 
\[
I_{n-l}=\{a_1, \cdot, a_l, w_1,\cdots, w_{n-l}\}\text{ con }w_1, \cdots, w_{n-l}\in\mathscr{B}.
\]
$I_{n-l}$ è libero con $n$ vettori $\implies$ $I_{n-l}$ è una base\qed
\end{itemize}
}

\esempio{$\mathcal{S}(\R^{3,3})=\{A\in\R^{3,3}\:\tc\:{}^t\!A=A\}$

Cerco una base. Sia $A\in\mathcal{S}(\R^{3,3})$ generica:
\[
A=\begin{pmatrix}
a & b & c \\
b & d & e \\
c & e & f
\end{pmatrix}\text{ con }a,b,c,d,e,f\in\R
\]

\begin{multline*}
A=a\begin{pmatrix}
1 & 0 & 0\\
0 & 0 & 0\\
0 & 0 & 0
\end{pmatrix}+b\begin{pmatrix}
0 & 1 & 0\\
1 & 0 & 0\\
0 & 0 & 0
\end{pmatrix}+\\+c\begin{pmatrix}
0 & 0 & 1\\
0 & 0 & 0\\
1 & 0 & 0
\end{pmatrix}+d\begin{pmatrix}
0 & 0 & 0\\
0 & 1 & 0\\
0 & 0 & 0
\end{pmatrix}+\\+e\begin{pmatrix}
0 & 0 & 0\\
0 & 0 & 1\\
0 & 1 & 0
\end{pmatrix}+f\begin{pmatrix}
0 & 0 & 0\\
0 & 0 & 0\\
0 & 0 & 1
\end{pmatrix}
\end{multline*}

Siano $E_1=\begin{pmatrix}
1 & 0 & 0\\
0 & 0 & 0\\
0 & 0 & 0
\end{pmatrix}$, $E_2=\begin{pmatrix}
0 & 1 & 0\\
1 & 0 & 0\\
0 & 0 & 0
\end{pmatrix}$, $E_3=\begin{pmatrix}
0 & 0 & 1\\
0 & 0 & 0\\
1 & 0 & 0
\end{pmatrix}$, $E_4=\begin{pmatrix}
0 & 0 & 0\\
0 & 1 & 0\\
0 & 0 & 0
\end{pmatrix}$, $E_5=\begin{pmatrix}
0 & 0 & 0\\
0 & 0 & 1\\
0 & 1 & 0
\end{pmatrix}$, $E_6=\begin{pmatrix}
0 & 0 & 0\\
0 & 0 & 0\\
0 & 0 & 1
\end{pmatrix}$, e sia $\mathscr{B}=\{E_1, \cdots, E_6\}$

Dato
\begin{multline*}
I=\Biggl\{A_1=\begin{pmatrix}
1 & 2 & 0\\
2 & 0 & 0\\
0 & 0 & 0
\end{pmatrix},\\ A_2=\begin{pmatrix}
1 & 0 & 0\\
0 & -1 & 0\\
0 & 0 & 1
\end{pmatrix},\\ A_3=\begin{pmatrix}
0 & 1 & -1\\
1 & 0 & 0\\
-1 & 0 & 0
\end{pmatrix}\Biggr\}\subseteq\mathcal{S}(\R^{3,3})
\end{multline*}
insieme libero, si trovino tre elementi $w_1, w_2, w_3\in\mathscr{B}$ tali per cui $I\cup\{w_1, w_2, w_3\}$ sia una base di $\mathcal{S}(\R^{3,3})$.

\[
A_1=E_1+2E_2; A_2=E_1-E_4+E_6; A_3=E_2-E_3
\]
e rispetto alla base $\mathscr{B}$
\[\begin{gathered}
A_1=(1, 2, 0, 0, 0, 0), A_2=(1, 0, 0, -1, 0, 1), A_3=(0, 1, -1, 0, 0, 0)\\
E_1=(1, 0, \cdots, 0), E_2=(0, 1, 0, \cdots, 0), \cdots, E_6=(0, \cdots, 0, 1)
\end{gathered}\]

Si studia l'appartenenza di $E_1\in\mathscr{L}(A_1, A_2, A_3)$. Studio il sistema
\[
E_1=\lambda_1 A_1 + \lambda_2 A_2+ \lambda_3A_3
\]
\[
\begin{cases}
1=\lambda_1+\lambda_2\\
0=2\lambda_1+\lambda_3\\
0=-\lambda_3\\
0=-\lambda_2\\
0=0\\
0=\lambda_2
\end{cases}\implies \begin{cases}
\lambda_2=0\\
\lambda_3=0\\
\lambda_1=0\\
\lambda_1=1
\end{cases}\implies\text{Il sistema non ha soluzione}
\]$\implies$ $E_1\notin\mathscr{L}(A_1, A_2, A_3)$ $\implies$ $I_2=\{A_1, A_2, A_3, E_1\}$

Si studia l'appartenenza di $E_2\in\mathscr{L}(A_1, A_2, A_3, E_1)$. Studio il sistema
\[
E_2=\lambda_1 A_1 + \lambda_2 A_2+ \lambda_3A_3+\lambda_4E_1
\]
\[
\begin{cases}
0=\lambda_1+\lambda_2+\lambda_4\\
1=2\lambda_1+\lambda_3\\
0=-\lambda_3\\
0=-\lambda_2\\
0=0\\
0=\lambda_2
\end{cases}\implies \begin{cases}
\lambda_4=-\frac{1}{2}\\
\lambda_3=0\\
\lambda_2=0\\
\lambda_1=\frac{1}{2}
\end{cases}\implies\text{Il sistema ha soluzione}
\]$\implies$ $E_2\in\mathscr{L}(A_1, A_2, A_3, E_1)$ $\implies$ scarto $E_2$

Si studia l'appartenenza di $E_3\in\mathscr{L}(A_1, A_2, A_3, E_1)$. Studio il sistema
\[
E_3=\lambda_1 A_1 + \lambda_2 A_2+ \lambda_3A_3+\lambda_4E_1
\]
\[
\begin{cases}
0=\lambda_1+\lambda_2+\lambda_4\\
0=2\lambda_1+\lambda_3\\
1=-\lambda_3\\
0=-\lambda_2\\
0=0\\
0=\lambda_2
\end{cases}\implies \begin{cases}
\lambda_4=-\frac{1}{2}\\
\lambda_3=-1\\
\lambda_2=0\\
\lambda_1=\frac{1}{2}
\end{cases}\implies\text{Il sistema ha soluzione}
\]$\implies$ $E_3\in\mathscr{L}(A_1, A_2, A_3, E_1)$ $\implies$ scarto $E_3$

Si studia l'appartenenza di $E_4\in\mathscr{L}(A_1, A_2, A_3, E_1)$. Studio il sistema
\[
E_4=\lambda_1 A_1 + \lambda_2 A_2+ \lambda_3A_3+\lambda_4 E_1
\]
\[
\begin{cases}
0=\lambda_1+\lambda_2+\lambda_4\\
0=2\lambda_1+\lambda_3\\
0=-\lambda_3\\
1=-\lambda_2\\
0=0\\
0=\lambda_2
\end{cases}\implies \begin{cases}
\lambda_2=0\\
\lambda_2=-1\\
\cdots
\end{cases}\implies\text{Il sistema non ha soluzione}
\]$\implies$ $E_4\notin\mathscr{L}(A_1, A_2, A_3, E_1)$ $\implies$ $I_2=\{A_1, A_2, A_3, E_1, E_4\}$

Si studia l'appartenenza di $E_5\in\mathscr{L}(A_1, A_2, A_3, E_1, E_4)$. Studio il sistema
\[
E_5=\lambda_1 A_1 + \lambda_2 A_2+ \lambda_3A_3+\lambda_4 E_1 +\lambda_5 E_4
\]
\[
\begin{cases}
0=\lambda_1+\lambda_2+\lambda_4\\
0=2\lambda_1+\lambda_3\\
0=-\lambda_3\\
0=-\lambda_2+\lambda_5\\
1=0\\
0=\lambda_2
\end{cases}\implies \begin{cases}
1=0\\
\cdots
\end{cases}\implies\text{Il sistema non ha soluzione}
\]$\implies$ $E_5\in\mathscr{L}(A_1, A_2, A_3, E_1, E_4)$ $\implies$ $I_3=\{A_1, A_2, A_3, E_1, E_4, E_5\}$

La soluzione è $\mathscr{B}'=\{A_1, A_2, A_3, E_1, E_4, E_5\}$
}

\section{Operazioni tra sottospazi vettoriali}

Sia $V$ uno spazio vettoriale su un un campo $\K$, e siano $W_1$ e $W_2\subseteq V$ due sottospazi vettoriali.

Si consideri
\[
W_1 \cap W_2 = \{x\:|\: x\in w_1 \land x \in w_2\}
\]

\proposizione{inters}{$W_1 \cap W_2$ è sempre sottospazio vettoriale}
\dimostrazioneprop{inters}{Siano $x,y\in W_1 \cap W_2$
\[\implies\left.\begin{cases}
x,y \in W_1 \implies (x+y)\in W_1 \\
x,y \in W_2 \implies (x+y)\in W_2\end{cases}\right\rbrace\implies (x+y)\in W_1\cap W_2\qedhere\]}

% TODO FINIRE