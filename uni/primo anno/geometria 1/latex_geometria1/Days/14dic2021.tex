\todo{Manca la parte iniziale}

\todo{Manca l'inizio di un esercizio}
Abbiamo utilizzato il seguente cambio di coordinate: %questa è la fine esercizio mancante
\begin{gather*}
    \begin{pmatrix}
        x\\y
    \end{pmatrix}=P\begin{pmatrix}
        x'\\y'
    \end{pmatrix}\qquad \begin{pmatrix}
        X\\Y
    \end{pmatrix}=\begin{pmatrix}
        x'+1/2\\ y'+1/4
    \end{pmatrix}\\
    \implies\:\begin{pmatrix}
        x'\\y'
    \end{pmatrix}=\begin{pmatrix}
        X-1/2\\ Y-1/4
    \end{pmatrix} \,\implies
\end{gather*}
\begin{multline*}
    \begin{pmatrix}
        x\\y
    \end{pmatrix}= P \begin{pmatrix}
        X\\Y
    \end{pmatrix}+ P\begin{pmatrix}
        -1/2\\ -1/4
    \end{pmatrix}=\\=\frac{1}{\sqrt{2}}\begin{pmatrix}
        1 & 1\\-1 & 1
    \end{pmatrix}\begin{pmatrix}
        X\\Y
    \end{pmatrix}+\frac{1}{\sqrt{2}}\begin{pmatrix}
        1 & 1\\-1 & 1
    \end{pmatrix}\begin{pmatrix}
        -1/2\\ -1/4
    \end{pmatrix}
\end{multline*}
\[
    \begin{pmatrix}
        X\\Y
    \end{pmatrix}=\frac{1}{\sqrt{2}}\begin{pmatrix}
        X+Y\\-X+Y
    \end{pmatrix}+\frac{1}{\sqrt{2}}\begin{pmatrix}
        -3/4\\ 1/4
    \end{pmatrix}
\]

\rule{7em}{.4pt}

È possibile dedurre di che conica si tratta dall'algoritmo senza dover operare il cambio di coordinate.

L'equazione della conica data da \[
    p(x,y)=a_{11} x^{2} + 2 a_{12} xy + a_{22}y^{2} + 2 a_{13}x + 2 a_{23} y + a_{33}=0
\] 
riscrivibile come 
\[
    \begin{pmatrix}
        x & y
    \end{pmatrix} A \begin{pmatrix}
        x\\ y
    \end{pmatrix} + 2a_{13} x + 2a_{23}y+a_{33}\qquad A=\begin{pmatrix}
        a_{11} & a_{12}\\
        a_{12} & a_{22}
    \end{pmatrix} \in S(\R^{2,2})
\] 
ma anche come
\[
    \begin{pmatrix}
        x & y & 1
    \end{pmatrix} B \begin{pmatrix}
        x\\ y\\ 1
    \end{pmatrix}\qquad B=\begin{pmatrix}
        a_{11} & a_{12} & a_{13}\\
        a_{12} & a_{22} & a_{23}\\
        a_{13} & a_{23} & a_{33}
    \end{pmatrix}
\]

$ A $ è la matrice incompleta della coninca, mentre $ B $ è la matrice completa.

Si può osservare che operando un cambio di coordinate (come una rototraslazione) alcune informazioni relative alle matrici $ A $ e $ B $ rimangono invariate. Da $ \rank A $, $ \rank B $, $ \det A $, $ \tr(A)\det(B) $ si può risalire a che conica si tratta.

\teorema{terribileveramentetanto}{
    \begin{itemize}
        \item $ \rank B = 3 $
            \begin{itemize}
                \item $ \rank A = 2 $
                    \begin{itemize}
                        \item $ \det A > 0 $\footnote{
                            Se $ \det A>0 $, allora $ \tr(A)\neq 0 $, infatti $ A $ è diagonalizzabile, $ \spettro(A)=\lambda_1, \lambda_2 $, $ \tr(A)= \lambda_1 + \lambda_2 $ mentre $ \det A = \lambda_1 \lambda_2 $. Quindi $ \det A >0 $ implica $ \lambda_1 $ e $ \lambda_2 $ hanno lo stesso segno 
                            
                            $\implies$ $ \lambda_1 + \lambda_2 \neq 0 $.
                    }: 
                            \begin{itemize}
                                \item $ \tr(A)\det(B)<0 $: ellissi
                                \item $ \tr(A)\det(B)>0 $: ellissi immaginaria
                            \end{itemize}
                        \item $ \det A < 0 $: iperbole
                    \end{itemize}
                \item $ \rank A = 1 $: parabola
            \end{itemize}
        \item $ \rank B = 2 $
            \begin{itemize}
                \item $ \det A \neq 0 $
                    \begin{itemize}
                        \item $ \det A > 0$: due rette immaginarie incidenti
                        \item $ \det A < 0 $: due rette reali incidenti
                    \end{itemize}
                \item $ \det A = 0 $: due rette parallele
            \end{itemize}
        \item $ \rank B = 1 $ ($\implies$ $ \rank A = 1 $): due rette coincidenti
    \end{itemize}
}

\section{Geometria Analitica}

La geometria analitica\footnote{In questa sezione, per distinguere i punti dai vettori, si indicheranno quest'ultimi in neretto (es: $\mathbf{n}$)} è lo studio di punti, rette, piani, sfere (e oggetti simili) nello spazio.

\subsection{Descrizione di un piano in $ \R^{3} $} 

Il piano non è necessariamente vettoriale (ovvero passante per l'origine)

Fisso $ p_0=(x_0, y_0, z_0) \in \R^{3} $ un punto, e $ \mathbf{n}=(a,b,c) $ un vettore, e descrivo il piano passante per $ p_0 $ e ortogonale a $ \mathbf{n} $.

Piano ortogonale a $ \mathbf{n} $: $ \{(x,y,z) \in \R^{3}\,|\, (x,y,z) \cdot (a,b,c)=0\} $, ovvero \[
    \{(x,y,z) \in \R^{3}\,|\, ax+by+cz=0\}
\]
Ne risulta quindi che \begin{equation}
    ax+by+cz=0
\end{equation} è l'equazione del piano in $ \R^{3} $ ortogonale a $ (a,b,c) $. Deduco che \begin{equation}
    a(x-x_0)+b(y-y_0)+c(z-z_0)=0
\end{equation} è l'equazione del piano passante per $ x_0, y_0, z_0 $ e ortogonale a $ (a,b,c) $

Quindi \begin{equation}
    ax+by+cz+d=0\qquad d=-ax_0-by_0-cz_0
\end{equation} è l'\textit{equazione cartesiana di un piano}.

Un altro modo per descrivere un piano passante per l'origine (piano vettoriale), è di fissare due vettori $ \mathbf{u}, \mathbf{v} $ non linearmente dipendente, e scrivere il piano come l'insieme dei vettori in $ \mathscr{L}(\mathbf{u}, \mathbf{v}) $ \[
    \mathscr{L}(\mathbf{u}, \mathbf{v})=\{t\mathbf{u}+s\mathbf{v}\,|\, t, s \in \R\}
\]
Ne risulta quindi che ogni piano passante per l'origine lo si può scrivere come l'insieme dei punti di equazione \[
    p=t\mathbf{u}+s\mathbf{v}\qquad (t,s \in \R, \mathbf{u},\mathbf{v} \text{ vettori})
\]

Un piano passante per $ p_0=(x_0, y_0, z_0) $ si ottiene traslando un piano passante per l'origine in $ p_0 $, e quindi ha equazione \[
    p-p_0=t\mathbf{u}+s\mathbf{v}
\]
equazione parametrica di un piano passante per $ p_0 $ e parallelo a $ \mathbf{u} $ e $ \mathbf{v} $. Posto \[
    \mathbf{u}=\begin{pmatrix}
        l\\m\\n
    \end{pmatrix}\qquad \mathbf{v}=\begin{pmatrix}
        l'\\m'\\n'
    \end{pmatrix}
\]
l'equazione diventa
\begin{equation}
    \begin{pmatrix}
        x-x_0\\ y-y_0\\ z-z_0
    \end{pmatrix}=t \begin{pmatrix}
        l\\m\\n
    \end{pmatrix}+s\begin{pmatrix}
        l'\\m'\\n'
    \end{pmatrix}\label{launo}
\end{equation}
al variare di $ s, t \in \R $

$ (x,y,z) $ soddisfa la \eqref{launo} $ \iff $ $ (x-x_0, y-y_0, z-z_0) \in \mathscr{L}(\mathbf{u}, \mathbf{v}) $
\begin{gather}
    \iff \quad\rank \begin{pmatrix}
        x-x_0 & y-y_0 & z-z_0\\
        l & m &n \\
        l' & m' & n'
    \end{pmatrix}=2\nonumber \\
    \iff\quad \det\begin{pmatrix}
        x-x_0 & y-y_0 & z-z_0\\
        l & m &n \\
        l' & m' & n'
    \end{pmatrix}=0\label{equazionededd}
\end{gather}

La \eqref{equazionededd} è un'altra equazione (cartesiana) del piano passante per $ (x_0, y_0, z_0) $ e parallelo a $ (l, m, n) $, $ (l', m', n') $

Dati tre punti non allineati $ A, B, C $, si può trovare l'equazione di un piano passante per questi tre punti. \begin{align*}
    A &= (x_{A}, y_{A}, z_{A}   )\\
    B &= (x_{B}, y_{B}, z_{B}   )\\
    C &= (x_{C}, y_{B}, z_{B}   )
\end{align*}

In particolare il piano $ \pi $ passa per $ A $ ed è parallelo a $ \mathbf{u}=B-A $ e $ \mathbf{v}=C-A $. Ricavo l'equazione \[
    \det \begin{pmatrix}
        x-x_{A} & y-y_{A} & z-z_{A}\\
        x_{B} -x_{A} & y_{B} -y_{A} & z_{B} -z_{A}\\
        x_{C} -x_{A} & y_{C} -y_{A} & z_{C} -z_{A}
    \end{pmatrix}=0
\]

\esercizio{
    Si trovi l'equazione del piano passante per $ A=(1, 0, 1) $, $ B=(2, -1, 0) $ e $ C=(1,1,1) $
}{
    \[
        B-A=(1,-1, -1)\qquad C-A=(0, 1, 0)  
    \] e $ \pi $ ha equazione \[
        \pi:\det\begin{pmatrix}
            x-1 & y & z-1\\
            1 & -1 & -1\\
            0 & 1 & 0
        \end{pmatrix}=0
    \] cioè \[
        \det \begin{pmatrix}
            x-1 & z-1 \\
            1 & -1
        \end{pmatrix}=0
    \] ovvero \[x+z-2=0\]
}{}

\subsection{Rette nello spazio}

Un primo modo è l'\textit{equazione cartesiana}: $ r $ è interesezione di due piani: $ \pi $ e $ \pi' $: \[
    \begin{cases}
        ax+by+cz+d=0\\
        a'x+by'+cz'+d'=0
    \end{cases}
\]

Nell'\textit{equazione parametrica} $ r $ ha equazione \[
    \begin{pmatrix}
        x \\ y \\ z
    \end{pmatrix}=\begin{pmatrix}
        x_0 \\ y_0 \\ z_0
    \end{pmatrix} + t \begin{pmatrix}
        l \\ m \\ n
    \end{pmatrix}
\]
Vale che $ r $ passa per $ \bigl(\begin{smallmatrix}
    x \\ y \\ z
\end{smallmatrix}\bigr) $ (basta porre $ t=0 $), e $ r $ è parallela al vettore $ \bigl(\begin{smallmatrix}
    x_0 \\ y_0 \\ z_0
\end{smallmatrix}\bigr) $

\esempio{
    Equazione della retta passante per $ p_0=(1,2,-1) $ e parallela a $ (0,1,-1) $ \[
        \begin{pmatrix}
            x \\ y \\ z
        \end{pmatrix}=\begin{pmatrix}
            1 \\ 2 \\ -1
        \end{pmatrix} + t \begin{pmatrix}
            0 \\ 1 \\ -1
        \end{pmatrix}
    \]

    Per dedurre la cartesiana si ``risolve'' eliminando il parametro $ t $.\begin{gather*}
        \begin{pmatrix}
            x-1\\ y-2\\ z-1
        \end{pmatrix}=\begin{pmatrix}
            0 \\ t \\ -t
        \end{pmatrix}\\
        \begin{cases}
            x-1=0\\
            y-2=t\\
            -1-z=t
        \end{cases} \,\implies\, \begin{cases}
            x-1=0\\
            y-2=-1-z
        \end{cases}\implies \\
        \begin{cases}
            x-1=0\\
            y+z-1=0
        \end{cases}
    \end{gather*}

    Ho trovato sia l'equazione parametrica che cartesiana del sistema.
}

Si può trovare anche l'\textit{equazione della retta per due punti distinti} $ A, B $ \begin{align*}
    A &= (x_{A}, y_{A}, z_{A}   )\\
    B &= (x_{B}, y_{B}, z_{B}   )
\end{align*}

La retta passa per $ A $ ed è parallela a $ B-A $, cioè ha equazione \[
    \begin{pmatrix}
        x \\ y \\ z
    \end{pmatrix}=\begin{pmatrix}
        x_A \\ y_A \\ z_A
    \end{pmatrix} + t \begin{pmatrix}
        x_{B}-x_{A}   \\ y_{B}-y_{A}   \\ z_{B}-z_{A}  
    \end{pmatrix}
\]

\subsection{Posizioni reciproche di due piani nello spazio}

Siano $ \pi_1 $ e $ \pi_2 $ due piani in $ \R^{3} $, ci sono solo tre possibilità:
\begin{enumerate}
    \item $ \pi_1= \pi_2 $: coincidono;
    \item $ \pi_1 \parallel \pi_2 $: sono paralleli;
    \item $ \pi_1\cap \pi_2 $ è una retta.
\end{enumerate}

Supponiamo \begin{align*}
    \pi_1: &a_1x+b_1y+c_1z+d_1=0\\
    \pi_2: &a_2x+b_2y+c_2z+d_2=0
\end{align*}

Studio \[
    \begin{cases}
        a_1x+b_1y+c_1z=-d_1\\
        a_1x+b_2y+c_2z=-d_2
    \end{cases}
\] sistema lineare non omogeneo in due equazioni e tre incognite. Le matrici associate sono \[
    A=\begin{pmatrix}
        a_1 & b_1 & c_1\\
        a_2 & b_2 & c_2
    \end{pmatrix}\qquad (A|B)=\left(\begin{array}{@{}ccc|c@{}}
        a_1 & b_1 & c_1 & -d_1\\
        a_2 & b_2 & c_2 & -d_2
    \end{array}
    \right)
\]
\begin{align*}
    \pi_1= \pi_2 \,&\iff\, \rank (A )= \rank(A|B)= 1  \\      
    \pi_1\parallel \pi_2\, &\iff\, \rank(A)=1,\: \rank(A|B)=2
\end{align*}