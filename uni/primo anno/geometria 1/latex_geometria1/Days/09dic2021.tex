\section{Spazio vettoriale quoziente}
Sia\marginnote{9 dic 2021}
$ V $ spazio vettoriale su un campo $ \K $, $ H \subseteq V $ sottospazio vettoriale. Si introduce $ \sim $ relazione di equivalenza, dove $ v, w \in V $ soddisfano $ v\sim w $ se $ v-w \in H $. %DOMANDA: come definiamo l'operazione "-"?

\esercizio{
    Dimostrare che $ \sim $ è una relazione di equivalenza
}{
    Risolvere per esercizio %TODO risolvere esercizio
}{}

Dato $ v \in V $ indico con $ [v] $ la classe di $ v $ rispetto a $ \sim $, \begin{equation}
    [v]=\{v+w\,\tc\, w \in H\}%DOMANDA: capire come sono equivalenti
\end{equation}

Indico con $ V/H $ il quoziente di $ V $ rispetto a $ \sim $. $ V/H $ ha a sua volta una struttura di spazio vettoriale su $ \K $, dove \begin{align*}
    [v_1]+[v_2]&:= [v_1+v_2]\qquad \forall\,v_1, v_2 \in V\\
    \lambda[v]&:= [\lambda v]\qquad \forall\, \lambda \in \K, v \in V
\end{align*}

\esercizio{
    Le operazioni di spazio vettoriale in $ V/H $ sono ben definite.
}{
    Risolvere per esercizio %TODO risolvere esercizio
}{}

\osservazione{
    \[\underline{0}_{V/H}=[\underline{0}]=\{\underline{0} + w\,\tc\, w \in H\}=H \] con $ \underline{0} $ vettore nullo in $ V $. 

    Quindi il vettore nullo in $ V/H $ si identifica in modo naturale con $ H $.
}

\esempio{
    Dato $ \R^{n} $, sia \[H=\{(x_1, \cdots, x_{h}, 0, \cdots, 0 )\,|\, x_1, \cdots, x_{h} \in \R \}\] $ H $ è un sottospazio vettoriale $ h $-dimensionale di $ \R^{n} $. Esiste una identificazione naturale di $ H $ con $ \R^{h} $ \begin{align*}
    H & \to \R^{h}  \\
    (x_1, \cdots, x_{h}, 0, \cdots, 0 ) & \mapsto (x_1, \cdots, x_{h} )
    \end{align*} Studio $ \R^{n}/H $ \[
        x\sim y \iff x-y \in H \iff x_i=y_i \: \forall\, i = h+1, \cdots, n
    \] Considero \begin{align*}
    \phi:\R^{n}/H & \to \R^{n-h} \\
    [(x_1, \cdots, x_{n})] & \mapsto (x_{h+1}, \cdots, x_{n}  )
    \end{align*}

    $ \phi $ è un isomorfismo, ovvero è ben definita, lineare, iniettiva e suriettiva.
    \begin{itemize}
        \item $ \phi $ è ben definita, infatti, se $ x \in [y] $, 
        
        $\implies$ $ x_{i}=y_{i}   $ $ \forall\, i = h+1, \cdots, n $ 
        
        $\implies$ $ (x_{h+1}, \cdots, x_{n}  )=(y_{h+1}, \cdots, y_{n}) $, quindi $ \phi[x] $ non dipende dal rappresentante $ x $;
        \item $ \phi $ è lineare, infatti \begin{multline*}\phi(\lambda [x]+\mu [y])=\phi([\lambda x]+ [\mu y])=\phi([\lambda x + \mu y])=\\
        =\lambda (x_{h+1}, \cdots, x_{n}  )+ \mu (y_{h+1}, \cdots, y_{n})= \lambda \phi ([x]) + \mu\phi([y]);
    \end{multline*}
    \item $ \phi $ è iniettiva, infatti se $ \phi([x])=\underline{0} \in \R^{n-h} $ 
    
    $\implies$ $ x_{h+1}=x_{h+1}=\cdots=x_{n}=0    $ 
    
    $\implies$ $ x \in H $ 
    
    $\implies$ $ [x]=\underline{0}_{V/H} $;
    \item $ \phi $ è suriettiva, infatti se $ (x_{h+1}, \cdots, x_{n}  ) \in \R^{n-h} $ 
    
    $\implies$ $ \phi[(0, \cdots, 0, x_{h+1}, \cdots, x_{n}  )] =(x_{h+1}, \cdots, x_{n} ) $
    \end{itemize}

    $ \R^{n} $, $ H \subseteq \R^{n} $ con $ \dim H = h $, $ \R^{n}/H $ è isomorfo a $ \R^{n-h} $, $ \dim \R^{n}/H=n-h $
}

\proposizione{oijooijosodijfoaodfsijfa}{
    Sia $ V $ uno spazio vettoriale su un campo $ \K $, con $ V $ finitamente generato. $ H \subseteq V $ un sottospazio vettoraile 
    
    $\implies$ $ \dim V/H=\dim V - \dim H $
}
\dimostrazioneprop{oijooijosodijfoaodfsijfa}{
    Sia $ \{v_1, \cdots, v_{h}\}  $ una base di $ H $, e completiamola ad una base di $ V $: \[
        \{v_1, \cdots, v_{h}, w_{1}, \cdots, w_{n-h}\}
    \]

    Sia $ L= \mathscr{L}(w_1, \cdots, w_{n-h} ) $ e dimostriamo che $ V/H $ è isomorfo ad $ L $.

    Sia $ \phi$ tale che \begin{align*}
    \phi: V/H & \to L  \\
    \Biggl[\sum_{i=1}^{h} x_{i}v_{i} +\sum_{j=1}^{n-h}y_{j}w_{j}     \Biggr] & \mapsto \sum_{j=1}^{n-h} y_{j} w_{j}   
    \end{align*} Si dimostra che $ \phi $ è un isomorfismo, ovvero che $ \phi $ è ben definita, lineare, iniettiva e suriettiva.
    \begin{itemize}
        \item $ \phi $ è ben definita: siano $ v, v' \in V $ tali che $ v\sim v' $. Scrivo \[v=\sum_{i=1}^{h}x_{i}v_{i} + \sum_{j=1}^{n-h}y_{j}w_{j}\] e \[
            v'=\sum_{i=1}^{h}x_{i}'v_{i} + \sum_{j=1}^{n-h}y_{j}'w_{j}
        \]
        $ v\sim v' $ $ \iff $ $ v-v' \in H $ $ \iff $ $ v-v' \in \mathscr{L}(v_1, \cdots, v_{h} ) $

        $ \iff $ $ y_{j}=y_{j}'$ $ \forall\, j = 1, \cdots, n-h $ 
        
        $\implies$ $ v'=\sum_{i=1}^{h}x_{i}'v_{i} + \sum_{j=1}^{n-h}y_{j}w_{j} $ 
        
        $\implies$ $ \phi[v] $ non dipende dal rappresentante $ v $.
        \item $ \phi $ è lineare, infatti 
        \begin{multline*}
            \phi([v]+[v'])=\phi([v+v'])=\\
            =\phi\Biggl(\biggl[\sum_{i=1}^{h}(x_{i}+x_{i}')v_{i}+\sum_{j=1}^{n-h}(y_{j}+y_{j}'  )w_{j}   \biggl]\Biggl)=\\
            = \sum_{j=1}^{n-h} (y_{j}+y_{j}')w_{j}=  \sum_{j=1}^{n-h} y_{j}w_{j}+\sum_{j=1}^{n-h} y_{j}'w_{j}=\\
            =\phi[v]+\phi[v'].
        \end{multline*}
        La dimostrazione di $ \phi(\lambda[v])=\lambda\phi([v]) $.
        \item $ \phi $ è iniettiva: infatti $ \phi([v])=\underline{0} $ 
        
        $\implies$ $ v=\sum_{i=1}^{h} x_{i}v_{i}    $ 
        
        $\implies$ $ v \in \mathscr{L}(v_1, \cdots, v_{n} ) $ 
        
        $\implies$ $ v \in H  $ $ \implies $ $ [v]=\underline{0}_{V/H} $ 
        
        $\implies$ $ \ker \phi =\{\underline{0}_{V/H}\} $ 
        
        $\implies$ $ \phi $ iniettiva.
        \item $ \phi $ è suriettiva: infatti se $ z \in L $ 
        
        $\implies$ $ \phi([z])=z $ 
        
        $\implies$ $ \phi $ è suriettiva.
    \end{itemize}

    Dal momento che $ V/H $ è isomorfo ad $ L $ avranno stessa dimensione, segue la tesi.\qed
}

\definizione{}{
    Sia $ V $ spazio vettoriale su $ \K $, $ H \subseteq V $ sottospazio vettoriale. La funzione \begin{align*}
    \pi: V & \to V/H \\
    v & \mapsto [v]
    \end{align*} si dice la \textit{proiezione}. $ \pi $ è sempre lineare, suriettiva, e $ \ker(\pi)=H $
}

\teorema[(primo teorema di Isomorfismo)]{primoteoiso}{
    Siano $ V $ e $ W $ spazi vettoriali su un campo $ \K $, sia $ F:V \to W $ lineare.

    Allora la funzione \begin{align*}
    \tilde{F}:V/\ker(F) &\to \im(F)\\
    [v] &\mapsto F(v)
    \end{align*} è ben definita ed è un isomorfismo.
}
\dimostrazione{primoteoiso}{
    $ \tilde{F} $ è ben definita: infatti se $ v\sim w $ in $ V/\ker(F) $ 
    
    $\implies$ $ v-w \in \ker(F) $ 
    
    $\implies$ $ F(v-w)=\underline{0} $ 
    
    $\underset{\footnotemark}{\implies}$ $ F(v)=F(w) $. Quindi $ \tilde{F} $ non dipende dal rappresentante, ma solo dalle classi.\footnotetext{$F$ è lineare}

    $ \tilde{F} $ è suriettiva, infatti $ F:V\to \im(F) $ è suriettiva.

    $ \tilde{F} $ è iniettiva: infatti se $ \tilde{F}([v])=\underline{0} $ 
    
    $\implies$ $ F(v)=\underline{0} $ 
    
    $\implies$ $ v \in \ker(F) $ 
    
    $\implies$ $ [v]= \underline{0}_{V/\ker(F)} $ 
    
    $\implies$ $ \ker(\tilde{F}) $ è banale, e $ \tilde{F} $ è iniettiva.

    %TODO manca diagramma (da foto): https://photos.app.goo.gl/hk6u1kQ8wu3ypxfm7
    Il diagramma è commutativo
}

\section{Spazio vettoriale duale}

Si consideri $ V $ spazio vettoriale su un campo $ \K $, si definisce il \textit{duale} \begin{equation}
    V^{*}=\bigl\{\alpha: V\to \K\,|\, \alpha\text{ lineare}\bigr\}
\end{equation} $ V^{*} $ eredita una struttura di spazio vettoriale su $ \K $, tramite \begin{align*}
    (\alpha_1+\alpha_2)(v) &:= \alpha_1(v)+\alpha_2(v)\\
    (\lambda\alpha)(v)&:= \lambda\bigl(\alpha(v)\bigr)\\
    \forall\, \alpha, \alpha_1, \alpha_2 \in V^{*}, &v \in V e \lambda \in \K
\end{align*}

$ V^{*} $ con questa struttura di spazio vettoriale su $ \K $ si dice lo \textit{spazio duale} di $ V $.
\esempio{
    Sia $ V= \K^{n} $, \begin{align*}
    \alpha_{i} : \K^{n} & \to \K \\
    (x_1, \cdots, x_{n} ) & \mapsto x_{i} 
    \end{align*} $ \alpha_{i} \in ( \K^{n})^{*}  $
}
\esempio{
    Sia $ V= \K^{n,n} $, \begin{align*}
    \alpha: \K^{n,n} & \to \K \\
    A & \mapsto \tr(A)
    \end{align*} $ \alpha \in ( \K^{n,n})^{*}  $
}

\rule{7em}{.4pt}

D'ora in avanti $ V $ è finitamente generato. In questo caso se $ \mathscr{B} =\{v_1, \cdots, v_{n}\} $ è una base fissata, ogni elemento $ \alpha \in V^{*} $ è determinato dai valori che assume nei vettori di $ \mathscr{B} $.

\teorema{iidioijsoidiotaciccioso}{
    Se $ V $ è finitamente generato, $ V $ e $ V^{*} $ hanno la stessa dimensione.
}
\dimostrazione{iidioijsoidiotaciccioso}{
    Sia $ \mathscr{B}=\{v_1, \cdots, v_{n}\}  $ una base di $ V$ e consideriamo $ \alpha_{1}, \alpha_{n} \in V^{*}  $ definiti come \[
        \alpha_{i}(v_{j} )=\begin{cases}
            1 &\text{se } i=j\\
            0 &\text{se } i\neq j
        \end{cases} \qquad \alpha_{i}(v_{j} )=\delta_{ij}  
    \] Ora si dimostra che $ \{\alpha_1, \cdots, \alpha_{n} \} $ è una base di $ V^{*} $ (da cui segue $ \dim V=\dim V^{*}$).

    Si dimostra $ \alpha_1, \cdots, \alpha_{n}  $ linearmente indipendenti e che sono dei generatori di $ V^{*}$.
    \begin{itemize}
        \item $ \alpha_1, \cdots, \alpha_{n}  $ linearmente indipendenti: siano $ \lambda_1, \cdots, \lambda_{n} \in \K $ tali che \[
            \lambda_1\alpha_1+\cdots+\lambda_{n}\alpha_{n}=\underline{0}_{V^{*}}.
        \] Questo significa che $ (\lambda_1\alpha_1+\cdots+\lambda_{n}\alpha_{n})(v)=0 $ $ \forall\, v \in V $. In particolare se $ v=v_{j}  $ ottengo \[
            (\lambda_1\alpha_1+\cdots+\lambda_{n}\alpha_{n})(v_{j} )=0
        \] 
        
        $\implies$ $ \parentesi{\lambda_{j} }{\lambda_1 \alpha_{1}(v_{j} )+ \cdots + \lambda_{n}\alpha_{n}(v_{j} )} =0 $ 
        
        $\implies$ $ \lambda_{j}=0  $, da cui deduco che $ \lambda_1=\lambda_2=\cdots=\lambda_{n}=0  $ 
        
        $\implies$ $ \alpha_1, \cdots, \alpha_{n}  $ linearmente indipendenti.
        \item $ \alpha_1, \cdots, \alpha_{n}  $ generano $ V^{*} $: infatti sia $ \alpha \in V^{*} $ e poniamo $ \lambda_{j}:=\alpha(v_{j})$. Si dimostra che \[
            \alpha=\lambda_1\alpha_1+\cdots+\lambda_{n}\alpha_{n}
        \] Infatti $ \alpha(v_{j} )=\lambda_{j}  $ e \[
            (\lambda_1\alpha_1+\cdots+\lambda_{n}\alpha_{n})(v_{j}) =\lambda_{j}    
        \] 
        
        $\implies$ $ \alpha $ e $ \lambda_1\alpha_1+\cdots+\lambda_{n}\alpha_{n} $ coincidono nei vettori di $ \mathscr{B} $, e quindi su tutto $ V$.\qed
    \end{itemize}
}

Ne segue che da una base $ \mathscr{B}=\{v_1, \cdots, v_{n}\}  $ di $ V $ è possibile sempre estrapolare \[
    \mathscr{B}^{*}=\{\alpha_1, \cdots, \alpha_{n}\} 
\] base di $ V^{*} $ detta base duale di $ V $.

Ciò permette di costruire un isomorfismo $ V\to V^{*} $ tramite la relazione che a $ v_{i}  $ fa corrispondere $ \alpha_{i}$ ($\implies$ $ V $ è isomorfo a $ V^{*} $).

Non esiste un isomorfismo canonico\footnote{definito senza dover scegliere una base di partenza} da $ V $ in $ V^{*} $.

\teorema{augustataurinorum}{
    $ V $ e $ V^{*} $ sono isomorfi (in dimensione finita), ma non canonicamente isomorfi.
}

Sia $ V $ spazio vettoriale su $ \K $, $ V^{*} $ spazio vettoriale su $ \K $. Posso considerare \begin{equation}
    (V^{*})^{*}=\{\gamma:V^{*}\to \K\,|\, \gamma\text{ lineare}\}
\end{equation} $ (V^{*})^{*} $ è lo spazio \textit{biduale} di $ V $

\teorema{evvivaqualcosadidimostrato}{
    $ V $ e $ (V^{*})^{*} $ in dimensione finita sono \textit{canonicamente isomorfi}.
}
\dimostrazione{evvivaqualcosadidimostrato}{
    Si definisce un isomorfismo $ \phi:V\to (V^{*})^{*} $ tramite la relazione \begin{equation}
        \phi(v)(\alpha):=\alpha(v)\qquad \forall\, v \in V, \alpha \in V^{*}
    \end{equation} Si dimostra che $ \phi $ è un isomorfismo: lineare, iniettiva.\footnote{Non c'è bisogno della suriettività perché hanno stessa dimensione.}
    \begin{itemize}
        \item $ \phi(v) \in (V^{*})^{*} $, cioè $ \phi(v): V^{*}\to \K $ è lineare, infatti \begin{multline*}
            \phi(v)(\lambda\alpha+\mu\beta)\underset{\footnotemarkk{elefantino}}{=}(\lambda \alpha+\mu\beta)=\\
            =\lambda \alpha(v)+\mu\beta(v)\underset{\footnotemark}{=} \lambda\phi(v)(\alpha)+\mu \phi(v)(\beta)
        \end{multline*}
        \consecfoottext{1}{elefantino}{definizione di $ \phi $}
        \footnotetext{definizione di $ \phi $}
        \item $ \phi $ è lineare: devo verificare che \[\phi(\lambda v + \mu w)= \lambda\phi(v)+\mu\phi(v)\qquad\forall\, \lambda, \mu \in \K, v, w \in V \] cioè che \[
            \phi(\lambda v + \mu w)(\alpha)=\lambda\phi(v)(\alpha)+\mu\phi(v)(\alpha)\quad \forall\, \alpha \in V^{*}
        \] ma \[
            \phi(\lambda v + \mu w)(\alpha)=\alpha(\lambda v + \mu w)\underset{\footnotemark}{=}\lambda \alpha (v)+ \mu \alpha(w)=\lambda\phi(v)(\alpha)+\mu\phi(w)(\alpha)
        \]\footnotetext{$\alpha$ è lineare}
        \item $ \phi $ è iniettiva: infatti se $ \phi(v)=\underline{0} \in (V^{*})^{*} $ 
        
        $\implies$ $ \phi(v)(\alpha)=0 $ $ \forall\,\alpha \in V^{*}  $ 
        
        $\implies$ $ \alpha(v) = 0 $ $ \forall\, \alpha \in V^{*} $ 
        
        $\implies$ $ v=\underline{0} $ $ \implies $ $ \ker(\phi)=\{\underline{0}\} $ e $ \phi $ è iniettiva. \qed
    \end{itemize}
}

\osservazione{
    In dimensione infinita, la stessa dimostrazione funziona per trovare una funzione lineare $ V\to (V^{*})^{*} $ iniettiva. Se $ V $ e $ (V^{*})^{*} $ sono isomorfi, $ V $ si dice \textit{riflessivo}.}