\subsection{Forme quadratiche reali ($ \K=\R$)}
Sia\marginnote{7 dic 2021} $ V $ spazio vettoriale su $ \R $, e $ \xi \in B_{S}(V, \R)  $. $ \xi $ si dice\begin{enumerate}
    \item \textit{definita positiva} se $ \xi(v,v)>0 $ $ \forall\, v \in V, v\neq \underline{0} $;
    \item \textit{definita negativa} se $ \xi(v,v)<0 $ $ \forall\, v \in V, v\neq \underline{0} $;
    \item \textit{semidefinita positiva} se $ \xi(v,v)\ge 0 $ $ \forall\, v \in V $;
    \item \textit{semidefinita negativa} se $ \xi(v,v)\le 0 $ $ \forall\, v \in V $;
    \item \textit{indefinita} se non rientra nei casi precedenti.
\end{enumerate}
\esempi{}{
    \begin{enumerate}
        \item Se $ \xi $ è un prodotto scalare, $ \xi $ è definita positiva, e $ -\xi  $ è definita negativa;
        \item considero $ \xi \in B_{S}(\R^{3}, \R)  $, tale che $ Q_{\xi}(x)=x_1^{2}+x_2^{2} $, semidefinita positiva, non definita positiva;
        \item considero $ \xi \in B_{S}(\R^{3}, \R)  $, tale che $ Q_{\xi}(x)=-x_1^{2}-x_2^{2} $, semidefinita negativa, non definita negativa;
        \item considero $ \xi \in B_{S}(\R^{3}, \R)  $, tale che $ Q_{\xi}(x)=x_1^{2}-x_2^{2} $, indefinita.
    \end{enumerate}
}

\osservazione{
    Se $ \xi  $ è definita (positiva o negativa), allora $ \xi $ è non degenere. Infatti $ \xi(v,v)\neq 0 $ se $ v\neq \underline{0} $, quindi $ \ker\xi=\{\underline{0}\} $. 
    
    Ma esistono forme bilineari indefinite non degeneri, come $ \xi \in B_{S}(\R^{3}, \R )  $, $ Q_{\xi}(x)=x_1^{2}+x_2^{2}-x_3^{2}  $
}

\teorema{eleferelicuresaij}{
    Sia $ \xi \in B_{S}(V, \R)  $, semidefinita positiva. Allora vale la disuguaglianza (di Cauchy-Schwartz) \begin{equation}
        |\xi(v,w)|^{2}\le Q_{\xi}(v) \cdot Q_{\xi}(w)\qquad \forall\, v, w \in V  
    \end{equation}
}
\dimostrazione{eleferelicuresaij}{
    \begin{itemize}
        \item Se $ Q_{\xi}(v)=0 $ $ \forall\, v \in V $ 
    
    $\implies$ $ \xi(v, w)=0 $ $ \forall\, v, w \in V $, vale il teorema.
    \item Se $ Q_{\xi}  $ non è identicamente nulla: si considerano tre casi\begin{enumerate}
        \item $ Q_{\xi}(v)\neq 0  $, $ Q_{\xi}(w)=0  $;
        \item $ Q_{\xi}(v)= 0  $, $ Q_{\xi}(w)\neq 0  $;
        \item $ Q_{\xi}(v)\neq 0 \neq Q_{\xi}(w) $;
    \end{enumerate}

    \textit{Caso 1.} $ \forall\, \lambda \in \R $, $ Q_{\xi}(v+ \lambda w)\ge 0 $ poiché $ \xi $ semidefinita positiva \[
        Q_{\xi}(v+\lambda\, w)=\xi(v+\lambda\, w, v+\lambda\, w) =\xi (v,v)+2 \lambda\xi(v, w)\ge 0
    \]
    Quindi $ Q_{\xi}(v)+2 \lambda \xi(v,w)  $ $ \forall\, \lambda \in \R $ 
    
    $\implies$ $ \xi(v,w)=0 $ e vale l'uguaglianza.

    Il \textit{caso 2.} si ricava dal caso 1. invertendo $ v $ e $ w $.

    \textit{Caso 3.} $ Q_{\xi}(v)\neq 0 \neq Q_{\xi}(w) $. So che $ \forall\, \lambda \in \R $, $ Q_{\xi}(v+\lambda\,w)\ge 0  $ \begin{multline*}
        0\le Q_{\xi}(v+\lambda\,w)=\xi(v+\lambda\,w, v+\lambda\,w)=\\=Q_{\xi}(v)+\lambda^{2}Q_{\xi}(w)+2 \lambda \xi(v,w)   
    \end{multline*} Sia \[
        p:= \lambda^{2} Q_{\xi}(w)+ 2 \lambda \xi (v,w) + Q_{\xi}(v)  
    \] $ p $ polinomio di secondo grado in $ \R_{2}[\lambda]  $.
    Poiché $ p\ge 0 $, il $ \Delta\le 0 $ 
    
$\implies$ $ \xi(v,w)^{2}- Q_{\xi}(v) \cdot Q_{\xi}(w)\le 0  $ 

$\implies$ $ |\xi(v,w)| \le Q_{\xi}(v) \cdot Q_{\xi}(w)   $\qed
    \end{itemize}
}

\proposizione{elefanterosacandido}{
    Nelle ipotesi del teorema precedente vale la disuguaglianza triangolare:\begin{equation}
        \sqrt{Q_{\xi}(v+w)}\le \sqrt{Q_{\xi}(v) }+\sqrt{Q_{\xi}(w) }\quad \forall\, v, w \in V
    \end{equation}
}
\dimostrazioneprop{elefanterosacandido}{
    Usiamo $ Q_{\xi}(v+w)=Q_{\xi}(v)+Q_{\xi}(w)+2\xi(v,w)$.
    \begin{align*}
        \left|Q_{\xi}(v+w) \right|&\le \left|Q_{\xi}(v)\right|+\left|Q_{\xi}(w) \right|+2\,\left|\xi(v,w)\right|\\
        &\le \left|Q_{\xi}(v)\right|+\left|Q_{\xi}(w) \right|+2\,\sqrt{Q_{\xi}(v) }\,\sqrt{Q_{\xi}(w) }\\
        &= \left(\sqrt{Q_{\xi}(v) }+\sqrt{Q_{\xi}(w)}\right)^{2}
    \end{align*} segue la tesi.\qed
}

\rule{7em}{.4pt}

Dato $ V $ spazio vettoriale su $ \R$, finitamente generato, e $ \xi \in B_{S}(V, \R)  $. Fisso $ \mathscr{B}=\{v_1, \cdots,v_{n}\} $ base di $ V $. Il teorema di Lagrange afferma che $ \exists\, \mathscr{B}' $ base di $ V $ ortogonale rispetto a $ \xi $, cioè tale che $ M^{ \mathscr{B}'}(\xi) $ diagonale, e quindi se $ (v)_{ \mathscr{B}'}=(x_1, \cdots, x_{n} ) $, vale $ Q_{\xi}(v)=x_1^{2}a_{11}+\cdots+ x_{n}^{2}a_{nn}    $.

Si può utilizzare il \textit{teorema spettrale}. $ \mathscr{B} $ da origine a $ M^{ \mathscr{B}}(\xi) $, se $ \mathscr{B}' $ è un'altra base $\implies$ $ M^{ \mathscr{B}'}(\xi)=\null^{t}\!P\,M^{ \mathscr{B}}(\xi)\, P $ con $ P \in \gl(n, \R) $ matrice del cambiamento di base. 

Qui si può utilizzare il teorema spettrale, $ M^{ \mathscr{B}}(\xi) $ è diagonalizzabile, ed esiste $ P \in O(n) $ tale che \[P^{-1}\, M^{ \mathscr{B}}(\xi)\,P=\begin{pmatrix}
    \lambda_1 & & \\
    & \ddots &\\
    & & \lambda_{n} 
\end{pmatrix}\] dove $ \lambda_1, \cdots, \lambda_{n}  $ sono gli autovalori di $ M^{ \mathscr{B}}(\xi) $. 

Poiché $ P \in O(n) $, $ \null^{t}\!P=P^{-1} $ 

$\implies$ $ \null^{t}\!P\,M^{ \mathscr{B}}(\xi) P = \biggl(\begin{smallmatrix}
    \lambda_1 & & \\
    & \ddots &\\
    & & \lambda_{n} 
\end{smallmatrix}\biggr) $, quindi si trova sempre una forma canonica di $ Q_{\xi}  $ in cui i coefficenti sono gli autovalori di $ M^{ \mathscr{B}}(\xi) $.

\todo{Manca un esercizio (chiesto a Simone Pacini)}

\osservazione{
    Sia $ \xi \in B_S(V, \R) $, esiste una base ortogonale di $ \xi $, $\mathscr{B}=\{v_1, \cdots, v_{n}\}$, $ \xi(v_i, v_j)=0$ $\forall\, i\neq j $ \[
        \xi(v_{i}, v_{i}  )\begin{cases}
            >0\\
            =0\\
            <0
        \end{cases}
    \] sia $ a_i:=\xi(v_i, v_i) $; si definisce $\mathscr{B}=\{v_1',\cdots, v_{n}'\}  $ dove 
    \begin{align*}
        v_i'&=v_i&\text{se }a_i=0\\
        v_i'&=\frac{v_i}{\sqrt{a_i}} &\text{se } a_i>0\\
        v_i'&=\frac{v_i}{\sqrt{-a_i}} &\text{se }a_i<0
    \end{align*}
        Risulta \[
        \xi(v_{i}', v_{i}'  )=\begin{cases}
            1 &\text{se }\xi(v_{i}, v_{i}  )>0\\
            0 &\text{se }\xi(v_{i}, v_{i}  )=0\\
            -1 &\text{se }\xi(v_{i}, v_{i}  )<0
        \end{cases}
        \]
        Si trova una base $\mathscr{B}'=\{v_1', \cdots, v_{n}'\}  $ ortogonale tale che $ \xi(v_i', v_i') \in \{-1, 0, 1\} $. Rispetto a questa base, $ Q_\xi $: \[
            Q_{\xi}(x)=x_1^{2}+ \cdots + x_{r}^{2}- x_{r+1}^{2}-\cdots-x_{r+s}^{2}\qquad r+s\le n = \dim V    
        \]

        In questo caso si dice che $ Q_{\xi} $ è in \textit{forma normale}.
} 

\esempio{}{
    Nell'esempio precedente la forma normale di $ Q_{\xi}(x)=-x_1^{2}+x_2^{2}+x_3^{2}  $
}

\teorema[(di Sylvester)]{silversdffsdf}{
    Sia $ V $ uno spazio vettoriale su $ \R $, finitamente generato. Sia $ \xi \in B_{S}(V, \R) $.
    
    Allora tutte le forme canoniche associate a $ Q_{\xi}  $ hanno lo stesso numero di coefficienti positivi ($p$) e lo stesso numero di coefficienti negativi ($q$).

    $ p+q=\rank (\xi) $ e la coppia $ (p,q) $ si dice la segnatura di $ \xi $.
}
\dimostrazione{silversdffsdf}{
    Sia $ \mathscr{C}=\{v_1, \cdots, v_{n}\}  $ una base ortogonale di $ V $ rispetto a $ \xi $, tale che $ Q_{\xi}  $ si scriva in forma normale, \[
        Q_{\xi}(v)= y_1^{2}+\cdots+ y_{p}^{2}- y_{p+1}^{2}- \cdots- y_{r}^{2}    
    \] dove $ (v)_{ \mathscr{C}}=(y_1, \cdots, y_{n} ) $, $ r=\rank \xi $.

    Quindi $ \xi(v_{i}, v_{i}  )=1 $ per $ i=1, \cdots, p $, $ \xi(v_{j}, v_{j})=-1 $ se $ j=p+1, \cdots, r $, $ \xi(v_{k}, v_{k}  )=0 $ se $ k=r+1, \cdots, n $.

    Sia $ \mathscr{C}'=\{v_1', \cdots, v_{n}'\}  $ un'altra base di $ V $ rispetto a cui $ Q_{\xi}  $ si scrive in forma normale \[
        Q_{\xi}(w) =z_1^{2}+\cdots+z_{t}^{2}-z_{t+1}^{2}-\cdots-z_{r}^{2}   
    \] dove $ (v)_{ \mathscr{C}'}=(z_1, \cdots, z_{n} ) $.

    Devo dimostrare che $ t=p $. Supponiamo per assurdo $ t<p $. Si considerino \[
        W_1= \mathscr{L}(v_1, \cdots, v_{p} )\qquad W_2= \mathscr{L}(v_{t+1}', \cdots, v_{n}'  )
    \] Vale che $ \dim W_1=p $, $ \dim W_{2}=n-t  $, \[\dim W_1+\dim W_2 = p+n-t>0\] 
    
    $\implies$ $ W_1\cap W_2 \neq \{\underline{0}\}$ 
    
    $\implies$ $ \exists v \in W_1\cap W_2 $, $ v\neq 0 $.
    
    Si ha quindi $ Q_{\xi}(v)>0  $ poiché $ \restriction{\xi}{W_1\times W_1}$ è definita positiva, ma inoltre $ Q_{\xi}(v)\le 0  $, poiché $ \restriction{\xi}{W_2\times W_2} $ è semidefinita negativa. Assurdo.
    
    $\implies$ $ t\ge p $. 
    
    In modo analogo si dimostra che $ t>p $ porta ad un assurdo (esercizio%ESERCIZIO fare per esercizio
    ) 
    
    $\implies$ $ t=p $\qed
}

Si definisce \textit{segnatura} di una $ \xi \in B_{S}(V, \R)$ la coppia di numeri $ (p,q) $, con $p $ il numero di coefficienti positivi di una forma canonica di $ \xi $ e $ q $ il numero di coefficienti negativi di una forma canonica di $ \xi $.

Per calcolare la forma canonica si fissa $ \mathscr{B} $ base di $ V $ e si trovano gli autovalori di $ M^{ \mathscr{B}}(\xi) \in \R^{n,n}$ simmetrica.

\definizione{}{
    Sia $ A \in \R^{n,n} $ una matrice simmetrica. Si definisce la \textit{segnatura} di $ A $ come la coppia di numeri $ (p,q) $, dove $ p $ è il numero di autovalori positivi di $ A $, mentre $ q $ è il numero degli autovalori negativi si $ A $, entrambi contati con la loro molteplicità.
}

\corollario{lklkmlkmlsdfjoijoijoijsdf}{
    Siano $ A, B \in \R^{n,n}$ due matrici simmetriche. Se $ A $ e $ B $ sono congruenti $\implies$ $ A $ e $ B $ hanno la stessa segnatura.
}{}

Il viceversa non è vero, controesempio: $ A=\bigl(\begin{smallmatrix}
    1 & 0 \\ 0 & 2
\end{smallmatrix}\bigr) $, $ B=\bigl(\begin{smallmatrix}
    1 & 0 \\ 0 & 1
\end{smallmatrix}\bigr) $ hanno la stessa segnatura $ (2, 0)$ ma non sono congruenti poiché $ \det A \neq \det B $

\rule{7em}{.4pt}

Data $ \xi \in B_{S}(V, \R)$ voglio trovare la forma normale di $ Q_{\xi}  $ ``a occhio'' (senza dover implementare il calcolo degli autovalori).

Fisso $ \mathscr{B} $ base di $ V $, considero $ M^{ \mathscr{B}}(\xi)=A $ e $ r(\lambda)=\det (A-\lambda\id) $.

La segnatura di $ \xi $ è $ (p, q) $, dove $ p $ è il numero di radici positive di $ r(\lambda) $, mentre $ q $ è il numero di radici negative di $ r(\lambda) $.

\paragraph{Regola di Cartesio} Sia $ r \in \R[x] $ un polinomio di grado $ n $, \[
        r(x)=a_0+a_1\,x+\cdots+a_{n}\, x^{2} 
    \] 
    
    $\implies$ il numero di radici positive di $ r(x) $ è il numero di variazione di segno dei coefficienti $ a_0, a_1, \cdots, a_{n}$, consecutivi tralasciando i coefficienti nulli.

\esempio{
    Sia $ \xi \in B_{S}(\R^{3}, \R)  $ definita da $ Q_{\xi}(x)=2x_1^{2}+4x_1x_3+x_2^{2}-x_3^{2}   $. Calcolo la forma normale di $ Q_{\xi}$.
    \[
        M^{ \mathscr{B}}(\xi)=\begin{pmatrix}
            2 & 0 & 2\\
            0 & 1 & 0\\
            2 & 0 & -1
        \end{pmatrix}\qquad p(\lambda)=-\lambda^{3}+2 \lambda^{2}+5 \lambda-6
    \]

    Si deduce che $ p $ ha due radici positive. $ p(0)=-6\neq 0 $, quindi $ p $ non ha radici nulle, allora $ p $ ha una radice negativa.

    La forma normale di $ Q_{\xi}  $ è \[
        Q_{\xi}(y)=y_1^{2}+y_2^{2}-y_3^{2}
    \]
}