\dimostrazione{oodoikjfkkekodino}{
    \begin{itemize}
        \item [``$1. \,\implies\,4.$''] Per ipotesi $ F $\marginnote{29 nov 2021} diagonalizzabile 
        
        $\implies$ $ \exists\, \mathscr{B} $ base di $ B $ rispetto a cui $ A=M^{ \mathscr{B}, \mathscr{B}}(F) $ è diagonale. \[
            A=\begin{pmatrix}
                a_1 & & & &\\
                & \ddots \\
                & & \ddots\\
                & & & \ddots\\
                & & & & a_{n} 
            \end{pmatrix}\qquad \begin{gathered}
                p(\lambda)=\det (A-\lambda\Id)=(a_1-\lambda)\\
                (a_{n}-\lambda )=\prod_{i=1}^{n}(a_{i}- \lambda )
            \end{gathered}
        \]
        
        $\implies$ $ p(\lambda) $ ha tutte le radici in $ \K $.

        Siano $ \{\lambda_1, \cdots, \lambda_{l}\}  $ le radici di $ p $. A meno di riordinare i vettori di $ B  $ posso supporre \[
            \mathscr{B}=\{\underset{\footnotemark}{\underbrace{v_1, \cdots, v_{k_1}}}, \underset{\footnotemark}{\underbrace{v_{k_1+1} , \cdots, v_{k_2+k_1}}}, \cdots, \underset{\footnotemark}{\underbrace{v_{k_{l-1}+kl }}}\}
        \]
        \addtocounter{footnote}{-2}
        \footnotetext{autovettori rispetto a $ \lambda_1 $}
        \stepcounter{footnote}
        \stepcounter{footnote}
        \addtocounter{footnote}{-1}
        \footnotetext{autovettori rispetto a $ \lambda_2 $}
        \stepcounter{footnote}
        \footnotetext{autovettore rispetto a $ \lambda_{l}  $}
        %TODO manca matrice 
        
        $\implies$ $ p(\lambda)=\det (A-\lambda\Id)=(\lambda_1-\lambda)^{k_1}(\lambda_2-\lambda)^{k_2} \cdot \cdots \cdot (\lambda_{l}-\lambda )^{k_{l} } $ con $ \lambda_{i}\neq \lambda_{j} $ per $ i\neq j $ 
        
        $\implies$ $ m_{a}(\lambda_{i} )=k_1   $

        $ \{v_1, \cdots, v_{k}\}  $ è una basee di $ V_{\lambda_1}  $ 
        
        $\implies$ $ \dim V_{\lambda_1} =k_1 $ 
        
        $\implies$ $ m_{g}(\lambda_1)=k_1  $

        $ \{v_{k_1+1}, \cdots, v_{k_1+k_2}\}  $ è una basee di $ V_{\lambda_1}  $ 
        
        $\implies$ $ \dim V_{\lambda_2} =k_2 $ 
        
        $\implies$ $ m_{g}(\lambda_2)=k_2 $

        Si itera il processo per ogni $ \lambda_{i}  $. 
        
        $\implies$ $ m_{a}(\lambda_{i} )= m_{g}(\lambda_{i} )   $ $ \forall\, i =1, \cdots, l $
        \item [``$4. \,\implies\,3.$''] Per ipotesi $ p $ ha tutte le radici in $ \K $ e $ m_{g}(\lambda_i)=m_{a}(\lambda_i)$ per ogni $ \lambda_{i}  $ radice di $ p $.
        
        Per ipotesi $ \{\lambda_1, \cdots, \lambda_{l} \} $ sono le radici di $ p $ e soddisfano \[m_{a}(\lambda_1)+\cdots+m_{a}(\lambda_{l} ) =n  \] con $ n= \dim V=\deg p $.

        $ m_{g}(\lambda_i)\le m_{a}(\lambda_i) \, \forall\, \lambda_i  $ 
        
        $\implies$ $ m_{g}(\lambda_1)+\cdots+ m_{g}(\lambda_{l} )=n=\dim V   $
        \item [``$3. \,\implies\,2.$''] Per ipotesi \[\dim V=n= m_{g}(\lambda_1)+\cdots+ m_{g}(\lambda_{l} )\] sappiamo che $ m_{g}(\lambda_{i})= \dim V_{\lambda_{i} }   $ \[
            V_{\lambda_{i} }\cap V_{\lambda_{j} }=\{\underline{0}\}\quad\text{se } \lambda_{i}\neq \lambda_j      
        \] 
        
        $\implies$ $ V=V_{\lambda_1}\oplus V_{\lambda_2}\oplus \cdots\oplus V_{\lambda_{l} }     $
        \item [``$2. \,\implies\,1.$''] Per ipotesi $ V=V_{\lambda_1}\oplus V_{\lambda_2}\oplus \cdots\oplus V_{\lambda_{l} } $. Considero una base $ \mathscr{B}_i $ di $ V_{\lambda_{i} }$ 
        
        $\implies$ $ \displaystyle\mathscr{B}=\bigcup_{i=1}^{l} \mathscr{B}_i  $ è una base di $ V $ fatta di autovettori di $ F $ 
        
        $\implies$ $ F $ è diagonalizzabile.\qed
    \end{itemize}
}

%TODO manca definizione

Una matrice $ A \in \K^{n,n}$ è diagonalizzabile se l'endomorfismo $ F_{A}: \K^{n}\to \K^{n}  $, $ x \mapsto Ax$ è diagonalizzabile.

$ \iff $ esiste $ P \in \gl (n, \K) $ tale che $ P^{-1}AP $ è diagonale.

\esercizio{
    Dire se la matrice $ A \in \R^{3,3} $, \[
        A=\begin{pmatrix}
            -1 & 0 & 0 \\
            0 & 1 & 1\\
            0 & 0 & 1
        \end{pmatrix}
    \] è diagonalizzabile.
}{
Si trova lo spettro di $ A $. \[
    p(\lambda)=\det(A-\lambda\id)= \det\begin{pmatrix}
        -1-\lambda & 0 & 0 \\
        0 & 1-\lambda & 1\\
        0 & 0 & 1-\lambda
    \end{pmatrix}=(-1-\lambda)(1-\lambda)^{2}
\] 

$\implies$
gli autovalori di $ A $ sono $ \pm 1 $, $ m_{a}(-1)=1  $, $ m_{a}(+1)=2  $.

\[m_{a}(-1)=1 \,\implies\, m_{g}(-1)=1  \,\implies\, m_{a}(-1)=m_{g}(-1)\]

Ci sono due possibilità:\begin{itemize}
    \item $ m_{g}(1)=1  $, $ A $ non è diagonalizzabile
    \item $ m_{g}(1)=2  $, $ A $ è diagonalizzabile
\end{itemize}


Si calcola $ m_{g}(1)  $, \[
    m_{g}(1)= \dim V_{1}=\dim \nulls (A-\I)  
\] \[
    A-I=\begin{pmatrix}
        -2 & 0 & 0\\
        0 & 0 & 1\\
        0 & 0 & 0
    \end{pmatrix}
\]
Imposto \[
    (A-I)\begin{pmatrix}
        x_1\\x_2\\x_3
    \end{pmatrix}=\underline{0} \,\implies\, \begin{pmatrix}
        -2 & 0 & 0\\
        0 & 0 & 1\\
        0 & 0 & 0
    \end{pmatrix}\begin{pmatrix}
        x_1\\x_2\\x_3
    \end{pmatrix}=\underline{0}
\] \[
    \,\implies\, \begin{cases}
        -2x_1=0\\x_3=0
    \end{cases} \,\implies\, \begin{cases}
        x_1=0\\ x_2=x_2\\ x_3=0
    \end{cases}
\] 

$\implies$ $ V_1= \mathscr{L}((0,1,0)) $ 

$\implies$ $ \dim V_1=1 $ 

$\implies$ $ m_{g} (1) =1$ 

$\implies$ $ A $ non è diagonalizzabile.}{}

\osservazione{
    Se $ \dim V =n$, e $ F \in\End(V) $ ha $ n $-autovalori (distinti) 
    
    $\implies$ $ F $ è diagonalizzabile. 
    
    Infatti la molteplicità algebrica di ogni autovalore è necessariamente $ 1 $ 
    
    $\implies$ la molteplicità geometrica di ogni autovalore è $ 1 $ 
    
    $\implies$ $ m_{g}(\lambda_i)=m_{a}(\lambda_i)  $ per ogni $ \lambda_i \in \spettro(F) $ 
    
    $\implies$ $ F $ è diagonalizzabile.
}

\osservazione{
    Sia $ V $ di dimensione finita, $ F \in \End(V)$, supponiamo che il polinomio $ p $ di $ F $ abbia tutte le radici in $ K $. Fisso $ \mathscr{B} $ base di $ V $, e sia $ A=M^{ \mathscr{B}, \mathscr{B}}(F) $, \begin{align*}
        p(\lambda)&=(\lambda_1-\lambda) \cdot \cdots \cdot (\lambda_{n}-\lambda )\\
        p(\lambda)&=(-1)^{n}\lambda^{n}-\tr(A)\lambda^{n-1}+\cdots+ (-1)^{n}\det A 
    \end{align*} con $\lambda_1, \cdots, \lambda_{n}$ non necessariamente distinti.
    
    Uguagliando le due scritture si ottiene \[
        \det A = \lambda_1 \cdot \lambda_2 \cdot \cdots \cdot \lambda_{n}\qquad \tr (A) =\lambda_1 + \lambda_2+ \cdots+ \lambda_{n}  
    \]
}
\subsection{Teorema di Caley-Hamilton}

Sia $ \K $ un campo e consideriamo $ \K^{n,n} $, $ \K[x] $. Se $ p \in \K[x] $, \[
        p(x)=a_{n}x^{n}+a_{n-1}x^{n-1}+\cdots+ a_1x+a_0\qquad p: \K\to \K  
    \]
    $ p $ induce una funzione $ p: \K^{n,n}\to \K^{n,n} $ \[
        p(A)=a_{n}A^{n}+a_{n-1}A^{n-1}+\cdots+ a_1A+a_0\id \in \K^{n,n}\, \forall\, A \in \K^{n,n}.
    \]

\teorema[(di Caley-Hamilton)]{dnondoinodin}{
    Se $ p $ è il polinomio caratteristico di $ A $, allora $ p(A)=\underline{0} $
}

Si definisce \[
        I=\{1 \in \K[x]\,|\, q(A)=\underline{0}\}, p \in I
    \] e da qui si arriva al polinomio minimo di $ A $.

\subsection{Teorema Spettrale}

\teorema[(versione geometrica)]{spettreucl}{
    Sia $ (V, \cdot ) $ uno spazio vettoriale euclideo ($V$ spazio vettoriale su $ \R $, $ \dim V< + \infty $ e $ \cdot  $ è un prodotto scalare di $ V $). 
    \begin{enumerate}
        \item Sia $ F \in \End(V) $ simmetrico (cioè $ F(v) \cdot w=v \cdot F(w) $ $ \forall\, v, w \in V $) 
    
        $\implies$ $ F $ è diagonalizzabile e i relativi autospazi sono 2 a 2 ortogonali.
        \item Se $ F \in \End(V) $ diagonalizzabile, e i suoi autospazi sono 2 a 2 ortogonali 
        
        $\implies$ $ F $ è simmetrico.
    \end{enumerate}
}

\nota{$ F \in \End(V) $ è simmetrico $ \iff $ la matrice che rappresenta $ F $ rispetto ad una base ortonormale è simmetrica}

\teorema[(versione algebrica)]{spetttrialgege}{
    Sia $ A \in \R^{n,n} $ \begin{enumerate}
        \item Se $ A $ è simmetrica ($\null^{t}\!A=a$) 
        
        $\implies$ $ \exists\, P \in O(n) $ ($\null^{t}\!P=P^{-1}$) tale che $ P^{-1}AP = \null^{t}\!PAP $ è diagonale.
        \item Se esiste $ P \in O(n) $ tale che $ P^{-1}AP $ è diagonale 
        
        $\implies$ $ A $ è simmetrica.
    \end{enumerate}
}

\lemma{diiidiiididididoij}{
    Sia $ (V, \cdot )$ uno spazio vettoriale Hermitiano ($V$ spazio vettoriale su $\C$ con $ \dim V< +\infty $, $ \cdot  $ prodotto Hermitiano), $ F \in \End(V) $ un endomorfismo Hermitiano (i.e. $ F(v) \cdot w = v \cdot F(w)$ $ \forall\, v, w \in V $) 
    
    $\implies$ tutti gli autovalori di $ F $ sono reali. 
}
\dimostrazionelem{diiidiiididididoij}{
    Sia $ \lambda \in \C $ un autovalore di $ F $ 
    
    $\implies$ $ \exists\, v\neq \underline{0} \in V  $ tale che $ F(v)= \lambda v $ \[
        \lambda ||v||^{2} = F(v) \cdot v = v \cdot F(v) = v (\lambda v)=\overline{\lambda} ||v||^{2}.
    \]
    Poiché $ ||v||\neq \underline{0} $ ottengo $ \lambda=\overline{\lambda} $, cioè $ \lambda \in \R $\qed
}

\corollario{maiochenesoscusi}{
    Se $ A \in \C^{n,n} $ è una matrice Hermitiana (i.e. $ A=\null^{t}\!\overline{A} $) 
    
    $\implies$ gli autovalori di $ A $ sono reali.
}{}

\dimostrazionecrl{maiochenesoscusi}{
    Ogni endomorfismo Hermitiano rispetto ad una base unitaria è rappresentato da una matrice Hermitiana, quindi si applica il lemma precedente.\qed
}

\lemma{diiidiioijoooijoajkfjlkjdf}{
    Sia $ (V, \cdot ) $ uno spazio vettoriale Euclideo, sia $ F \in \End(V) $ simmetrico e $ \lambda_1, \lambda_2 \in \spettro(F) $ con $ \lambda_1\neq \lambda_2 $. 
    
    $\implies$ $ V_{\lambda_1}  $ e $ V_{\lambda_2}  $ sono ortogonali. ($v \cdot w =0$ $ \forall\, v \in V_{\lambda_1}, w \in V_{\lambda_2}   $)
}

\dimostrazionelem{diiidiioijoooijoajkfjlkjdf}{
    Siano $ v \in V_{\lambda_1}  $ e $ w \in V_{\lambda_2}  $. $ F(v) \cdot w =v \cdot F(w) $ poiché $ F $ è simmetrico.
    \[
        \lambda_1 v \cdot w = F(v) \cdot  = v \cdot (F(w)) = v \cdot (\lambda_2 w) = \lambda_2 v \cdot w
    \] 
    
    $\implies$ $ (\lambda_1-\lambda_2)v \cdot w = 0 $ con $ \lambda_1-\lambda_2\neq 0 $ 
    
    $\implies$ $ v \cdot w =0$.\qed
}

\dimostrazione{spettreucl}{
    \begin{enumerate}
        \item $ F \in \End(V)$ simmetrico. Considero $ \spettro(F)=\{\lambda_1, \cdots, \lambda_{l}\} $. 

        Per il lemma precedente so che $ V_{\lambda_i} \bot V_{\lambda_{j} }  $ se $ \lambda_i\neq \lambda_j $. Dimostro che 
        \[
            V = V_{\lambda_1}\oplus\cdots\oplus V_{\lambda_{l} }  
        \] 
        da cui si ottiene la tesi.

        Considero $ H=V_{\lambda_1} \oplus\cdots\oplus V_{\lambda_{l} }   $ sottospazio vettoriale di $ V $ tale che $ F(H) \subseteq H $ (poiché $ F(V_{\lambda_i} ) \subseteq V_{\lambda_{i} }  $).

        Suppongo per assurdo che $ H \varsubsetneq V $. 
        
        $\implies$ $ H^{\bot}\neq \{\underline{0}\} $

        Si verifica che $ F(H^{\bot}) \subseteq H^{\bot} $, infatti se $ v \in H^{\bot} $ 
    
        $  \implies$ $ V h = 0 $ $ \forall\, h \in H $ \[
            F(v) \cdot h \underset{\footnotemark}{=} v \cdot F(h) \underset{\footnotemark}{=}0
        \]
        \addtocounter{footnote}{-1}
        \footnotetext{poiché $ F $ è simmetrico}
        \stepcounter{footnote}
        \footnotetext{poiché $ F(h) \in H $}

        Sia $ F'=\restriction{F}{H^{\bot}}$, $ F' :H\bot \to H^{\bot} $ ed è un endomorfismo simmetrico. 

        Per un lemma precedente $ F' $ ha almeno un autovalore.
    \end{enumerate}
    
}

\dimostrazione{spetttrialgege}{
    
}

% TODO creare pdf singolo + pagina obsidian
