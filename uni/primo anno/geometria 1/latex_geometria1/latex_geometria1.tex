\documentclass[twoside, 11pt, titlepage]{article}

bookmark88�anag��LK+��Archi�Usersdavidepeccioli	DocumentsPersonal Knowledge Managementnote archiveuni
primo anno	unito.tex  8Lt���(�
�����n�%���b��� ��$4DTA�'���	file:///Macintosh HD� htA��$E81A49BF-F8E1-4658-8356-C54D8CF4E1D4���/3dnibtex????�����d�@�T�U�V� | � � 0     \0 ���� �������"��

\begin{document}

\begin{titlepage}
\null
\vfill
\begin{center}
{\Huge \textbf{Geometria 1}}\\
\vspace{1em}
{\large Davide Peccioli}\\
\vspace{0.6em}
{\large Anno accademico 2021-2022}
\vfill
Università degli studi di Torino
\end{center}
\end{titlepage}
{\pagestyle{empty}
\null\newpage}

\section{Matrici}

Una matrice è una tabella rettangolare di numeri reali ($\in\R$)

\[
A=\begin{pmatrix}
a_{1 1} & a_{1 2} & \cdots & a_{1 n} \\
a_{2 1} & a_{2 2} & \cdots & a_{2 n}\\
\vdots & \vdots & \vdots & \vdots \\
a_{m 1} & \cdots & \cdots & a_{m n}
\end{pmatrix}\qquad \begin{aligned}
\text{contiene } &m\cdot n \text{ numeri}\\
\text{contiene } &m \text{ righe}\\
\text{contiene } &n \text{ colonne}
\end{aligned}
\]

$a_{ij}$ è l'elemento della matrice nella $i$-esima riga e nella $j$-esima colonna. $a_{ij}\in\R$.

$A$ è una matrice $m\cdot n$. Se $m=n$ allora $A$ è una \textbf{matrice quadrata}.

Le matrici servono per:
\begin{itemize}
\item risolvere sistemi lineari
\item studiare spazi vettoriali
\item classificarre strutture geometrice (es. coniche)
\item presentare funzioni (semplificandone lo studio)
\end{itemize}

$\R^{m,n}$ è l'insieme delle matrici $m\cdot n$:
\begin{itemize}
\item $\Q^{m, n}$ è l'insieme delle matrici $m\cdot n$ le cui entrate sono elementi di $\Q$.
\end{itemize}

\esempi{
\begin{itemize}
\item $\R^{2,2}$: matrici $2\cdot 2$
	\[
	\begin{pmatrix}
		a_{11} & a_{12} \\
		a_{21} & a_{22}
	\end{pmatrix},\quad
	\begin{pmatrix}
		1 & 2 \\
		0 & 3
	\end{pmatrix},\quad
	\begin{pmatrix}
		5 & 6 \\
		-1 & \frac{1}{2}
	\end{pmatrix}\cdots\in\R^{2,2}
	\]
\item $\R^{1,1}=\R$
\item $\R^{m,1}$:
	\[
	A=\begin{pmatrix}
	a_{11} \\
	a_{21} \\
	\vdots \\
	\vdots \\
	a_{m1}
	\end{pmatrix}\in\R^{m,1}\qquad\text{anche \textbf{vettori colonna}}
	\]
\item $\R^{1,n}$:
	\[
	A=\begin{pmatrix}
	a_{11} &
	a_{12} &
	\cdots&
	a_{1n}
	\end{pmatrix}\in\R^{1,n}\qquad\text{anche \textbf{vettori riga}}
	\]
\end{itemize}
}

In  $\R^{m,n}$ è sempre definita la \textbf{matrice nulla}, in cui tutte le entrate sono nulle. In $\R^{n,n}$ è sempre definita la \textbf{matrice identità}:
\[
I=\begin{pmatrix}
1 & 0 & \cdots & \cdots & \cdots & 0 \\
0 & 1 & 0 & \cdots & \cdots & 0 \\
0 & 0 & 1 & 0 &\cdots & 0 \\
\vdots & & & \ddots & & \vdots \\
\vdots & & & & \ddots & \vdots \\
0 & \cdots & \cdots & \cdots & 0 & 1
\end{pmatrix}
\]
\begin{itemize}
\item In $\R^{1,1}$, $I=1$
\item In $\R^{2,2}$
\[
I=\begin{pmatrix}
1 & 0 \\
0 & 1
\end{pmatrix}
\]
\item In $\R^{3,3}$
\[
I=\begin{pmatrix}
1 & 0 & 0 \\
0 & 1 & 0 \\
0 & 0 & 1
\end{pmatrix}
\]
\end{itemize}

La diagonale composta unicamente da $1$ nella matrice identità è ila \textbf{diagonale principale} della matrice.

\subsection{Somma}

Siano $A, B \in \R^{m,n}$
\[
A=\qmatrice{a}\qquad B=\qmatrice{b}
\]
\[
A+B=\begin{pmatrix}
a_{11}+b_{11} & a_{12}+b_{12} & \cdots & a_{1n}+b_{1n} \\
\vdots & & & \vdots \\
a_{m1}+b_{m1} & \hdotsfor{2} & a_{mn}+b_{mn}
\end{pmatrix}
\]

\esempi{
\begin{itemize}
\item In $\R^{1,1}$ la somma tra matrici coincide con la somma usuale di numeri reali.
\item $\displaystyle \begin{pmatrix}
1 & 2 & 3 \\
0 & -1 & 4
\end{pmatrix}+\begin{pmatrix}
0 & -2 & 1 \\
3 & -1 & 4
\end{pmatrix}=\begin{pmatrix}
1 & 0 & 4 \\
3 & -2 & 8
\end{pmatrix}$
\end{itemize}
}

\proprieta[della somma]{
\begin{itemize}
\item [(\textit{i})] La somma è \textbf{associativa}:\[\forall A,B,C\in\R^{m,n} \qquad (A+B)+C=A+(B+C)\] e posso scrivere $A+B+C$ senza ambiguità.
\item [(\textit{ii})] La somma è \textbf{commutativa} (o abeliana):
\[
\forall A,B\in\R^{m,n}\qquad A+B=B+A
\]
\item [(\textit{iii})] Se $A\in\R^{m,n}$ e $B\in\R^{m,n}$ è la matrice nulla ($B=\underline{0}$), allora $A+B=B+A=A$
\item [(\textit{iv})] $A-A=\underline{0}$: 
\[
\forall A \in \R^{m,n} \exists -A\in \R^{m,n} \:\tc\: A-A=0
\]
\definizione{
Data $A\in\R^{m,n}$,
\[
\text{con }A=\qmatrice{a}
\]
si definisce $-A$,
\[
\text{con }-A=\qmatrice{-a}
\]
}
\notazione{
In genere si scrive $A-B$ in luogo di $A+(-B)$, e si considera come una sottrazione di matrici
}
\end{itemize}}

\definizione{
Due matrici $A,B\in\R^{m,n}$ sono uguali se hanno le stesse entrate ($A=B$)
}
\proprieta{$A=B\iff B-A=0$}

\section{Gruppo}

\definizione{Siano $A, B$ due insiemi, si definisce \textbf{prodotto cartesiano}:
\[
A\times B=\{(a,b)\:\tc\: a\in A, b\in B\}
\]
in cui conta l'ordine: $(a,b)\neq_(b,a)$

\[
A\times A = \{(a_1, a_2)\:\tc\: a_1, a_2\in A\}
\]}

\definizione{Sia $G$ un insieme. Una \textbf{operazione} in $G$ è una funzione
\begin{align*}
\star : G\times G &\to G\\
(g, h) &\mapsto g\star h
\end{align*}}
\proprieta{
\begin{itemize}
\item[(\textit{i})] L'operazione è \textbf{associativa} se $(g\star h)\star k=g\star (h\star k)$
\item[(\textit{ii})] L'operazione ha un \textbf{elemento neutro} se \[\exists\, e\in G \:\tc\: g\star e=e\star g = g, \:\forall g \in G\]
\item [(\textit{iii})] Se $g\in G$ chiamiamo \textbf{inverso di $g$} un elemento
\[
k\in G \:\tc\: g\star k=k\star g = e
\]
\end{itemize}
}
\definizione{
Un \textbf{gruppo} è un insieme $G$ con un'operazione~$\star\:\tc$
\begin{enumerate}
\item $\star$ è associativa
\item esiste un elemento neutro
\item ogni elemento ha un inverso
\end{enumerate}
}

\esempi{Sono gruppi
\[
(\R, +),\: (\Z, +),\: (\Q, +),
\]
\[
\cancel{(\R, \cdot)}\text{: lo zero non ha un inverso},\]
\[
(\R\setminus\{0\}, \cdot),\: (\R^{m,n}, +)
\]
}
\definizione{Un gruppo $(G, \star)$ è \textbf{abeliano} se
\[
g\star h=h\star g \:\forall\: g,h\in G
\]
Nel caso di un gruppo abeliano l'operazione è indicata con $+$ e l'elemento neutro con $0$.
\[
(\R^{m,n}, +)\text{ è un gruppo abeliano}
\]}

\section{Operazioni con le matrici}
\subsection{Moltiplicazione}

Si può moltiplicare $\lambda \in \R$ con matrici $A\in\rmn$
\[
\lambda A=\lambda\qmatrice{a}=\qmatrice{\lambda a}
\]
\[
-1\cdot A = -A\qquad\text{coerente con la definizione di }-A
\]

\esempio{
\[
2\begin{pmatrix}
3 & 1 & 0\\
-1 & 4 & 1
\end{pmatrix}=\begin{pmatrix}
6 & 2 & 0\\
-2 & 8 & 2
\end{pmatrix}
\]
}

\osservazione{$0\cdot A$ è la matrice nulla $\forall A\in \rmn$}

\proprieta[del prodotto per scalari]{
\begin{itemize}
\item [(\textit{i})] $\displaystyle \lambda(A+B)=\lambda A + \lambda B \qquad\forall\lambda\in\R,\:A,B\in\rmn$
\item [(\textit{ii})] $\displaystyle (\lambda+\mu)\cdot A=\lambda\cdot A+\mu\cdot A\qquad\forall\lambda\mu\in\R,\:A\in\rmn$
\item [(\textit{iii})] $\displaystyle(\lambda\mu)A=\lambda(\mu A)\qquad\forall\lambda\mu\in\R,\:A\in\rmn$
\item [(\textit{iv})] $\displaystyle 1\cdot A = A\qquad\forall A\in\rmn$
\end{itemize}
}

$(\rmn, +)$ è un \textbf{gruppo abeliano} in cui è definita una moltiplicazione per scalari in cui valgono le proprietà \textit{i}-\textit{iv} (prototipo per gli spazi vettoriali).

\subsection{Prodotto tra matrici}
\[
A,B\:\tc\: A\in\R^{m,\textcolor{red}{n}}, B\in\R^{\textcolor{red}{n}, k}\implies AB\in \R^{m, k}
\]
Questo è definito come il prodotto \textbf{righe per colonne}. Il numero di colonne della prima matrice deve corrispondere con il numero di righe della seconda matrice.

%%%%%% APPUNTI GIORNO PER GIORNO

%% 5 ottobre 2021

\teorema{primo}{Sia $V$ uno spazio vettoriale su campo $\K$, e $W\subseteq V$ un sottospazio vettoriale:
\begin{enumerate}
\item \label{thm_1:1} se $V$ è finitamente generato $\implies$ $W$ è finitamente generato;
\item se  $V$ è finitamente generato $\implies$ $\dim W\le\dim V$
\item se $V$ è finitamente generato e $\dim W = \dim V$ $\implies$ $W=V$
\end{enumerate}
 }
\dimostrazione{primo}{
	\begin{enumerate}
	\item Supponiamo che $V$ sia finitamente generato, e per assurdo che $W$ non lo sia.
	
	$V$ è finitamente generato $\implies$ $V$ ha una base \[\mathscr{B}=\{v_1,\cdots, v_n\}\]
	
	$W$ non è finitamente generato, e sia $w_1\in W,\:w_1\neq \underline{0}$, considero $\mathscr{L}(w_1)\subseteq W$, ma $W\neq \mathscr{L}(w_1)$, altrimenti $W$ sarebbe generato da $w_1$. $\implies$ $\exists\: w_2\in W \land w_2 \notin \mathscr{L}(w_1)$.
	
	Considero $\mathscr{L}(w_1, w_2)\subseteq W$, ma $W\neq \mathscr{L}(w_1, w_2)$, altrimenti $W$ sarebbe generato da $\{w_1, w_2\}$. $\implies$\\$\implies$ $\exists\: w_3\in W \land w_3 \notin \mathscr{L}(w_1, w_2)$.
	
	Itero il procedimento e trovo \begin{multline*}\{w_1, \cdots, w_{n+1}\}\subseteq W\:\tc\: w_{n+1}\notin \mathscr{L}(w_1, \cdots, w_n)\:\implies \\ \implies\{w_1, \cdots, w_{n+1}\}\text{ è un insieme libero}\end{multline*}e contiene più elementi di una base $\mathscr{B}$. Assurdo per teorema precedente. % TODO aggiungere numero del teorema
	
	\item Supponiamo $V$ finitamente generato, e sia $W\subseteq V$ un sottospazio vettoriale. $W$ è finitamente generato (per \ref{thm_1:1}.) 
	$\implies$ $\exists \:\mathscr{B}=\{w_1, \cdots, w_m \}$ base di $W$ $\implies$ $\mathscr{B}\subseteq V$ è un sottoinsieme libero $\implies$ $m\le \dim V$ $\implies$ $\dim W \le \dim V$
	
	\item Sia $W\subseteq V$ uno spazio vettoriale, con $V$ finitamente generato. $\dim W = \dim V$.
	
	$W$ ha una base $\mathscr{B}$ con $n$ vettori, dove $n=\dim V$ $\implies$ $\mathscr{B}$ è una base di $V$.
	
	Se $\mathscr{B}=\{w_1, \cdots, w_n\}$ $\implies$ $W=\mathscr{L}(w_1, \cdots, w_n)=V$ $\implies$ $W=V$
	\end{enumerate}
}

\osservazione{
Se $V$ è uno spazio vettoriale finitamente generato, e $\dim V=n$ $\implies$ ogni insieme libero con $n$ elementi è una base. Infatti se $\mathscr{B}=\{v_1, \cdots, v_n\}$ è un insieme libero, se per assurdo esistesse $v\in V\land v\notin \mathscr{L}(v_1, \cdots, v_n)$ $\implies$ $\{v_1, \cdots, v_n, v\}\subseteq V$ è un insieme libero di cardinalità $n+1$ (ovvero con $n+1$ elementi). Assurdo.
}

\teorema[del completamento di una base]{complbs}{
Sia $V$ uno spazio vettoriale su un campo $\K$ finitamente generato. Sia $\mathscr{B}=\{v_1,\cdots,v_n\}$ una base di $V$ e sia $I=\{a_1, \cdots,a_l\}\subseteq V$ un sottoinsieme libero. Esiste sempre $\mathscr{B}'$ base di $V$ i cui primi $l$-elementi sono $a_1,\cdots,a_l$ e i restanti $n-l$-elementi sono elementi di $\mathscr{B}$.
\[
\mathscr{B}'=\{a_1,\cdots,a_l, w_1, \cdots, w_{n-l}\}\text{ con }w_1, \cdots, w_{n-l}\in\mathscr{B}
\]}

\dimostrazione{complbs}{
Applico il metodo degli scarti successivi
\begin{itemize}
\item [$l=n$] l'enunciato è banale ($I$ è già una base e non va completata);
\item [$l<n$] $\implies$ $\mathscr{L}(a_1,\cdots, a_l)\varsubsetneq V$ \\ $\implies$ $\exists\: w_1 \in \mathscr{B} \:\tc\: w_1\notin \mathscr{L}(a_1,\cdots, a_l)$. Infatti, se tutti i generatori appartenenti a $\mathscr{B}$ fossero combinazioni lineari di $a_1, \cdot, a_l$, non sarebbero più tutti linearmente indipendenti. $\implies$ $I_1=\{a_1,\cdot,a_l,w_1\}$ è libero.

Se $I_1$ è una base, la dimostrazione si conclude, altrimenti $\exists\: w_2 \in \mathscr{B} \:\tc\: w_2\notin \mathscr{L}(a_1,\cdots, a_l, w_2)$ \\ $\implies$ $I_1=\{a_1,\cdot,a_l,w_1, w_2\}$ è libero.

Se $I_2$ è una base la dimostrazione si conclude, altrimenti si itera fino a 
\[
I_{n-l}=\{a_1, \cdot, a_l, w_1,\cdots, w_{n-l}\}\text{ con }w_1, \cdots, w_{n-l}\in\mathscr{B}.
\]
$I_{n-l}$ è libero con $n$ vettori $\implies$ $I_{n-l}$ è una base
\end{itemize}
}

\esempio{$\mathcal{S}(\R^{3,3})=\{A\in\R^{3,3}\:\tc\:{}^t\!A=A\}$

Cerco una base. Sia $A\in\mathcal{S}(\R^{3,3})$ generica:
\[
A=\begin{pmatrix}
a & b & c \\
b & d & e \\
c & e & f
\end{pmatrix}\text{ con }a,b,c,d,e,f\in\R
\]

\begin{multline*}
A=a\begin{pmatrix}
1 & 0 & 0\\
0 & 0 & 0\\
0 & 0 & 0
\end{pmatrix}+b\begin{pmatrix}
0 & 1 & 0\\
1 & 0 & 0\\
0 & 0 & 0
\end{pmatrix}+\\+c\begin{pmatrix}
0 & 0 & 1\\
0 & 0 & 0\\
1 & 0 & 0
\end{pmatrix}+d\begin{pmatrix}
0 & 0 & 0\\
0 & 1 & 0\\
0 & 0 & 0
\end{pmatrix}+\\+e\begin{pmatrix}
0 & 0 & 0\\
0 & 0 & 1\\
0 & 1 & 0
\end{pmatrix}+f\begin{pmatrix}
0 & 0 & 0\\
0 & 0 & 0\\
0 & 0 & 1
\end{pmatrix}
\end{multline*}

Siano $E_1=\begin{pmatrix}
1 & 0 & 0\\
0 & 0 & 0\\
0 & 0 & 0
\end{pmatrix}$, $E_2=\begin{pmatrix}
0 & 1 & 0\\
1 & 0 & 0\\
0 & 0 & 0
\end{pmatrix}$, $E_3=\begin{pmatrix}
0 & 0 & 1\\
0 & 0 & 0\\
1 & 0 & 0
\end{pmatrix}$, $E_4=\begin{pmatrix}
0 & 0 & 0\\
0 & 1 & 0\\
0 & 0 & 0
\end{pmatrix}$, $E_5=\begin{pmatrix}
0 & 0 & 0\\
0 & 0 & 1\\
0 & 1 & 0
\end{pmatrix}$, $E_6=\begin{pmatrix}
0 & 0 & 0\\
0 & 0 & 0\\
0 & 0 & 1
\end{pmatrix}$, e sia $\mathscr{B}=\{E_1, \cdots, E_6\}$

Dato
\begin{multline*}
I=\Biggl\{A_1=\begin{pmatrix}
1 & 2 & 0\\
2 & 0 & 0\\
0 & 0 & 0
\end{pmatrix},\\ A_2=\begin{pmatrix}
1 & 0 & 0\\
0 & -1 & 0\\
0 & 0 & 1
\end{pmatrix},\\ A_3=\begin{pmatrix}
0 & 1 & -1\\
1 & 0 & 0\\
-1 & 0 & 0
\end{pmatrix}\Biggr\}\subseteq\mathcal{S}(\R^{3,3})
\end{multline*}
insieme libero, si trovino tre elementi $w_1, w_2, w_3\in\mathscr{B}$ tali per cui $I\cup\{w_1, w_2, w_3\}$ sia una base di $\mathcal{S}(\R^{3,3})$.

\[
A_1=E_1+2E_2; A_2=E_1-E_4+E_6; A_3=E_2-E_3
\]
e rispetto alla base $\mathscr{B}$
\[\begin{gathered}
A_1=(1, 2, 0, 0, 0, 0), A_2=(1, 0, 0, -1, 0, 1), A_3=(0, 1, -1, 0, 0, 0)\\
E_1=(1, 0, \cdots, 0), E_2=(0, 1, 0, \cdots, 0), \cdots, E_6=(0, \cdots, 0, 1)
\end{gathered}\]

Si studia l'appartenenza di $E_1\in\mathscr{L}(A_1, A_2, A_3)$. Studio il sistema
\[
E_1=\lambda_1 A_1 + \lambda_2 A_2+ \lambda_3A_3
\]
\[
\begin{cases}
1=\lambda_1+\lambda_2\\
0=2\lambda_1+\lambda_3\\
0=-\lambda_3\\
0=-\lambda_2\\
0=0\\
0=\lambda_2
\end{cases}\implies \begin{cases}
\lambda_2=0\\
\lambda_3=0\\
\lambda_1=0\\
\lambda_1=1
\end{cases}\implies\text{Il sistema non ha soluzione}
\]$\implies$ $E_1\notin\mathscr{L}(A_1, A_2, A_3)$ $\implies$ $I_2=\{A_1, A_2, A_3, E_1\}$

Si studia l'appartenenza di $E_2\in\mathscr{L}(A_1, A_2, A_3, E_1)$. Studio il sistema
\[
E_2=\lambda_1 A_1 + \lambda_2 A_2+ \lambda_3A_3+\lambda_4E_1
\]
\[
\begin{cases}
0=\lambda_1+\lambda_2+\lambda_4\\
1=2\lambda_1+\lambda_3\\
0=-\lambda_3\\
0=-\lambda_2\\
0=0\\
0=\lambda_2
\end{cases}\implies \begin{cases}
\lambda_4=-\frac{1}{2}\\
\lambda_3=0\\
\lambda_2=0\\
\lambda_1=\frac{1}{2}
\end{cases}\implies\text{Il sistema ha soluzione}
\]$\implies$ $E_2\in\mathscr{L}(A_1, A_2, A_3, E_1)$ $\implies$ scarto $E_2$

Si studia l'appartenenza di $E_3\in\mathscr{L}(A_1, A_2, A_3, E_1)$. Studio il sistema
\[
E_3=\lambda_1 A_1 + \lambda_2 A_2+ \lambda_3A_3+\lambda_4E_1
\]
\[
\begin{cases}
0=\lambda_1+\lambda_2+\lambda_4\\
0=2\lambda_1+\lambda_3\\
1=-\lambda_3\\
0=-\lambda_2\\
0=0\\
0=\lambda_2
\end{cases}\implies \begin{cases}
\lambda_4=-\frac{1}{2}\\
\lambda_3=-1\\
\lambda_2=0\\
\lambda_1=\frac{1}{2}
\end{cases}\implies\text{Il sistema ha soluzione}
\]$\implies$ $E_3\in\mathscr{L}(A_1, A_2, A_3, E_1)$ $\implies$ scarto $E_3$

Si studia l'appartenenza di $E_4\in\mathscr{L}(A_1, A_2, A_3, E_1)$. Studio il sistema
\[
E_4=\lambda_1 A_1 + \lambda_2 A_2+ \lambda_3A_3+\lambda_4 E_1
\]
\[
\begin{cases}
0=\lambda_1+\lambda_2+\lambda_4\\
0=2\lambda_1+\lambda_3\\
0=-\lambda_3\\
1=-\lambda_2\\
0=0\\
0=\lambda_2
\end{cases}\implies \begin{cases}
\lambda_2=0\\
\lambda_2=-1\\
\cdots
\end{cases}\implies\text{Il sistema non ha soluzione}
\]$\implies$ $E_4\notin\mathscr{L}(A_1, A_2, A_3, E_1)$ $\implies$ $I_2=\{A_1, A_2, A_3, E_1, E_4\}$

Si studia l'appartenenza di $E_5\in\mathscr{L}(A_1, A_2, A_3, E_1, E_4)$. Studio il sistema
\[
E_5=\lambda_1 A_1 + \lambda_2 A_2+ \lambda_3A_3+\lambda_4 E_1 +\lambda_5 E_4
\]
\[
\begin{cases}
0=\lambda_1+\lambda_2+\lambda_4\\
0=2\lambda_1+\lambda_3\\
0=-\lambda_3\\
0=-\lambda_2+\lambda_5\\
1=0\\
0=\lambda_2
\end{cases}\implies \begin{cases}
1=0\\
\cdots
\end{cases}\implies\text{Il sistema non ha soluzione}
\]$\implies$ $E_5\in\mathscr{L}(A_1, A_2, A_3, E_1, E_4)$ $\implies$ $I_3=\{A_1, A_2, A_3, E_1, E_4, E_5\}$

La soluzione è $\mathscr{B}'=\{A_1, A_2, A_3, E_1, E_4, E_5\}$
}

\section{Operazioni tra sottospazi vettoriali}

Sia $V$ uno spazio vettoriale su un un campo $\K$, e siano $W_1$ e $W_2\subseteq V$ due sottospazi vettoriali.

Si consideri
\[
W_1 \cap W_2 = \{x\:|\: x\in w_1 \land x \in w_2\}
\]

\proposizione{inters}{$W_1 \cap W_2$ è sempre sottospazio vettoriale}
\dimostrazioneprop{inters}{Siano $x,y\in W_1 \cap W_2$
\[\implies\left.\begin{cases}
x,y \in W_1 \implies (x+y)\in W_1 \\
x,y \in W_2 \implies (x+y)\in W_2\end{cases}\right\rbrace\implies (x+y)\in W_1\cap W_2\]}

% TODO FINIRE

% 7 ott 2021

\proposizione{somvet}{Sia $V$ uno spazio vettoriale e $W, W_1$ e $W_2$ sottospazi di $V$.

	Se $W$ contiene $W_1$ e $W$ contiene $W_2$ allora $W$ contiene $W_1+W_2$ (cioè $W_1+W_2$ è il più piccolo sottospazio di $V$ che contiene sia $W_1$ che $W_2$)}

\dimostrazioneprop{somvet}{Sia $x+y \in W_1+W_2$, $x \in W_1$ $\implies$ $x \in W, y \in W_2$ $\implies$ $y \in W$ $\implies$ $x+y \in W$, poiché $W$ è un sottospazio vettoriale. Quindi ogni $v \in W_1+W_2$ è elemento di $W$ $\implies$ $W_1+W_2 \subseteq W$.

La somma si generalizza a più sottospazi. Siano $W_1, \cdots, W_l \subseteq V$ sottospazi vettoriali, allora si definisce \[W_1+\cdots+W_l=\{x_1+\cdots+x_l | x_1 \in W_1, \cdots, x_l \in W_l\} \subseteq V\] è un sottospazio vettoriale ed è il più piccolo sottospazio che contiene tutti i $W_1,\cdots, W_l$}

\esercizio{
	Si trovino somma e intersezione dei seguenti sottospazi vettoriali di $\R^4$
	\begin{itemize}
		\item [a.] $W_1=\{(x_1, x_2, 0, 0) | x_1,x_2 \in \R\}$, $W_2=L(e_4)$
		\item [b.] $W_1=\{(x_1, x_2, 0, 0) | x:1,x:2 \in \R\}$,\\ $Z_2=\{(0, x_2, 0, x_4) | x_2, x_4 \in \R\}$
	\end{itemize}}{
	\begin{itemize}
		\item [a.] $W_1+W_2=\{(x_1, x_2, 0, x_4) | x_1, x_2, x_4 \in \R\}$, $W_1 \cap W_2=\{\underline{0}\}$
		\item [b.] $W_1+Z_2=\{(x_1, x_2, 0, x_4) | x_1, x_2, x_4 \in \R\}$,\\$W_1 \cap Z_2=\{(0, x_2, 0, 0) | x_2 \in \R\}$
	\end{itemize}}

\proposizione{caa}{Sia $V$ spazio vettoriale su un campo $\K$ e $W_1, W_2 \subseteq V$ due sottospazi. Sono fatti equivalenti le seguenti proposizioni:
	\numerato{\item $W_1 \cap W_2 = \{\underline{0}\}$ (hanno intersezione banale)
	\item ogni $v \in W_1+W_2$ si scrive in modo unico come $v=x+y$ con $x \in W_1$ e $y \in W_2$}}

\dimostrazioneprop{caa}{\elencop{\item [1. $\implies$ 2.] Suppongo $W_1 \cap W_2 = \{\underline{0}\}$ e considero $v \in W_1+W_2$. Scrivo $v=x_1+y_1$, $v=x_2+y_2$ e dimostro che $x_1=x_2$ e $y_1=y_2$

		$\underline{0}=v-v=(x_1+y_1)-(x_2+y_2)=(x_1-x_2)+(y_1-y_2)$ $\implies$ $x_1-x_2=y_2-y_1$, $x_1-x_2 \in W_1$ mentre $y_2-y_1 \in W_2$. Per l'uguaglianza risulta che 
		\[
		\begin{cases}
		x_1-x_2 \in W_2 \implies x_1-x_2 \in W_1 \cap W_2\\
		y_2-y_1 \in W_1 \implies y_2-y_1 \in W_1 \cap W_2
		\end{cases}\]
		\[
		\implies \begin{cases}
		x_1-x_2=\underline{0} \implies x_1=x_2\\
		y_2-y_1=\underline{0} \implies y_1=y_2
		\end{cases}\]
	
	\item [2. $\implies$ 1.] Suppongo che ogni $v \in W_1+W_2$ si scriva in modo unico come $v=x+y$ con $x \in W_1$ e $y \in W_2$ e dimostro che $W_1 \cap W_2 = \{\underline{0}\}$

		Sia $v \in W_1 \cap W_2$. Sia $v \in W_1+W_2$, $v=x+y=x+v+y-v$, con $x+v \in W_1$, $y-v \in W_2$. Quindi se $v \neq \underline{0}$, le due scritture $v=x+y$, $v=(x+v)+(y-v)$ sono diverse e ciò non è possibile per ipotesi}}

		
\notazione{
	Se $ W_1  \cap W_2 = \{\underline{0}\} $ si scrive $ W_1 \oplus W_2 $ invece che $ W_1+W_2 $
	$ \oplus $ si legge ``somma diretta''}
	
\esempio{ $\K^{n,n} = S(\K^{n,n}) \oplus A(\K^{n,n})$}
\esempio{$ R^2 = \mathscr{L}(e_1) \oplus \mathscr{L}(e_2) $}

\proposizione{vetteq}{
	Sia $ V $ uno spazio vettoriale su un campo $\K$. Siano $ W_1, \cdots, W_l \subseteq V $ sottospazi vettoriali. Sono fatti equivalenti le seguenti proposizioni
	\begin{enumerate}
	\item $ W_i \cap (W_1+\cdots+W_\{i-1\}+W_\{i+1\}+\cdots+W_l) =\{\underline{0}\} $ $ \forall i = 1,\cdots,l $
	\item Ogni $ v \in W_1+\cdots+W_l $ si scrive in modo unico come $ v=x_1+\cdots+x_l$ con $x_1 \in W_1, \cdots, x_l \in W_l$
	\end{enumerate}
	Se vale 1. si scrive $ W_1 \oplus W_2 \oplus \cdots \oplus W_l $}
	
%TODO Dimostrazione per esercizio

\esempio{
	Considero $ V $ spazio vettoriale di dimensione finita e $ \mathscr{B}=\{v1, \cdots, v_n\} $ $\implies$ $ V=\mathscr{L}(v_1) \oplus \cdots \oplus \mathscr{L}(v_l) $} %TODO controllare che l'esercizio sia finito

Sia $V$ spazio vettoriale su un campo $\K$, finitamente generato. Sia $ W \subseteq V $ un sottospazio vettoriale, sia $ \mathscr{B}=\{w_1, \cdots, w_l\} $ una base di $ W $. Possiamo completare $ \mathscr{B} $ con una base dello spazio $ \mathscr{B}'=\{w_1, \cdots, w_l, v_1, \cdots, v_m\} $. Sia \[ Z=\mathscr{L}(v_1,\cdots, v_m) \subseteq V\] un sottospazio vettoriale, e per costruzione $ V=W \oplus Z $

\osservazione{
	Sia $ V $ spazio vettoriale di dimensione finita con $ V=W \oplus Z $
	Siano $ \mathscr{B}=\{w1, \cdots, w_l\} $ una base di $ W $ e $ C=\{z_1, \cdots, z_m\} $ una base di $ Z $.
	Ogni elemento di $ V $ si scrive in modo unico come $ v=x+y $ con $ x \in W $ e $ y \in Z $
	$ \mathscr{B} $ base di $ W $ 
	
	$\implies$ $ x $ si scrive in modo unico come $ x=\lambda_1w_1+\cdots+\lambda_l w_l $
	
	$ \mathscr{C} $ base di $ Z $ $\implies$ $y$ si scrive in modo unico come \[ y=\mu_1z_1+\cdots+\mu_n z_n \]
	
	$\implies$ $ v $ si scrive in modo unico come \[v=\lambda_1w_1+\cdots+\lambda_l w_l+\mu_1z_1+\cdots+\mu_n z_n\]
	
	$\implies$ $ B \cup C=\{w1, \cdots, w_l, z_1, \cdots, z_l\} $ è una base di $ V  $
	
	$\implies$ $ \dim V = \dim W + \dim Z $}
	
\teorema{grassmann}{
	Sia $ V $ uno spazio vettoriale su un campo $\K$ finitamente generato. Siano $ W_1, W_2 \subseteq V $ due sottospazi vettoriali $ \tc V=W_1+W_2 $. Allora 
	\[\dim V = \dim W_1 + \dim W_2 - \dim (W_1 \cap W_2)\]
	Questa è la \textbf{Formula di Grassmann}.}
	
\dimostrazione{grassmann}{
	Chiamo $ \dim V=n $, $ \dim W_1=l $, $ \dim W_2=p $, $ \dim (W_1 \cap W_2) = r $
	
	In particoalre $ l,p \leq n $, $ r\leq l,p $
	\begin{enumerate}
	\item $ r=l $ $\implies$ $ W_1 \cap W_2 = W_1 $ $\implies$ $ W_1 \subseteq W_2 $ $\implies$ $ W_1+W_2=W_2=V $ %TODO mostrare tutte implicazioni
	\item $ r=p $ $\implies$ $ W_1 \cap W_2 = W_2 $ $\implies$ $ W_2 \subseteq W_1 $ $\implies$ $ W_1+W_1=W_1=V $ %TODO mostrare tutte le implicazioni
	\item si assume $ r\leq l,p $ e sia $ \mathscr{B}=\{a_1,\cdots,a_r\ $} una base di $ W_1 \cap W_2 $
	
		Completo $ \mathscr{L} $ con una base $ \mathscr{C} $ di $ W_1 $, $ \mathscr{C}=\{a_1, \cdots, a_r, b_{r+1}, \cdots, b_l\} $, e completo $ \mathscr{B} $ con una base $ \mathscr{D} $ di $ W_2 $, $ \mathscr{D}=\{a_1, \cdots, a_r, c_{r+1}, \cdots, c_p\} $
%		Si verifica che l'insieme \{a_1, \cdots, a_r, b_\{r+1\}, \cdots, b_l, c_\{r+1\}, \cdots, c_p\} è una base di V. In questo modo si ottiene dim V= l + (p-r), cioè la tesi. Ovviamente risulta L(a_1, \cdots, a_r, b_\{r+1\}, \cdots, b_l, c_\{r+1\}, \cdots, c_p)=V, in quanto contiene i generatori sia di W_1 che di W_2, e quindi anche della loro somma. Verifichiamo che l'insieme \{a_1, \cdots, a_r, b_\{r+1\}, \cdots, b_l, c_\{r+1\}, \cdots, c_p\} è libero.
%		Supponiamo 
%		lambda1a1+ \cdots. + lambda_r a_r + mu_\{r+1\} b_\{r+1\} + \cdots + mu_l b_l + gamma_\{r+1\} c_\{r+1\} + \cdots + gamma_p c_p = \underline{0} **
%		lambda1a1+ \cdots. + lambda_r a_r + mu_\{r+1\} b_\{r+1\} + \cdots + mu_l b_l = -gamma_\{r+1\} c_\{r+1\} - \cdots - gamma_p c_p
%		Sia c=-gamma_\{r+1\} c_\{r+1\} - \cdots - gamma_p c_p=lambda1a1+ \cdots. + lambda_r a_r + mu_\{r+1\} b_\{r+1\} + \cdots + mu_l b_l, sicuramente c in w2
%		lambda1a1+ \cdots. + lambda_r a_r + mu_\{r+1\} b_\{r+1\} + \cdots + mu_l b_l in W_1 
%		$\implies$ c in W_1 \cap W_2 = L(a1,\cdots, a_r) $\implies$ c = beta1 a1 +\cdots+beta_r a_r, vado a sostituire in **
%		(beta1 a1 +\cdots+beta_r a_r)+(gamma_\{r+1\} c_\{r+1\} + \cdots + gamma_p c_p) = \underline{0} $\implies$ 
%			beta1=\cdots=beta_r=0
%			gamma_\{r+1\}=\cdots=gamma_p=0 %TODO chiarire meglio perché
%		Ho ottenuto gamma_\{r+1\}=\cdots=gamma_p=0 e
%			lambda1a1+ \cdots. + lambda_r a_r + mu_\{r+1\} b_\{r+1\} + \cdots + mu_l b_l=\underline{0}, poiché l'insieme C=\{a1, \cdots, a_r, b_\{r+1\}, \cdots, b_l\} è libero $\implies$ lambda 1=\cdots=lambda r = mu_\{r+1\}=\cdots=mu_l=0
%			$\implies$ \{a_1, \cdots, a_r, b_\{r+1\}, \cdots, b_l, c_\{r+1\}, \cdots, c_p\} è libero
		\end{enumerate}}

%TODO sistemare appunti con quelli del prof
%TODO sistemare sistema di numerazione teoremi e proposizioni

\end{document}