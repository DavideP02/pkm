% 7 ott 2021

Proposizione
	Sia V uno spazio vettoriale e W, W_1 e W_2 sottospazi di V.
	Se W contiene W_1 e W contiene W_2 allora W contiene W_1+W_2 (cioè W_1+W_2 è il più piccolo sottospazio di V che contiene sia W_1 che W_2)

Dimostrazione
	Sia x+y in W_1+W_2, x in W_1 implies x in W, y in W_2 implies y in W implies x+y in W, poiché W è un sottospazio vettoriale. Quindi ogni v in W1+W2 è elemento di W implies W1+W2 subseteq W

La somma si generalizza a più sottospazi. Siano W1, ... W_l subseteq V sottospazi vettoriali, allora si definisce W1+...+W_l={x1+...+x_l | x1 in W1, ..., x_l in W_l} subseteq in V è un sottospazio vettoriale ed è il più piccolo sottospazio che contiene tutti i W_1,...,W_l.

Esercizio
	Si trovino somma e intersezione dei seguenti sottospazi vettoriali di \R^4
	
		a. W1={(x1, x2, 0, 0) | x1,x2 in \R}, W2=L(e4)
		b. W1={(x1, x2, 0, 0) | x1,x2 in \R}, Z2={(0, x2, 0, x4) | x2, x4 in R}
	
Soluzione
	a. W1+W2={(x1, x2, 0, x4) | x1, x2, x4 in R}, W1 cap W2={00}
	b. W1+Z2={(x1, x2, 0, x4) | x1, x2, x4 in R)} W1 cap Z2={(0, x2, 0, 0) | x2 in R}

Proposizione
	V spazio vettoriale su un campo K e W1 e W2 subseteq V due sottospazi. Sono fatti equivalenti le seguenti proposizioni
	1. W1 cap W2 = {00} (hanno intersezione banale)
	2. ogni v in W1+W2 si scrive in modo unico come v=x+y con x in W1 e y in W2

Dimostrazione

	1 implies 2 - Suppongo W1 cap W2 = {00} e considero v in W1+W2. Scrivo v=x1+y1, v=x2+y2 e dimostro che x1=x2 e y1=y2
		00=v-v=(x1+y1)-(x2+y2)=(x1-x2)+(y1-y2) implies x1-x2=y2-y1, x1-x2 in W1 mentre y2-y1 in W2 implies per l'uguaglianza risulta che 
		x1-x2 in W2 implies x1-x2 in W1 cap W2 implies x1-x2=00 implies x1=x2\\ 
		y2-y1 in W1 implies y2-y1 in W1 cap W2 implies y2-y1=00 implies y1=y2
	
	2 implies 1 - Suppongo che ogni v in W1+W2 si scriva in modo unico come v=x+y con x in W1 e y in W2 e dimostro che W1 cap W2 = {00}
		Sia v in W1 cap W2. Sia v in W1+W2, v=x+y=x+v+y-v, con x+v in W1, y-v in W2. Quindi se v neq 00, le due scritture v=x+y, v=(x+v)+(y-v) sono diverse e ciò non è possibile per ipotesi
		
Notazione
	Se W1 cap W2 = {00} si scrive W1 sommadiretta (più cerchiato) W2 invece che W1+W2
	più_cerchiato si legge ``somma diretta''
	
Esempio K^n,n = S(k^n,n) sommadiretta A(K^n,n)
Esempio R^2 = L(e1) sommadiretta L(e2)

Proposizione
	Sia V uno spazio vettoriale su un campo K. Siano W1, ..., W_l subseteq V sottospazi vettoriali. Sono fatti equivalenti le seguenti proposizioni
	1. W_i cap (W1+...+W_{i-1}+W_{i+1}+...+W_l) ={00} forall i = 1,...,l
	2. Ogni v in W1+...+W_l si scrive in modo unico come v=x_1+...+x_l con x_1 in W1, ..., x_l in W_l
	Se vale 1. si scrive W1 sommadiretta W2 sommadiretta ... sommadiretta W_l
	
Dimostrazione per esercizio %TODO

Esempio
	Considero V spazio vettoriale di diimensione finita e B={v1, ..., v_n} implies V=L(v1) sommadiretta ... sommadiretta L(v_l) %TODO controllare che l'esercizio sia finito
	
Sia V spazioo vettoriale su un campo K, finitamente generato. Sia W subseteq V un sottospazio vettoriale, sia B={w_1, ..., w_l} una base di W. Possiamo completare B con una base dello spazio B'={w_1, ..., w_l, v_1, ..., v_m}. Sia Z=L(v_1,..., v_m) subseteq V è un sottospazio vettoriale, e per costruzione V=W sommadiretta Z

Osservazione
	Sia V spazio vettoriale di dimensione finita con V=W sommadiretta Z
	Siano B={w1, ..., w_l} una base di W e C={z_1, ..., z_m} una base di Z.
	Ogni elemento di V si scrive in modo unico come v=x+y con x in W e y in Z
	B base di W implies x si scrive in modo unico come x=lambda1w1+...+lambda_l w_l
	C base di Z implies y si scrive in modo unico come y=mu1z1+...+mu_n z_n
	implies v si scrive in modo unico come v=lambda1w1+...+lambda_l w_l+mu1z1+...+mu_n z_n
	implies B cup C={w1, ..., w_l, z_1, ..., z_l} è una base di V implies dim V = dim W + dim Z
	
Teorema
	Sia V uno spazio vettoriale su un campo K finitamente generato. Siano W1, W2 subseteq V due sottospazi vettoriali t.c. V=W1+W2. Allora 
	dim V = dim W1 + dim W2 - dim (W1 cap W2)
	Questa è la Formula di Grassmann.
	
Dimostrazione
	Chiamo dim V=n, dim W1=l, dim W2=p, dim (W1 cap W2) = r
	In particoalre l,p <= n, r<= l,p
	1. r=l implies W1 cap W2 = W1 implies W1 subseteq W2 implies W1+W2=W2=V %TODO mostrare tutte implicazioni
	2. r=p implies W1 cap W2 = W2 implies W2 subseteq W1 implies W1+W1=W1=V %TODO mostrare tutte le implicazioni
	3. si assume r<= l,p e sia B={a1,...,a_r} una base di W1 cap W2
		Completo B con una base C di W1, C={a_1, ..., a_r, b_{r+1}, ..., b_l}, e completo B con una base D di W2, D={a_1, ..., a_r, c_{r+1}, ..., c_p}
		Si verifica che l'insieme {a_1, ..., a_r, b_{r+1}, ..., b_l, c_{r+1}, ..., c_p} è una base di V. In questo modo si ottiene dim V= l + (p-r), cioè la tesi. Ovviamente risulta L(a_1, ..., a_r, b_{r+1}, ..., b_l, c_{r+1}, ..., c_p)=V, in quanto contiene i generatori sia di W1 che di W2, e quindi anche della loro somma. Verifichiamo che l'insieme {a_1, ..., a_r, b_{r+1}, ..., b_l, c_{r+1}, ..., c_p} è libero.
		Supponiamo 
		lambda1a1+ .... + lambda_r a_r + mu_{r+1} b_{r+1} + ... + mu_l b_l + gamma_{r+1} c_{r+1} + ... + gamma_p c_p = 00 **
		lambda1a1+ .... + lambda_r a_r + mu_{r+1} b_{r+1} + ... + mu_l b_l = -gamma_{r+1} c_{r+1} - ... - gamma_p c_p
		Sia c=-gamma_{r+1} c_{r+1} - ... - gamma_p c_p=lambda1a1+ .... + lambda_r a_r + mu_{r+1} b_{r+1} + ... + mu_l b_l, sicuramente c in w2
		lambda1a1+ .... + lambda_r a_r + mu_{r+1} b_{r+1} + ... + mu_l b_l in W1 
		implies c in W1 cap W2 = L(a1,..., a_r) implies c = beta1 a1 +...+beta_r a_r, vado a sostituire in **
		(beta1 a1 +...+beta_r a_r)+(gamma_{r+1} c_{r+1} + ... + gamma_p c_p) = 00 implies 
			beta1=...=beta_r=0
			gamma_{r+1}=...=gamma_p=0 %TODO chiarire meglio perché
		Ho ottenuto gamma_{r+1}=...=gamma_p=0 e
			lambda1a1+ .... + lambda_r a_r + mu_{r+1} b_{r+1} + ... + mu_l b_l=00, poiché l'insieme C={a1, ..., a_r, b_{r+1}, ..., b_l} è libero implies lambda 1=...=lambda r = mu_{r+1}=...=mu_l=0
			implies {a_1, ..., a_r, b_{r+1}, ..., b_l, c_{r+1}, ..., c_p} è libero