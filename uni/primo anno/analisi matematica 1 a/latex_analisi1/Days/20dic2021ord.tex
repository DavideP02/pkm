\subsubsection{Algebra degli ``o'' piccoli}

Siamo\marginnote{20 dic 2021} nel caso $ o(x^{n})_{x\to 0}  $.

Dati $ n, m \in \N $: \begin{enumerate}
    \item $ n<m $: $ x^{m}=o(x^{n}) $;
    \item $ n\le m $: $ o(x^{n} ) + o(x^{m}) = o(x^{n}) $;
    \item dato $ \lambda \in \R $, $ \lambda o (x^{n})=o(x^{n}) $, e $ o( \lambda x^{n}) = o(x^{n}) $;
    \item $ x^{m}\,o(x^{n}) = o(x^{m+n}) $ e $ o(x^{m})\,o(x^{n})= o(x^{m+n}) $;
    \item $ \left(o(x^{n})\right)^{m}=o(x^{mn}) $;
    \item se $ f $ limitata in $ U(0) $, allora $ f(x)o(x^{n})=o(x^{n}) $;
    \item 
\end{enumerate}

Si noti che \[
    f=o(x^{n})\,\iff\, \lim_{x\to 0} \parentesi{\omega(x)}{\frac{f(x)}{x^{n}}} =0\,\iff\, f(x)= x^{n}\omega(x)
\] con $ \omega(x) \xrightarrow[x\to 0]{} 0 \displaystyle $

\begin{proof}
    Di seguito la dimostrazione delle proprietà:
    \begin{enumerate}
        \item \todo{Manca la dimostrazione}
        \item $ f(x)=o(x^{n}) $, $ g(x)=o(x^{m}) $, $ m\ge n $ \[
            f(x)=x^{n} \cdot \omega_{1}(x)\qquad g(x)=x^{m} \cdot \omega_2(x)  
        \] quindi \[
            f(x)+g(x)=x^{n} \omega_1(x)+ x^{m}\omega_2(x)=x^{n}\bigl(\parentesi{\omega(x)}{\omega_1(x)+x^{n-m}\omega_2(x)}\bigr)
        \] poiché $ m\ge n $, \[
            \omega(x)=\omega_1(x)+x^{n-m}\omega_2(x) \xrightarrow[x\to 0]{} 0 \displaystyle
        \] da cui \[
            f(x)+g(x)=x^{n}\omega(x) \,\iff\, o(x^{n}) + o(x^{m})=o(x^{n}).
        \]
        \item Dimostrare per esercizio %ESERCIZIO lasciata per esercizio
        \item \todo{Manca la dimostrazione}
        \item Dimostrare per esercizio %ESERCIZIO lasciata per esercizio
        \item Dimostrare per esercizio %ESERCIZIO lasciata per esercizio
        \qedhere
    \end{enumerate}
\end{proof}

\subsubsection{Calcolo di ordine di infinitesimo e parte principale}

Ricordiamo che $ f $ è infinitesima di ordine $ \alpha>0$ per $ x\to x_0 $ se \[
    \lim_{x\to x_0} \frac{f(x)}{(x-x_0)^{\alpha}} = \lambda \in \R 
\]
Possiamo scrivere che \[
    \lim_{x\to x_0} \frac{f(x)-\lambda (x-x_0)^\alpha}{(x- x_0 )^\alpha} =0
\] ovvero \[
    f(x)-\lambda (x-x_0)^\alpha = o((x- x_0 )^\alpha)_{x\to x_0}
\] ovvero \[
    f(x)= \parentesi{\footnotemark}{\lambda (x-x_0)^{\alpha}} + o \left((x-x_0)^{\alpha}\right)_{x\to x_0}
\]\footnotetext{parte principale di $ f $ (indicata con $ ppf $)}

Allora, date $ f $ infinitesima di ordine $ \alpha $ in $ x_0 $, 
\[
    f(x)= \lambda(x-x_0)^{\alpha} o \left((x-x_0)^{\alpha}\right)_{x\to x_0}
\]
$ g $ infinitesima di ordine $ \beta $ in $ x_0 $, ovvero \[
    g(x)= \mu(x-x_0)^{\beta}+ o\left((x-x_0)^{\beta}\right)_{x\to x_0}
\]
\begin{multline}
    \lim_{x\to x_0} \frac{f(x)}{g(x)} =\\
    = \lim_{x\to x_0} \frac{\lambda(x-x_0)^{\alpha} o \left((x-x_0)^{\alpha}\right)_{x\to x_0}}{\mu(x-x_0)^{\beta}+ o\left((x-x_0)^{\beta}\right)_{x\to x_0}}=\\
    = \lim_{x\to x_0} \frac{\lambda(x-x_0)^{\alpha}\left(1+\frac{o\left((x-x_0)^{\alpha}\right)}{\lambda(x-x_0)^{\alpha}}\right)}{\mu(x-x_0)^{\beta}\left(1+\frac{o\left((x-x_0)^{\beta}\right)}{\mu(x-x_0)^{\beta}}\right)}\label{questaquielelle}
\end{multline}
Per definizione si ha che \[
    \frac{o\left((x-x_0)^{\alpha}\right)}{\lambda(x-x_0)^{\alpha}} \xrightarrow[x\to x_0]{}  0 \qquad \frac{o\left((x-x_0)^{\beta}\right)}{\mu(x-x_0)^{\beta}} \xrightarrow[x\to x_0]{}  0
\] quindi \eqref{questaquielelle} equivale a
\begin{equation}
    \lim_{x\to x_0} \frac{\lambda(x-x_0)^{\alpha}}{\mu(x-x_0)^{\beta}}=\frac{\lambda}{\mu} \lim_{x\to x_0} \frac{(x-x_0)^{\alpha}}{(x-x_0)^{\beta}}
\end{equation}

L'obiettivo ora è la ricerca di ordine di infinitesimo e parte principale di $ f $.

\esempio{
    Determinare l'ordine di infinitesimo e parte principale per $ x\to 0 $ di \[
        f(x)=\sin x -x
    \]
    Calcolare \[
        \lim_{x\to 0} \frac{f(x)}{x^k}
    \] con $ k\ge 0 $

    Dagli sviluppi di MacLaurin del seno, \[
        \sin x = x- \frac{x^{3}}{3!}+o(x^{4})
    \] da cui \[
        f(x)=\sin x - x = \cancel{x}- \frac{x^{3}}{3!}+o(x^{4}) -\cancel{x}
    \] per cui \[
        f(x)=\frac{x^{3}}{3!}+o(x^{4})
    \] e la parte principale di $ f $ è $ \frac{x^{3}}{3!} $.
    L'ordine di infinitesimo di $ f $ per $ x\to 0 $ è $ \alpha=3 $.

    \[
        \lim_{x\to x_0} \frac{f(x)}{x^k} = -\frac{1}{6} \lim_{x\to 0^{+}} \frac{x^{3}}{x^{k}} = \begin{cases}
            + \infty &k>3\\
            -\frac{1}{6} &k=3\\
            0 &0\le k<3 
        \end{cases}
    \]
}

\esempio{
    Sia $ f(x)=(\sin x)^{2}-x^{2} $, calcolarne ordine di infinitesimo, parte principale e \[
        \lim_{x\to 0} \frac{f^{x}}{x^{k}}
    \]
    Dagli sviluppi di MacLaurin del seno, \[
        \sin x = x- \frac{x^{3}}{3!}+o(x^{4})
    \]
    \begin{multline*}
        (\sin x)^{2}=\left(x- \frac{x^{3}}{3!}+o(x^{4})\right)^{2}= \\
        \cdots\\
        =x^{2}+\frac{x^{3}}{36}-\frac{x^{4}}{3} +o(x^{8})+o(x^{5})+o(x^{7})=\\
        = x^{2}-\frac{x^{4}}{3}+ \parentesi{o(x^{5})}{o(x^{5})+o(x^{5})+o(x^{7})+o(x^{8})}=x^{2}-\frac{x^{4}}{3}+o(x^{5})
    \end{multline*}
    \todo{Svolgere i calcoli al posto di quei $\cdots$}
    Da cui \[
        f(x)=(\sin x)^{2}-x^{2}=\cancel{x^{2}}-\frac{x^{4}}{3}+o(x^{5})-\cancel{x^{2}}
    \]
    \[
        f(x)=-\frac{1}{3}x^{4}+o(x^{5})
    \] il cui ordine di infinitesimo è $ \alpha= 4 $, e la parte principale è $ -x^{4}/3 $
    %ESERCIZIO finire il limite
}

\osservazione{
    Dato $ n\le m $
    \[
        \frac{o(x^{m})}{x^{n}}=\frac{1}{x^{n}}x^{m}\,\omega(x)
        = x^{m-n}\,\omega(x)=o(x^{m-n})
    \]
}

\esercizio{
    Calcolare \begin{gather*}
        \lim_{x\to 0} \frac{\ln(\cos^{2}x)}{\sin^{2}x}\\
        \lim_{x\to 0} \frac{\ln(\cos^{2}x+x^{2})}{x^{\alpha}}
    \end{gather*}
}{
    Lasciato per esercizio%ESERCIZIO 
}{}