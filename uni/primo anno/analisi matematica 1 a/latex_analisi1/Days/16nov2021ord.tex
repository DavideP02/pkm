\lemma{dddddddsdfsdfs}{
    Data\marginnote{16 nov 2021} $ \{a_{k} \}_{k=0}^\infty $ a valori in $ \R^{n} $, 
    
    $ \{a_{k} \}_{k=0}^\infty $ è di Cauchy $ \implies $ $ \{a_{k} \}_{k=0}^\infty $ è limitata
}
\dimostrazionelem{dddddddsdfsdfs}{
    Consideriamo $ \varepsilon=1 $: 
    \[
        \exists \,\varkappa>0:\, \forall\, k>\varkappa
    \] si ha $ |a_{k} - a_{\varkappa}|<1 $, $ \forall\, k\ge \varkappa $
    \[
        |a_{k}-a_{\varkappa}|\ge ||a_{k}|-|a_{\varkappa}||  
    \]
    Allora per $ k>\varkappa $ \[
        ||a_{k}|-|a_{\varkappa}||<1
    \]
    \[
        |a_{\varkappa}|-1<|a_{k}|<|a_{\varkappa}| +1  
    \]

    Consideriamo\begin{align*}
        m&=\min\{|a_0|,|a_1|,\cdots,|a_{\varkappa}|, |a_{\varkappa}|-1  \}\\
        M&=\max\{|a_0|, |a_1|, \cdots, |a_{\varkappa}|, |a_{\varkappa}|+1  \}
    \end{align*}

    Dunque $ \forall\, k \in \N $, \[
        m < |a_{k}| < M 
    \] 
    
    $\implies$ $ \{a_{k} \}_{k=0}^\infty $ è limitata \qed 
}

\teorema[(Criterio di convergenza di Cauchy per le successioni)]{dfgdfgdfgdfgdfg}{
    Data $ \{a_{k} \}_{k=0}^\infty $ a valori in $ \R^{n} $, si ha

    $ a_{k} $ convergente $ \iff $ $ a_{k}  $ è di Cauchy
}
\dimostrazione{dfgdfgdfgdfgdfg}{
    \begin{itemize}
        \item [``$\implies$''] $ \{a_{k} \} $ è convergente, allora \[
            \exists\, l \in \R^{n}
        \] tale che \[
            \forall\, \varepsilon>0\quad \exists\, \overline{k}:\, \forall\,k>\overline{k}:\, |a_{k}-l|< \varepsilon 
        \]
        possiamo scrivere
        \[
            |a_{k}-a_{m}|\le |a_{k}-l|+|a_{m}-l|    
        \]
        \[
            \exists\,\overline{k}\quad\forall\, k\quad m\ge \overline{k}:
        \]
        \begin{align*}
            |a_{k}-l|&< \varepsilon/2\\ 
            |a_{m}-l|&< \varepsilon/2
        \end{align*}
        ossia
        \[
            |a_{k}-a_{m}|\le |a_{k}-l|+|a_{m}-l|<\varepsilon/2+\varepsilon/2= \varepsilon
        \]

        $\implies$ $ \{a_{n}\} $ è di Cauchy
        \item [``$\impliedby$''] $ \{a_{k} \}_{k=0}^\infty $ è di Cauchy
        
        $ \underset{Lemma}{\underbrace{\implies}} $ $ \{a_{k} \}_{k=0}^\infty $ è limitata

        $\underset{B-W}{\underbrace{\implies}}$ ammette una sotosuccessione convergente, ossia esiste $ h_{k} \in \N$, $ \{a_{h_{k} } \} $ è convergente, ossia $ \exists \,l \in \R^{n} $: \[
            \forall\, \varepsilon \quad \exists\,\overline{k}:\, \forall\, k>\overline{k}:\, |a_{h_{k} }-l|< \varepsilon/2 
        \]

        Osserviamo \[
            |a_{k}-l|\le |a_{k}-a_{h_{k} } |+|a_{h_{k} }-l|
        \]

        Poiché la successione è di Cauchy \[
            \exists\,\overline{\overline{k}}:\,\forall\, m, h > \overline{\overline{k}}:\, |a_{m}-a_{h}|< \varepsilon  
        \] \[
            \exists\,\overline{\overline{\overline{k}}}:\,\forall\,k\ge\overline{\overline{\overline{k}}}:\, h_{k}> \overline{\overline{k}}
        \]

        Allora preso \[
            \varkappa =\max\{\overline{k}, \overline{\overline{k}}, \overline{\overline{\overline{k}}}\}
        \]

        Otteniamo $ \forall\,k\ge\varkappa $ \[
            |a_{k}-l|\le \overset{< \varepsilon/2}{\overbrace{|a_{k}-a_{h_{k} }|}}+\overset{< \varepsilon/2}{\overbrace{|a_{h_{k} } -l|}}< \varepsilon\qedd
        \]
    \end{itemize}
}

\osservazione{
    Nella dimostrazione si è usato il Teorema di Bolzano-Weirestrass, ossia la completezza di $ \R $, dunque il criterio di convergenza di Cauchy non vale per successioni a valori in $ \Q $ o in $ \Q^{n} $.
}