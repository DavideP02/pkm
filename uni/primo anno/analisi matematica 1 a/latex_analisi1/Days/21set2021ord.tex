\subsection{Insieme $ \R $}\marginnote{21 set 2021}

\proposizione{sdfsdfsdfdd}{
    Sia $ d $ la diagonale del quadrato di lato $ 1 $, ovvero $ d^{2}=2 $. $ d \notin \Q$
}
\dimostrazioneprop{sdfsdfsdfdd}{
    Assumiamo per assurdo che $ d \in \Q $ 
    
    $\implies$ $ \exists\, m,n \in \Z, n \neq 0 $ primi tra loro tali che $ d=\frac{m}{n} $ 
    
    $\implies$ $ \frac{m^{2}}{n^{2}}=2 $ 
    
    $\implies$ $ m^{2}=2n^{2} $ 
    
    $\implies$ $ m^{2} $ è pari $\implies$ $ m $ è pari\footnote{\hyperref[prp:m2parimpari]{dimostrazione successiva}}

    $\implies$ $ \exists\, k \in \Z $ tale che $ m=2k $ 
    
    $\implies$ $ m^{2}=4k^{2} $ 
    
    $\implies$ $ 2n^{2}=4k^{2} $ 
    
    $\implies$ $ n^{2}=2k_2 $ 
    
    $\implies$ $ n^{2} $ è pari $ \implies $ $ n $ è pari;

    si ha contradizione dell'ipotesi che $ m,n $ fossero primi tra di loro (in quanto entrambi pari hanno almeno un divisore in comune, ovvero 2).\qed
}

\proposizione{m2parimpari}{
    $ m \in \Z $, $ m^{2} $ pari $\implies$ $ m $ pari
}               
\dimostrazioneprop{m2parimpari}{
    Per assurdo, assumiamo $ m $ dispari 
    
    $\implies$ $ \exists\, k \in \Z | m=2k+1 $ 
    
    $\implies$ $ m^{2}=(2k+1)^{2}=4k^{2}+4k+1 $ 
    
    $\implies$ $ m^{2}=\underset{pari}{\underbrace{4k(k+1)}} + 1 $ 
    
    $\implies$ $ m^{2} $ è dispari.

    Si ha contraddizione, pertanto $ m $ è pari.\qed
}               

Dal momento che si è utilizzata nelle ultime dimostrazioni, è bene aprire una parentesi sulle \textit{dimostrazioni per assurdo}

\subsubsection*{Schema dimostrativo per assurdo}

\proposizione[(schema I)]{assurdo1}{
    Siano $ p, q $ preposizioni
\[
    (p \,\implies\,q)\underset{?}{\iff}\bigl((p\land \lnot q) \,\implies\, \lnot p \bigr)
\]
}
\dimostrazioneprop{assurdo1}{
\[
    \begin{array}{ccccccc}
        \toprule
             p  &  q  &  \lnot p  &  \lnot q  &  p \implies q  &  p\land \lnot q  & (p\land \lnot q)\implies\lnot p\\
        \midrule
            1 & 1 & 0 & 0 & 1 & 0 & 1 \\
            1 & 0 & 0 & 1 & 0 & 1 & 0 \\
            0 & 1 & 1 & 0 & 1 & 0 & 1 \\
            0 & 0 & 1 & 1 & 1 & 0 & 1 \\
        \bottomrule
    \end{array}
\]

Si noti come la quinta e l'ultima colonna siano uguali. \qed
}

\proposizione[(schema II)]{assurdo2}{
    Siano $ p, q $ preposizioni
    \[
        (p \,\implies\,q)\underset{?}{\iff}\bigl((p\land \lnot q) \,\implies\, q \bigr)
    \]
}
\dimostrazioneprop{assurdo2}{
    \[
    \begin{array}{cccccc}
        \toprule
             p  &  q  & \lnot q  & p\land \lnot q & p \implies q  & (p\land \lnot q)\implies q\\
        \midrule
            1 & 1 & 0 & 0 & 1 & 1\\
            0 & 0 & 1 & 0 & 1 & 1\\
            1 & 0 & 1 & 1 & 0 & 0\\
            0 & 1 & 0 & 0 & 1 & 1\\
        \bottomrule
    \end{array}
\]

Si noti come la quinta e l'ultima colonna siano uguali. \qed
}

\proposizione[(schema III)]{assurdo3}{
    Siano $ p, q $ preposizioni
    \[
        (p \,\implies\,q)\underset{?}{\iff}(\lnot q \,\implies\, \lnot p)
    \]
}
\dimostrazioneprop{assurdo3}{
    \[
    \begin{array}{cccccc}
        \toprule
             p  &  q  & \lnot q  & \lnot p & p \implies q  & \lnot q \implies \lnot p\\
        \midrule
            1 & 1 & 0 & 0 & 1 & 1\\
            1 & 0 & 1 & 0 & 0 & 0\\
            0 & 1 & 0 & 1 & 1 & 1\\
            0 & 0 & 1 & 1 & 1 & 1\\
        \bottomrule
    \end{array}
\]

Si noti come la quinta e l'ultima colonna siano uguali. \qed
}

\rule{7em}{.4pt}

Dalle dimostrazioni precedenti \hyperref[prp:sdfsdfsdfdd]{(\textit{p}.\roman{prpsdfsdfsdfdd})} si è reso evidente che necessitiamo di un insieme numerico che permetta di risolvere il problema di trovare la diagonale di un quadrato di lato 1: infatti, questo semplice caso ci dimostra che la retta euclidea non è in corrispondenza biunivoca con $ \Q $, ma che anzi la retta di $ \Q $ ha "un buco"

Vogliamo trovare $ X $ tale che $ \Q \subseteq X $, $ X \leftrightarrow $ retta

Per trovare questo insieme è necessario introdurre le \textit{relazioni} all'interno di un insieme

\subsubsection{Relazioni}

Sia $ A $ un insieme generico: diciamo $ \mathcal{R}  $ relazione su A tale che \[
    \mathcal{R} \subseteq A\times A
\]

Dati $ a, b \in A $ si scrive $ a\mathcal{R}b \,\iff\, (a, b) \in \mathcal{R} $. Diciamo che $ a $ è in corrispondenza con $ b $ se $ a\mathcal{R}b $

\proprieta{}{
    \begin{itemize}
        \item $ \mathcal{R} $ si dice \textit{simmetrica} se $ a,b \in A $, $ a\mathcal{R}b $ $\implies$ $ b\mathcal{R}a $
        \item $ \mathcal{R} $ si dice \textit{riflessiva} se $ \forall\, a \in A $, $ a\mathcal{R}a $
        \item $ \mathcal{R} $ si dice \textit{transitiva} se dati $ a,b,c \in A$, $ a\mathcal{R}b \,\land\, b\mathcal{R}c $ $\implies$ $ a\mathcal{R}c $
        \item $ \mathcal{R} $ si dice \textit{antisimmetrica} se dati $ a,b \in A$, $ a\mathcal{R}b \,\land\, b\mathcal{R}a $ $\implies$ $ a=b $
    \end{itemize}
}

\definizione{}{Una relazione $ \mathcal{R} $ su $ A $ è detta \textit{di ordine} se soddisfa le proprietà \textit{riflessiva}, \textit{antisimmetrica} e \textit{transitiva}}

\definizione{}{Una relazione $ \mathcal{R} $ su $ A $ è detta di \textit{ordine totale} (o anche $ A $ è totalmente ordinato rispetto ad $ \mathcal{R} $) se è una relazione d'ordine e vale \[
    \forall\, a, b \in A\quad a\mathcal{R}b\lor b\mathcal{R}a
\]}
\esempi{}{
    \begin{itemize}
        \item $ A $ insieme delle parole del dizionario italiano, $ \mathcal{R} $ ordine lessicografico
        
        $a, b \in A\quad a\mathcal{R}b$ se $a$ viene prima o coincide con $b$ nell'ordine alfabetico.

        $ \mathcal{R} $ è riflessiva, transitiva e antisimmetrica, $ \mathcal{R} $ è di ordine totale.
        \item Sia $ U $ insieme universo, $ \mathscr{P}(U) $ l'insieme delle parti di $ U $\footnote{Si è fatto così e non si è scelto $ V $ (insieme di tutti gli insiemi) per evitare i paradossi; in particolare, vedasi \textit{paradosso di Russel}}, $ \mathcal{R} $ relazione di inclusione ($ \subset $)
        
        $ A, B \in \mathscr{P}(U)$, $ A \subset B $ $\iff$ $ \forall \, x \in A \,\implies\, x \in B $

        $ \mathcal{R} $ è di ordine su $ \mathscr{P}(U) $ ma non è di ordine totale
        \item Nell'insieme $ \Q $ si consideri la relazione \begin{itemize}
            \item minore stretto
            
            $ a<b $ se $ a $ precede strettamente $ b $ nell'ordine da sinistra a destra della retta euclidea

            \item minore uguale
            
            $ a\le b $ se $ a $ precede o coincide $ b $ nell'ordine da sinistra a destra della retta euclidea
        \end{itemize} 
        Si noti che \begin{itemize}
            \item [$ < $] non è di ordine (non soddisfa né la proprietà riflessiva né la proprietà antisimmetrica)
            \item [$\le$] è di ordine totale
        \end{itemize}

        La relazione $<$ non è di ordine in quanto\begin{enumerate}
            \item non soddisfa la proprietà riflessiva: ogni numero non è minore a se stesso 
            \item non soddisfa la proprietà di antisimmetria, in quanto non esiste nessuna coppia di numeri per cui valgano le
            relazioni $a<b$ e $b<a$
        \end{enumerate}
        
        La relazione $ \le $ è di ordine totale, in quanto soddisfa tutte e tre le proprietà:
        \begin{enumerate}
            \item è riflessiva, in quanto ogni numero è minore o uguale a se stesso
            \item è antisimmetrica, in quanto l’unico modo per cui valga la relazione $ a\le b $ e $ b \le a $ è che $a=b$
            \item è transitiva, in quanto se $a\le b$ è $b\le c$ allora $a\le c$
            \item inoltre, per ogni coppia (non ordinata) di numeri reali, è sempre possibile stabilire almeno un ordine che permetta di soddisfare la relazione.
        \end{enumerate}
    \end{itemize}
}

\definizione{}{
    La relazione $ \mathcal{R} $ su A è detta \textit{relazione di equivalenza} se soddisfa le proprietà \textit{riflessiva}, \textit{simmetrica} e \textit{transitiva}. Si indica generalmente con $ x \sim y $ invece di $ x\mathcal{R}y $
}

Una classe di equivalenza di $ u \in A $ (dove $ u $ è detto ``rappresentante'') è
\[
    [u]=\{v \in A:\, v \sim u\}
\]

L'insieme quoziente di $ A $ rispetto a $ \sim $: \[
    A/\sim := \{[u]: u \in A\}
\]