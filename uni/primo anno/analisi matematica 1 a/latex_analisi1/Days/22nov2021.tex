\esempio{\marginnote{22 nov 2021}
    \begin{align*}
    f:(0,1) & \to \R \\
    x & \mapsto x^{2}
    \end{align*}
    \begin{align*}
    g:(0,1) & \to \R \\
    x & \mapsto \frac{1}{x}
    \end{align*}

    Entrambe le funzioni sono continue in $ (0,1) $: $ \forall\,x_0 \in (0,1) $, \[ \forall\, \varepsilon>0\, \exists\,\delta=\delta( \varepsilon, x_0)>0 \,\tc\, |x-x_0|< \delta\quad \begin{cases}
        |f(x)-f(x_0)|< \varepsilon\\
        |g(x)-g(x_0)|< \varepsilon
    \end{cases}\]
    \begin{itemize}
        \item [$f$.] fissiamo $ x_0 \in(0,1) $ \begin{multline*}
            |f(x)-f(x_0)|=|x^{2}-x_0^{2}|=|x-x_0|\,|x+x_0| \underset{x \in (0,1)}{=}\\
            \le \underset{\le 2}{\underbrace{(|x|+|x_0|)}} \cdot |x-x_0| \le 2 |x-x_0|
        \end{multline*}
        Preso $ \delta < \varepsilon/2 $ si ha che
        \[
            |x-x_0|<\delta \,\implies\, |f(x)-f(x_0)|< \varepsilon
        \]

        Ne risulta che $ \delta $ non dipende da $ x_0 \in (0,1) $
        \item [$g$.] fissiamo $ x_0 \in(0,1) $, fissiamo $ \varepsilon>0 $
        \[
            |g(x)-g(x_0)|< \varepsilon\,\iff\, |1/x-1/x_0|< \varepsilon
        \]

        $ \underset{x \in(0,1)}{\iff} $ $ \frac{|x-x_0|}{x\,x_0} < \varepsilon$

        Consideriamo $ x \in (x_0-x_0/2, x_0+x_0/2) = (x_0/2, 3x_0/2) $, $ x \in B_{\frac{x_0}{2}}(x_0) $ ha $ \delta = x_0/2 $, otteniamo 
        %TODO manca un pezzo
    \end{itemize}

    Riassumiamo: \begin{itemize}
        \item [$g$.] $ \forall\, x_0 \in (0,1) $, $ \forall\, \varepsilon>0 $ $ \exists \delta =\delta (x_0, \varepsilon) $ tale che \[
            \forall\,x \quad |x-x_0|<\delta \,\implies\, |g(x)-g(x_0)|< \varepsilon
        \]
        \item [$f$.] $ \forall\, \varepsilon>0 $ $ \exists \delta =\delta (\varepsilon) $ tale che \[
        \begin{aligned}
            \forall\,x_0 & \in (0,x_1)\\
            \forall\, x & \in(0,1)
        \end{aligned}
            \quad |x-x_0|<\delta \,\implies\, |f(x)-f(x_0)|< \varepsilon
        \]
        $ f $ è \textit{uniformemente continua} in $ (0, 1) $
    \end{itemize}
}

\definizione{}{
    Data \begin{align*}
    f:D & \to \R \\
    x & \mapsto f(x)
    \end{align*} con $ D \subseteq \R $, diciamo che $ f $ è \textit{uniformemente continua su} $ D $ se \[
        \forall\, \varepsilon >0\,\exists\, \delta=\delta( \varepsilon)>0\,\tc\, \forall\,x_0, x \in D \quad |x-x_0|<\delta \,\implies\, |f(x)-f(x_0)|< \varepsilon
    \]
}

\esercizio{
    Verificare che su $ [0, + \infty) $ $ f(x)=e^{-x} $ è uniformemente continua e $ g(x)=e^{x} $ non è uniformemente continua.
}{
    Verificare per esercizio %TODO risolvere
}{}

\osservazione{
    Sia \begin{align*}
    f:E & \to \R \\
    x & \mapsto f(x)
    \end{align*} con $ E \subseteq \R $. 
    
    Indichiamo con \textit{diamentro} di $ E $ \[
        \text{diam} E =\sup_{x,y \in E} \{|x-y|\}
    \]

    Indichiamo con \textit{oscillazione} di $ f $ in $ E $ \[
        \omega_E (f)=\sup_{x,y \in E} \{|f(x)-f(y)|\}
    \]

    Sia ha che $ f $ è uniformemente continua $ \iff $ \[
        \forall\, \varepsilon >0\, \exists\, \delta >0 \,\tc\, \text{diam}E <\delta \,\implies\, \omega_E (f)< \varepsilon
    \]
}

\section{Successioni e topologia in $ \R^{n} $}

Breve riassunto:
\begin{enumerate}
    \item $ E $ limitato 
    
    $\implies$ ogni successione a valori in $ E $ ammette sottosuccessioni convergenti
    \item $ E $ chiuso 
    
    $ \iff $ il limite di una successione a valori in $ E $, se esiste, appartiene ad $ E $
\end{enumerate}

\definizione{}{
    Dato $ K \subseteq \R^{n} $ diciamo $ K $ \textit{sequenzialmente compatto} (o compatto per successioni) $
        \forall\, \{x_{k} \}_{k=0}^\infty$ a valori in $ K $ ammette una sottosuccessione $ \{x_{h_{k} } \}_{k=0}^\infty $ convergente a $ y \in K $

    $ K \subseteq \R $ è compatto se ogni sua successione ammette una sottosuccessione convergente in $ K $ stesso
}

\esempi{}{
    \begin{enumerate}
        \item $ E=\{1+1/x, x \in \R, |x| \ge 1\} $, $ E $ limitato, non chiuso.
        
        Consideriamo $ x_{k}=1+1/k \xrightarrow{k\to + \infty} 1 \displaystyle \notin E $ 
        
        $\implies$ ogni sua sottosuccessione $ x_{h_{k}} \xrightarrow{k\to + \infty} 1 \displaystyle \notin E  $ 
        
        $\implies$ $ E $ non è compatto.
        \item $ A=\{\cos k \pi/2\}_{k \in \Z}=\{0, 1, -1\} $ %TODO manca
        \item $ I=[-1, 1] $, chiuso e limitato. $ I $ è limitato 
        
        $\implies$ $ \forall\, x_{k}  $ ammette $ x_{h_{k}} \xrightarrow{k\to + \infty} l \in \R\displaystyle  $. Inoltre $ I $ è chiuso 
        
        $\implies$ $ l \in I $ 
        
        $\implies$ $ I $ è compatto.
        \item $ J=[0, +\infty) $, chiuso non limitato. Sia $ x_{k}=k $ a valori in $ J $. 
        
        $ x_{k} \xrightarrow{k\to +\infty} +\infty \displaystyle $ 
        
        $\implies$ ogni sua sottosuccessione $ x_{h_{k} } \xrightarrow{k\to +\infty} +\infty \notin J   \displaystyle $ 
        
        $\implies$ $ J $ non è compatto
    \end{enumerate}
}

\teorema[(caratt. degli insiemi compatti)]{carcompins}{
    Sia $ K \subseteq \R^{n} $, 
    
    $ K $ è sequenzialmente compatto 

    $ \iff $ $ K $ è chiuso e limitato.
}
\dimostrazione{carcompins}{
    \begin{itemize}
        \item [``$\impliedby$''] Assumiamo $ K $ limitato 
        
        $\implies$ $ \forall\, \{x_{k} \}_{k=0}^\infty $ a valori in $ K $ ammette $ \{x_{h_{k} } \}_{k=0}^\infty $, $ x_{h_{k} } \xrightarrow{k\to +\infty} l \in \R^{n} \displaystyle$

        Inoltre, poiché $ K $ è chiuso si ha $ l \in K $ 
        
        $\implies$ $ K $ è compatto sequenzialmente.
        \item [``$\implies$''] $ K $ compatto.
            \begin{enumerate}
                \item Verifichiamo che $ K $ è limitato; per assurdo assumiamo $ K $ non limitato 
                
                $\implies$ $ \forall\, k \in \N $, $ \exists\, x_{k} \in K  $ tale che $ |x_{k}| > k  $ %non è chiaro

                Sia ora $ \{x_{k} \}_{k=0}^\infty $ a valori in $ K $ data dagli $ x_{k}  $ visti sopra.

                $ |x_{k}| \xrightarrow{k\to +\infty} +\infty \displaystyle  $ ($x_{k}$ è divergente a $\infty $ ) 
                
                $\implies$ ogni sua sottosuccessione $ \{x_{h_{k} } \}_{k=0}^\infty $ diverge a $ \infty $ 
                
                $\implies$ $ K $ non è compatto.

                Abbiamo dimostrato che $ K $ compatto 
                
                $\implies$ $ K $ è limitato.
                \item Verifichiamo che $ K $ è chiuso; usiamo la proprietà \[
                    K\text{ è chiuso}\,\iff\, \delta K \subseteq K
                \]

                Sia $ z \in \delta K $. Per definizione di frontiera $ \forall\, k \in \N\setminus\{0\} $ $ \exists\, x_{k} \in K $ tale che $ x_{k}\in B_{1/k}(z)   $

                Sia $ \{x_{k} \}_{k=0}^\infty $ la successione così ottenuta. $\implies$ \[\forall\, \varepsilon>0\,\exists\, \overline{k} \,|\, \forall\, k \ge \overline{k}\quad x_{k} \in B_{1/k}(z) \subseteq B_{1/\overline{k}}(z) \subseteq B_{ \varepsilon}(z)   \] dunque $ x_{k} \xrightarrow{k\to +\infty} z \in \delta K \displaystyle $

                Osserviamo che poiché $ K $ è compatto esiste una sottosuccessione $ \{x_{h_{k} } \}_{k=0}^\infty $ di $ \{x_{k} \}_{k=0}^\infty $ che converge a $ w \in K $, ossia \[\{x_{h_{k} } \}_{k=0}^\infty \xrightarrow{k\to +\infty} w \in K\displaystyle.\]

                Poiché $ x_{k} \xrightarrow{k\to +\infty} z \displaystyle $ si ha che ogni sua sottosuccessione converge a $ z $.

                Allora $ x_{h_{k}} \to w  $, $ x_{h_{k} } \to z  $ 
                
                $\implies$ per il teorema di unicità del limite, $ w=z $ 
                
                $\implies$ $ z \in K $ 
                
                $\implies$ $ \delta K \subseteq K $ 
                
                $\implies$ $ K $ è chiuso.\qed
            \end{enumerate}
    \end{itemize}
}

In $ \R^{n} $ sono sequenzialmente compatti tutti e soli gli insiemi chiusi e limitati.
\esempi{}{
    \begin{itemize}
        \item $ [a,b] $ compatto;
        \item $ [a,b), (a,b] $ non compatti (non chiusi);
        \item $ [a, +\infty), (-\infty, a] $ non compatti (non limitati);
        \item $ \overline{B}_{r}(x_0) =\{x \in \R^{n}\,\tc\, |x-x_0|\le r\} $ compatta;
        \item $ B_{r}(x_0) =\{x \in \R^{n}\,\tc\, |x-x_0|< r\} $ non compatta (non chiusa);
        \item $ B_{r}^{c}=\{x \in \R^{n}\,\tc\, |x-x_0|\ge r\} $ non compatta (non limitata).
    \end{itemize}
}

\subsection{Continuità e compattezza in $ \R^{n} $}

\paragraph{Continuità}
Sia
\begin{align*}
f:D & \to \R^{m} \\
x & \mapsto f(x)
\end{align*} con $ D \subseteq \R^{n}$ \[
    x=(x_1, \cdots, x_{n} )\qquad f(x)=(f_1(x), \cdots, f_{m}(x) )
\]

Diciamo che $ f $ è continua su $ D $ se \[
    \forall\, x_0 \in D\, \forall\, \varepsilon>0\,\exists\, \delta >0 \,\tc\, \forall\, x \in D\quad |x-x_0|<\delta \,\implies\, |f(x)-f(x_0)|< \varepsilon
\]

\osservazione{
$ f $ è continua su $ D $ se tutte le funzioni componenti $ f_1(x), \cdots, f_{m}(x)  $ sono continue su $ D $.
}

\teorema{compcont1}{
    Sia $ K \subseteq \R^{n} $, $ K $ compatto sequenzialmente. Consideriamo $
    f:K \to \R^{m}$ continua su tutto $ K $ 
    
    $\implies$ $ f(K) $ è un insieme sequenzialmente compatto in $ \R^{m}$

    L'immagine continua di un compatto è compatta.
}

%TODO creare pdf singolo e pagina obsidian