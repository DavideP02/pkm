\definizione{}{Un\marginnote{22 set 2021} insieme $ U $ si dice totalmente ordinato con la relazione d'ordine ``$\preceq$''}

Consideriamo $ A \subseteq U$
\begin{enumerate}
    \item $ A $ è \textit{limitato superiormente} se 
    \[
        \exists\, k \in U\,\tc\, \forall a \in A, a \preceq k
    \] 
    
    $\implies$ $ k $ è detto \textit{maggiorante} di $ A $
    \item $ A $ è \textit{limitato inferiormente} se 
    \[
        \exists\, h \in U\,\tc\, \forall a \in A, h \preceq a
    \] 
    
    $\implies$ $ k $ è detto \textit{minorante} di $ A $
\end{enumerate}
Possono esistere infiniti maggioranti e infiniti minoranti

\definizione{}{$M$ è il massimo di $ A $ se $ M $ è un maggiorante ($ a \preceq M \forall\,a \in A$) e $ M \in A $}

\definizione{}{$m$ è il minimo di $ A $ se $ m $ è un minorante ($ m \preceq a \forall\,a \in A$) e $ m \in A $}

Si dice che $ M=\max A $ e $ m = \min A $

\esempi{}{
    Per tutti gli esempi successivi si consideri $ U= \Q $ e $ \preceq = \le $\begin{enumerate}
        \item $ A = \{5, 7, 9, -4, 588\} $. $ \min A = -4 $, $ \max A = 588 $
        
        Con $ A \subseteq Q $ e $ A $ contenente un numero finito di valori 
        
        $\implies$ $ A $ ammette $ \max $ e $ \min $
        \item $ B=\{2^{n}\,|\,n \in \N\} $, $ B $ è limitato inferiormente 
        
        $\implies$ $ \min B = 1 $, $ B $ non è limitato superiormente
        \item $ C=\{1+1/n\,|\, n \in \N\setminus \{0\}\} $, $ C $ è limitato: $ \forall\, x \in C $, $ 1<x\le 2 $
        
        $ C $ ammette un massimo ($\max C=2$), $ C $ non ammette un minimo
        \item $ D=\{1-1/n \,|\, n \in \N\setminus\{0\}\} $
        
        $ \forall\,x \in D $, $ 0\le x < 1 $
        
        $ \min D = 0 $, $ D $ non ammette $ \max $
    \end{enumerate}
}

\definizione{}{
    Sia $ U $ totalmente ordinato con relazione d'ordine $ \preceq  $, e sia $ a \in U $.
    \begin{itemize}
        \item Diciamo \textit{estremo superiore} di $ A $ ($\sup A$) il più piccolo dei maggioranti.
        \item Diciamo \textit{estremo inferiore} di $ A $ ($\inf A$) il più grande dei minoranti
    \end{itemize}
    \[
        \sup A = \min\{M \in U\,|\, \forall\,x \in A, x \preceq M\},\quad\inf A = \max\{m \in U\,|\, \forall\,x \in A, m \preceq a\}
    \]
    Se esistono $ \max A $ e/o $ \min A $ 
    
    $\implies$ $ \sup A = \max A $, $ \inf A = \min A $ 
}

\esempio{}{
    Sia $ C =\{1+1/n\,|\, n \in \N\setminus\{0\}\} $

    $ \max C = 2 = \sup C$

    $ \min C = \nexists $ 
    
    $\implies$ se $ m $ è minorante di $ C $ 
    
    $\implies$ $ m\le 1 $ $\implies$ $ \inf C =1 $.

    Sia $ D=\{1-1/n\,|\,n \in \N\setminus\{0\}\} $

    $ \min D = 0 = \inf D $

    $ \max D = \nexists $ 
    
    $\implies$ se $ M $ è maggiorante di $ D $ 
    
    $\implies$ $M\ge 1$  $\implies$ $ \sup D = 1 $
}   

\esempio{}{
    \[
        E=\{r \in \Q; r\ge 0, r^{2}<2\} \subseteq \Q
    \]
    \begin{itemize}
        \item $ E $ è limitato: $ \forall\, r \in E, 0\le r<2 $
        \item $ \inf E = \min E = 0 $
        \item $ \sup E$? Se $ x^{2}<2 $ 
        
        $\implies$ $ 0\le x < \sqrt{2}\notin \Q $. Un candidato $ \sup E =\sqrt{2}\notin \Q $ 
        
        $\implies$ $ \sup E = \nexists $
    \end{itemize}
}

\rule{7em}{.4pt}

L'obiettivo, quindi, è quello di costruire un insieme numerico $ X $ (con $ \Q \subseteq X $) con operazioni $ + $ e $ \cdot  $ tale che ogni sottoinsieme limitato ammetta estremo superiore e inferiore.

\subsubsection{Definizione assiomatica dei numeri reali}

\begin{itemize}
    \item [$\mathcal{R}_1$.] È definita un'applicazione $ \R\times\R\to \R$, indicata con il segno ``$ + $'' detta \textit{addizione} o \textit{somma}, che soddisfa le seguenti proprietà:
        \begin{itemize}
            \item $ \forall\,a,b,c \in \R $, $ (a+b)+c=a+(b+c) $ (associativa);
            \item $ \forall\, a,b \in \R $, $ a+b=b+a $ (commutativa);
            \item esiste un elemento in $ \R $ indicato con $ 0 $ (zero) tale che $ \forall\,a \in \R $, $ a+0=a $ (esistenza elemento neutro per $ + $);
            \item $ \forall\,a \in \R $, $ \exists * $ tale che $ a+*=0 $, si indica $ *=-a $, detto \textit{inverso}, \textit{opposto} di $a$ (esistenza dell'inverso per $ + $).
        \end{itemize}
    \item [$\mathcal{R}_2$.] È definita un'applicazione $ \R\times\R\to \R$, indicata con il segno ``$ \cdot  $'' detta \textit{prodotto} o \textit{moltiplicazione}, che soddisfa le seguenti proprietà:
    \begin{itemize}
        \item $ \forall\,a,b,c \in \R $, $ (a \cdot b) \cdot c=a \cdot (b \cdot c) $ (associativa);
        \item $ \forall\, a,b \in \R $, $ a \cdot b=b \cdot a $ (commutativa);
        \item esiste un elemento in $ \R $ indicato con $ 1 $ (uno) tale che $ \forall\,a \in \R $, $ a \cdot 1=a $ (esistenza elemento neutro per $ + $);
        \item $ \forall\,a \in \R $, $ a\neq 0 $ $ \exists * $ tale che $ a \cdot *=1 $, si indica $ *=a^{-1} $, detto \textit{inverso}, \textit{reciproco} di $a$ (esistenza dell'inverso per $ \cdot  $);
        \item $ \forall\, a, b, c \in \R $, $ (a+b) \cdot c=(a \cdot c)+(b \cdot c) $ (distrubutiva).
    \end{itemize}
    \item [$\mathcal{R}_3$.] È definita in $ \R $ una relazione di ordine totale, indicata con ``$\le$'', che soddisfa le seguenti proprietà:
        \begin{itemize}
            \item $ \forall\, a,b,c \in \R $: $ a\le b $ $ \implies $ $ a+c\le b+c $;
            \item $ \forall\,a,b,c \in\R  $, $ 0\le c $: $ a \cdot c\le b \cdot c $.
        \end{itemize}
    \item [$\mathcal{R}_4$.] Sia $ A \subset R $, $ A \neq \emptyset $
    
    Se $ A $ è limitato superiormente, allora $ A $ ammette un estremo superiore.
    
    Se $ A $ è limitato inferiormente, allora $ A $ ammette un estremo inferiore

\end{itemize}

$ \mathcal{R}_1 $ garantisce che $ (\R, +) $ è un gruppo

Queste proprietà possono essere definite anche per $ \Q $, in cui valgono però solo le proprietà corrispondenti a $ \mathcal{R}_1 $, $ \mathcal{R}_2 $, $ \mathcal{R}_3 $.

Se valgono le proprietà $ \mathcal{R}_1 $, $ \mathcal{R}_2 $, $ \mathcal{R}_3 $ per un qualche insieme $ \K $, questo insieme prende il nome di \textit{campo totalmente ordinato}.

$ \R $ e $ \Q $ sono campi totalmente ordinati, e $ \R $ è un \textit{campo ordinato completo}

\subsection{Campi ordinati completi}

Si è costruito un insieme $ \R $ con $ (+, \cdot, \ge) $, che soddisfa $ \mathcal{R}_1 $, $ \mathcal{R}_2 $, $ \mathcal{R}_3 $ e $ \mathcal{R}_4 $.
\begin{itemize}
    \item Quanti insiemi con queste proprietà esistono?
    \item Che relazione c'è tra di loro?
    \item Come li rappresentiamo?
\end{itemize}

\rule{7em}{.4pt}

\definizione{}{
    Dati $ B $ e $ B' $ campi ordinati (soddisfano $ \mathcal{R}_1 $, $ \mathcal{R}_2 $, $ \mathcal{R}_3 $), si definisce \textit{isomorfismo} tra $ B $ e $ B' $ una relazione
    \begin{align*}
    \varphi: B & \to B'  \\
    a & \mapsto a'=\varphi(a)
    \end{align*}
    che gode delle seguenti proprietà:
    \begin{itemize}
        \item $ \varphi$ è biunivoca 
        \item $ \forall a,b \in B $ \begin{itemize}
            \item [\textit{i}.] $ \varphi(a+b)=\varphi(a)+\varphi(b) $
            \item [\textit{ii}.] $ \varphi(a \cdot b)=\varphi(a) \cdot \varphi(b) $
            \item [\textit{iii}.] $ a \le b $ $\implies$ $ \varphi(a)\le \varphi(b) $
        \end{itemize}
    \end{itemize}
}

\teorema{ddddd}{
    Siano $ B $ e $ B' $ campi ordinati $ (+, \cdot , \le, \mathcal{R}_1,\mathcal{R}_2,\mathcal{R}_3) $, con $ B $ completo e $ B' $ completo 
    
    $\implies$ $ \exists $ un isomorfismo $ \varphi: B \to B' $

    Si dice che $ B $ è isomorfo a $ B' $ (e viceversa) poiché la relazione di isomorfismo è di equivalenza: $ B \sim B' $

    Non lo dimostreremo
}

Scelto un campo $ B $ a piacere possiamo costruire la classe di equivalenza \[
    [B]=\{\text{campi ordinati completi}\}
\]
\[
    \R=[B]
\]

\subsubsection{Rappresentazione}
Modello decimale: $ x \in \R $  si rappresenta come \[
    x=p,\alpha_1\alpha_2\cdots\alpha_{n}\cdots 
\]
dove $ p \in \Z $ e $ [\alpha_1\alpha_2\cdots\alpha_{n}\cdots ] $ è un allineamento infinito di cifre tra $ \{1, \cdots, 9\} $

Modello binario: $ y \in \R $ si rappresenta come \[
    y=p,\beta_1 \beta_2\cdots\beta_{n}\cdots
\]
dove $ p \in \Z $ e $ [\beta_1 \beta_2\cdots\beta_{n}\cdots ] $ è un allineamento infinito di cifre tra $ \{1, 2\} $

Non conta il modello che si usa; è necessario dimostrare che questi modelli soddisfino gli assiomi: fare riferimento al libro di testo