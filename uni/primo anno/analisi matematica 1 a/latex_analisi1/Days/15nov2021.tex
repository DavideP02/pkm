\marginnote{15 nov 2021}

% Fine lezione 13

\teorema[(legame limite successione e sottosuccessione)]{ddfsfsfsdfs}{
    Consideriamo $ \{a_{n} \}_{n=0}^\infty $, $ l \in \R* $, 
    
    \[
        \lim_{n\to \infty} a_{n}  = l
    \]

    $ \iff $ ogni sottosuccessione di $a_{n}$ ammette una sottosuccessione che tende a $l$
}
\dimostrazione{ddfsfsfsdfs}{
    \begin{itemize}
        \item [``$\implies$''] La prima implicazione è vera, pertanto \[
            \forall\,V(l) \,\exists \,\overline{n}\,  \forall\,n\ge \overline{n}:\, a_{n} \in V(l) 
        \] 
        Sia $ n\to k_{n}$ crescente, e $ b_{n}=a_{k_{n} } $, allora \[
            \exists\, \overline{\overline{n}} \,\forall\,n\ge \overline{\overline{n}}:\,  k_{n} \ge \overline{n}   
        \]
        allora $ b_{n}=a_{k_{n} } \in V(l) $.

        Dunque \[
            \forall\,V(l)\, \exists\,\overline{\overline{n}} \in \N \,\tc\, \forall\,n > \overline{\overline{n}}:\,b_{n} \in V(l)
        \] 
        
        $\implies$ $ \lim_{n\to +\infty} b_{n}  =l $

        Abbiamo anche dimostrato che $ a_{n} \xrightarrow{n\to \infty} l \displaystyle$ implica che qualsiasi sua sottosuccessione $ b_{k_{n}}\to l  $
        \item [``$\impliedby$''] Assumiamo vera la seconda implicazione, e procedendo per assurdo assumiamo vera la negazione della prima implicazione, ossia \[
            \forall\, V(l) \forall\,n \in \N \exists n'\ge n | a_{n'} \notin V(l) 
        \]

        Consideriamo $ n=1 $; $ \exists n_{1}'>1 $ tale che $ a_{n_1'} \notin V(l) $; $ k_1=n_1' $

        Consideriamo $ n=k_1+1 $; $ \exists n_2' \ge k_1+1>k_1 $ tale che $ a_{n_2'} \notin V(l) $; $ k_2=n_2' $

        Consideriamo $ n=k_2+1 $; $ \exists n_3' \ge k_2+1>k_1 $ tale che $ a_{n_3'} \notin V(l) $; $ k_3=n_3' $

        $ \dots $

        Otteniamo una successione di indici \begin{align*}
        \N & \to  \N \\
        n & \mapsto k_{n} 
        \end{align*}
        strettamente crescente, e una successione $ b_{n}=a_{k_{n} } $ tale che \[
            \exists\, V(l) | \forall\, n, b_{n} \notin V(l) 
        \]

        Allora $ b_{n}  $ non può ammettere sottosuccessioni che tendono a $ l $ 
        
        $\implies$ abbiamo dimostrato la negazione della seconda implicazione, partendo dalla negazione della prima, ovvero la prima implicazione implica la seconda \qed
    \end{itemize}
}

% Lezione 14 (nuova)

\subsection{Successioni a valori in $ \R^{n} $}

\[ \{a_{k} \}_{k=0}^\infty\quad a_{k}=(a_1^{k}, a_2^{k}, a_3^{k}, \cdots, a_{n}^{k}) \in \R^{n} \]

\esempio{}{
    Fissato $ x \in \R^{n} $, \[
        a_{k}=kx =(kx_1, kx_2, kx_3, \cdots, kx_{n}     )
    \]  
}

$ \{a_{k} \}_{k=0}^\infty $ a valori vettoriali è convergente a $ l \in \R^{n} $ se \[
    \forall\, \varepsilon>0\, \exists \overline{k} \in \N \,\forall\, k\ge\overline{k}:\,\underset{\big(\sum_{j=1}^{n}(a_{j}^{k}-l )^{2}\big)^{1/2}}{\underbrace{|a_{k} - l |}}< \varepsilon
\]

$ \{a_{k} \}_{k=0}^\infty $ a valori vettoriali è divergente a $ l \in \R^{n} $ se \[
    \forall\, M>0\, \exists \overline{k} \in \N \,\forall\, k\ge\overline{k}:\,|a_{k}|>M
\]

$ \{a_{k} \}_{k=0}^\infty $ si dice irregolare (oscillante) se non è né convergente né divergente

\osservazione{
    Per $ \{a_{k} \}_{k=0}^\infty $ a valori in $ \R^{n} $ vale il teorema di legame tra limiti di successione e sottosuccessioni

    Valgono tutti i teoremi sui limiti che non coinvolgono l'ordinamento del codominio. (In particolare, non si definiscono le successioni monotone, e quindi non vale il teorema sui limiti delle successioni monotone)
}

\proposizione{asgdfgdffger}{
    Sia $ E \subseteq \R^{n} $, sia $ y \in \R^{n} \cup \{\infty\}$

    Se $ y $ è di accumulazione per $ E $ 
    
    $\implies$ $ \exists\, \{x_{k} \}_{k=0}^\infty $ a valori in $ E $, con $ x_{k} \neq y $ $ \forall\, k \in \N $ e tale che \[
        \lim_{k\to +\infty} x_{k} = y
    \]
}
\dimostrazioneprop{asgdfgdffger}{
    \begin{itemize}
        \item [caso 1.] $ y \in \R^{n} $: $ y \in E' $, si ha \[
            \forall\, r>0 \exists x \in E, x \neq y, x \in B_{r}(y) 
        \]

        Consideriamo $ k=1,2,3,\dots $; possiamo determinare $ x_{k} \in E $, con $ x_{k}\neq y$ e $ x_{k} \in B_{1/k}(y) $

        Abbiamo ottenuto una successione $ \{x_{k} \}_{k=0}^\infty $ a valori in $ E $ tale che $ \forall\,\varepsilon>0 $ $ \exists\, \overline{k} \,|\, \forall\, k\ge \overline{k}:$ $x_{k} \in B_{1/k}(y)\subset B_{{1/\overline{k}}}(y) \subset B_{\varepsilon}(y)   $ 

        Allora $ x_{k} \xrightarrow{k\to +\infty} y \displaystyle  $, $ x_{k}\neq y  $
        \item [caso 2.] $ y =\infty $, $ y \in E' $
        \[
            \forall\, M >0 \exists x \in E:\, |x| > M
        \]

        Per $ k=1,2,3,\dots $ consideriamo $ x_{k} \in E $, con $ |x_{k}|\ge k$ allora \[
            \forall\, \varepsilon>0 \,\exists \overline{k} \in \N \, \forall\, k \ge \overline{k}:\, |x_{k}| \ge k \ge \overline{k}>M 
        \] 
        
        $\implies$ $ x_{k}\to \infty  $ \qed
    \end{itemize}
}

\teorema[(di Bolzano-Weierstrass per le successioni)]{centrale}{
    Data $ \{a_{k} \}_{k=0}^\infty $ a valori in $ \R^{n} $ (valori vettoriali), si ha che 
    
    se $ \{a_{k} \}_{k=0}^\infty $ è limitata 
    
    $\implies$ $ \exists \{a_{h_{k} } \}_{k =0}^\infty $ sottosuccessione tale che $ a_{h_{k}}$ è \textit{convergente} a $ l \in \R $

    Ogni successione limitata ammette sempre una sottosuccessione convergente
}
\dimostrazione{centrale}{
    Indichiamo con $ E=\{a_{k} \} =$ insieme dei valori della successione. $ E $ è limitato per ipotesi;
    \begin{itemize}
        \item [caso 1.] assumiamo che $ E $ abbia un numero infinito di elementi. 
        
        $\implies$ per il teorema di Bolzano-Weiesrtrass sui sottoinsiemi infiniti di $ \R^{n} $ $ \implies $ $ E $ ammette almeno un punto di accumulazione $ \lambda \in \R^{n}$ 
        
        $\implies$ $ \exists \{b_{k} \}_{k=0}^\infty $ a valori in $ E $, tale che $ b_{k} \xrightarrow{k\to +\infty} \lambda \displaystyle  $

        Ma $ E \equiv $ i valori di $ \{a_{k} \}_{k=0}^\infty $
        
        dunque $ b_{k} $ è sottosuccessione di $ a_{k}  $.

        Allora esiste una sottosuccessione di $ a_{k}$ convergente.

        \item [caso 2.] assumiamo che $ E $ abbia un numero finito di elementi. 
        
        $\implies$ esisterà sicuramente un valore di $ E $ assunto infinite volte dalla successione $ \{a_{k} \}_{k=0}^\infty  $. Sia $ a_{k}=l  $ per infiniti indici.

        Consideriamo $ b_{k}=l $, $ \forall\,k \in \N $, $ b_{k}  $ è successioni a valori in $ E $, ed essendo costante: $ b_{k} \xrightarrow{k\to +\infty} l \displaystyle $, dunque $ b_{n}  $ è convergente \qed
    \end{itemize}
}

\osservazione{
    Il teorema di Bolzano-Weierstrass per le successioni utilizza il teorema di Bolzano-Weierstrass per gli insiemi in $ \R^{n} $. Dunque è necessaria la completezza di $ \R $

    Se $ \{a_{n} \} \subset \R^{n} $ ed è limitata $\implies$ $ \{a_{n} \} $ convergente

    Se $ \{a_{n} \} \subset \R$ ed è limitata $\implies$ $ \{a_{n} \} $ convergente

    Se $ \{a_{n} \} \subset \C $ ed è limitata $\implies$ $ \{a_{n} \} $ convergente

    Se $ \{a_{n} \} \subset \Q $ ed è limitata $\nRightarrow $ $ \{a_{n} \} $ convergente
}

\subsubsection{Successioni e chiusura di $ E \subset \R^{n} $}

Si ricorda che la chiusura è \[
    \overline{E}=E\cup \delta E
\]

\proprieta{}{
    Data $ E \in \R^{n} $ e $ y \in \R $

    $ y \in \overline{E} $ $ \iff $ $ \exists \{x_{k} \}_{k=0}^\infty $ a valori in $ E $ tale che $ x_{k} \xrightarrow{k\to +\infty} y \displaystyle  $
}
\begin{proof}
    Procediamo spezzando le due implicazioni
    \begin{itemize}
        \item [``$\implies$''] Ricordiamo che $ \overline{E}=E\cup E' $
        
        $ y \in \overline{E}=E\cup E' $
        \begin{itemize}
            \item se $ y \in E $, allora consideriamo $ x_{k}\equiv  y \in E$ si ha $ x_{k} \xrightarrow{k\to +\infty} y \displaystyle $
            \item se $ y \in E' $ e $ y \notin E $, per la proposizione \hyperref[prp:asgdfgdffger]{(\textit{p.}\roman{prpasgdfgdffger})}, $ \exists\, \{x_{k} \}_{k=0}^\infty $ a valori in $ E $ tale che $ x_{k} \xrightarrow{k\to +\infty} y \displaystyle $
        \end{itemize}

        \item [``$\impliedby$''] Assumiamo per assurdo che esista $ x_{k} \xrightarrow{k\to +\infty} y \displaystyle $ e $ y\notin \overline{E} $, con $ x_{k} \in E $.
        
        $ \overline{E} $ è un insieme chiuso, allora $ (\overline{E})^{C} $ è aperto, ovvero $ \exists\,r>0 $ tale che $ B_{r}(y) \subset (\overline{E})^{C}  $

        Allora $ B_{r}(y)\cap \overline{E}=\emptyset $, allora poiché $ E \subset \overline{E} $ \[
            \exists\,r>0:\, B_{r}(y)\cap E = \emptyset 
        \]
        allora qualsiasi successione a valori in $ E $ non può convergere a $ y $, dunque neghiamo $ x_{k} \xrightarrow{k\to +\infty} y \displaystyle $, si ha contraddizione, dunque \[ y \in \overline{E} \qedhere\]
    \end{itemize}
\end{proof}

\teorema{dsfsdfsdddddd}{
    Dato $ E \in \R^{n} $

    $ E $ è chiuso\referenze{(A)}{\label{x:A}}
    
    $\iff$ se esiste $ \{x_{k} \}_{k=0}^\infty $ a valori in $ E $ tale che $ x_{k} \xrightarrow{k\to +\infty} y \displaystyle $ allora $ y \in E $ \referenze{(B)}{\label{x:B}}

    Equivalentemente:
    
    $ E $ è chiuso \referenze{(A)}{\label{x:A1}}

    $\iff$ tutte le sue successioni convergenti hanno limite in $ E $ stesso \referenze{(B)}{\label{x:B1}}
}
\dimostrazione{dsfsdfsdddddd}{
    \begin{itemize}
        \item [``$\implies$''] $ E $ è chiuso. Ricordiamo che $ E $ è chiuso $ \iff $ $ E=\overline{E} $
        
        Allora per proprietà precedente \[
            \{x_{k} \}_{k=0}^\infty \subset E \,\land\, x_{k}\to y \,\implies\, y \in \overline{E}=E 
        \] 
        \item [``$\impliedby$''] Ricordiamo che $ E $ chiuso $ \iff $ $ E' \subset E $. Dimostriamo che $ E' \subset E $.
        
        Consideriamo $ y \in E' $, $ \implies $ $ \exists \{x_{k} \}_{k=0}^\infty\subset E $, con $ x_{k} \neq y $, $ x_{k} \to y $, allora per \hyperref[x:B1]{(B)}, $ y \in E $

        Dunque $ E' \subset E $, ed $ E $ chiuso \qed

    \end{itemize}
}

\subsection{Successioni di Cauchy}

\definizione{}{
    Sia $ \{a_{k} \}_{k=0}^\infty $ a valori in $ \R^{n} $. Questa successione è detta \textit{successione di Cauchy} (o successione fondamentale) se \[
        \forall\, \varepsilon>0\, \exists\,\overline{k} \in \N\,|\, \forall\, k,m\ge \overline{k}:\, |a_{k}-a_{m}|< \varepsilon  
    \]

    O, equivalentemente
    \[
        \forall\, \varepsilon>0\, \exists\, \overline{k} \in \N\, \forall\,k>\overline{k}\, \forall\, p \in \N:\, |a_{k}-a_{k+p}|< \varepsilon    
    \]
    (Definitivamente $ |a_{k}-a_{k+p}|< \varepsilon $)
}

Intuitivamente, da un certo punto in poi i valori della successione di Cauchy sono vicini a piacere

Studieremo il legame tra l'essere di Cauchy l'essere convergente.