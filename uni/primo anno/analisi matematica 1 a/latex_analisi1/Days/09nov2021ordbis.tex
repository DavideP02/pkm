% Lezione 13 https://math.i-learn.unito.it/mod/resource/view.php?id=102586

\subsection{Un limite notevole}\marginnote{9 nov 2021}

\[
    \lim_{n\to +\infty} \sqrt[n]{n^{\alpha}}\qquad\text{con }\alpha \in \R
\]
\begin{itemize}
    \item $ \alpha = 0$ $\implies$ il limite vale 1
    \item $ \alpha >0 $; ricordiamo che \[
        \lim_{n\to +\infty} \frac{n^{\alpha}}{(1+ \varepsilon)^{n}} =0
    \]

    Allora $ \forall\, \varepsilon>0 $ \[
        -(1- \varepsilon)^{n}<n^{\alpha}<(1+ \varepsilon)^{n}
    \] definitivamente

    Ma è facile vedere \[
        1<n^{\alpha}<(1+ \varepsilon)^{n}
    \] definitivamente 
    
    $\implies$ $ 1<\sqrt[n]{n^{\alpha}}<1+ \varepsilon $ definitivamente

    Per $ \varepsilon\to 0 $ si ha che
    \[
        \lim_{n\to +\infty} \sqrt[n]{n^{\alpha}} =1
    \]
    \item $ \alpha < 0 $
    \[
        \sqrt[n]{n^{\alpha}}=\frac{1}{\sqrt[n]{n^{-\alpha}}}=\frac{1}{\sqrt[n]{n^{\beta}}}
    \]

    Ma $ \sqrt[n]{n^{\beta}} \xrightarrow{n\to +\infty} 1 \displaystyle$, con $ \beta =-\alpha>0 $
    Quindi \[
        \frac{1}{\sqrt[n]{n^{\beta}}}=1
    \]
\end{itemize}
Ne segue che $ \forall\, \alpha \in \R $
\[
    \lim_{n\to +\infty} \sqrt[n]{n^{\alpha}}=1
\]
%TODO manca un pezzo 


\subsection{Sottosuccessioni}

Si ha l'obiettivo di indagare più a fondo il comportamento delle successioni irregolari

\esempi{
\begin{enumerate}
\item Si consideri \[
    a_{n} = (-1)^{n}=1,-1,1,-1
\]
\begin{itemize}
    \item con gli indici pari \[
        a_{2n} = (-1)^{2n}= 1, 1, 1 \qquad n \in \N
    \]
    si ha che $ a_{2n} \xrightarrow{n\to +\infty} 1 \displaystyle  $
    \item con gli indici dispari \[
        a_{2n+} = (-1)^{2n+1}= -1, -1, .1 \qquad n \in \N
    \]
    si ha che $ a_{2n+1} \xrightarrow{n\to +\infty} -1 \displaystyle  $
\end{itemize}
%TODO manca una successione
\end{enumerate}}

\definizione{}{
    Sia $ a:\{a_{n} \}_{n=0}^\infty $ successione a valori reali. Consideriamo una successione di indici \begin{align*}
    k:\N & \to \N \\
    n & \mapsto k_{n}  
    \end{align*}
    con $ k $ strettamente crescente, ovvero \[
        k_{n} < k_{n+1} \quad \forall\, n \in \N    
    \]

    Diciamo \textit{sottosuccessione di} $ a $ la successione \[
        b_{n}=a_{k_{n} }  
    \]
}

Concretamente per costruire $ \{b_{n} \}_{n=0}^\infty $ cancelliamo ad $ \{a_{n} \}_{n=0}^\infty $ una quantità infinita di termini lasciando gli altri invariati.

Ogni successione è sottosuccessione di se stessa, basta prendere $ k_{n}=n$

\esercizio{
Dati \begin{gather*}a_{n}=\sin \biggl(\frac{ \pi}{2}n \biggr)\\ b_{n}=n\sin \biggl(\frac{ \pi}{2}n \biggr) \end{gather*} estrarre le possibili sottosuccessioni regolari
}{
    DA FARE %ESERCIZIO terminare a casa
}{}