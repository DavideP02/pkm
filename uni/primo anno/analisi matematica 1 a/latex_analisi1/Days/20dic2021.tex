\todo{Sistemare appunti di oggi}

\teorema{darbouxdi}{
Sia\marginnote{20 dic 2021} $ f $ derivabile in $ I\setminus\{x_0\} $, $ f $ continua in $ x_0 $
\begin{align*}
    \lim_{x\to x_0^{+}} f'(x) = l \in \R &\implies\, f\text{ derivabile da destra in }x_0\\
    \lim_{x\to x_0^{-}} f'(x) = m \in \R &\implies\, f\text{ derivabile da sinistra in }x_0
\end{align*} 
Vale \[
    f'_{+}(x_0)=l\qquad f'_{-}(x_0)=m
\]
}
\dimostrazione{darbouxdi}{
    \[
        \lim_{x\to x_0} \frac{f(x)-f(x_0)}{x-x_0} = \left(\frac{0}{0}\right)
    \] poiché $ f $ è continua in $ x_0 $.
    \begin{multline*}
        \lim_{x\to x_0} \frac{f'(x)}{1} = \lim_{x\to x_0} f'(x) = l \in \R\\ 
        \underset{\footnotemark}{\implies}\, \lim_{x\to x_0^{+}} \frac{f(x)-f(x_0)}{x-x_0} = l\\
        \implies\, \lim_{x\to x_0^{+}} f'_+(x_0) = l\qedd
    \end{multline*}
    \footnotetext{de L'H\^opital}
}

\rule{7em}{.4pt}

\definizione{}{
    Sia $ \{a_{k} \}_{k=0}^\infty $ successione a valori reali, \begin{equation}
        \Lambda = \{l \in \R^{*}\,|\: \exists\, a_{k_{n} } \longrightarrow l\}
    \end{equation}
    Si dice $ l $ valore limite di $ \{a_{k} \}_{k=0}^\infty $, e $ \Lambda $ classe limite.
}

\osservazione{
    \begin{enumerate}
        \item $ a_{n} \xrightarrow[n\to +\infty]{} \lambda \in \R^{*} $ 
        
        $\implies$ $ \forall\,a_{k_{n} } \longrightarrow \lambda $ 
        
        $\implies$ $ \Lambda=\{\lambda\} $.
        \item $ \Lambda\neq\emptyset $, infatti preso $ A=\{a_{n} \} $ insieme dei valori della successione $ \{a_{k} \}_{k=0}^\infty $, può avvenire che:
        \begin{itemize}
            \item [(\textit{i})] $ A $ è finito, e almeno uno dei suoi elementi $ \lambda $ è assunto infinite volte dalla successione. Allora \[
                \exists\,a_{n_{k} } \equiv \lambda\qquad a_{k_{n} } \xrightarrow[n\to + \infty]{} \lambda \,\implies\, \lambda \in \Lambda 
            \]
            \item [(\textit{ii})] $ A $ è infinito, allora per il teorema di Bolzano Weierstrass ammette almeno un punto di accumulazione $ \lambda \in \R^{*} $: allora \[
                \exists\, a_{k_{n} } \longrightarrow \lambda \,\implies\, \lambda \in \Lambda
            \]
        \end{itemize}
        \item $ \Lambda $ è chiusa in $ \R^{*} $. 
        
        Ricordiamo che $ X $ è chiuso $ \iff $ $ X' \subseteq X $.

        Sia $ l \in \R^{*} $ punto di accumulazione di $ \Lambda $ ($l \in \Lambda'$), ovvero \[
            \forall\, \varepsilon>0\quad \exists\, \lambda \in \Lambda,\: \lambda\neq l\:\tc\quad \lambda \in (l- \varepsilon; l+ \varepsilon)
        \]

        $\lambda \in \Lambda$, $ \lambda $ è punto limite \begin{multline*}
            \exists\, a_{n_{k} } \longrightarrow \lambda \,\implies\\\implies\, \forall\,\eta\quad (\lambda-\eta; \lambda+\eta)\text{ contiene infiniti punti di } \{a_{k} \}_{k=0}^\infty
        \end{multline*}
        Sia $ \eta $ tale che $ (\lambda-\eta; \lambda+\eta) \subseteq (l- \varepsilon; l+ \varepsilon) $ 
        
        $\implies$ $ \forall\, \varepsilon>0 $, $ (l- \varepsilon; l+ \varepsilon) $ contiene infiniti punti di $ \{a_{k} \}_{k=0}^\infty $ 
        
        $\implies$ $ \exists $ $ a_{k_{n} }\longrightarrow l $ 
        
        $\implies$ $ l \in \Lambda $ 
        
        $\implies$ $ \Lambda' \subseteq\Lambda $ 
        
        $\implies$ $ \Lambda $ è chiuso.
    \end{enumerate}
}

Grazie alle osservazioni $ 2.$ $ 3. $ si ha che $ \Lambda $ ammette in $ \R^{*} $ massimo $ M $ e minimo $ m $ (con $ m,M \in [- \infty, + \infty] $).

Indichiamo con \begin{align*}
    \limsup_{n\to + \infty} a_{n} &= M =\max_{ \in \R^{*}} \Lambda\\
    \liminf_{n\to + \infty} a_{n} &= m =\min_{ \in \R^{*}} \Lambda
\end{align*}
Altre notazioni sono \[
    \limsup_{n\to + \infty} a_{n} = \overline{\lim_{n\to \infty} } a_{n}\qquad \liminf_{n\to + \infty} a_{n} = \varliminf_{n\to \infty} a_{n}    % TODO rendere decente questa formula
\]

Ne consegue che $ \{a_{k} \}_{k=0}^\infty $ può non ammettere limite in $ \R^{*} $, ma ammette sempre $ \liminf $ e $ \limsup $.

\esempio{
    Sia $ a_{n}=q^{n}$: allora \begin{itemize}
        \item se $ |q|<0 $ allora $ a_{n}\longrightarrow 0  $ 
        
        $\implies$ $ \Lambda=\{0\} $, e \[
            \limsup a_{n} =\liminf a_{n} =0 
        \]
        \item se $ q=1 $, $ a_{n} \equiv 1 $, $ \Lambda=\{1\} $, e \[
            \limsup a_{n} =\liminf a_{n} = 1 
        \]
        \item se $ q=-1 $, $ a_{n}=1,-1, 1, -1, \cdots$: \[
            \exists\, a_{k_{n} }\longrightarrow 1\qquad \exists\, a_{k_{n} } \longrightarrow -1 
        \] quindi $ \Lambda=\{+1, -1\} $. Inoltre \[
            \lim_{n\to + \infty} a_{n}  = \nexists\qquad \begin{aligned}
                \limsup a_{n} &=1\\
                \liminf a_{n} &=-1  
            \end{aligned}
        \]
        \item se $ q>1 $, $ \lim_{n\to + \infty} =+ \infty $, $ \Lambda =\{+ \infty\} $, e \[
            \limsup a_{n} =\liminf a_{n} = + \infty
        \]
        \item se $ q<-1 $, $ \Lambda=\{+ \infty, -\infty\} $ e \[
            \limsup a_{n} = + \infty\qquad \liminf a_{n} = - \infty
        \]
    \end{itemize}
}

\osservazione{
    In generale vale che \[
        \lim_{n\to + \infty} a_{n}  =l \in \R^{*} \,\iff\, \liminf_{n\to + \infty} a_{n} = \limsup_{n\to +\infty} a_{n} =l 
    \]  
}

\esempio{
    Sia $ \displaystyle a_{n}=\sin\left(n\frac{\pi}{2}\right)$, $ a_{n} $ assume solo i valori $ \{-1, 0, 1\} $. Si hanno le seguenti sottosuccessioni: \begin{align*}
        a_{2n}=b_{n}&= \sin\left(2n\frac{\pi}{2}\right)\equiv 0\\
        a_{4n+1}=c_{n}&= \sin\left((4n+1)\frac{\pi}{2}\right)\equiv 1\\
        a_{4n+3}=d_{n}&= \sin\left((4n+3)\frac{\pi}{2}\right)\equiv -1 
    \end{align*} 

    $ \implies $ $ \Lambda =\{-1, 0, 1\} $, e \[
        \liminf a_{n} = -1\qquad \limsup a_{n} = +1   
    \]
}

\esempio{
    Sia $\displaystyle a_{n}=n\,\sin\left(n\frac{\pi}{2}\right)  $: \begin{align*}
        a_{2n}&= 2n\, \sin\left(2n\frac{\pi}{2}\right) =0\\
        a_{4n+1}&= (4n+1)\,\sin\left((4n+1)\frac{\pi}{2}\right) = 1, 5, 9, 13,\cdots\\
        a_{4n+3}&= (4n+3)\,\sin\left((4n+3)\frac{\pi}{2}\right) = -3, -7, -11,\cdots
    \end{align*}
    Vale che \[
        a_{2n} \longrightarrow  0\qquad a_{4n+1} \longrightarrow + \infty\qquad a_{4n+3}\longrightarrow - \infty
    \] quindi $ \Lambda=\{- \infty, 0, \infty\} $ e \[
        \liminf a_{n} = - \infty\qquad \limsup a_{n} = + \infty  
    \]
}

\proprieta{}{
    Vale sempre che \[
        \liminf a_{n} \le \limsup a_{n}  
    \]

    Inoltre se $ \{a_{k} \}_{k=0}^\infty $ è limitata superiormente (inferiormente), allora 
    \begin{align*}
        \exists\, a_{k_{n}} &\longrightarrow \limsup a_{n} = M \in \R\\  
        \Bigl(\exists\, a_{k_{n}} &\longrightarrow \liminf a_{n} = m \in \R\Bigr)
    \end{align*}
}   
\rule{7em}{.4pt}

\subsection{Formula di Taylor}
\subsubsection{Algebra degli ``o'' piccoli}

Siamo nel caso $ o(x^{n})_{x\to 0}  $.

Dati $ n, m \in \N $: \begin{enumerate}
    \item $ n<m $: $ x^{m}=o(x^{n}) $;
    \item $ n\le m $: $ o(x^{n} ) + o(x^{m}) = o(x^{n}) $;
    \item dato $ \lambda \in \R $, $ \lambda o (x^{n})=o(x^{n}) $, e $ o( \lambda x^{n}) = o(x^{n}) $;
    \item $ x^{m}\,o(x^{n}) = o(x^{m+n}) $ e $ o(x^{m})\,o(x^{n})= o(x^{m+n}) $;
    \item $ \left(o(x^{n})\right)^{m}=o(x^{mn}) $;
    \item se $ f $ limitata in $ U(0) $, allora $ f(x)o(x^{n})=o(x^{n}) $;
    \item 
\end{enumerate}

Si noti che \[
    f=o(x^{n})\,\iff\, \lim_{x\to 0} \parentesi{\omega(x)}{\frac{f(x)}{x^{n}}} =0\,\iff\, f(x)= x^{n}\omega(x)
\] con $ \omega(x) \xrightarrow[x\to 0]{} 0 \displaystyle $

\begin{proof}
    Di seguito la dimostrazione delle proprietà:
    \begin{enumerate}
        \item Manca la dimostrazione %TODO da aggiungere
        \item $ f(x)=o(x^{n}) $, $ g(x)=o(x^{m}) $, $ m\ge n $ \[
            f(x)=x^{n} \cdot \omega_{1}(x)\qquad g(x)=x^{m} \cdot \omega_2(x)  
        \] quindi \[
            f(x)+g(x)=x^{n} \omega_1(x)+ x^{m}\omega_2(x)=x^{n}\bigl(\parentesi{\omega(x)}{\omega_1(x)+x^{n-m}\omega_2(x)}\bigr)
        \] poiché $ m\ge n $, \[
            \omega(x)=\omega_1(x)+x^{n-m}\omega_2(x) \xrightarrow[x\to 0]{} 0 \displaystyle
        \] da cui \[
            f(x)+g(x)=x^{n}\omega(x) \,\iff\, o(x^{n}) + o(x^{m})=o(x^{n}).
        \]
        \item Dimostrare per esercizio %ESERCIZIO lasciata per esercizio
        \item Manca la dimostrazione%TODO da aggiungere
        \item Dimostrare per esercizio %ESERCIZIO lasciata per esercizio
        \item Dimostrare per esercizio %ESERCIZIO lasciata per esercizio
        \qedhere
    \end{enumerate}
\end{proof}

\subsubsection{Calcolo di ordine di infinitesimo e parte principale}

Ricordiamo che $ f $ è infinitesima di ordine $ \alpha>0$ per $ x\to x_0 $ se \[
    \lim_{x\to x_0} \frac{f(x)}{(x-x_0)^{\alpha}} = \lambda \in \R 
\]
Possiamo scrivere che \[
    \lim_{x\to x_0} \frac{f(x)-\lambda (x-x_0)^\alpha}{(x- x_0 )^\alpha} =0
\] ovvero \[
    f(x)-\lambda (x-x_0)^\alpha = o((x- x_0 )^\alpha)_{x\to x_0}
\] ovvero \[
    f(x)= \parentesi{\footnotemark}{\lambda (x-x_0)^{\alpha}} + o \left((x-x_0)^{\alpha}\right)_{x\to x_0}
\]\footnotetext{parte principale di $ f $ (indicata con $ ppf $)}

Allora, date $ f $ infinitesima di ordine $ \alpha $ in $ x_0 $, 
\[
    f(x)= \lambda(x-x_0)^{\alpha} o \left((x-x_0)^{\alpha}\right)_{x\to x_0}
\]
$ g $ infinitesima di ordine $ \beta $ in $ x_0 $, ovvero \[
    g(x)= \mu(x-x_0)^{\beta}+ o\left((x-x_0)^{\beta}\right)_{x\to x_0}
\]
\begin{multline}
    \lim_{x\to x_0} \frac{f(x)}{g(x)} =\\
    = \lim_{x\to x_0} \frac{\lambda(x-x_0)^{\alpha} o \left((x-x_0)^{\alpha}\right)_{x\to x_0}}{\mu(x-x_0)^{\beta}+ o\left((x-x_0)^{\beta}\right)_{x\to x_0}}=\\
    = \lim_{x\to x_0} \frac{\lambda(x-x_0)^{\alpha}\left(1+\frac{o\left((x-x_0)^{\alpha}\right)}{\lambda(x-x_0)^{\alpha}}\right)}{\mu(x-x_0)^{\beta}\left(1+\frac{o\left((x-x_0)^{\beta}\right)}{\mu(x-x_0)^{\beta}}\right)}\label{questaquielelle}
\end{multline}
Per definizione si ha che \[
    \frac{o\left((x-x_0)^{\alpha}\right)}{\lambda(x-x_0)^{\alpha}} \xrightarrow[x\to x_0]{}  0 \qquad \frac{o\left((x-x_0)^{\beta}\right)}{\mu(x-x_0)^{\beta}} \xrightarrow[x\to x_0]{}  0
\] quindi \eqref{questaquielelle} equivale a
\begin{equation}
    \lim_{x\to x_0} \frac{\lambda(x-x_0)^{\alpha}}{\mu(x-x_0)^{\beta}}=\frac{\lambda}{\mu} \lim_{x\to x_0} \frac{(x-x_0)^{\alpha}}{(x-x_0)^{\beta}}
\end{equation}

L'obiettivo ora è la ricerca di ordine di infinitesimo e parte principale di $ f $.

\esempio{
    Determinare l'ordine di infinitesimo e parte principale per $ x\to 0 $ di \[
        f(x)=\sin x -x
    \]
    Calcolare \[
        \lim_{x\to 0} \frac{f(x)}{x^k}
    \] con $ k\ge 0 $

    Dagli sviluppi di MacLaurin del seno, \[
        \sin x = x- \frac{x^{3}}{3!}+o(x^{4})
    \] da cui \[
        f(x)=\sin x - x = \cancel{x}- \frac{x^{3}}{3!}+o(x^{4}) -\cancel{x}
    \] per cui \[
        f(x)=\frac{x^{3}}{3!}+o(x^{4})
    \] e la parte principale di $ f $ è $ \frac{x^{3}}{3!} $.
    L'ordine di infinitesimo di $ f $ per $ x\to 0 $ è $ \alpha=3 $.

    \[
        \lim_{x\to x_0} \frac{f(x)}{x^k} = -\frac{1}{6} \lim_{x\to 0^{+}} \frac{x^{3}}{x^{k}} = \begin{cases}
            + \infty &k>3\\
            -\frac{1}{6} &k=3\\
            0 &0\le k<3 
        \end{cases}
    \]
}

\esempio{
    Sia $ f(x)=(\sin x)^{2}-x^{2} $, calcolarne ordine di infinitesimo, parte principale e \[
        \lim_{x\to 0} \frac{f^{x}}{x^{k}}
    \]
    Dagli sviluppi di MacLaurin del seno, \[
        \sin x = x- \frac{x^{3}}{3!}+o(x^{4})
    \]
    \begin{multline*}
        (\sin x)^{2}=\left(x- \frac{x^{3}}{3!}+o(x^{4})\right)^{2}= \\
        \cdots\\%TODO svolgere i calcoli terribili
        =x^{2}+\frac{x^{3}}{36}-\frac{x^{4}}{3} +o(x^{8})+o(x^{5})+o(x^{7})=\\
        = x^{2}-\frac{x^{4}}{3}+ \parentesi{o(x^{5})}{o(x^{5})+o(x^{5})+o(x^{7})+o(x^{8})}=x^{2}-\frac{x^{4}}{3}+o(x^{5})
    \end{multline*}
    Da cui \[
        f(x)=(\sin x)^{2}-x^{2}=\cancel{x^{2}}-\frac{x^{4}}{3}+o(x^{5})-\cancel{x^{2}}
    \]
    \[
        f(x)=-\frac{1}{3}x^{4}+o(x^{5})
    \] il cui ordine di infinitesimo è $ \alpha= 4 $, e la parte principale è $ -x^{4}/3 $
    %ESERCIZIO finire il limite
}

\osservazione{
    Dato $ n\le m $
    \[
        \frac{o(x^{m})}{x^{n}}=\frac{1}{x^{n}}x^{m}\,\omega(x)
        = x^{m-n}\,\omega(x)=o(x^{m-n})
    \]
}

\esercizio{
    Calcolare \begin{gather*}
        \lim_{x\to 0} \frac{\ln(\cos^{2}x)}{\sin^{2}x}\\
        \lim_{x\to 0} \frac{\ln(\cos^{2}x+x^{2})}{x^{\alpha}}
    \end{gather*}
}{
    Lasciato per esercizio%ESERCIZIO 
}{}