\definizione{}{Un\marginnote{22 set 2021} insieme $ U $ si dice totalmente ordinato con la relazione d'ordine ``$\preceq$''}

Consideriamo $ A \subseteq U$
\begin{enumerate}
    \item $ A $ è \textit{limitato superiormente} se 
    \[
        \exists\, k \in U\,\tc\, \forall a \in A, a \preceq k
    \] 
    
    $\implies$ $ k $ è detto \textit{maggiorante} di $ A $
    \item $ A $ è \textit{limitato inferiormente} se 
    \[
        \exists\, h \in U\,\tc\, \forall a \in A, h \preceq a
    \] 
    
    $\implies$ $ k $ è detto \textit{minorante} di $ A $
\end{enumerate}
Possono esistere infiniti maggioranti e infiniti minoranti

\definizione{}{$M$ è il massimo di $ A $ se $ M $ è un maggiorante ($ a \preceq M \forall\,a \in A$) e $ M \in A $}

\definizione{}{$m$ è il minimo di $ A $ se $ m $ è un minorante ($ m \preceq a \forall\,a \in A$) e $ m \in A $}

Si dice che $ M=\max A $ e $ m = \min A $

\esempi{}{
    Per tutti gli esempi successivi si consideri $ U= \Q $ e $ \preceq = \le $\begin{enumerate}
        \item Sia $ A = \{5, 7, 9, -4, 588\} $. $ \min A = -4 $, $ \max A = 588 $
        
        Con $ A \subseteq Q $ e $ A $ contenente un numero finito di valori 
        
        $\implies$ $ A $ ammette $ \max $ e $ \min $
        %TODO finire gli appunti da pagina due del pdf
    \end{enumerate}
}