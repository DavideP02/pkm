\esempio{}{$f(x)=x^{2}$\marginnote{29 nov 2021}, $ x_0 \in \R $, 
\begin{multline*}
    \lim_{x\to x_0} \frac{f(x)-f(x_0)}{x-x_0} = \lim_{h\to 0} \frac{f(x_0+h)-f(x_0)}{h} =\\
    \lim_{h\to 0} \frac{(x_0+h)^{2}-x_{0}^{2}}{h} = \lim_{h\to 0} \frac{\cancel{x_0^{2}}+2x_0h+h_2-\cancel{x_0^{2}}}{h} = \\
    \lim_{h\to 0} \frac{\cancel{h}}{\cancel{h}}(2x_0+h) = 2x_0
\end{multline*}

$ f(x)=x^{2} $ è derivabile $ \forall\,x_0 \in \R $. La retta tangente è \[
    y = 2x_0 \,x-x_0^{2}
\]
%TODO aggiungere grafico
}

\esempio{}{
    \[H(x)=\begin{cases}
        0 & x<0\\
        1 & x\ge 0
    \end{cases}\qquad \dom H = \R, x_0 \in \R\]
    % TODO manca questo esempio
}

$ f $ continua e derivabile ovunque, $ H $ non è continua e non è derivabile in $ x_0 =0$

\teorema{contederiv}{
Sia $ f:I\to \R $, $ x_0 \in I $. $ f $ derivabile in $ x_0 $ 

$\implies$ $ f $ continua in $ x_0 $
}
\dimostrazione{contederiv}{
    $ f $ continua in $ x_0 $ \[
        \iff \lim_{h\to 0} f(x_0+h) = f(x_0) \iff \lim_{h\to 0} \bigl(f(x_0)-f(x_0)\bigr) =0
    \]
    Dimostriamo che vale se $ f $ è derivabile in $ x_0 $
   \[
        \lim_{h\to 0} \bigl(f(x_0)-f(x_0)\bigr) =\\ \lim_{h\to 0} \underset{\to f'(x_0) \in \R}{\frac{f(x_0+h)-f(x_0)}{h}} \cdot h = 0\qedhere\]
}

Quindi $ f $ derivabile in $ x_0 $ implica $ f $ continua, dunque $ f $ non continua in $ x_0 $ implica $ f $ non derivabile in $ x_0 $.

Si ha che $ f $ continua in $ x_0 $ \textit{non} implica $ f $ derivabile in $ x_0 $
\esempio{
    \[
      f(x)=|x|=\begin{cases}
          x & x\ge 0\\
          -x & x<0
      \end{cases}  
    \]
    $ f $ continua in $ x_0=0 $. \begin{align*}
        \lim_{x\to h^{+}} \frac{f(0+h)-f(0)}{h}& = +1 \\
        \lim_{x\to h^{-}} \frac{f(0+h)-f(0)}{h}& = -1 
    \end{align*} ovvero la funzione non è derivabile per $ x_0=0 $
}{}

\osservazione{
    $ f(x)=|x| $ non è derivabile in $ x_0=0 $ e \begin{align*}
        \lim_{x\to h^{+}} \frac{f(0+h)-f(0)}{h}& = +1 \\
        \lim_{x\to h^{-}} \frac{f(0+h)-f(0)}{h}& = -1 
    \end{align*}
    La funzione $ |x| $ ammette in $ x_0=0 $ limite destro e sinistro del rapporto incrementale.
}

\definizione{}{
    Consideriamo $ f:I\to \R $, $ x_0 \in I $, $ f $ è derivabile da destra (sinistra) in $ x_0 $ se \[
        \lim_{x\to h^{\pm}} \frac{f(x_0+h)-f(x_0)}{h}=\begin{cases}
            m \in \R & h\to 0^{+}\\
            l \in \R & h\to 0^{-}
        \end{cases}
    \]

    Diciamo $ x_0 $ \textit{punto angoloso}. 

    Si dirà $ m=f_{+}'(x_0)  $ \textit{derivata destra} in $ x_0 $ e $ l=f_{-}'(x_0)  $ \textit{derivata sinistra} in $ x_0 $
}

\osservazione{
    $ f $ è derivabile in $ x_0 \in \dom f $ 

    $ \iff $ $ f $ è derivabile da destra e da sinistra in $ x_0 $ e $ f_{+}'(x_0)=f_{-}'(x_0)  $
}

\proprieta{}{
    $ f:I\to \R $, $ x_0 \in I $, $ x_0 $ punto angoloso per $ f $ 
    
    $\implies$ $ f $ continua in $ x_0 $
}
\begin{proof}
    \[
        \lim_{h\to 0} f(x_0+h)-f(x_0) = \lim_{h\to 0^{\pm}} h\underset{\tiny\begin{gathered}
            \xrightarrow{h\to 0^{+}} f'_+(x_0) \in \R\\
            \xrightarrow{h\to 0^{-}} f'_-(x_0) \in \R
        \end{gathered}}{\underbrace{\frac{f(x_0+h)-f(x_0)}{h}}} =0\qedhere
    \]
\end{proof}

\esempio{
    Consideriamo  \[H(x)=\begin{cases}
        0 & x<0\\
        1 & x\ge 0
    \end{cases}\]

    $ x_0=0 $ è un punto di salto. $ H $ non è derivabile in $ x_0=0 $, è possibile che ammetta derivata destra e sinistra?
    \[
        \lim_{h\to 0^{+}} \frac{H(0+h)-H(0)}{h} = \lim_{h\to 0^{+}} 0 = 0
    \]
    $ H $ ammette derivata destra in $ x_0=0 $, $ H'_+ (0)=0$.
    \[
        \lim_{h\to 0^{-}} \frac{H(0+h)-H(0)}{h} = \lim_{h\to 0^{+}} \frac{0-1}{h} = +\infty
    \]
    $ H $ non ammette derivata sinistra in $ x_0=0 $.
}{}

\osservazione{
    Se $ x_0 $ è un punto di discontinuità di prima specie (eliminabile o salto) $ f $ non può ammettere in $ x_0 $ sia derivata destra che derivata sinistra.
}

\proprieta[(Algebra delle derivate)]{
    Siano $ f, g: I \to \R$, consideriamo $ x \in I $, $ f, g $ derivabili in $ x $. Allora si ha
    \begin{enumerate}
        \item [(\textit{i})] $ f+g $ è derivabile in $ x $, \[(f+g)'\,(x)=f'(x)+g'(x);\]
        \item [(\textit{ii})] $ fg $ è derivabile in $ x $, \[(fg)'\,(x)=f'(x)\,g(x)+f(x)\,g'(x)\] (regola di Leibniz)
        \item [(\textit{iii})] $ k \in \R $, $ kf $ derivabile in $ x $, \[(kf)'\,x =k \,f'(x)\]
        \item [(\textit{iv})] se $ g(x)\neq 0 $ allora $ f/g $ è derivabile in $ x $ e vale \[
            \biggl(\frac{f}{g}\biggr)'\,(x)=\frac{f'(x)\,g(x)-f(x)\,g'(x) }{g^{2}(x)}.
        \]
    \end{enumerate}
}
\begin{proof}
    % TODO manca dimostrazione
\end{proof}
\osservazione{
    Le proprietà (\textit{i}) e (\textit{iii}) garantiscono che l'inisme delle funzioni derivabili su $ x  $ è uno spazio vettoriale su campo reale. Vale \[
        (\alpha f +\beta g)'\,(x)=\alpha f'(x)+\beta g'(x)
    \]
}

\definizione{}{
Data $ f:I \to \R$ diciamo $ f $ derivabile su $ I $ se $ \forall\, x \in I $, $ f$ derivabile in $ x $

Possiamo scrivere il rapporto incrementale $ \forall\, x \in I $ \[
    \frac{f'(x+h)-f'(x)}{h}
\]

Se \[
    \lim_{h\to 0} \frac{f'(x+h)-f'(x)}{h} = L \in \R
\] diciamo $ f $ derivabile 2 volte in $ x $, e \[
    f''(x)=L
\] detta \textit{derivata seconda} di $ f $ in $ x $.

Assunta $ f'' $ derivabile su tutto $ I $ possiamo allo stesso modo definire \[
    f'''(x)
\]

Se $ f $ derivabile $ n-1 $ volte su I, definiamo la derivata $ n $-esima di $ f $ in $ x \in I $ \[
    f^{(n)}= \lim_{h\to 0} \frac{f^{(n-1)}(x+h)-f^{(n-1)}(x)}{h}
\] se tale limite esiste.
}

\notazione{}{
    Si indica \begin{gather*}
        f'(x)=\dot{f}=D(f(x))=\frac{\mathrm{d}f}{\mathrm{d}x}\\
        f''(x)=\ddot{f}=D^{2}(f(x))=\frac{\mathrm{d}^{2}f}{\mathrm{d}x^{2}}\\
        f'''(x)=D^{3}(f(x))=\frac{\mathrm{d}^{3}f}{\mathrm{d}x^{3}}\\
        f^{(n)}(x)=D^{n}(f(x))=\frac{\mathrm{d}^{n}f}{\mathrm{d}x^{n}}
    \end{gather*}
}

%TODO creare pdf e pagina obsidian