\lemma{dddddddsdfsdfs}{
    Data\marginnote{16 nov 2021} $ \{a_{k} \}_{k=0}^\infty $ a valori in $ \R^{n} $, 
    
    $ \{a_{k} \}_{k=0}^\infty $ è di Cauchy $ \implies $ $ \{a_{k} \}_{k=0}^\infty $ è limitata
}
\dimostrazionelem{dddddddsdfsdfs}{
    Consideriamo $ \varepsilon=1 $: 
    \[
        \exists \,\varkappa>0:\, \forall\, k>\varkappa
    \] si ha $ |a_{k} - a_{\varkappa}|<1 $, $ \forall\, k\ge \varkappa $
    \[
        |a_{k}-a_{\varkappa}|\ge ||a_{k}|-|a_{\varkappa}||  
    \]
    Allora per $ k>\varkappa $ \[
        ||a_{k}|-|a_{\varkappa}||<1
    \]
    \[
        |a_{\varkappa}|-1<|a_{k}|<|a_{\varkappa}| +1  
    \]

    Consideriamo\begin{align*}
        m&=\min\{|a_0|,|a_1|,\cdots,|a_{\varkappa}|, |a_{\varkappa}|-1  \}\\
        M&=\max\{|a_0|, |a_1|, \cdots, |a_{\varkappa}|, |a_{\varkappa}|+1  \}
    \end{align*}

    Dunque $ \forall\, k \in \N $, \[
        m < |a_{k}| < M 
    \] 
    
    $\implies$ $ \{a_{k} \}_{k=0}^\infty $ è limitata \qed 
}

\teorema[(Criterio di convergenza di Cauchy per le successioni)]{dfgdfgdfgdfgdfg}{
    Data $ \{a_{k} \}_{k=0}^\infty $ a valori in $ \R^{n} $, si ha

    $ a_{k} $ convergente $ \iff $ $ a_{k}  $ è di Cauchy
}
\dimostrazione{dfgdfgdfgdfgdfg}{
    \begin{itemize}
        \item [``$\implies$''] $ \{a_{k} \} $ è convergente, allora \[
            \exists\, l \in \R^{n}
        \] tale che \[
            \forall\, \varepsilon>0\, \exists\, \overline{k}:\, \forall\,k>\overline{k}:\, |a_{k}-l|< \varepsilon 
        \]
        possiamo scrivere
        \[
            |a_{k}-a_{m}|\le |a_{k}-l|+|a_{m}-l|    
        \]
        \[
            \exists\,\overline{k}\,\forall\, k,m\ge \overline{k}:
        \]
        \begin{align*}
            |a_{k}-l|&< \varepsilon/2\\ 
            |a_{m}-l|&< \varepsilon/2
        \end{align*}
        ossia
        \[
            |a_{k}-a_{m}|\le |a_{k}-l|+|a_{m}-l|<\varepsilon/2+\varepsilon/2= \varepsilon
        \]

        $\implies$ $ \{a_{n}\} $ è di Cauchy
        \item [``$\impliedby$''] $ \{a_{k} \}_{k=0}^\infty $ è di Cauchy
        
        $ \underset{Lemma}{\underbrace{\implies}} $ $ \{a_{k} \}_{k=0}^\infty $ è limitata

        $\underset{B-W}{\underbrace{\implies}}$ ammette una sotosuccessione convergente, ossia esiste $ h_{k} \in \N$, $ \{a_{h_{k} } \} $ è convergente, ossia $ \exists \,l \in \R^{n} $: \[
            \forall\, \varepsilon \, \exists \overline{k}:\, \forall\, k>\overline{k}:\, |a_{h_{k} }-l|< \varepsilon/2 
        \]

        Osserviamo \[
            |a_{k}-l|\le |a_{k}-a_{h_{k} } |+|a_{h_{k} }-l|
        \]

        Poiché la successione è di Cauchy \[
            \exists\,\overline{\overline{k}}:\,\forall\, m, h > \overline{\overline{k}}:\, |a_{m}-a_{h}|< \varepsilon  
        \] \[
            \exists\,\overline{\overline{\overline{k}}}:\,\forall\,k\ge\overline{\overline{\overline{k}}}:\, h_{k}> \overline{\overline{k}}
        \]

        Allora preso \[
            \varkappa =\max\{\overline{k}, \overline{\overline{k}}, \overline{\overline{\overline{k}}}\}
        \]

        Otteniamo $ \forall\,k\ge\varkappa $ \[
            |a_{k}-l|\le \overset{< \varepsilon/2}{\overbrace{|a_{k}-a_{h_{k} }|}}+\overset{< \varepsilon/2}{\overbrace{|a_{h_{k} } -l|}}< \varepsilon\qedhere
        \]
    \end{itemize}
}

\osservazione{
    Nella dimostrazione si è usato il Teorema di Bolzano-Weirestrass, ossia la completezza di $ \R $, dunque il criterio di convergenza di Cauchy non vale per successioni a valori in $ \Q $ o in $ \Q^{n} $.
}

\section{Teoremi per le funzioni continue}

\notazione{
    Un punto $ x_0 $ tale che $ f(x_0)=0 $ è detto \textit{zero} di $ f $
}

\teorema[(Teorema di esistenza degli zeri)]{teoexzeri}{
    Consideriamo $ f:[a,b] \to \R $, e assumiamo $ f $ continua su $ [a,b] $, e assumiamo che $ f(a)f(b)<0 $ 

    $\implies$ $ \exists\, c \in (a,b) \,|\, f(c)=0$
}
\dimostrazione{teoexzeri}{
    Assumiamo $ f(a)>0 $ e $ f(b)<0 $.

    %TODO aggiungere grafico 

    Poniamo $ a_0=a $ e $ b_0=b $; consideriamo il punto medio $ c_0=\frac{a_0+b_0}{2} $. 
    
    Abbiamo tre possibilità sul segno di $ f(c_0) $: 
    \begin{enumerate}
        \item $ f(c_0)>0 $: poniamo $ a_1=c_0 $ e $ b_1=b_0 $;
        \item $ f(c_0)<0 $: poniamo $ a_1=a_0 $ e $ b_1=c_0 $;
        \item $ f(c_0)=0 $: la dimostrazione è terminata ponendo $ c=c_0 $: $ f(c)=0 $.
    \end{enumerate}

    Consideriamo $ c_1=\frac{a_1+b_1}{2} $
    
    Abbiamo tre possibilità sul segno di $ f(c_1) $: 
    \begin{enumerate}
        \item $ f(c_1)>0 $: poniamo $ a_2=c_1 $ e $ b_2=b_1 $;
        \item $ f(c_1)<0 $: poniamo $ a_2=a_1 $ e $ b_2=c_1 $;
        \item $ f(c_1)=0 $: la dimostrazione è terminata ponendo $ c=c_1 $: $ f(c)=0 $.
    \end{enumerate}

    Procedendo in questo modo, vi sono due possibilità
    \begin{itemize}
        \item $ \exists\, n $ tale che $ f(c_{n} )=0 $: $ c=c_n $ e $ f(c)=0 $;
        \item si ottengono due successioni a valori reali in $ [a,b] $, che chiamiamo $ \{a_{n} \}_{n=0}^\infty $ e $ \{b_{n} \}_{n=0}^\infty $ tali che: \begin{itemize}
            \item $ \{a_{n} \}_{n=0}^\infty $ crescente e $ \forall\, n $, $ a_{n}\le b_0=b  $;
            \item $ \{b_{n} \}_{n=0}^\infty $ decrescente e $ \forall\, n $, $ b_{n}\ge a_0=a  $;
            \item $ \forall\,n $, $ a_{n}\le b_{n}   $
        \end{itemize}

        Otteniamo inoltre una sequenza di intervalli $ [a_{n}, b_{n}]  $ tali che \[
            [a_0, b_0]\supset [a_1, b_1] \supset \cdots \supset [a_{n-1}, b_{n-1}  ]\supset [a_{n}, b_{n}  ]\cdots
        \]

        Inoltre \[
            b_{n}-a_{n}=\frac{b_{n-1}-a_{n-1}}{2}\, \forall\,n
        \] allora \[
            b_{n}-a_{n}= \frac{b_0-a_0}{2^{n}}.
        \]

        Si verifica che $ a_{n} \longrightarrow l $, in quanto $ a_{n} $ crescente e limitata superiormente, e $ b_{n} \longrightarrow m $, in quanto $ b_{n} $ è decrescente e limitata inferiormente: allora \[
            \forall\,n:\, a_{n}\le l,\,b_{n}\ge m  
        \] allora \[
            \forall\, n:\, 0\le m-l\le b_{n}-a_{n}=\frac{b_0-a_0}{2^{n}} \xrightarrow{n\to +\infty} 0 \displaystyle  
        \] 
        
        $\implies$ $ m-l=0 $ 
        
        $\implies$ $ m=l $.

        Poniamo $ c=m=l $, e consideriamo $ c $ candidato zero della funzione. Verifichiamo che vale $ f(c)=0 $.

        Infatti,\[
            \forall\,n\quad f(a_{n}) >0\quad f(b_{n} )<0
        \]
        inoltre \[
            \lim_{n\to +\infty} f(a_{n} ) = f(c)
        \]
        perché $ f $ continua e vale il teorema di relazione, e \[
            \lim_{n\to +\infty} f(b_{n} ) = f(c).
        \]
        
        Inoltre \[
            \lim_{n\to +\infty} f(a_{n} ) \ge 0
        \]
        per il teorema di permanenza del segno, e 
        \[
            \lim_{n\to +\infty} f(b_{n} ) \le 0.
        \]

        Risulta quindi che $ \begin{cases}
            f(c)\ge 0 \\
            f(c)\le 0
        \end{cases} $ 
        
        $\implies$ $ f(c)=0 $\qed
    \end{itemize}
}

\osservazione{
    Sotto l'ipotesi $ f $ continua su un intervallo $ [a,b] $, $ c $, lo zero $ c $ non è unico.
}
\osservazione{
    Il teorema vale solo su intervalli
}

\esempio{
    Preso \[
      f(x)=\begin{cases}
        1 & x \in [a,b]  \\
        -1 & x \in [c, d]
      \end{cases}  
    \]
    con $ b\lneqq c $, vale che $ f(a)>0 $, $ f(d)<0 $, $ f $ è continua su $ [a,b]\cup[c,d]=D $, $ \nexists\, c \in D $ tale che $ f(c)=0 $
}

\osservazione{
    Data $ f: [a,b]\to \R $, l'ipotesi di $ f $ continua non è eliminabile
}

\teorema[(dei valori intermedi)]{valinter}{
    Sia $ f:(a,b)\to \R $, $ a,b \in \R^{*} $, continua su $ (a,b) $; indichiamo \[
        i=\inf_{x \in (a,b)}  f(x)\qquad s=\sup_{x \in (a,b)} f(x) 
    \] con $ i, s \in \R^{*} $

    $ \implies $ $ \forall\, \lambda \in (i, s) $, $ \exists\, c \in (a,b) $ tale che $ f(c)= \lambda $
}
\dimostrazione{valinter}{
    Prendiamo $ \lambda \in (i,s) $. \[
        \exists\, x_1,x_2 \in (a,b)\,\tc\, i<\underset{m}{\underbrace{f(x_1)}}<\lambda<\underset{M}{\underbrace{f(x_2)}}<s.
    \]

    Consideriamo $ g(x)= f(x)-\lambda $: $ g $ continua su $ (a,b) $, e $ g(x_1)<0 $ e $ g(x_2)>0 $; inoltre $ x_1, x_2 \in (a,b) $, quindi $ g $ contiua su $ [x_1, x_2] $ oppure $ [x_2, x_1] $. 

    Allora, per il teorema di esistenza degli zeri, si ha che \[
        \exists\, c\text{ tra }x_1, x_2\,\tc\, g(c)=0
    \] ossia \[
        f(c)-\lambda=0 \,\implies\, f(c)=\lambda\qedhere
    \]
}

\corollario{ciaociao}{Sia $ f: (a,b)\to \R $ continua su $ (a,b) $, $ a,b \in \R^{*} $; si indica con \[
    i=\inf_{x \in (a,b)}  f(x)\qquad s=\sup_{x \in (a,b)} f(x) 
\]con $ i, s \in \R^{*} $, si ha che \[
    f\bigl((a,b)\bigr)=(i,s)
\]}

Possiamo dire che $ f $ continua mappa intervalli in intervalli, ovvero \[
    f(I)=J
\] con $ J=(i,s) $, e 
\[
    i=\inf_{x \in I}  f(x)\qquad s=\sup_{x \in I} f(x) 
\]
