\section{Limite successione}\marginnote{2 nov 2021}

Data $ \{a_{n} \}_{n=0}^\infty, \N \to \R, a: n\to a_{n}  $, $l \in \R*$, diciamo che 
\[
    \lim_{n\to\infty}a_{n}=l  
\]
se $\forall V(l) \exists U(+\infty) n \in(\N intersezione D) \implies a_{n \in V(l)}$
Scriviamo $\forall V(l) \exists \overline{n} \in N \,|\, \forall n>\overline{n} a_{n} \in V(l)$

$l\in \R$, diciamo che $ \{a_{n} \}_{n=0}^\infty $ è \textbf{convergente} a $ l $ se $ \forall\, \varepsilon \exists \overline{n} \in \N | \forall n>\overline{n} |a_{n} -l |<\varepsilon  $

Se $ l=\pm \infty $ $a_{n}$ è divergente a $ \pm\infty $, se $ \lim_{n\to+\infty} a_{n}=\nexists $ allora $ \{a_{n}\}_{n=0 }^\infty$ è irregolare (o oscillante).


\esempio{
\begin{itemize}
    \item $ \{a_{n} \}_{n=0}^\infty=(-1)^n $ con $ n \in \N $ è irregolare e limitata
    \item $ \{b_{n} \}_{n=0}^\infty=(-1)^n\cdot n $ con $ n = 0, -1, 2, -3, 4, \cdots$ è irregolare e non limitata
\end{itemize}
}

Si dice di una successione $\{a_{n} \}_{n=0}^\infty$

\begin{itemize}
    \item $\forall \{a_{n} \}$ è crescente se $ \forall n \in \N $, $ a_{n} \le a_{n+1}  $
    \item $\forall\{a_{n} \}$ è strettamente crescente se $ \forall n \in \N $, $ a_{n} < a_{n+1}  $
    \item $\forall\{a_{n} \}$ è decrescente se $ \forall n \in \N $, $ a_{n} \ge a_{n+1}  $
    \item $\forall \{a_{n} \}$ è strettamente decrescente se $ \forall n \in \N $, $ a_{n} > a_{n+1}  $
\end{itemize}
Una successione crescente o decrescente si dice monotona, se strettamente crescente o decrescente si dice strettamente monotona.

Un predicato $ P(n) $ è verificato definitivamente se $ \exists \overline{n} \forall n\le \overline{n}$ $ P(n) $  è vero

Valgono per $ \{a_{n} \}_{n=0}^\infty$ i seguenti teoremi
\begin{itemize}
    \item Teorema di unicità del Limite
    \item Teorema di permanenza del segno 
    \item Teorema di limitatezza:
        \teorema{limitatezza}{
            \[
                \lim_{n\to \infty} a_{n} =l \in \R \,\implies\, \{a_{n} \}_{n=0}^\infty\text{ è convergente e limitata}
            \]
        }
    \item Teorema del confronto
    \item Teorema di esistenza del Limite per successioni definitivamente monotone
        \teorema{esistenzalim}{
            $ \{a_{n} \}_{n=0}^\infty $ è definitivamente crescente 
            
            $\implies$ ammette limite in $ \R* $

            Precisamente se 
            \begin{itemize}
                \item $\{a_{n} \}_{n=0}^\infty$ è definitivamente monotona e limitata 
            
                $\implies$ è convergente
                \item $ \{a_{n} \}_{n=0}^\infty $ è definitivamente monotona e non limitata 
            
                $\implies$ è divergente
            \end{itemize}
        }
\end{itemize}

\teorema[(Principio di Archimede)]{archi}{
    $ \forall a,b \in \R_{+} , a,b>0 $ 
    
    $\implies$ $ \exists n \in \N $ tale che $ na>b $
}
\dimostrazione{archi}{
    Utilizziamo la funzione parte intera: 
    
    $ x \in \R $ si dice $ [x]=\max_{n \in \Z} \{n\le x\} $

    Si verifica che $ \forall x \in \R $, $ [x]< x\le [x] +1 $ 

    Se $ x\ge 0 $, $[x]\ge 0 $, $ [x] \in \R $

    Considerato $ x=\frac{b}{a} $ \[
        \bigg[\frac{b}{a}\bigg]\le \frac{b}{a}<\bigg[\frac{b}{a}\bigg]+1
    \]

    Posto $ \overline{n}=\big[\frac{b}{a}\big] +1 \in \N$ \[
        \frac{b}{a}<\overline{n}\,\implies\, \overline{n} a > b  
    \]
}
Osserviamo che posto $ a=1 $ si ha che $ \forall b \in \R $, $\exists n \in \N$ t.c. $ n>b $

\paragraph{Applicazione del Principio di Archimede} Verifichiamo che \[
    \lim_{n\to \infty}\frac{1}{n}=0 
\]

Fissiamo $ \varepsilon>0 $, vogliamo verificare che definitivamente $ \big|\frac{1}{n}\big|<\varepsilon$

$\iff$ $\frac{1}{n}<\varepsilon$

$ \iff $ $ n>\frac{1}{\varepsilon} $

$ \frac{1}{\varepsilon} \in \R $ allora per il principio di archimede \[
    \exists \overline{n} \in \N, \overline{n}>\frac{1}{\varepsilon}  
\]

Allora $ \forall n \ge \overline{n}  $, $ n>\frac{1}{\varepsilon} $ 

$\implies$ $ \frac{1}{n}\le \varepsilon $ dunque $ \frac{1}{n}<\varepsilon $ definitivamente

Dunque \[
    \lim_{n\to \infty} \frac{1}{n}=0
\]

\teorema[(Disugualiganza di Bernoulli)]{bern}{
    \[
        \forall n \in \N, \forall x \in \R, x>-1
    \] 
    si ha che
    \[
        (1+x)^n \ge 1+nx
    \]
}
\dimostrazione{bern}{
    Dimostrazione per induzione
    \[
        P(n): (1+x)^n\ge 1+nx, x>-1
    \]
    \begin{enumerate}
        \item $ P(0) $ \[
            1+x>0\, (1+x)^0 = 1 =1+n0
        \]
        P(0) è vera 
        \item Assumiamo vera $P(n)$ e verifichiamo $ P(n+1) $ \begin{gather*}
            P(n):\qquad (1+x)^{n}\ge 1+nx\quad \land \quad x>-1\\
            \textcolor{blue}{(1+x) \cdot }(1+x)^{n}\ge 1+nx \textcolor{blue}{ \cdot  (1+x)}\\
            (1+x)^{n+1}\ge (1+nx)(1+x)\\
            (1+nx)(1+x) = 1+nx+x+\underset{\ge 0}{\underbrace{nx^{2}}} \ge 1+(n+1)x
        \end{gather*}
        Dunque $ P(n+1): $ 
        \[
            (1+x)^{n+1}\ge 1+ (n+1)x
        \]
        è verificata
    \end{enumerate}
    Allora per induzione \[
        \forall\, n \in \N\qquad (1+x)^{n}\ge 1+nx \quad\land\quad x>-1\qedhere
    \]
}

\subsection{Limiti}
Progressione geometrica \[
        q \in \R, \lim_{n \to \infty} q^n=?, n \in \N 
    \]
    \begin{itemize}
        \item $q>1, q=1+p$ con $p>0$
            $ q^n = (1+p)^n \ge 1+np$ per la disuguaglianza di Bernoulli

            $ 1+np \to +\infty$ per $ n \to +\infty $

            Per confronto \[
                \lim_{n\to + \infty} q^n=+ \infty
            \]
        \item $ q=1 $ $ q^n = 1 $ $ \forall n $
        \[
            \lim_{n\to + \infty} q^n=1
        \]
        \item $ -1<q<1 $ $ \iff $ $ |q|<1 $ 
        
        $\implies$ $|q|=\frac{1}{1+p}$ con $ p>0 $ \[
            |q^n|=|q|^n=\frac{1}{(1+p)^n} \le \frac{1}{1+np}
        \]

        $ 1+np \to +\infty$ per $ n \to +\infty $ 
        
        $\implies$ $ \frac{1}{1+np} \to 0$

        Per confronto \[
            \lim_{n\to + \infty} |q^n| = 0 \,\implies\,\lim_{n\to + \infty} q^n = 0
        \]
        \item $ q=-1 $
        
        $ q^n $ è irregolare e limitata
        \item $ q<-1 $ \[
            q^n=(-1)^n|q|^n
        \]
        ma $ |q|>1 $ quindi $ |q|^n\to + \infty $ per $ n\to + \infty $, e quindi $ q^n $ è irregolare non limitata
    \end{itemize}
    Riassumendo \[
    q^n\begin{cases}
        \text{divergente a } + \infty &q>1\\
        \text{convergente a } 1 &q=1\\
        \text{convergente a } 0 &|q|<1\\
        \text{irregolare limitata} &q=-1\\
        \text{irregolare non limitata} &q<-1
    \end{cases}
    \]
    \esercizio{
    Posto $ q \in \R $ e \[
        b_{n}=\sum_{k=0}^n q^k
    \]
    calcolare \[
        \lim_{n\to + \infty} b_{n}    \]
    }{
        Da risolvere %TODO risolvere esercizio
    }

\teorema[(di relazione)]{rel}{
    Sia $f:D\to \R$: $ x\to f(x) $, $ x_{0} \in D' $ e $ x_{0} \in \R* $, $ l \in \R* $ 
    
    Allora $ \lim_{x\to x_{0} } f(x) = l $ \referenze{(A)}{\label{vi:A}}
    
    $ \iff $ 
    
    per ogni successione $ a: \{a_{n}\}_{n=0}^\infty $ a valori in $ D\setminus \{x_{0} \} $ 
    
    $a_{n} \xrightarrow{n\to + \infty} x_{0}  \displaystyle $ $\implies$ $f(a_{n}) \xrightarrow{n\to + \infty} l \displaystyle $ \referenze{(B)}{\label{vi:B}}
}
\dimostrazione{rel}{
    \begin{itemize}
        \item [(A) $\implies$ (B)] Sappiamo che $\displaystyle \lim_{x\to x_{0} } f(x) =l $ ovvero \begin{equation}
            \forall\, V(l)\, \exists\, U(x_{0} ) \,|\, x \in U\, \land \,x \in V \,\land\, x \neq x_{0} \,\implies\, f(x) \in V(l)\label{eq1}
        \end{equation}

        Consideriamo $ \{a_{n} \}_{n=0}^\infty $ con $ a_{n} \xrightarrow{n\to + \infty} x_{0}  \displaystyle  $ con $ a_{n} \in D $ e $ a_{n}\neq x_{0} $ ossia \[
            \exists\, \overline{n}  \in \N \,|\,\forall\, n \ge \overline{n},\, a_{n} \in D\,\land\, a_{n} \neq x_{0} \,\land\, a_{n} \in U(x_{0})
        \]
        allora \begin{equation}f(a_{n} ) \in V(l)\label{eq2}\end{equation}
        
        Concludendo, unendo \eqref{eq1} e \eqref{eq2}: \[
            \forall V(l) \exists \overline{n} \in \N | \forall n > \overline{n} f(a_{n} ) \in V(l)
        \]
        ossia
        \[
            \lim_{n\to + \infty} f(a_{n} ) =l
        \]
        
        \item [(B) $\implies$ (A)] Procediamo per assurdo: verificando $ \neg A \implies \neg B $
        
        $ \neg B $: esiste una successione $ \{a_{n} \}_{n=0}^\infty$ tale che $ a_{n} \in D\setminus \{x_{0} \} $ per cui $ a_{n} \xrightarrow{n\to + \infty} x_0  \displaystyle $ con $ a_{n}\neq x_0  $, e $ f(a_{n}) \nrightarrow l $

        Abbiamo ipotizzato $\neg$\hyperref[vi:A]{(A)}, ossia \[
            \lim_{x\to x_0} f(x) \neq l
        \] ossia \[
            \exists\, V(l)\,|\, \forall\, U(x_0)\, \exists\, x \in U \,\land\, x \in D\,\land\, x \neq x_0\,\tc\, f(x)\notin V(l)
        \]

        Ci poniamo nel caso particolare $ x_0 \in \R $ (il caso $ x_0=\pm \infty $ funziona analogamente).
        
        $\neg$\hyperref[vi:A]{(A)} $ \,\implies\,  $
        \[
            \exists\, V(l)\,|\, \forall\, \delta >0\, \exists\, x\,|\, 0<|x-x_0|<\delta \,\land\, x \in D \,\land\, f(x)\notin V(l)
        \]

        Consideriamo $\delta=1$ $ \exists\, x_{1}  \, 0<|x-x_{0} |<1 $ $\land$ $ f(x_1) \notin V(l) $
        
        Consideriamo $ \delta=\frac{1}{2} $ $ \exists\, x_{2}  \, 0<|x_{2} -x_{0} |<1 $ $\land$ $ f(x_2) \notin V(l) $

        \dots

        Consideriamo $ \delta=\frac{1}{n} $ $ \exists\, x_{n}  \, 0<|x_{n} -x_{0} |<1 $ $\land$ $ f(x_n) \notin V(l) $

        \dots

        Allora abbiamo costruito una successione $ \{x_{n} \}_{n=0}^\infty $ tale che $ x_{n} \in D $, $ x_{n}\neq x_0  $ e $ f(x_{n})\notin V(l) $ 

        inoltre $ \forall \varepsilon >0 \exists \overline{n} | \forall n>\overline{n} 0<|x_{n}-x_{0}|<\varepsilon $ ($\overline{n}>\frac{1}{\varepsilon}$)

        ossia $ x_{n} \xrightarrow{n\to + \infty} x_{0}  \displaystyle  $

        Abbiamo costruto una successione $ \{x_{n} \}_{n=0}^\infty $ con $ x_{n}\to x_0  $, $ x_{n} \neq x_0 $ e $ \lim_{n\to + \infty} f(x_{n} ) =l $ 

        ossia abbiamo ottenuto che $ \neg B $ è vera \qedhere
    \end{itemize}
}

\subsection{Confronti tra infiniti}
\begin{enumerate}
    \item Dati $ a>1 $ e $ n \in \N $ osserviamo che 
    \begin{multline*}
        0\le \frac{\sqrt{n}}{a^n}=\frac{\sqrt{n}}{(1+h)^n}\le\\
        \le \frac{\sqrt{n}}{1+hn}\le \frac{\sqrt{n}}{hn}=\\
        =\frac{1}{h}\cdot\frac{1}{\sqrt{n}}
    \end{multline*}
    e $ \frac{1}{\sqrt{n}} \xrightarrow{n\to + \infty} 0 \displaystyle $ allora per confronto \[
        \lim_{n\to +\infty} \frac{\sqrt{n}}{a^n} =0
    \]
    ovvero \[
        \sqrt{n}=o(a^n)_{n\to +\infty}
    \]
    \item Dato $ a>1 $
    \[
        0\le \frac{n}{a^n}=\bigg(\frac{\sqrt{n}}{(\sqrt{a})^n}\bigg)^2
    \]
    ma $ \frac{\sqrt{n}}{(\sqrt{a})^n} \xrightarrow{n\to + \infty} 0 \displaystyle $

    Otteniamo che \[
        \lim_{n\to + \infty} \frac{n}{a^n} =0
    \]
    ovvero \[
        n=o(a^n)_{n\to + \infty}
    \]
    \item Dato $ k \in \N\setminus \{0,1\} $
    \[
        0\le \frac{n^k}{a^n}=\bigg(\frac{n}{(\sqrt[k]{a})^n}\bigg)^k
    \]
    ma $ \frac{n}{(\sqrt[k]{a})^n} \xrightarrow{n\to + \infty} 0 \displaystyle $

    Dato che $ a>1 $ e $ \sqrt[k]{a}>1 $ concludiamo che \[
        \lim_{n\to + \infty} \frac{n^k}{a^n} = 0 
    \]
    ovvero \[
        n^k = o(a^n)_{n\to + \infty}
    \]
    \item %TODO manca un pezzo: https://math.i-learn.unito.it/pluginfile.php/160452/mod_resource/content/2/UNO_lezione13_2020_21.pdf
\end{enumerate}