\subsection{Derivate di funzioni elementari}

\begin{itemize}
    \item Sia $ f(x)=x^{\alpha} $\marginnote{30 nov 2021}, $ x \in (0,+ \infty) $, $ \alpha \in \R $. \begin{multline*}
        \lim_{h\to 0} \frac{f(x+h)-f(x)}{h} = \lim_{h\to 0} \frac{(x+h)^{\alpha}-x^{\alpha}}{h} =\\
        = \lim_{h\to 0} \frac{x^{\alpha}(1+h/x)^{\alpha}-x^{\alpha}}{h} = x^{\alpha}\lim_{h\to 0} \frac{(1+h/x)^{\alpha}-1}{(h/x) \cdot x} = \\
        x^{\alpha-1} \lim_{h\to 0} \frac{(1+t)^{\alpha}-1}{t} = \alpha x^{\alpha-1}
    \end{multline*}
    
    Abbiamo effettuato una sostituzione: $ t=h/x $, mentre l'ultimo è un limite notevole % TODO aggiungere nella formula
    \item Sia $ f(x)=e^{x} $, $ x \in \R $\begin{multline*}
        \lim_{h\to 0} \frac{f(x+h)-f(x)}{h} = \\
        \lim_{h\to 0} \frac{e^{x+h}-e^{x}}{h} = e^{x} \lim_{h\to 0} \frac{e^{h}-1}{h} = \\
        = e^{x}
    \end{multline*}

    Vale quindi $ \forall\, x \in \R $, \begin{equation}
        D(e^{x})=e^{x}
    \end{equation}

    Di verifica inoltre che \begin{equation}
        a>0\qquad D(a^{x})=a^{x}\ln x
    \end{equation}
    \item Sia $ f(x)=\ln x $, $ x>0 $ \begin{multline*}
        \lim_{h\to 0} \frac{f(x+h)-f(x)}{h} = \lim_{h\to 0} \frac{\ln (x+h)-\ln x}{h} =\\
        \lim_{h\to 0} \frac{\ln [x(1+h/x)]-\ln x}{h}=\lim_{h\to 0} \frac{\cancel{\ln x} + \ln(1+h/x)]-\cancel{\ln x}}{x(h/x)} =\\
        \frac{1}{x} \lim_{h\to 0} \frac{\ln (1+t)}{t} = \frac{1}{x}
    \end{multline*}

    Sostituendo $ t=h/x $, e ricordando che $ \lim_{h\to 0} \frac{\ln (1+t)}{t}=1 $ % TODO aggiungere queste cose in display
    \item Sia $ f(x)=\sin x $, $ x \in \R $.\begin{multline}
        \lim_{h\to 0} \frac{\sin(h+x)-\sin(x)}{h} = \lim_{h\to 0} \frac{\sin x \,\cos h + \sin h\,\cos x-\sin(x)}{h} =\\
        = \lim_{h\to 0} \biggl( \sin x \frac{\cos h -1}{h} + \cos x \frac{\sin h}{h} \biggr)=\\
        = \cos x
    \end{multline}

    % TODO manca scrivere i limiti notevoli

    %TODO sistemare in numeri equazione
\end{itemize}

\subsection{Prima formula dell'incremento finito}

\osservazione{
    Sia $ f:I\to \R $, $ x \in I  $, assumiamo $ f $ derivabile in $ x $.
    \begin{align*}
        \iff &\lim_{ h\to 0} \frac{f(x+h)-f(x)}{h} = f'(x)\\
        \iff &\lim_{ h\to 0} \frac{f(x+h)-f(x)}{h} - f'(x) = 0
    \end{align*}

    Poniamo \begin{equation}
        \varepsilon(h) =\frac{f(x+h)-f(x)}{h} - f'(x)
    \end{equation}
    Vale $ \dom \varepsilon(h)=\dom f \setminus \{0\} $. Notiamo che $ \lim_{h\to 0} \varepsilon(h) = 0$, quindi $ \varepsilon(h)  $ è estendibile per continuità in $ h=0 $.

    Possiamo scrivere che $ f $ è derivabile in $ x $

    $ \iff $ $ \exists\, \varepsilon(h) $ continua in $ I(0) $, con \[
        \lim_{h\to 0} \varepsilon(h) =0
    \] tale che \begin{equation}
        f(x+h)-f(x)=f'(x)\,h + h\, \varepsilon(h)
    \end{equation}

    Questa è la prima formula dell'\textit{incremento finito}.

    Equivalentemente, poiché $ \lim_{h\to 0} \bigl(h\, \varepsilon(h)\bigr)/h = 0 $ si ha \begin{equation}
        f(x+h)-f(x)=f'(x)\,h+o(h)
    \end{equation}
}

\teorema[(di derivazione delle funzioni composte)]{diderfunccomi}{
    Consideriamo $ f:D\to \R $, $ D \subseteq \R $, e consideriamo $ g:E\to \R $, con $ f(D) \subseteq E $.
    Assumiamo $ f $ derivabile in $ x \in D $, e $ g $ derivabile in $ y=f(x) \in E $.

    Allora $ w=g\circ f $ è derivabile in $ x $ e vale \begin{equation}
        w'(x)=g'(f(x))\, f'(x)\label{chain rule}
    \end{equation}

    La \eqref{chain rule} prende il nome di \textit{chain rule}.
}
\dimostrazione{diderfunccomi}{
    $ f $ derivabile in $ x $, poniamo $ y = f(x) $, e assumiamo $ g $ derivabile in $ y $. \begin{equation}
        \exists\, \varepsilon(k) \xrightarrow{k\to 0} 0 \displaystyle\,|\, g(y+k)-g(y)=g'(y)\,k+k\,\varepsilon(k) \label{cicio}
    \end{equation}

    Poniamo $ k =f(x+h) - f(x) \xrightarrow{h\to 0} 0 \displaystyle $, si ha allora \[
        k=f(x+h)-y 
    \] e, poiché $ f $ continua in $ x $ \[
        y+k=f(x+h)
    \]

    Sostituiamo in \eqref{cicio} \[
        g  \bigl(f(x+h) \bigr)-g \bigl(f(x) \bigr) = g'(y)\,k+k\,\varepsilon_{k} 
    \] dunque, sostituendo e dividendo per $ h$
    \begin{multline*}
        \frac{g \bigl(f(x+h) \bigr)-g \bigl(f(x)\bigr)}{h}=\\= g'\bigl(f(x)\bigr)\,\parentesi{ \xrightarrow[h\to 0]{} f'(x) }{\frac{\bigl(f(x+h)-f(x)\bigr)}{h}}+ \parentesi{ \xrightarrow[h\to 0]{} f'(x) \in \R }{\frac{\bigl(f(x+h)-f(x)\bigr)}{h}}\, \parentesi{ \xrightarrow[h\to 0]{} 0 }{\varepsilon(k) }
    \end{multline*} allora \begin{equation*}
        \lim_{h\to 0} \frac{g \bigl(f(x+h)\bigr)-g\bigl(f(x)\bigr)}{h} = g'(f(x))\, f'(x)\qedhere
    \end{equation*}
}

%TODO manca l'iterazione del processo

\esempio{}{
    Consideriamo $ f(x)= \ln(\sin x) $, con $ x \in (0,\pi) $ \[
        f'(x)=\frac{1}{\sin x} \cos x =\frac{\cos x}{\sin x}
    \]

    In generale, se $ f(x)=\ln g(x) $, $ x \in \dom f$, $g(x)>0 $, vale \[
        f'(x)=\frac{1}{g(x)}\,g'(x)=\frac{g'(x)}{g(x)}
    \]
}

\esercizio{
    Calcolare \[
        D(\ln |x|)
    \]
}{
   Da svolgere % TODO risolvere 
}{}

\teorema[(di derivazione della funzione inversa)]{dderfuncinv}{
    Sia \[f:I\to \R\] $ I $ intervallo, $ f $ invertibile su $ I $ (strettamente monotona su $ I $), indichiamo $ J=f(I) $, e assumiamo $ f $ derivabile in $ x_0 \in I $, e $ f'(x_0)\neq 0 $, allora posto $ y_0=f(x_0) $, si ha che la funzinoe inversa \[
        f^{-1}:J\to I   
    \] è derivabile in $ y_0 $ e vale \begin{equation}
        (f^{-1})'\,(y_0)=\frac{1}{f'(x_0)}.
    \end{equation}
}

\esempio{}{Sia $ f(x)=x^{3} $, $ f'(0)=0 $ \begin{align*}
    f^{-1}(0)& =0\\
    f^{-1}(y) &=\sqrt[3]{y}
\end{align*} $ f^{-1}(y) $ è derivabile per qualsiasi $ y \neq 0 $. Osserivamo che in $ y=0 $ la tangente a $ y=\sqrt[3]{x} $ è verticale, dunque $ \sqrt[3]{x} $ non è derivabile in $ 0 $.
%TODO aggiugnere grafico di f e f^{-1}
}

\paragraph{Applicazione} Sia $ f(x)=\tan x $, $ x \in \bigl(-\frac{ \pi}{2}, \frac{ \pi}{2}\bigr) $, e $ f^{-1}=\arctan x $ \begin{multline*}
    f'(x)=D(\tan x)=D \biggl(\frac{\sin x}{\cos x}\biggr)=\\=\frac{\cos^{2}x+\sin^{2}x}{\cos^{2}x}=\\= 1+\tan^{2}x>0
\end{multline*}

Poniamo $ y=\tan x $, $ f^{-1} $ derivabile in $ y $ \[
    \bigl(f^{-1}\bigr)'(y)=\frac{1}{f'(x)}=\frac{1}{1+\tan^{2}x}=\frac{1}{1+y^{2}}
\]
Dunque \begin{equation}
    D(\arctan x) = \frac{1}{1+x^{2}}
\end{equation}% TODO aggiungere grafici tan e arctan


\paragraph{Applicazione} Sia $ f(x) = \sin x $, $ x \in \bigl[-\frac{ \pi}{2}, \frac{ \pi}{2}\bigr]$. Consideriamo $ f^{-1} = \arcsin x$

%TODO aggiungere grafici sin e arcsin

\[
    f'(x)=D(\sin x)=\cos x \neq 0 \text{ per } x \neq \pm\frac{ \pi}{2}
\]
Posto $ y=f(x)=\sin x $, $ \arcsin y $ è derivabile per $ y\neq \pm 1 = f\bigl(\pm\frac{ \pi}{2}\bigr)$ \[
    \bigl(f^{-1}\bigr)'(y)=\frac{1}{f'(x)}=\frac{1}{\cos x}
\]

Osserviamo che \begin{gather*}
    \cos^{2}x + \sin^{2}x )=1\\
    \cos^{2}x =1- \sin^{2}x )\\
    \cos x =\pm \sqrt{1- \sin^{2}x )}\\
    x \in \biggl[-\frac{\pi}{2},\frac{\pi}{2}\biggr]\quad \cos x =+\sqrt{1- \sin^{2}x )}
\end{gather*}

Quindi
\[
    \bigl(f^{-1}\bigr)'(y)=\frac{1}{\cos x}=\frac{1}{\sqrt{1- \sin^{2}x )}}=\frac{1}{\sqrt[]{1-y^{2}}}.
\] 

Allora \begin{equation}
    \forall\,x \in (-1, 1)\quad D(\arcsin x)=\frac{1}{\sqrt[]{1-x^{2}}}
\end{equation}

Inoltre \begin{equation}
    \forall\,x \in (-1, 1)\quad D(\arccos x)=\frac{1}{-\:\sqrt[]{1-x^{2}}}
\end{equation}

\subsection{Studio dei punti di dubbia derivabilità}

Sia $ f:I\to \R $, assumiamo $ f $ derivabile in $ I\setminus \{x_0\} $

L'obiettivo è studiare la derivabilità in $ x_0 $

\teorema[(di Darboux)]{darbouxdi}{
    Sia $ f: I\to \R $, , assumiamo $ f $ derivabile in $ I\setminus \{x_0\} $, $ f $ continua in $ x_0 $ e esistano \begin{align*}
        \lim_{x\to x_0^{+}} f'(x) &= l \in \R^{*}\\
        \lim_{x\to x_0^{-}} f'(x) &= m \in \R^{*}
    \end{align*}

    Allora \begin{enumerate}
        \item se $ l=m \in \R $ 
        
        $\implies$ $ f $ è derivabile in $ x_0 $ e $ f'(x_0)=l=m $;
        \item se $ l, m \in \R $, $ l\neq m $ 
        
        $\implies$ $ f $ è derivabile da destra e sinistra in $ x_0 $ e si ha \[
            f_{+}'(x)=l \qquad f_{-}'(x)=m;
        \]

        \item se anche solo uno tra $ m $ e $ l $ è $ \pm \infty $;
        
        $\implies$ $ f $ non è derivabile in $ x_0 $.
    \end{enumerate}
}
Se \[
    \lim_{x\to x_0} f'(x_0) = \nexists
\] (al di fuori dei casi precedenti) l'esistenza di $ f'(x_0) $ varia da caso a caso.

\esempi{}{
    \begin{enumerate}
    \item Sia $ f(x)=\begin{cases}
        x \,\sin\frac{1}{x} & x\neq 0\\
        0 & x =0
    \end{cases} $, $ f $ continua in $ 0 $
    \[
        x\neq 0\qquad f'(x)=\sin \frac{1}{x}+x\,\cos x\,\frac{1}{x^{2}}=\parentesi{\footnotemark}{\sin \frac{1}{x}}+\parentesi{\footnotemark}{\cos x\,\frac{1}{x}}
    \]
    \addtocounter{footnote}{-1}
    \footnotetext{oscilla tra $ \pm 1 $ per $ x\to 0 $}
    \stepcounter{footnote}
    \footnotetext{oscilla tra $ \pm \infty $ per $ x\to 0 $}

    % TODO aggiungere limite rapporto incrementale
    \item Sia $ g(x)=\begin{cases}
        x^{2}\,\sin \frac{1}{x} & x \neq 0\\
        0 & x=0
    \end{cases} $, $ g $ continua in $ 0 $.
    \[
        x\neq 0\qquad g'(x)=\parentesi{\xrightarrow[x\to 0]{} 0}{2x \sin \frac{1}{x}}-\parentesi{\footnotemark}{\frac{\cancel{x^{2}}}{\cancel{x^{2}}}}
    \]
    \footnotetext{oscilla tra $ \pm 1 $ per $ x\to 0 $}

    % TODO manca il limite del rapporto incrementale
    % TODO manca la fine dell'esercizio
\end{enumerate}
}

\esercizio{
    Studiare la derivabilità in $ x_0=0 $ di \[
        f(x)=\begin{cases}
            \sin x & x\ge 0\\
            \alpha x & x<0
        \end{cases}
    \] al variare di $ \alpha \in \R $
}{
    Da risolvere % TODO risolvere esercizio
}{}