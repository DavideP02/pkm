\subsection{Limite superiore e limite inferiore}
\definizione{}{
    Sia\marginnote{20 dic 2021} $ \{a_{k} \}_{k=0}^\infty $ successione a valori reali, \begin{equation}
        \Lambda = \{l \in \R^{*}\,|\: \exists\, a_{k_{n} } \longrightarrow l\}
    \end{equation}
    Si dice $ l $ valore limite di $ \{a_{k} \}_{k=0}^\infty $, e $ \Lambda $ classe limite.
}

\osservazione{
    \begin{enumerate}
        \item $ a_{n} \xrightarrow[n\to +\infty]{} \lambda \in \R^{*} $ 
        
        $\implies$ $ \forall\,a_{k_{n} } \longrightarrow \lambda $ 
        
        $\implies$ $ \Lambda=\{\lambda\} $.
        \item $ \Lambda\neq\emptyset $, infatti preso $ A=\{a_{n} \} $ insieme dei valori della successione $ \{a_{k} \}_{k=0}^\infty $, può avvenire che:
        \begin{itemize}
            \item [(\textit{i})] $ A $ è finito, e almeno uno dei suoi elementi $ \lambda $ è assunto infinite volte dalla successione. Allora \[
                \exists\,a_{n_{k} } \equiv \lambda\qquad a_{k_{n} } \xrightarrow[n\to + \infty]{} \lambda \,\implies\, \lambda \in \Lambda 
            \]
            \item [(\textit{ii})] $ A $ è infinito, allora per il teorema di Bolzano Weierstrass ammette almeno un punto di accumulazione $ \lambda \in \R^{*} $: allora \[
                \exists\, a_{k_{n} } \longrightarrow \lambda \,\implies\, \lambda \in \Lambda
            \]
        \end{itemize}
        \item $ \Lambda $ è chiusa in $ \R^{*} $. 
        
        Ricordiamo che $ X $ è chiuso $ \iff $ $ X' \subseteq X $.

        Sia $ l \in \R^{*} $ punto di accumulazione di $ \Lambda $ ($l \in \Lambda'$), ovvero \[
            \forall\, \varepsilon>0\quad \exists\, \lambda \in \Lambda,\: \lambda\neq l\:\tc\quad \lambda \in (l- \varepsilon; l+ \varepsilon)
        \]

        $\lambda \in \Lambda$, $ \lambda $ è punto limite \begin{multline*}
            \exists\, a_{n_{k} } \longrightarrow \lambda \,\implies\\\implies\, \forall\,\eta\quad (\lambda-\eta; \lambda+\eta)\text{ contiene infiniti punti di } \{a_{k} \}_{k=0}^\infty
        \end{multline*}
        Sia $ \eta $ tale che $ (\lambda-\eta; \lambda+\eta) \subseteq (l- \varepsilon; l+ \varepsilon) $ 
        
        $\implies$ $ \forall\, \varepsilon>0 $, $ (l- \varepsilon; l+ \varepsilon) $ contiene infiniti punti di $ \{a_{k} \}_{k=0}^\infty $ 
        
        $\implies$ $ \exists $ $ a_{k_{n} }\longrightarrow l $ 
        
        $\implies$ $ l \in \Lambda $ 
        
        $\implies$ $ \Lambda' \subseteq\Lambda $ 
        
        $\implies$ $ \Lambda $ è chiuso.
    \end{enumerate}
}

Grazie alle osservazioni $ 2.$ $ 3. $ si ha che $ \Lambda $ ammette in $ \R^{*} $ massimo $ M $ e minimo $ m $ (con $ m,M \in [- \infty, + \infty] $).

Indichiamo con \begin{align*}
    \limsup_{n\to + \infty} a_{n} &= M =\max_{ \in \R^{*}} \Lambda\\
    \liminf_{n\to + \infty} a_{n} &= m =\min_{ \in \R^{*}} \Lambda
\end{align*}
Altre notazioni sono \[
    \limsup_{n\to + \infty} a_{n} = \varlimsup_{n\to \infty} a_{n}\qquad \liminf_{n\to + \infty} a_{n} = \varliminf_{n\to \infty} a_{n} 
\]

Ne consegue che $ \{a_{k} \}_{k=0}^\infty $ può non ammettere limite in $ \R^{*} $, ma ammette sempre $ \liminf $ e $ \limsup $.

\esempio{
    Sia $ a_{n}=q^{n}$: allora \begin{itemize}
        \item se $ |q|<0 $ allora $ a_{n}\longrightarrow 0  $ 
        
        $\implies$ $ \Lambda=\{0\} $, e \[
            \limsup a_{n} =\liminf a_{n} =0 
        \]
        \item se $ q=1 $, $ a_{n} \equiv 1 $, $ \Lambda=\{1\} $, e \[
            \limsup a_{n} =\liminf a_{n} = 1 
        \]
        \item se $ q=-1 $, $ a_{n}=1,-1, 1, -1, \cdots$: \[
            \exists\, a_{k_{n} }\longrightarrow 1\qquad \exists\, a_{k_{n} } \longrightarrow -1 
        \] quindi $ \Lambda=\{+1, -1\} $. Inoltre \[
            \lim_{n\to + \infty} a_{n}  = \nexists\qquad \begin{aligned}
                \limsup a_{n} &=1\\
                \liminf a_{n} &=-1  
            \end{aligned}
        \]
        \item se $ q>1 $, $ \lim_{n\to + \infty} =+ \infty $, $ \Lambda =\{+ \infty\} $, e \[
            \limsup a_{n} =\liminf a_{n} = + \infty
        \]
        \item se $ q<-1 $, $ \Lambda=\{+ \infty, -\infty\} $ e \[
            \limsup a_{n} = + \infty\qquad \liminf a_{n} = - \infty
        \]
    \end{itemize}
}

\osservazione{
    In generale vale che \[
        \lim_{n\to + \infty} a_{n}  =l \in \R^{*} \,\iff\, \liminf_{n\to + \infty} a_{n} = \limsup_{n\to +\infty} a_{n} =l 
    \]  
}

\esempio{
    Sia $ \displaystyle a_{n}=\sin\left(n\frac{\pi}{2}\right)$, $ a_{n} $ assume solo i valori $ \{-1, 0, 1\} $. Si hanno le seguenti sottosuccessioni: \begin{align*}
        a_{2n}=b_{n}&= \sin\left(2n\frac{\pi}{2}\right)\equiv 0\\
        a_{4n+1}=c_{n}&= \sin\left((4n+1)\frac{\pi}{2}\right)\equiv 1\\
        a_{4n+3}=d_{n}&= \sin\left((4n+3)\frac{\pi}{2}\right)\equiv -1 
    \end{align*} 

    $ \implies $ $ \Lambda =\{-1, 0, 1\} $, e \[
        \liminf a_{n} = -1\qquad \limsup a_{n} = +1   
    \]
}

\esempio{
    Sia $\displaystyle a_{n}=n\,\sin\left(n\frac{\pi}{2}\right)  $: \begin{align*}
        a_{2n}&= 2n\, \sin\left(2n\frac{\pi}{2}\right) =0\\
        a_{4n+1}&= (4n+1)\,\sin\left((4n+1)\frac{\pi}{2}\right) = 1, 5, 9, 13,\cdots\\
        a_{4n+3}&= (4n+3)\,\sin\left((4n+3)\frac{\pi}{2}\right) = -3, -7, -11,\cdots
    \end{align*}
    Vale che \[
        a_{2n} \longrightarrow  0\qquad a_{4n+1} \longrightarrow + \infty\qquad a_{4n+3}\longrightarrow - \infty
    \] quindi $ \Lambda=\{- \infty, 0, \infty\} $ e \[
        \liminf a_{n} = - \infty\qquad \limsup a_{n} = + \infty  
    \]
}

\proprieta{}{
    Vale sempre che \[
        \liminf a_{n} \le \limsup a_{n}  
    \]

    Inoltre se $ \{a_{k} \}_{k=0}^\infty $ è limitata superiormente (inferiormente), allora 
    \begin{align*}
        \exists\, a_{k_{n}} &\longrightarrow \limsup a_{n} = M \in \R\\  
        \Bigl(\exists\, a_{k_{n}} &\longrightarrow \liminf a_{n} = m \in \R\Bigr)
    \end{align*}
}   