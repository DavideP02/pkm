\section{Teoremi per la risoluzione di limiti}

\subsection{Teorema di De L'H\^opital}

\teorema[(di De L'H\^opital)]{hopital}{
    Siano\marginnote{14 dic 2021} $ - \infty\le a < b \le + \infty $ e $ f, g: (a,b)\to \R $, che soddisfano le seguenti condizioni:
    \begin{itemize}
        \item [(\textit{i})] $ \displaystyle \lim_{x\to a^{+}} f(x) = \lim_{x\to a^{+}} g(x) = 0$ oppure $ + \infty $, $ - \infty $, $ \infty $;
        \item [(\textit{ii})] $ f, g$ derivabili in $ (a,b) $ e $ g'(x)\neq 0 $, $ \forall\, x \in (a,b) $
        \item [(\textit{iii})] $ \displaystyle \lim_{x\to a^{+}} \frac{f'(x)}{g'(x)} = L \in R^{*} $
    \end{itemize}

    Allora \begin{equation}
        \lim_{x\to a^{+}} \frac{f(x)}{g(x)} = L
    \end{equation}
}
  
\osservazione{
    Lo stesso risultato vale per \[
        \lim_{x\to b^{-}} \frac{f(x)}{g(x)}
    \] nonché per \[
        \lim_{x\to x_0} \frac{f(x)}{g(x)}
    \] con $ x_0 \in (a,b) $
}

\osservazione{
    Dimostriamo il teorema per un caso particolare, ovvero \[
        \lim_{x\to a^{+}} f(x) = \lim_{x\to a^{+}} g(x) = 0\qquad f(a)=g(a)=0
    \]

    Si ha che $ \forall\, x \in (a,b) $ esiste $ y \in (a,x) $ tale che \[
        f(x)\,g'(y) = [f(x)-f(a)]\,g'(y)=\\
        = [g(x)-g(a)]\,f'(y)=g(x)\,f'(y)
    \] per il teorema di Cauchy applicato a $ (a,x) $. Segue che \[
        \frac{f(x)}{g(x)}=\frac{f'(y)}{g'(y)}
    \]

    Inoltre, vale $ y=y(x) $ e $ a<y(x)<x $, quindi $ \displaystyle \lim_{x\to a^{+}} y(x) = a $, e, per il teorema sui limiti di funzioni composte \[
        \lim_{x\to a^{+}} \frac{f'(y)}{g'(y)} = \lim_{x\to a^{+}} \frac{f'(y\bigl(x)\bigr)}{g'\bigl(y(x)\bigr)}=L
    \]
    di conseguenza \[
        \lim_{x\to a^{+}} \frac{f(x)}{g(x)} = \lim_{x\to a^{+}} \frac{f'(x)}{g'(x)} = L\qedd
    \]
}

\esempi[(alcuni limiti notevoli)]{
    \begin{itemize}
        \item $ \forall\, \alpha>1 $ vale $ \displaystyle \lim_{x\to +\infty} \frac{x}{\alpha^{x}} = 0 $, infatti \[
            \lim_{x\to + \infty} \frac{x}{\alpha^{x}} \overset{H}{=} \lim_{x\to + \infty} \frac{1}{\alpha^{x}\,\ln\alpha} = 0
        \]
        \item segue che $ \displaystyle \lim_{x\to + \infty} \frac{x^{b}}{a^{x}} = 0 $, $ \forall\, a>1, b>0 $, infatti \[
            \frac{x^{b}}{a^{x}}=\Bigl(\frac{x}{\alpha^{x}}\Bigr)^{b}\qquad\text{con }\alpha=a^{1/b}>1
        \]
        \item $ \forall\, a>1, b>0 $ vale $ \displaystyle \lim_{x\to + \infty} \frac{\log_{a} x }{x^{b}} = 0 $, infatti sia $ t=\log_{a}x  $: \begin{align*} x&=a^{t},\\ x^{b}&=(a^{b})^{t},\\ a^{b}&>1\end{align*} pertanto \[
            \lim_{x\to + \infty} \frac{\log_{a} x }{x^{b}} = \lim_{t\to + \infty} \frac{t}{(a^{b})^{t}} = 0
        \]
        \item segue dal caso precedente che $ \forall\,b>0,\: a>1 $ vale $ \displaystyle\lim_{x\to 0^{+}} x^{b}\log_{a} x  = 0$. Infatti, posto $ t=1/x $ \[
            \lim_{x\to 0^{+}} x^{b}\log_{a} x  = \lim_{t\to + \infty} \frac{-\log_{a}t }{t^{b}} = 0
        \]
    \end{itemize}
}

Il teorema di De L'H\^opital si può applicare più e più volte, all'occorrenza. 
\esempio{
    \begin{align*}
        \lim_{x\to 0} \frac{x-\sin x}{x^{3}} &= \left[\frac{0}{0}\right]\\
        &\overset{H}{=} \lim_{x\to 0} \frac{1-\cos x}{3 x^{2}} \overset{H}{=} \lim_{x\to 0} \frac{\sin x}{6}
        = \frac{1}{6} \lim_{x\to 0} \frac{\sin x}{x} = \frac{1}{6}
    \end{align*}
}

\attenzione{
    Il teorema di De L'H\^opital non si può utilizzare sempre, ma bisogna farlo con la testa.
    \begin{itemize}
        \item È bene sottolineare che il teorema valga solo per le forme indeterminate, infatti \[
            \lim_{x\to 0^{+}} \frac{x+1}{x} = + \infty
        \] ma, applicando De l'H\^opital \[
            \lim_{x\to 0^{+}} \frac{x+1}{x} \overset{H}{=} \lim_{x\to 0^{+}} \frac{1}{1} = 1
        \]
        Ovviamente il risultato corretto è il primo.
        \item Se il limite del rapporto delle derivate non esiste non è possibile affermare che il limite iniziale non esista (ovvero il teorema in questione è una condizione sufficiente, ma non necessaria). Per esempio 
        \begin{align*}
            \lim_{x\to + \infty} \frac{2x+\sin x}{2x-\sin x} &\overset{H}{=} \lim_{x\to + \infty} \frac{2+ \cos x}{2-\cos x} = \nexists\\
            &= \lim_{x\to + \infty} \frac{2+\frac{\sin x}{x}}{2- \frac{\sin x}{x}} = \frac{2}{2} = 1
        \end{align*}
        In questo caso il valore del limite è ``1''.
        \item A volte, applicando il teorema di De l'H\^opital subito, si ``complicano'' solo le cose \[
            \lim_{x\to 0^{+}} \frac{x}{e^{-1/x}} \overset{H}{=} \lim_{x\to 0^{+}} \frac{x^{2}}{e^{-1/x}} = \cdots\]
            Invece, applicandolo dopo:\[
            \text{sia } t=1/x\qquad \lim_{t\to + \infty} \frac{1/t}{e^{-t}} = \lim_{t\to + \infty} \frac{e^{t}}{t}\overset{H}{=} \lim_{t\to + \infty} \frac{e^{t}}{1}=+ \infty
        \]
    \end{itemize}
}

\subsection{Formula di Taylor}

\definizione{}{
    Sia $ f:(a,b)\to \R $, derivabile $ n $ volte. Sia $ x_0 \in (a,b)$. Si dice \textit{polinomio di Taylor} di grado $ n $, con centro in $ x_0 $: \begin{equation}
        T_{n}(x)=\sum_{k=0}^{n} \frac{f^{(k)}(x_0)}{k!}(x-x_0)^{k}  
    \end{equation} 
    ovvero\footnote{per convenzione $f^{(0)}(x)=f(x)$} 
    \[
        T_{n}(x)=f(x_0)+f'(x_0)(x-x_0)+\frac{1}{2}\,f''(x_0)(x-x_0)^{2}+\cdots+\frac{1}{n!}f^{(n)}(x_0)(x-x_0)^{n}
    \]
}

\teorema{delcavolo}{
    Sia $ f:(a,b)\to \R $, derivabile $ n $ volte in $ x_0 \in (a,b) $. Allora se $ T_{n}(x)$ è il polinomio di Taylor di grado $ n $ generato da $ f $ con centro in $ x_0 $, posto $ E_{n}(x)=f(x)-T_{n}(x)$ si ha:
    \begin{itemize}
        \item [(\textit{a})] Per ogni $ n\ge 1 $
        \begin{equation}
            E_{n}(x)=o\bigl((x-x_0)^{n}\bigr)_{x\to x_0}\qquad\text{(formula di Peano)}
        \end{equation}
        \item [(\textit{b})] Se $ f $ è derivabile $ (n+1) $ volte in $ (a,b) $, $ n\ge 0 $, per ogni $ x \in (a,b) $ esiste $ x $ compreso tra $ x_0 $ e $ x $ tale che:\begin{equation}
            E_{n}(x)=\frac{f^{(n+1)}(c)}{(n+1)!}(x-x_0)^{(n+1)}\qquad\text{(formula di Lagrange)}
        \end{equation}
    \end{itemize}
}

\esempio{
    Sia $ f(x)=\sin x $, dato $ x_0=0 $: \begin{align*}
        f'(x)&= \cos x & f''(x)&=-\sin x & f^{(3)}(x)&=-\cos(x)\\
        f'(0)&=1 & f''(0)&=0 & f^{(3)}(0)&= -1
    \end{align*}
    \begin{align*}
        f(x)&=f(0)+f'(0)\,x + \frac{1}{2}\,f''(0)\,x^{2}+\frac{1}{6}f^{(3)}(0)\, x^{3}+o(x^{3})_{x\to 0}\\
        \sin x &= x-\frac{1}{6}\,x^{3}+o(x^{3})_{x\to x_0}
    \end{align*}

    Volendo calcolare $ \displaystyle \lim_{x\to 0} \frac{x-\sin x}{x^{3}} $:\begin{multline*}
        \lim_{x\to 0} \frac{x-\sin x}{x^{3}} =\\
        = \lim_{x\to 0} \frac{x-\bigl(x-\frac{1}{6}\,x^{3}+o(x^{3})\bigr)}{x^{3}} = \\
        = \lim_{x\to 0} \left[\frac{(1/6)\,\cancel{x^{3}}}{\cancel{x^{3}}}-\parentesi{\longrightarrow 0}{\frac{o(x^{3})}{x^{3}}}\right] = \frac{1}{6}
    \end{multline*}
}

Gli sviluppi di Taylor con punto di base $ x_0=0 $ prendono il nome di \textit{sviluppi di Mac Laurin}.
