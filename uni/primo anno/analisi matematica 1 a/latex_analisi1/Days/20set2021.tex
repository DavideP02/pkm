\part{Insiemi}

\section{Introduzione agli insiemi}

Gli\marginnote{20 set 2021} insiemi numerici a cui siamo abituati da sempre sono
\begin{gather*}
    \N=\{0, 1, 2, \cdots\}\\
    \Z=\{\cdots, -2, -1, 0, 1, 2, \cdots\}\\
    \Q=\{r=\frac{m}{n}:m \in \Z, n \in \N\setminus\{0\}, m,n \text{ primi tra loro}\}
\end{gather*}

Per l'insieme $ \Q $ esiste una rappresentazione decimale:
\[
    r=n,\alpha_1 \alpha_2 \alpha_3 \cdots\alpha_{j} \cdots
\]
con $ n \in \Z $, $ a_{i} \in\{0, 1, 2,\cdots,9\} $. "$ \alpha_1 \alpha_2 \alpha_3 \cdots\alpha_{j} \cdots $" prende il nome di allineamento periodico (o finisce o si ripete all'infinito).

\subsection{Corrispondenza biunivoca}

Due insiemi \textit{finiti} possono essere messi in corrispondenza biunivoca se e solo se hanno lo stesso numero di oggetti.

\subsubsection{Corrispondenza $ \N-\Z $}

\begin{center}
\begin{tikzpicture}
    \matrix (m) [matrix of math nodes,row sep=3em,column sep=0em,minimum width=0.3em]
    {
        \N=\{ & 0; & 1; & 2; & 3; & 4; & \dots &\dots\} \\
        \Z=\{ & \dots & -2; & -1; & 0; & 1; & 2; &\dots\}\\};
    \node [below] (n0) at (m-1-2) {};
    \node [below] (n1) at (m-1-3) {};
    \node [below] (n2) at (m-1-4) {};
    \node [below] (n3) at (m-1-5) {};
    \node [below] (n4) at (m-1-6) {};
    \node [above] (z-2) at (m-2-3) {};
    \node [above] (z-1) at (m-2-4) {};
    \node [above] (z0) at (m-2-5) {};
    \node [above] (z1) at (m-2-6) {};
    \node [above] (z2) at (m-2-7) {};
    \draw [<->] (n0) -- (z0);
    \draw [<->] (n1) -- (z1);
    \draw [<->] (n2) -- (z-1);
    \draw [<->] (n3) -- (z2);
    \draw [<->] (n4) -- (z-2);
    % \path[-stealth]
    %   (m-1-1) edge node [left] {} (m-2-4);
\end{tikzpicture}
\end{center}

\subsubsection{Corrispondenza $ \N - \N\times \N $}

In rosso è segnato l'insieme $ \N $, mentre in nero le coppie di $ \N\times \N$, che sono state ordinate dalle freccie rosse:
\begin{center}
    \begin{tikzpicture}
        \matrix (m) [matrix of math nodes,row sep=1em,column sep=1em,minimum width=0.3em]
        {
            00^{\textcolor{red}{1}} & 01^{\textcolor{red}{2}} & 02^{\textcolor{red}{6}} & 03^{\textcolor{red}{7}} & \dots\\
            10^{\textcolor{red}{3}} & 11^{\textcolor{red}{5}} & 12^{\textcolor{red}{8}} & 13 & \dots\\
            20^{\textcolor{red}{4}} & \dots & \dots & \dots & \dots \\
            \vdots & \vdots & \vdots & \vdots & \ddots\\};
        \node [right] (00) at (m-1-1) {};
        \node [left] (01a) at (m-1-2) {};
        \node [below left] (01b) at (m-1-2) {};
        \node [above right] (10a) at (m-2-1) {};
        \node [below] (10b) at (m-2-1) {};
        \node [above] (20a) at (m-3-1) {};
        \node [above right] (20b) at (m-3-1) {};
        \node [below left] (11a) at (m-2-2) {};
        \node [above right] (11b) at (m-2-2) {};
        \node [below left] (02a) at (m-1-3) {};
        \node [right] (02b) at (m-1-3) {};
        \node [left] (03a) at (m-1-4) {};
        \node [below left] (03b) at (m-1-4) {};
        \node [above right] (12a) at (m-2-3) {};
        \node [below left] (12b) at (m-2-3) {};
        \node [above right] (dd) at (m-3-2) {};
        
        \draw [->, red] (00) -- (01a);
        \draw [->, red] (01b) -- (10a);
        \draw [->, red] (10b) -- (20a);
        \draw [->, red] (20b) -- (11a);
        \draw [->, red] (11b) -- (02a);
        \draw [->, red] (02b) -- (03a);
        \draw [->, red] (03b) -- (12a);
        \draw [red, dotted, thick] (12b) -- (dd);
    \end{tikzpicture}
    \end{center}

In generale, se $ K \leftrightarrow \N $ (dove $ \leftrightarrow $ si legge "in corrispondenza biunivoca") $\implies$ 
\begin{align*}
    K & \leftrightarrow K\times K = K^{2} \\
    K & \leftrightarrow K\times K \times K = K^{3} \\
    K & \leftrightarrow K\times K\times\cdots\times K = K^{n} \\
\end{align*}

\definizione{}{Un insieme $ A $ è detto \textit{numerabile} se può essere messo in corrispondenza biunivoca con $ \N $}

Gli insiemi $ \N $, $ \Z $, $ \Q $ e $ \N^{n} $, $ \Z^{n} $, $ \Q^{n} $ sono numerabili