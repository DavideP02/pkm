%TODO capire sistemazione
\section{Funzione convessa}
\definizione{}{
    Un insieme $ E \subseteq \R^{n} $\marginnote{7 dic 2021} è \textit{convesso} se $ \forall\, x,y \in E $, il segmento $ [x,y] \subseteq E$.

    Si noti che in $ \R^{n} $ un segmento $ [x,y] $ è definito come \[
        [x,y]:= x+ty
    \]al variare di $ t \in [0,1] $.
}
\definizione{}{
    Data $ f:I\to \R $, con $ I $ intervallo, diciamo \textit{epigrafo} (o epigrafico), l'insieme \[
        \epi f =\big\{(x,y)\in \R^{2}\,\tc\, x \in I, y\ge f(x)\big\}
    \]

    Diciamo che una funzione $ f $ è convessa su $ I $, se $ \epi(f) $ è convesso in $ \R^{2} $
}

%TODO aggiungere grafico dell'epigrafo

%TODO manca l'identificazione analitica

Poniamo adesso nel caso in cui $ f $ sia derivabile su $ I $.

\teorema[(di caratt. delle funzioni convesse)]{elefante11ddedoijooodidiego}{
    Sia $ f: I\to \R $ derivabile su $ I $, allora sono equivalenti le seguenti proprietà:
    \begin{enumerate}
        \item $ f $ convessa su $ I $;
        \item $ \forall\, x_0, x \in I $, $ f(x)\ge f(x_0)+f'(x_0)(x-x_0) $;
        \item $ f' $ è crescente su $ I $.
    \end{enumerate}
}
%TODO aggiungere interpretazione grafica
\dimostrazione{elefante11ddedoijooodidiego}{
    Dimostreremo che $ 1. \,\implies\,2. \,\implies\, 3. \,\implies\, 1. $
    \begin{itemize}
        \item [$1. \,\implies\,2.$] Consideriamo $ f $ convessa su $ I $, e fissiamo $ x_0, x \in I $, con $ x_0<x $. È noto che $ \forall\, t \in [0,1] $ 
        \begin{align*}
            f\bigl((1-t)x_0+tx\bigr)&\le (1-t)f(x_0)+t\,f(x)\\
            f(x_0-tx_0+tx)&\le f(x_0)-t\,f(x_0)+t\,f(x)\\
            f\bigl(x_0+t(x-x_0)\bigr)&\le  f(x_0)+t\bigl(f(x)-f(x_0)\bigr)\\
            f\bigl(x_0+t(x-x_0)\bigr)-f(x_0)&\le t\,\bigl(f(x)-f(x_0)\bigr)
        \end{align*}
        Per $ t \in (0,1) $ dividiamo per $ t(x-x_0)>0 $
        \[
            \frac{f\bigl(x_0+t(x-x_0)\bigr)-f(x_0)}{t(x-x_0)}\le \frac{\cancel{t}\,\bigl(f(x)-f(x_0)\bigr)}{\cancel{t}(x-x_0)}
        \]
        Posto $ h=t(x-x_0) $ abbiamo che \[
            \frac{f(x_0+h)-f(x_0)}{h} \le \frac{f(x)-f(x_0)}{(x-x_0)}
        \]
        Quando $ t\to 0 $ $\implies$ $ h\to 0 $, quindi questa proprietà è valida definitivamente per $ h\to 0 $.
        \[
            \parentesi{f'(x_0)}{\lim_{h\to 0} \frac{f(x_0+h)-f(x_0)}{h}} \underset{\footnotemark}{\le} \frac{f(x)-f(x_0)}{(x-x_0)}
        \]\footnotetext{per permanenza del segno} 
        
        $\implies$ $ f(x)\ge f(x_0)+f'(x_0)(x-x_0) $.
    \item [$2. \,\implies\, 3.$] Noi sappiamo che $ f $ è derivabile su $ I $, e $ f(x)\ge f(x_0)+f'(x_0)(x-x_0) $.
    
    Allora si ha che fissati $ x_0, x \in I $, con $ x_0<x $, \[
        f(x)\ge f(x_0)+f'(x_0)\parentesi{>0}{(x-x_0)}
    \] inoltre vale anche (scambiando $ x $ e $ x_0 $) \[
        f(x_0)\ge f(x)+f'(x)\parentesi{<0}{(x_0-x)}.
    \]

    Allora abbiamo che \begin{gather*}
        f'(x_0)\le \frac{f(x)-f(x_0)}{x-x_0}\\
        f(x_0)-f(x)\ge f'(x)(x_0-x) \,\implies\, f(x)-f(x_0)\le f'(x)(x-x_0).
    \end{gather*}
    Otteniamo quindi, dalla seconda equazione: \begin{gather*}
        f'(x)\ge \frac{f(x)-f(x_0)}{x-x_0}\\
        f'(x_0)\le \frac{f(x)-f(x_0)}{x-x_0}\le f'(x).
    \end{gather*}
    
    
    Allora per $ x<x_0 $ generico in $ I $ si ha \[
        f'(x_0)\le f'(x)
    \] 
    
    $\implies$ per genericità di $ x_0, x \in I $, $ x<x_0 $, $ f' $ è crescente su $ I $.
    \item [$3. \,\implies\, 1.$] Data $ f' $ crescente su $ I $, dobbiamo far vedere che \begin{multline}
        \forall\, t \in [0,1], \forall\, x_0,x \in I \\
        f\bigl((1-t)x_0+tx\bigr)\le (1-t)f(x_0)+t\,f(x)\label{star}
    \end{multline}
        \eqref{star} è ovviamente vera per $ t=0\,\land\,t=1 $

        Consideriamo ora $ t \in(0,1)  $. Preso \[z_{t}=(1-t)\,x_0+t\,x=x-t\,x_0+t\,x=x_0+t\,(x-x_0)\]
        Dato $ t \in (0,1) $, assumendo $ x_0<x $, vale che $ x_0<z_{t}<x  $.

        Applichiamo il Teorema di Lagrange agli intervalli $ I_1=[x_0, z_{t}] $ e $ I_2=[z_{t}, x  ] $, in quanto $ f $ derivabile (e quindi continua) su $ I_1, I_2 $.

        Allora $ \exists \, z \in I_1 $ e $ w \in I_2  $ tale che \[
            f'(z)=\frac{f(z_{t})-f(x_0)}{(z_{t}-x_0 )}\qquad f'(w)=\frac{f(x)-f(z_{t} )}{(x-z_{t} )}
        \]
        \begin{gather}
            f'(z)= \frac{f\bigl(x_0+t\,(x-x_0)\bigr)-f(x_0)}{\cancel{x_0}+t\,(x-x_0)-\cancel{x_0}}\label{AAAAd}\\
            f'(w)=\frac{f(x)-f\bigl(x_0+t\,(x-x_0)\bigr)}{x-x_0-t\,x+t\,x_0)}=\frac{f(x)-f\bigl(x_0+t\,(x-x_0)\bigr)}{(1-t)(x-x_0)}\label{BBBd}
        \end{gather}

        Da \eqref{AAAAd} otteniamo che \begin{equation}
            f\bigl(x_0+t\,(x-x_0)\bigr)-f(x_0)=t\,f'(z)\,(x-x_0).\label{etereum}
        \end{equation}
        Da \eqref{BBBd} otteniamo che \begin{equation}
            f(x)-f\bigl(x_0+t\,(x-x_0)\bigr)=(1-t)\,f'(w)\,(x-x_0)\label{eterumd}
        \end{equation}

        Moltiplichiamo \eqref{etereum} per $ (1-t) $ e \eqref{eterumd} per $ t $
        \begin{gather*}
            (1-t)\bigl[f\bigl(x_0+t\,(x-x_0)\bigr)-f(x_0)\bigr]=\parentesi{>0}{t\,(1-t)}\,f'(z)\,\parentesi{>0}{(x-x_0)}\\
            t\bigl[f(x)-f\bigl(x_0+t\,(x-x_0)\bigr)\bigr]=\parentesi{>0}{t\,(1-t)}\,f'(w)\,\parentesi{>0}{(x-x_0)}.
        \end{gather*}

        Inoltre $ z \in [x_0, z_{t} ] $, $ w \in[z_{t}, x ] $ 
        
        $\implies$ $ z<w $ 
        
        $\implies$ $ f'(z)\le f'(w) $, in quanto per ipotesi $ f' $ è crescente.

        Allora si ha, per $ x_0<x $ \begin{multline*}
            (1-t)\bigl[f\bigl(x_0+t\,(x-x_0)\bigr)-f(x_0)\bigr]\le\\\le t\bigl[f(x)-f\bigl(x_0+t\,(x-x_0)\bigr)\bigr]
        \end{multline*}

        Con banali passaggi algebrici si ottiene \[
            f\bigl((1-t)\,x_0+t\,x\bigr)\le (1-t)\,f(x_0)+t\,f(x)
        \] ossia $ f $ convessa su $ I $.

        Per $ x<x_0 $ si fanno passaggi simili.\qed
    \end{itemize}
}

\teorema[(test della derivata seconda)]{testdelladerivata}{
    Data $ f $ derivabile due volte su $ I $, $ f $ è convessa su $ I $ 

    $ \iff $ $ f''(x)\ge 0 $ $ \forall\, x \in I $.
}
\dimostrazione{testdelladerivata}{
   È sufficiente applicare il test della derivata prima ad $ f' $.\qed 
}

\definizione{
    $ f:I\to \R $ è detta \textit{concava} se $ g:=-f $ è convessa.
}{}

$ f $ derivabile su $ I $ è concava 

$ \iff $ la tangente in ogni punto giace sopra il grafico;

$\iff$ $ f' $ è decrescente (e se $ f $ derivabile due volte $ \iff $ $ f''(x)\le 0 $ $ \forall\, x \in I $).

\definizione{}{
    Sia $ f:I\to \R $, $ x_0 \in \mathring{I} $, $ f $ derivabile in $ x_0 $, $ x_0 $ è un \textit{punto di flesso} per $ f $ 
    \begin{itemize}
        \item se $ f $ convessa in $ (x_0-\delta, x_0) $ e $ f $ concava in $ (x_0, x_0+\delta) $, ed è detto \textit{flesso discendente};
        %TODO aggiungere grafico di esempio
        \item se $ f $ concava in $ (x_0-\delta, x_0) $ e $ f $ convessa in $ (x_0, x_0+\delta) $, ed è detto \textit{flesso ascendente}.
    \end{itemize}
}
\attenzione{
    In $ x_0 $ punto di flesso, la tangente in $ x_0 $ attraversa (taglia) il grafico.
}

\definizione{}{
    Se la tangente è orizzontale in $ x_0 $ ($f'(x_0)=0$), $ x_0 $ è detto \textit{flesso a tangente orizzontale}.
}
\attenzione{
    Per definire il flesso la funzione deve essere derivabile in quel punto. 

    Caso particolare: $ f $ derivabile in $ I\setminus \{x_0\} $, continua in $ x_0 $ \[
        \lim_{x\to x_0} f'(x) = \begin{aligned}
            +\infty\\
            -\infty
        \end{aligned}
    \] $ x_0$ è detto flesso a tangente verticale. 

    In $ x_0 $ flesso a tangente verticale, la tangente è verticale e taglia il grafico.
}
\proprieta{}{
    Data $ f:I\to \R $, $ I $ intervallo, $ x_0 \in I $, $ f $ derivabile due volte in $ x_0$. Allora vale
    
    $ x_0 $ è punto di flesso (non verticale) 
    
    $\implies$ $ f''(x_0)=0 $
}
\begin{proof}
    Si applica il teorema di Fermat a $ f' $
\end{proof}

I candidati punti di flesso (se $ f $ è derivabile due volte) sono i punti con $ f''(x_0)=0 $.
\attenzione{
    $ f''(x)=0 $ $ \nRightarrow $ $ x_0 $ punto di flesso
}
\esempio{}{
    Sia $ f(x)=x^{4} $, \[
        f'(x)=4x^{3}\qquad f''(x)=12x^{2}
    \]
    Si ha che $ f''(x)=0 $ $ \iff $ $ x=0 $.

    Quindi $ x=0 $ è punto di minimo e non di flesso.
}   