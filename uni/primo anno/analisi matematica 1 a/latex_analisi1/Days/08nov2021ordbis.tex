\section{Continuità}

Sia\marginnote{8 nov 2021} $ f:D\to \R $, $ D \subseteq \R $ $ x_{0} \in D'$, $ x_0 \in \R $, $ l \in \R $

Diciamo $ \lim_{x\to x_0} f(x) =l $
\[
  \iff\quad \forall \varepsilon>0 \exists \delta >0 \,\tc\, 0<|x-x_0|<\delta \,\implies\,|f(x)-l|<\varepsilon  
\]

Il valore di $ l $ non è in alcun modo legato ad $ f(x_0) $

Consideriamo $ x_0 \in D $
\esempi{}{
    \begin{itemize}
        \item $ f(x)=x^2 $
        \[
            \lim_{x\to 0} f(x) = 0 =f(0)
        \]
        \item $ f(x)= \begin{cases}
            x^2 & x \neq 0 \\
            1 & x=0
        \end{cases}$

        \[
            \lim_{x\to 0} f(x) =0 \neq f(0)
        \]
        \item $ \text{sgn}(x)= \begin{cases}
            -1 & x < 0 \\
            0 & x=0\\
            1 & x>0
        \end{cases}$

        \begin{align*}
            \lim_{x\to 0} \text{sgn}(x) & = \nexists\\
            \lim_{x\to 0^+} \text{sgn}(x) = 1 &\neq \text{sgn}(0)=0\\
            \lim_{x\to 0^-} \text{sgn}(x) = -1 &\neq \text{sgn}(0)=0
        \end{align*}
        \item $ H(x)=\begin{cases}
            0 & x<0\\
            1 & x\ge 0
        \end{cases} $
        \begin{gather*}
            \lim_{ x\to 0^{+}} H(x) =0=H(0)\\
            \lim_{x\to 0^{-}} H(x) =0\neq H(0)
        \end{gather*}
        \item $ f(x)= \begin{cases}
            x\sin \frac{1}{x} & x \neq 0 \\
            0 & x=0
        \end{cases}$

        \[
            \lim_{x\to 0} f(x) =0=f(0)
        \]
    \end{itemize}
}

\definizione{}{
    Consideriamo $ D \subseteq \R^n $
    \begin{align*}
    f:D & \to \R^m \\
    x & \mapsto f(x)
    \end{align*}
    con $ x=(x_1,\cdots,x_{n}) $, $ f(x)=(f_{1}(x), f_2(x), \cdots, f_{m}(x)  ) $

    Diciamo che $ f $ è continua in $ x_0 \in D $ se 
    \begin{itemize}
        \item [\textit{a}.] $ x_0 $ punto isolato di $ D $
        \item [\textit{b}.] $ x_0 \in D'$ e vale una delle seguenti affermazioni tra di loro equivalenti:
            \begin{itemize}
                \item [(\textit{i})] $ \forall V(f(x_0))$ $ \exists U(x_0)$ tale che $ x \in U \cap D $ 
                
                $\implies$ $f(x) \in V$
                \item [(\textit{ii})] $ \forall \varepsilon>0 $ $ \exists \delta > 0 $ tale che $ |x-x_0|<\delta $ 
                
                $\implies$ $ |f(x)-f(x_0)|<\varepsilon $
                \item [(\textit{iii})] $ \lim_{x\to x_0} f(x) =f(x_0) $
                \item [(\textit{iv})] data $ \{x_{n} \}_{n=0}^\infty $ a valori in $ D $ tale che $ x_{n} \xrightarrow{n\to \infty} x_0 \displaystyle  $ allora
                \[
                    \lim_{n\to \infty} f(x_{n} ) = f(x_0)
                \]
            \end{itemize}
    \end{itemize}}

    \lemma{dfgdfgdg}{
        Le quattro affermazioni precedenti sono equivalenti
    }
    \dimostrazionelem{dfgdfgdg}{
        \begin{itemize}
            \item [\textit{i}.$\iff$\textit{ii}.] è ovvio
            \item [\textit{ii}.$\implies$\textit{iii}.] è ovvio
            \item [\textit{iii}.$\iff$\textit{iv}.] per il teorema di relazione
            \item [\textit{iii}.$\implies$\textit{ii}.] $ \displaystyle\lim_{x\to x_0} f(x) =f(x_0) $ vale \[\forall \varepsilon >0 \,\exists \delta >0:\, |x-x_0|<\delta \land x \neq x_0 \,\implies\, |f(x)-f(x_0)|< \varepsilon\]
            
            se $ x=x_0 $ $ |f(x)-f(x_0)|= |f(x_0)-f(x_0)|=0< \varepsilon$ 
            
            $\implies$ $ \forall \varepsilon >0 \,\exists \delta>0:\, |x-x_0|<\delta \,\implies\,|f(x)-f(x_0)|< \varepsilon$ ossia $ f $ continua in $ x_0 $ \qed
        \end{itemize}
    }  

    Diciamo che $ f $ è continua in $ E \subseteq D$ se $ \forall x_0 \in E $ $ f $ è continua in $ x_0 $

    \esempi{}{
        Funzioni continue nel loro dominio\begin{itemize}
            \item $ f(x)=x $
            \item $ f(x)=x^{\alpha} $
            \item $ f(x)=a^{x} $, $ a>0 $, $ a\neq 1 $
            \item $ f(x)=\log_a x $, $ a>0 $, $ a\neq 1 $
        \end{itemize}
    }

    In generale dati $ f:D\to \R $ con $ D \subseteq \R $, e $ x_0 \in D $ se si ha 
    \[\begin{cases}
        \displaystyle\lim_{x\to x_0^+} f(x) = f(x_0)\text{ si dice che } f \text{ è continua da destra}\\
        \displaystyle\lim_{x\to x_0^-} f(x) = f(x_0)\text{ si dice che } f \text{ è continua da sinistra}
    \end{cases}\]