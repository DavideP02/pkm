\subsection{Costante di Nepero}\marginnote{8 nov 2021}

Consideriamo la successione $ a_{n} = \big(1+\frac{1}{n}\big)^n $

\[
  \lim_{n\to +\infty} (1+\frac{1}{n})^n =  1^{+\infty}
\] è una forma indeterminata

Verifichiamo la convergenza:
\begin{enumerate}
    \item $ a_{n}$ è crescente
    \item $ a_{n}$ è superiormente limitata
    \item applichiamo il teorema di esistenza del limite per le succesioni monotone
\end{enumerate}

\begin{enumerate}
    \item $ a_{1} = 2  $, per $ n\ge 2 $ stimiamo il rapporto
    \begin{multline*}
        \frac{a_{n}}{a_{n-1}}=\frac{(1+\frac{1}{n})^n}{(1+\frac{1}{n-1})^{n-1}}=\\
        =\frac{(\frac{1+n}{n})^n}{(\frac{n}{n-1})^{n-1}}=\frac{(\frac{1+n}{n})^n}{(\frac{n}{n-1})^{n}(\frac{n}{n-1})^{-1}}=\\
        =\frac{(\frac{1+n}{n})^n(\frac{n-1}{n})^n}{\frac{n-1}{n}}=\frac{(\frac{n^2-1}{n^2})^n}{\frac{n-1}{n}}=\\
        =\frac{(1-\frac{1}{n^2})^n}{\frac{n-1}{n}}=**
    \end{multline*}
    Applico la disuguaglianza di Bernoulli
    \[
        \frac{1}{n^2}<1, -\frac{1}{n^2}>-1
    \]
    \[
    \implies\,(1-\frac{1}{n^2})\ge 1-n\frac{1}{n^2}=1-\frac{1}{n}
    \] 

    $\implies$ $ **\ge \frac{1-\frac{1}{n}}{1-\frac{1}{n}}=1 $

    Quindi $ \forall n\ge 2 $, $ a_{n} \ge a_{n-1}   $, quindi $ a_{n} $ è crescente definitivamente
    \item Dimostriamo ora che $ a_{n}  $ è limitata superiormente.
    
    Consideriamo $ b_{n} = (1+\frac{1}{n})^{n+1} $ ($a_{n}\le b_{n} \forall n \in \N$ )

    Verifichiamo che $ b_{n} $ è decrescente.
    \[
        \frac{b_{n} }{b_{n-1}}=\frac{(1+\frac{1}{n})^n}{(1+\frac{1}{n-1})^n}=
        \cdots
        =\frac{1+\frac{1}{n}}{(1+\frac{1}{n^2-1})^n}
    \]

    Stimiamo $ (1+\frac{1}{n^2-1})^n $; per qualsiasi $ n\ge 2 $, $ \frac{1}{n^2-1}>0 $, e posso applicare Bernoulli:
    \[
        (1+\frac{1}{n^2-1})^n\ge 1 - \frac{n}{n^2-1}\ge 1+\frac{n}{n^2}=1+\frac{1}{n}
    \]

    Ottengo quindi che
    \[
        \frac{b_{n} }{b_{n-1}}=\frac{1+\frac{1}{n}}{(1+\frac{1}{n^2-1})^n}\le \frac{1+\frac{1}{n}}{1+\frac{1}{n}}=1
    \]

    Quindi $ \forall n\ge 2 $, $ b_{n} <b_{n-1}  $, quindi $ b_{n} $ decrescente definitivamente, ma $ b_{2} = 4 $ $ \,\implies\,$ $ b_{n} \le 4 $ definitivamente

    Poiché $a_{n}\le b_{n} \forall n \in \N$ si ha $ a_{n}  $ crescente e $ a_{n} \le 4 $ definitivamente

    \item Dunque, per il teorema di esistenza del limite per successioni monotone limitate, otteniamo che
    \[
      \lim_{n\to + \infty} \biggl(1+\frac{1}{n}\biggr)^n = \sup\biggl\{\biggl(1+\frac{1}{n}\biggr)^n\biggr\} \in \R
    \]
    (esiste ed è un numero reale), e lo chiamiamo $ e $, detta costante di Nepero\qed
\end{enumerate}

    Quindi \[
        e=\lim_{n\to + \infty} \biggl(1+\frac{1}{n}\biggr)^n
    \]

    Osserviamo che
    \[
      a_{1} = 2 \le \biggl(1+\frac{1}{n}\biggr)^n \le 4
    \]

    Una prima stima di $ e $ risulta essere
    \[
        2\le e \le 4
    \] 

    Con opportuni algoritmi di approssimazione si stima che
    \[
      e=2,7182818284\dots  
    \]

    \osservazione{
        $ e \in \R\setminus \Q$ (dimostrazione sul libro di testo)
    }

    \proposizione{sgrsgfdsgfsd}{
        \[\lim_{x\to \pm\infty} \biggl(1+\frac{1}{n}\biggr)^n = e\]
    }

    \lemma{gagaga}{
        Sia $ x_{n} \xrightarrow{n\to +\infty} \pm\infty \displaystyle  $ allora \[
            \lim_{n\to + \infty} \biggl(1+\frac{1}{x_n}\biggr)^{x_{n}}= e
        \]
    }
    \dimostrazioneprop{sgrsgfdsgfsd}{
        Applicando il teorema di relazione, a partire dal lemma \hyperref[lmm:gagaga]{(\textit{l.}\roman{lmmgagaga})} si ottiene 
        \[\lim_{x\to \pm\infty} \biggl(1+\frac{1}{n}\biggr)^n = e\]
    }
    \pagebreak[4]
    \dimostrazionelem{gagaga}{
        \begin{enumerate}
            \item $x_{n} \xrightarrow{n\to + \infty} +\infty$, ricordiamo $ [x_{n} ]\le x_{n} \le [x_{n}] +1  $ 
            \begin{gather*}
                \frac{1}{[x_{n} ]+1}<\frac{1}{x_{n} }\le \frac{1}{[x_{n} ]}\\
                1+\frac{1}{[x_{n} ]+1}<1+\frac{1}{x_{n} }\le 1+\frac{1}{[x_{n} ]}\\
                \parentesi{\alpha_{n} }{\biggl(1+\frac{1}{[x_{n} ]+1}\biggr)^{[x_{n} ]}}\le 1+\frac{1}{x_{n} }\le \parentesi{\beta_{n} }{\biggl(1+\frac{1}{[x_{n} ]}\biggr)^{[x_{n} ]+1}}
            \end{gather*}
             
            \[
                \beta_n = \parentesi{ \xrightarrow[n\to + \infty]{} e}{\biggl(1+\frac{1}{[x_{n} ]}\biggr)^{[x_{n} ]}}\:\parentesi{ \xrightarrow[n\to + \infty]{} 1}{\biggl(1+\frac{1}{[x_{n} ]}\biggr)}
            \] notando che $ [x_{n} ] \xrightarrow{n\to +\infty} +\infty \displaystyle $. Ne risulta che $ \beta_{n} \xrightarrow{n\to +\infty} e \displaystyle $.

            \[
                \alpha_{n} = \biggl(1+\frac{1}{[x_{n} ]+1}\biggr)^{[x_{n} ]} =\parentesi{ \xrightarrow[n\to + \infty]{} e}{\biggl(1+\frac{1}{[x_{n} ]+1}\biggr)^{[x_{n}]+1}}\:\parentesi{ \xrightarrow[n\to + \infty]{} 1}{\biggl(1+\frac{1}{[x_{n} ]}\biggr)^{-1}}
            \] Ne risulta che $ \alpha_{n} \xrightarrow{n\to +\infty} e \displaystyle $.

            Dunque, data $x_{n} \xrightarrow{n\to + \infty} +\infty$ \[
                \lim_{n\to +\infty} \biggl(1+\frac{1}{x_{n} }\biggr)^{x_{n} } = e
            \] per il teorema del confronto
            \item $ x_{n} \xrightarrow{n\to + \infty} - \infty \displaystyle $, $ y_{n}=-x_{n } \xrightarrow{n\to + \infty} \infty \displaystyle $ \begin{multline*}
                \biggl(1+\frac{1}{x_{n} }\biggr)^{x_{n} } = \biggl(1-\frac{1}{y_{n} }\biggr)^{-y_{n} } = \\=\biggl(\frac{y_{n} - 1}{y_{n} }\biggr)^{-y_{n} }= \biggl(\frac{y_{n}}{y_{n} -1}\biggr)^{y_{n} }= \biggl(1+ \frac{1}{y_{n} -1}\biggr)^{y_{n} }=\\
                = \parentesi{ \xrightarrow[n\to + \infty]{} e}{\biggl(1+ \frac{1}{y_{n} -1}\biggr)^{y_{n} -1}}\:\parentesi{ \xrightarrow[n\to +\infty]{} 1 \displaystyle}{\biggl(1+ \frac{1}{y_{n} -1}\biggr)}
            \end{multline*} poiché $ (y_{n}-1) \xrightarrow{n\to + \infty} +\infty \displaystyle  $.

            Dunque, data $x_{n} \xrightarrow{n\to + \infty} -\infty$ \[
                \lim_{n\to +\infty} \biggl(1+\frac{1}{x_{n} }\biggr)^{x_{n} } = e.
            \]
            \item La proprietà\[
                \lim_{x_{n} \to \infty} \biggl(1+\frac{1}{x_{n} }\biggr)^{x_{n} } = e
            \] discende direttamente da 1. e 2.
        \end{enumerate}
    }