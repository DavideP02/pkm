\section{Limiti}
\subsection{Introduzione al concetto di limite}

Data\marginnote{11 ott 2021} una funzione $ f:D\to \R $, con $ D \subseteq \R^{n} $ e $ x_0 \in \R^{n} $, ci poniamo l'obiettivo di descrivere il comportamento della funzione quando $ x $ si "avvicina" a $ x_0 $.

Indichiamo con $ x\to x_0 $: "avvicinarsi a $ x_0 $", "essere nei pressi di $ x_0 $". In alcuni casi è intuitivo.
\begin{itemize}
    \item caso 1: \begin{align*}
    f:\R & \to \R \\
    x & \mapsto x^{2}
    \end{align*}\begin{align*}
        x=2 \,&\implies\, f(x)=4 & x\to 2\quad &f(x) \to 4\\
        x=3 \,&\implies\, f(x)=9 & x\to 3\quad &f(x) \to 9\\
        x=-1 \,&\implies\, f(x)=1 & x\to -1\quad &f(x) \to 1
    \end{align*}

    Dobbiamo comunque chiarirlo in termini rigorosi.
    \item caso 2: $ f(t)$ rappresenta un impulso luminoso istantaneo al tempo $ t=2 $
    \begin{center}
        \begin{tikzpicture}
            \begin{axis}[
                xlabel=$t$,
                ylabel=$y$,
                axis equal,
                axis lines=middle,
                enlargelimits,
                xmax=5,
                xmin=0,
                ymax=4,
                ymin=-1,
                xtick={2},
                ytick={3},
                scale only axis, 
                height=4cm, 
                width=4cm
                ]
            \addplot [blue, no marks, thick] coordinates {(0,0) (7,0)};
            \addplot [white, thick, only marks] coordinates {(2,0)};
            \addplot [blue, thick, only marks] coordinates {(2,3)};
            \addplot [blue, thick, mark=o] coordinates {(2,0)};
            \end{axis}
        \end{tikzpicture}
    \end{center} dove \[
        f(t)=\begin{cases}
            0 & t\neq 2\\
            3 & t=2
        \end{cases}
    \]
\end{itemize}
\subsection{Estensione di $ \R $}
\subsection{Estensione di $ \R^{n} $}
\subsection{Limite di una funzione}