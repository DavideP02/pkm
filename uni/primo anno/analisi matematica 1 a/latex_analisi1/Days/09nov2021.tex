% Lezione 18: https://math.i-learn.unito.it/pluginfile.php/133502/mod_resource/content/1/UNO_lezione18_2020_21.pdf

\marginnote{9 nov 2021}

\begin{multline*}
    \lim_{x\to x_0} f(x) =f(x_0)\\
    \iff\,\lim_{h\to 0} f(x_0+h) =f(x_0)\,\iff\\ 
    \lim_{h\to 0} f(x_0+h) - f(x_0)=0
\end{multline*}

\esempio{
    Verifichiamo che $ \forall x_0 \in \R $, $ \sin x $ è continua in $ x_0 $. Sappiamo che $ \lim_{x\to 0} \sin x = 0 $.

    Per  $ x_0 \in \R $
    \begin{multline*}
        \lim_{h\to 0} \sin (x_0+h)-\sin (x_0) =\\
        = \lim_{h\to 0} \biggl(\sin x_0 \cos h + \sin h \cos x_0 - \sin x_0 \biggr)=\\
        = \lim_{h\to 0} \biggl(\sin x_0 (\cos h -1)+ \sin h \cos x_0 \biggr)= 
    \end{multline*}

    Dato che $ \sin h \xrightarrow{h\to 0} 0 \displaystyle $
    \[
        = \sin x_0 \lim_{h\to 0} \biggl(\cos h - 1\biggr)=0
    \]

    Allora $ \forall x_0 \in \R $ si ha \[
        \lim_{x\to x_0} \sin x =\sin x_0
    \] 
    
    $\implies$ $ \sin x $ continua su $ \R $. Allo stesso modo si verifica che $ \cos x $ è continua su $ \R $
}

\proprieta{(Algebra delle funzioni continue)}{
    Date $ f, g:D\to \R $, con $ x_0 \in D \subseteq \R $, $ f, g $ continue in $ x_0 $, allora $ \forall a \in \R $ si ha che $ af+g $ è continua in $ x_0 $

    Inoltre
    \begin{itemize}
        \item $ fg $ continua in $ x_0 $
        \item se $ g(x_0)\neq 0 $ allora $ \frac{f}{g} $ continua in $ x_0 $
        \item $ f_{+}(x)=\max \{0,f(x)\}$ è continua in $ x_0 $
        % TODO aggiungere grafico di f+
    \end{itemize}
}

\teorema[(Continuità della funzione composta)]{adgfdagf}{
    Sia $ f:D\to \R $, $ x_0 \in D $, $ g:f(D)\to \R $. Se $ f $ è continua in $ x_0 $ e $ g $ continua in $ f(x_0) $ 
    
    $\implies$ $ g\circ f $ è continua in $ x_0 $
}
\dimostrazione{adgfdagf}{
    % TODO aggiungere grafico
    \[ \forall\, V(g(f(x_0)))\,\exists\, W(f(x_0))\text{ tale che } \forall y \in W \cap f(D) \] 
    
    $\implies$ $ g(y) \in V$
    
    $ \exists U(x_0) $ tale che $ \forall x \in U\cap D $ $\implies$ $f(x) \in W$

    Allora $ \exists U(x_0) $ tale che $ \forall x \in U\cap D $ $ g(f(x)) \in V $ 
    
    $\implies$ $ g \circ f $ è continua in $ x_0 $
    % riguardare questa dimostrazione con gli appunti
}

\proprieta{}{
    Date $ f:D \to \R $, $ x_0 $ di accumulazione per $ D $, $ g:E\to \R$, con $ f(D) \subseteq E $, assumiamo
    \begin{itemize}
    \item [\textit{i}.] \[
        \lim_{ x\to x_0} f(x) =l \in E
    \]
    \item [\textit{ii}.] $ g $ continua in $ l $, $ l \in \R $ 
    \end{itemize}

    $\implies$ $ \lim_{x\to x_0} g(f(x)) =g(l) $

    Allora, date \textit{i}. e \textit{ii}., si ha
    \[
        \lim_{x\to x_0} g(f(x)) =g \biggl( \lim_{x\to x_0} f(x) \biggr)
    \]
}

Si dimostra che sono continue nel loro dominio 
\begin{itemize}
    \item i polinomi
    \item le frazioni algebriche
    \item le funzioni esponenziali
    \item le funzioni logaritmiche
    \item le funzioni goniometriche e le loro inverse
\end{itemize}
Tutte queste funzioni sono dette "funzioni elementari"

\attenzione{
    Data $ f:D\to \R $, $ f $ invertibile su $ D $, e $ f $ continua su $ D $

    $\nRightarrow $ $ f^{-1} $ sia continua du $ f(D) $
}

\esempio{}{
    %TODO grafico 1
    La funzione è analiticamente definita come
    \[
        f(x)= \begin{cases}
            x & 0 \le x\le 1 \\
            x-1 & 2 < x \le 3
        \end{cases}
    \]

    Notiamo che $ D=[0,1]\cup (2,3] $, e che $ f $ sia continua nel suo dominio.
    \[
        f(D)=[0,2]
    \]

    Invertendola:
    \[
        f^{-1}(x)= \begin{cases}
            x & 0 \le x \le 1 \\
            x+1 & 1 < x \le 2
        \end{cases}
    \]
    %TODO grafico finv
    Quindi $ f^{-1} $ non è continua su $ f(D) $, in particolare non è continua in $ x_0=1 $
}

\proprieta{}{
    Data $ f:I\to \R $, con $ I $ intervallo,

    se $f$ è invertibile e continua su $ I $ 
    
    $\implies$ $ f^{-1} $ è continua su $ J=f(I) $
}

\subsection{Discontinuità}

Consideriamo $ f:D\to \R $, $ x_0 \in D $ e $ f $ continua in $ D\setminus \{x_0\} $

Diciamo che:
\begin{enumerate}
    \item $ x_0 $ è una \textit{discontinuità eliminabile} se \[
        \lim_{x\to x_0} f(x) = l \in \R \,\land\, l\neq f(x_0)
    \]
    \esempio{}{
        \[
            f(x) = \begin{cases}
                \frac{\sin x}{x} & x \neq 0 \\
                0 & x=0
            \end{cases}
        \]
        $ f $ è continua in $ \R\setminus \{0\} $, vale \[
            \lim_{x\to 0} f(x) = 1 \in \R\neq 0
        \]

        Quindi $ x_0=0 $ è discontinuità eliminabile
    }
    \item $ x_0 $ è detto \textit{salto} o \textit{punto di salto} se \begin{gather*}
        \lim_{x\to x_0^{+}} f(x) = l \in \R\\
        \lim_{x\to x_0^{-}} f(x) = n \in \R\\
        l\neq n
    \end{gather*}

    Si definisce \textit{ampiezza del salto} la grandezza
    \[
        s = l-n
    \]
    \esempio{}{
        Data \[
            H(x)= \begin{cases}
                0 & x<0 \\
                1 & x\ge 0
            \end{cases}
        \] si ha che $ x_0 = 0 $ è salto. $ s=1 $
    }
    \esempio{}{
        Data \[
            \text{sgn}(x)= \begin{cases}
                1 & x>0 \\
                0 & x=0\\
                -1 & x<0
            \end{cases} 
        \]
        si ha che $ x_0 = 0 $ è salto. $ s=2 $
    }
    \notazione{}{Nel \textsc{Pagani Salsa} i punti di salto sono detti discontinuità di prima specie}
    \notazione{}{
        Nella terminologia a lezione, si intendono sia i salti che le discontinuità eliminabili come discontnuità di prima specie
    }
    \item $ x_0 $ è \textit{discontinuità di seconda specie} se si verifica una delle seguenti condizioni
    \begin{align*}\displaystyle\lim_{x\to x_0^{\pm}} f(x) = & \pm\infty\\
        & \mp\infty\\
    &+\infty\\
&-\infty\end{align*}
\[\lim_{x\to x_0^{+}} f(x) = \nexists\]
\[\lim_{x\to x_0^{-}} f(x) = \nexists\]

    % TODO manca esempio
\end{enumerate}

\subsection{Prolungamento per continuità di una funzione}

Sia $ f:D\to \R$ e $ x_0 \in D'$. 

Assumiamo che \[
    \lim_{x\to x_0} f(x) =l \in \R
\]

Diciamo \textit{prolungamento per continuità} di $ f $ in $x_0$ la funzione \[
    \tilde{f}(x)= \begin{cases}
        f(x) & x \in D\setminus \{x_0\} \\
        l & x=x_0
    \end{cases}
\] 
$\tilde{f}$ è continua in $x_0$

Ovviamente se $ x_0 \in D $ e $ f $ continua in $ x_0 $ allora \[
    \tilde{f}(x)=f(x)
\]

\esempi{}{
    \begin{itemize}
        \item Consideriamo \[
            f(x)= \begin{cases}
                x^{2} & x\neq 0 \\
                1 & x=0
            \end{cases}
        \]
        $ f $ non è continua in 0, con una discontinuità elimiinabile

        \[
            \tilde{f}(x)\begin{cases}
                x^{2} & x\neq 0 \\
                0 & x=0
            \end{cases}=x^{2}
        \]
        Questo è il prolungamento per continuità di $ f $
        \item Consideriamo \[
            f(x)=x\sin\frac{1}{x}
        \]

        Si ha che $ \text{dom} f = \R\setminus\{0\} $. $ f $ è continua nel suo dominio.
        \[
            \lim_{x\to 0} f(x) = \lim_{x\to 0} x\sin\frac{1}{x} =0
        \]

        Allora \[
            \tilde{f}(x)\begin{cases}
                x\sin\frac{1}{x} & x\neq 0 \\
                0 & x=0
            \end{cases}
        \]
        è il prolungamento per continuità di $ f $ in 0; $ \tilde{f} $ è continua su $ \R $
        \item Consideriamo $ f(x)=x^{x} $. Si ha che $ D=\text{dom}f=(0;+\infty) $.

        Osserivamo che \[
            \lim_{x\to 0^+} x^{x} = e^{l} = 1
        \] dove \[
            l= \lim_{x\to 0^+} x\ln x = \cdots = 0
        \]

        La funzione $ \tilde{f} $
        \[
            \tilde{f}(x)\begin{cases}
                x^{x} & x\neq 0 \\
                1 & x=0
            \end{cases}
        \] è l'estensione per continuità di $ f(x) $  in $ x_0=0 $. $ \tilde{f} $ è continua su $ [0;+\infty) $
    \end{itemize}
}

% Lezione 13 https://math.i-learn.unito.it/mod/resource/view.php?id=102586

\section{Successioni}

\subsection{Un limite notevole}
\[
    \lim_{n\to +\infty} \sqrt[n]{n^{\alpha}}\qquad\text{con }\alpha \in \R
\]
\begin{itemize}
    \item $ \alpha = 0$ $\implies$ il limite vale 1
    \item $ \alpha >0 $; ricordiamo che \[
        \lim_{n\to +\infty} \frac{n^{\alpha}}{(1+ \varepsilon)^{n}} =0
    \]

    Allora $ \forall\, \varepsilon>0 $ \[
        -(1- \varepsilon)^{n}<n^{\alpha}<(1+ \varepsilon)^{n}
    \] definitivamente

    Ma è facile vedere \[
        1<n^{\alpha}<(1+ \varepsilon)^{n}
    \] definitivamente 
    
    $\implies$ $ 1<\sqrt[n]{n^{\alpha}}<1+ \varepsilon $ definitivamente

    Per $ \varepsilon\to 0 $ si ha che
    \[
        \lim_{n\to +\infty} \sqrt[n]{n^{\alpha}} =1
    \]
    \item $ \alpha < 0 $
    \[
        \sqrt[n]{n^{\alpha}}=\frac{1}{\sqrt[n]{n^{-\alpha}}}=\frac{1}{\sqrt[n]{n^{\beta}}}
    \]

    Ma $ \sqrt[n]{n^{\beta}} \xrightarrow{n\to +\infty} 1 \displaystyle$, con $ \beta =-\alpha>0 $
    Quindi \[
        \frac{1}{\sqrt[n]{n^{\beta}}}=1
    \]
\end{itemize}
Ne segue che $ \forall\, \alpha \in \R $
\[
    \lim_{n\to +\infty} \sqrt[n]{n^{\alpha}}=1
\]
%TODO manca un pezzo 


\subsection{Sottosuccessioni}

Si ha l'obiettivo di indagare più a fondo il comportamento delle successioni irregolari

\esempi{
\begin{enumerate}
\item Si consideri \[
    a_{n} = (-1)^{n}=1,-1,1,-1
\]
\begin{itemize}
    \item con gli indici pari \[
        a_{2n} = (-1)^{2n}= 1, 1, 1 \qquad n \in \N
    \]
    si ha che $ a_{2n} \xrightarrow{n\to +\infty} 1 \displaystyle  $
    \item con gli indici dispari \[
        a_{2n+} = (-1)^{2n+1}= -1, -1, .1 \qquad n \in \N
    \]
    si ha che $ a_{2n+1} \xrightarrow{n\to +\infty} -1 \displaystyle  $
\end{itemize}
%TODO manca una successione
\end{enumerate}}

\definizione{}{
    Sia $ a:\{a_{n} \}_{n=0}^\infty $ successione a valori reali. Consideriamo una successione di indici \begin{align*}
    k:\N & \to \N \\
    n & \mapsto k_{n}  
    \end{align*}
    con $ k $ strettamente crescente, ovvero \[
        k_{n} < k_{n+1} \quad \forall\, n \in \N    
    \]

    Diciamo \textit{sottosuccessione di} $ a $ la successione \[
        b_{n}=a_{k_{n} }  
    \]
}

Concretamente per costruire $ \{b_{n} \}_{n=0}^\infty $ cancelliamo ad $ \{a_{n} \}_{n=0}^\infty $ una quantità infinita di termini lasciando gli altri invariati.

Ogni successione è sottosuccessione di se stessa, basta prendere $ k_{n}=n$

\esercizio{
Dati \begin{gather*}a_{n}=\sin \biggl(\frac{ \pi}{2}n \biggr)\\ b_{n}=n\sin \biggl(\frac{ \pi}{2}n \biggr) \end{gather*} estrarre le possibili sottosuccessioni regolari
}{
    DA FARE %TODO terminare a casa
}{}