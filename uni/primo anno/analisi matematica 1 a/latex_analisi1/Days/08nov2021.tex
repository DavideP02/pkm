\section{Costante di Nepero}\marginnote{8 nov 2021}

Consideriamo la successione $ a_{n} = \big(1+\frac{1}{n}\big)^n $

\[
  \lim_{n\to +\infty} (1+\frac{1}{n})^n =  1^{+\infty}
\] è una forma indeterminata

Verifichiamo la convergenza:
\begin{enumerate}
    \item $ a_{n}$ è crescente
    \item $ a_{n}$ è superiormente limitata
    \item applichiamo il teorema di esistenza del limite per le succesioni monotone
\end{enumerate}

\begin{enumerate}
    \item $ a_{1} = 2  $, per $ n\ge 2 $ stimiamo il rapporto
    \begin{multline*}
        \frac{a_{n}}{a_{n-1}}=\frac{(1+\frac{1}{n})^n}{(1+\frac{1}{n-1})^{n-1}}=\\
        =\frac{(\frac{1+n}{n})^n}{(\frac{n}{n-1})^{n-1}}=\frac{(\frac{1+n}{n})^n}{(\frac{n}{n-1})^{n}(\frac{n}{n-1})^{-1}}=\\
        =\frac{(\frac{1+n}{n})^n(\frac{n-1}{n})^n}{\frac{n-1}{n}}=\frac{(\frac{n^2-1}{n^2})^n}{\frac{n-1}{n}}=\\
        =\frac{(1-\frac{1}{n^2})^n}{\frac{n-1}{n}}=**
    \end{multline*}
    Applico la disuguaglianza di Bernoulli
    \[
        \frac{1}{n^2}<1, -\frac{1}{n^2}>-1
    \]
    \[
    \implies\,(1-\frac{1}{n^2})\ge 1-n\frac{1}{n^2}=1-\frac{1}{n}
    \] 

    $\implies$ $ **\ge \frac{1-\frac{1}{n}}{1-\frac{1}{n}}=1 $

    Quindi $ \forall n\ge 2 $, $ a_{n} \ge a_{n-1}   $, quindi $ a_{n} $ è crescente definitivamente
    \item Dimostriamo ora che $ a_{n}  $ è limitata superiormente.
    
    Consideriamo $ b_{n} = (1+\frac{1}{n})^{n+1} $ ($a_{n}\le b_{n} \forall n \in \N$ )

    Verifichiamo che $ b_{n} $ è decrescente.
    \[
        \frac{b_{n} }{b_{n-1}}=\frac{(1+\frac{1}{n})^n}{(1+\frac{1}{n-1})^n}=
        \cdots
        =\frac{1+\frac{1}{n}}{(1+\frac{1}{n^2-1})^n}
    \]

    Stimiamo $ (1+\frac{1}{n^2-1})^n $; per qualsiasi $ n\ge 2 $, $ \frac{1}{n^2-1}>0 $, e posso applicare Bernoulli:
    \[
        (1+\frac{1}{n^2-1})^n\ge 1 - \frac{n}{n^2-1}\ge 1+\frac{n}{n^2}=1+\frac{1}{n}
    \]

    Ottengo quindi che
    \[
        \frac{b_{n} }{b_{n-1}}=\frac{1+\frac{1}{n}}{(1+\frac{1}{n^2-1})^n}\le \frac{1+\frac{1}{n}}{1+\frac{1}{n}}=1
    \]

    Quindi $ \forall n\ge 2 $, $ b_{n} <b_{n-1}  $, quindi $ b_{n} $ decrescente definitivamente, ma $ b_{2} = 4 $ $ \,\implies\,$ $ b_{n} \le 4 $ definitivamente

    Poiché $a_{n}\le b_{n} \forall n \in \N$ si ha $ a_{n}  $ crescente e $ a_{n} \le 4 $ definitivamente

    \item Dunque, per il teorema di esistenza del limite per successioni monotone limitate, otteniamo che
    \[
      \lim_{n\to + \infty} \biggl(1+\frac{1}{n}\biggr)^n = \sup\biggl\{\biggl(1+\frac{1}{n}\biggr)^n\biggr\} \in \R
    \]
    (esiste ed è un numero reale), e lo chiamiamo $ e $, detta costante di Nepero\qed
\end{enumerate}

    Quindi \[
        e=\lim_{n\to + \infty} \biggl(1+\frac{1}{n}\biggr)^n
    \]

    Osserviamo che
    \[
      a_{1} = 2 \le \biggl(1+\frac{1}{n}\biggr)^n \le 4
    \]

    Una prima stima di $ e $ risulta essere
    \[
        2\le e \le 4
    \] 

    Con opportuni algoritmi di approssimazione si stima che
    \[
      e=2,7182818284\dots  
    \]

    \osservazione{
        $ e \in \R\setminus \Q$ (dimostrazione sul libro di testo)
    }

    \proposizione{sgrsgfdsgfsd}{
        \[\lim_{x\to \pm\infty} \biggl(1+\frac{1}{n}\biggr)^n = e\]
    }

    \lemma{gagaga}{
        Sia $ x_{n} \xrightarrow{n\to +\infty} \pm\infty \displaystyle  $ allora \[
            \lim_{n\to + \infty} \biggl(1+\frac{1}{x_n}\biggr)^{x_{n}}= e
        \]
    }
    \dimostrazioneprop{sgrsgfdsgfsd}{
        Applicando il teorema di relazione, a partire dal lemma \hyperref[lmm:gagaga]{(\textit{l.}\roman{lmmgagaga})} si ottiene 
        \[\lim_{x\to \pm\infty} \biggl(1+\frac{1}{n}\biggr)^n = e\]
    }
    \dimostrazionelem{gagaga}{
        \begin{enumerate}
            \item $x_{n} \xrightarrow{n\to + \infty} +\infty$, ricordiamo $ [x_{n} ]\le x_{n} \le [x_{n}] +1  $ 
            \begin{gather*}
                \frac{1}{[x_{n} ]+1}<\frac{1}{x_{n} }\le \frac{1}{[x_{n} ]}\\
                1+\frac{1}{[x_{n} ]+1}<1+\frac{1}{x_{n} }\le 1+\frac{1}{[x_{n} ]}\\
                \parentesi{\alpha_{n} }{\biggl(1+\frac{1}{[x_{n} ]+1}\biggr)^{[x_{n} ]}}\le 1+\frac{1}{x_{n} }\le \parentesi{\beta_{n} }{\biggl(1+\frac{1}{[x_{n} ]}\biggr)^{[x_{n} ]+1}}
            \end{gather*}
             
            \[
                \beta_n = \parentesi{ \xrightarrow[n\to + \infty]{} e}{\biggl(1+\frac{1}{[x_{n} ]}\biggr)^{[x_{n} ]}}\:\parentesi{ \xrightarrow[n\to + \infty]{} 1}{\biggl(1+\frac{1}{[x_{n} ]}\biggr)}
            \] notando che $ [x_{n} ] \xrightarrow{n\to +\infty} +\infty \displaystyle $. Ne risulta che $ \beta_{n} \xrightarrow{n\to +\infty} e \displaystyle $.

            \[
                \alpha_{n} = \biggl(1+\frac{1}{[x_{n} ]+1}\biggr)^{[x_{n} ]} =\parentesi{ \xrightarrow[n\to + \infty]{} e}{\biggl(1+\frac{1}{[x_{n} ]+1}\biggr)^{[x_{n}]+1}}\:\parentesi{ \xrightarrow[n\to + \infty]{} 1}{\biggl(1+\frac{1}{[x_{n} ]}\biggr)^{-1}}
            \] Ne risulta che $ \alpha_{n} \xrightarrow{n\to +\infty} e \displaystyle $.

            Dunque, data $x_{n} \xrightarrow{n\to + \infty} +\infty$ \[
                \lim_{n\to +\infty} \biggl(1+\frac{1}{x_{n} }\biggr)^{x_{n} } = e
            \] per il teorema del confronto
            \item $ x_{n} \xrightarrow{n\to + \infty} - \infty \displaystyle $, $ y_{n}=-x_{n } \xrightarrow{n\to + \infty} \infty \displaystyle $ \begin{multline*}
                \biggl(1+\frac{1}{x_{n} }\biggr)^{x_{n} } = \biggl(1-\frac{1}{y_{n} }\biggr)^{-y_{n} } = \\=\biggl(\frac{y_{n} - 1}{y_{n} }\biggr)^{-y_{n} }= \biggl(\frac{y_{n}}{y_{n} -1}\biggr)^{y_{n} }= \biggl(1+ \frac{1}{y_{n} -1}\biggr)^{y_{n} }=\\
                = \parentesi{ \xrightarrow[n\to + \infty]{} e}{\biggl(1+ \frac{1}{y_{n} -1}\biggr)^{y_{n} -1}}\:\parentesi{ \xrightarrow[n\to +\infty]{} 1 \displaystyle}{\biggl(1+ \frac{1}{y_{n} -1}\biggr)}
            \end{multline*} poiché $ (y_{n}-1) \xrightarrow{n\to + \infty} +\infty \displaystyle  $.

            Dunque, data $x_{n} \xrightarrow{n\to + \infty} -\infty$ \[
                \lim_{n\to +\infty} \biggl(1+\frac{1}{x_{n} }\biggr)^{x_{n} } = e.
            \]
            \item La proprietà\[
                \lim_{x_{n} \to \infty} \biggl(1+\frac{1}{x_{n} }\biggr)^{x_{n} } = e
            \] discende direttamente da 1. e 2.
        \end{enumerate}
    }

\section{Continuità}

Sia $ f:D\to \R $, $ D \subseteq \R $ $ x_{0} \in D'$, $ x_0 \in \R $, $ l \in \R $

Diciamo $ \lim_{x\to x_0} f(x) =l $
\[
  \iff\quad \forall \varepsilon>0 \exists \delta >0 \,\tc\, 0<|x-x_0|<\delta \,\implies\,|f(x)-l|<\varepsilon  
\]

Il valore di $ l $ non è in alcun modo legato ad $ f(x_0) $

Consideriamo $ x_0 \in D $
\esempi{}{
    \begin{itemize}
        \item $ f(x)=x^2 $
        \[
            \lim_{x\to 0} f(x) = 0 =f(0)
        \]
        \item $ f(x)= \begin{cases}
            x^2 & x \neq 0 \\
            1 & x=0
        \end{cases}$

        \[
            \lim_{x\to 0} f(x) =0 \neq f(0)
        \]
        \item $ \text{sgn}(x)= \begin{cases}
            -1 & x < 0 \\
            0 & x=0\\
            1 & x>0
        \end{cases}$

        \begin{align*}
            \lim_{x\to 0} \text{sgn}(x) & = \nexists\\
            \lim_{x\to 0^+} \text{sgn}(x) = 1 &\neq \text{sgn}(0)=0\\
            \lim_{x\to 0^-} \text{sgn}(x) = -1 &\neq \text{sgn}(0)=0
        \end{align*}
        \item $ H(x)=\begin{cases}
            0 & x<0\\
            1 & x\ge 0
        \end{cases} $
        \begin{gather*}
            \lim_{ x\to 0^{+}} H(x) =0=H(0)\\
            \lim_{x\to 0^{-}} H(x) =0\neq H(0)
        \end{gather*}
        \item $ f(x)= \begin{cases}
            x\sin \frac{1}{x} & x \neq 0 \\
            0 & x=0
        \end{cases}$

        \[
            \lim_{x\to 0} f(x) =0=f(0)
        \]
    \end{itemize}
}

\definizione{}{
    Consideriamo $ D \subseteq \R^n $
    \begin{align*}
    f:D & \to \R^m \\
    x & \mapsto f(x)
    \end{align*}
    con $ x=(x_1,\cdots,x_{n}) $, $ f(x)=(f_{1}(x), f_2(x), \cdots, f_{m}(x)  ) $

    Diciamo che $ f $ è continua in $ x_0 \in D $ se 
    \begin{itemize}
        \item [\textit{a}.] $ x_0 $ punto isolato di $ D $
        \item [\textit{b}.] $ x_0 \in D'$ e vale una delle seguenti affermazioni tra di loro equivalenti:
            \begin{itemize}
                \item [(\textit{i})] $ \forall V(f(x_0))$ $ \exists U(x_0)$ tale che $ x \in U \cap D $ 
                
                $\implies$ $f(x) \in V$
                \item [(\textit{ii})] $ \forall \varepsilon>0 $ $ \exists \delta > 0 $ tale che $ |x-x_0|<\delta $ 
                
                $\implies$ $ |f(x)-f(x_0)|<\varepsilon $
                \item [(\textit{iii})] $ \lim_{x\to x_0} f(x) =f(x_0) $
                \item [(\textit{iv})] data $ \{x_{n} \}_{n=0}^\infty $ a valori in $ D $ tale che $ x_{n} \xrightarrow{n\to \infty} x_0 \displaystyle  $ allora
                \[
                    \lim_{n\to \infty} f(x_{n} ) = f(x_0)
                \]
            \end{itemize}
    \end{itemize}}

    \lemma{dfgdfgdg}{
        Le quattro affermazioni precedenti sono equivalenti
    }
    \dimostrazionelem{dfgdfgdg}{
        \begin{itemize}
            \item [\textit{i}.$\iff$\textit{ii}.] è ovvio
            \item [\textit{ii}.$\implies$\textit{iii}.] è ovvio
            \item [\textit{iii}.$\iff$\textit{iv}.] per il teorema di relazione
            \item [\textit{iii}.$\implies$\textit{ii}.] $ \displaystyle\lim_{x\to x_0} f(x) =f(x_0) $ vale \[\forall \varepsilon >0 \,\exists \delta >0:\, |x-x_0|<\delta \land x \neq x_0 \,\implies\, |f(x)-f(x_0)|< \varepsilon\]
            
            se $ x=x_0 $ $ |f(x)-f(x_0)|= |f(x_0)-f(x_0)|=0< \varepsilon$ 
            
            $\implies$ $ \forall \varepsilon >0 \,\exists \delta>0:\, |x-x_0|<\delta \,\implies\,|f(x)-f(x_0)|< \varepsilon$ ossia $ f $ continua in $ x_0 $ \qed
        \end{itemize}
    }  

    Diciamo che $ f $ è continua in $ E \subseteq D$ se $ \forall x_0 \in E $ $ f $ è continua in $ x_0 $

    \esempi{}{
        Funzioni continue nel loro dominio\begin{itemize}
            \item $ f(x)=x $
            \item $ f(x)=x^{\alpha} $
            \item $ f(x)=a^{x} $, $ a>0 $, $ a\neq 1 $
            \item $ f(x)=\log_a x $, $ a>0 $, $ a\neq 1 $
        \end{itemize}
    }

    In generale dati $ f:D\to \R $ con $ D \subseteq \R $, e $ x_0 \in D $ se si ha 
    \[\begin{cases}
        \displaystyle\lim_{x\to x_0^+} f(x) = f(x_0)\text{ si dice che } f \text{ è continua da destra}\\
        \displaystyle\lim_{x\to x_0^-} f(x) = f(x_0)\text{ si dice che } f \text{ è continua da sinistra}
    \end{cases}\]