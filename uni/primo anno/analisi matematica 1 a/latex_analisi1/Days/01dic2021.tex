\definizione{
Data $ f: I\to \R $, $ f $\marginnote{1 dic 2021} derivabile in $ I\setminus \{x_0\} $, $ f $ continua in $ x_0 $, se \[
    \lim_{x\to x_0} f'(x) = \pm \infty
\]
allora $ x_0 $ è un \textit{flesso a tangente verticale}.

In $ x_0 $ la tangente al grafico è verticale}
\esempio{}{
    $ f(x)=\sqrt[3]{x} $, derivabile in $ \R\setminus\{0\} $ \[
        f'(x)_{x\neq 0}=\frac{1}{3\sqrt[3]{x}} \xrightarrow[x\to 0]{} +\infty \displaystyle
    \]
}

\definizione{
    Data $ f: I\to \R $, $ f $ derivabile in $ I\setminus \{x_0\} $, $ f $ continua in $ x_0 $, se \[
    \lim_{x\to x_0^{\pm}} f'(x) = \begin{aligned}
        +&\infty\\
        -&\infty
    \end{aligned}\begin{pmatrix}
        -\infty\\
        +\infty
    \end{pmatrix}
\]
allora $ x_0 $ è detto \textit{cuspide}.
}{}

\osservazione{
    È fondamentale che $ f $ sia continua in $ x_0 $, altrimenti può succedere che \[\lim_{x\to x_0} f'(x) = l \in \R\] e $ f $ non derivabile in $ x_0 $.
}

\esempio{
    $ f(x)=\begin{cases}
        x^{2} & x\neq 0\\
        1 & x=0
    \end{cases} $ non è derivabile in $ x_0=0 $, ma \[
        \lim_{x\to 0} f'(x) = \lim_{x\to 0} 2x =0 \in \R
    \]
}

%TODO mancano tutte le cose successive, non si capiva assolutamente niente
% appunti di Simone Pacini