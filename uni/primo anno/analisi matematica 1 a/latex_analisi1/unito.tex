% pacchetti fondamentali per qualsiasi documento
\usepackage[T1]{fontenc}
\usepackage[utf8]{inputenc}
\usepackage[italian]{babel}
\usepackage[babel]{csquotes}
\usepackage[style=numeric]{biblatex}
\usepackage{microtype}


\usepackage{graphicx} % inserire immagini
\usepackage{multicol} % due colonne
\usepackage{ulem} % sottolineare
\usepackage{lipsum} % lorem ipsum
\usepackage{xcolor} % colori in latex

\usepackage[hang]{footmisc} %per le note a pié pagina
\footnotemargin=0.8em

\usepackage{parskip} % rimuove l'indentazione dei nuovi paragrafi

%definizioni particolari
\newcommand{\straniero}[1]{\textit{#1}} %parole straniere
\newcommand{\titolo}[1]{\textsc{#1}} %titoli

%citazioni
\usepackage{lineno}

\newcommand{\citazione}[1]{%
  \begin{quotation}
  \begin{linenumbers}
  \modulolinenumbers[5]
  \begingroup
  \setlength{\parindent}{0cm}
  \noindent #1
  \endgroup
  \end{linenumbers}
  \end{quotation}\setcounter{linenumber}{1}
  }
%

%rimuovere header e footer dalle pagine vuote
\usepackage{ifthen}
\makeatletter
\def\cleardoublepage{\clearpage\if@twoside \ifodd\c@page\else
    \hbox{}
    \vspace*{\fill}
    \vspace{\fill}
    \thispagestyle{empty}
    \newpage
    \if@twocolumn\hbox{}\newpage\fi\fi\fi}
\makeatother

% pacchetti matematica
\usepackage[leqno,intlimits]{amsmath} 
\usepackage{amssymb}
\usepackage{amsthm}
\usepackage{yhmath}
\usepackage{dsfont}
\usepackage{mathrsfs}

\usepackage{pgfplots} % stampare le funzioni
	\pgfplotsset{/pgf/number format/use comma,compat=newest}
	
\usepackage{cancel} % semplificare

\usepackage{polynom} %divisione tra polinomi

\usepackage{forest} % grafi ad albero

\usepackage{booktabs} % tabelle

\usepackage{commath} %simboli e differenziali

% definizione comandi matematici

\DeclareMathOperator{\arcsec}{arcsec}
\DeclareMathOperator{\arccot}{arccot}
\DeclareMathOperator{\arccsc}{arccsc}
\DeclareMathOperator{\rank}{rank}
\DeclareMathOperator{\tr}{tr}
\DeclareMathOperator{\tc}{t.c.}
\newcommand{\R}{\mathds{R}}
\newcommand{\K}{\mathds{K}}
\newcommand{\Q}{\mathds{Q}}
\newcommand{\N}{\mathds{N}}
\newcommand{\C}{\mathds{C}}
\newcommand{\Z}{\mathds{Z}}
\newcommand{\rmn}{\R^{m,n}}

\newcommand{\qmatrice}[1]{\begin{pmatrix}
#1_{11} & \cdots & #1_{1n}\\
\vdots & \ddots & \vdots \\
#1_{m1} & \cdots & #1_{mn}
\end{pmatrix}}

% Comandi per la creazione del riquadro attorno alle equazioni
% \equazione{arg1} crea una equazione con riquadro colorato

\usepackage[leqno,fleqn,intlimits]{empheq}
\usepackage[most]{tcolorbox}
\usepackage{ifthen}

	\newif\ifmarg
	\margtrue
	\ifmarg
	\makeatletter
	\let\mytagform@=\tagform@
	\def\tagform@#1{\maketag@@@{\mbox{~}\hbox{\rlap{\hspace{0.5in}(\ignorespaces#1\unskip\@@italiccorr)}}}\kern1sp}
	\renewcommand{\eqref}[1]{{\mytagform@{\ref{#1}}}}
	\makeatother
	\fi
	
	\newcommand*\mygraybox[0]{%
		\tcbhighmath}
		
	\newcommand{\equazione}[1]{	\begin{empheq}[box=\mygraybox]{equation*}
			#1
		\end{empheq}}

\usepackage{geometry}
\geometry{papersize={15.24cm,22.86cm}, bottom=3cm,%
heightrounded}

\usepackage{hyperref}
\hypersetup{%
	pdfauthor={Davide Peccioli},
	pdftitle={Geometriai},
	pdfsubject={Appunti UniTO},
	allcolors=blue,
	citecolor=green,
	colorlinks=true, 
	bookmarksopen=true}

% comandi per appunti

\newcounter{esempi}[section]
\newcounter{teorema}
\newcounter{proposizione}
\newcounter{osservazione}[section]
\newcounter{lemma}

\newcommand{\esempi}[2][]{\stepcounter{esempi}\paragraph{Esempi #1 (\thesection.\theesempi)\label{es:\thesection.\theesempi}} #2}
\newcommand{\esempio}[2][]{\stepcounter{esempi}\paragraph{Esempio #1 (\thesection.\theesempi)\label{\thesection.\theesempi}} #2}
\newcommand{\proprieta}[2][]{\paragraph{Proprietà #1} #2}
\newcommand{\definizione}[2][]{\paragraph{Definizione #1} #2}
\newcommand{\notazione}[2][]{\paragraph{Notazione #1} #2}
\newcommand{\osservazione}[2][]{\stepcounter{osservazione}\paragraph{Osservazione (\thesection.\theosservazione)\label{oss:\thesection.\theosservazione}#1} #2}
\newcommand{\esercizio}[3][]{\paragraph{Esercizio #1} #2 
\paragraph{Soluzione} #3}

\newcommand{\teorema}[3][]{
    \stepcounter{teorema}
		\newcounter{thm#2}\addtocounter{thm#2}{\theteorema}
		\paragraph{Teorema
			\Roman{teorema} \label{thm:#2} #1} 
		#3}
\newcommand{\dimostrazione}[2]{\paragraph{\textit{dim.} \hyperref[thm:#1]{(\Roman{thm#1})}} #2}

\newcommand{\proposizione}[3][]{
    \stepcounter{proposizione}
		\newcounter{prp#2}\addtocounter{prp#2}{\theteorema}
		\paragraph{Proposizione
			\textit{p.}\roman{proposizione} \label{prp:#2} #1} 
		#3}
\newcommand{\dimostrazioneprop}[2]{\paragraph{\textit{dim.} \hyperref[prp:#1]{(\textit{p.}\roman{prp#1})}} #2}

%%%%

\newcommand{\numerato}[1]{\begin{enumerate}
#1
\end{enumerate}}

\newcommand{\elencop}[1]{\begin{itemize}
#1
\end{itemize}}