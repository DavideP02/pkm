\documentclass[12pt, a4paper,oneside]{book}
\usepackage[T1]{fontenc}
\usepackage[utf8]{inputenc}
\usepackage{braket}
\usepackage{geometry}
\geometry{a4paper,top=3cm}
\usepackage{titlesec}
\titleformat{\chapter}{\normalfont\huge}{\thechapter.}{20pt}{\huge\it}
\usepackage{newlfont}
\textwidth=450pt\oddsidemargin=0pt
\usepackage[english]{babel}
\newbox\bwk\edef\tempd#1pt{#1\string p\string t}\tempd\def\nbextr#1pt{#1}
\def\npts#1{\expandafter\nbextr\the#1\space}
\def\ttwplink#1#2{\special{ps:1 0 0 setrgbcolor}#2\special{ps:0 0 0 setrgbcolor}\setbox\bwk=\hbox{#2}\special{ps:( linkto #1)\space\npts{\wd\bwk} \npts{\dp\bwk} -\npts{\ht\bwk} true\space Cpos}}
\begin{document}
\begin{titlepage}
	\begin{center}
		{{\Large{\textsc{Liceo "I. Newton" di Chivasso}}}} \rule[0.1cm]{15.8cm}{0.1mm}
		\rule[0.5cm]{15.8cm}{0.6mm}
		{\small{\bf LICEO SCIENTIFICO\\
				Indirizzo scienze applicate}}
	\end{center}
	\vspace{15mm}
	\begin{center}
		\vspace{5cm}
		{\LARGE{\bf APPUNTI}}\\
		\vspace{19mm} {\large{\bf Corso di Matematica Olimpionica}}
	\end{center}
	\vspace{50mm}
	\par
	\noindent
	\begin{minipage}[t]{0.47\textwidth}
		{\large{\bf Davide PECCIOLI}}
	\end{minipage}
	\hfill
	\vspace{3cm}
	\begin{center}
		{\large{\bf $1^a$ e $2^a$ superiore }}
	\end{center}
\end{titlepage}
\chapter{Successioni}
	\section{Aritmetiche  $\to n_a=n_{a-1}+k$}
	\begin{itemize}
	\item punto di partenza$\to a_0$
	\item ragione $\Delta a=d$
	\end{itemize}
	\[
	\Set{a_0;a_0+d;a_0+2d; \dots}; a_n=a_{n-1}+d=a_0+nd
	\]
	\[
	\sum_{i=0}^n a_i=\frac{(n+1)(a_0+a_0+nd)}{2}
	\]
	\section{Geometriche $\to n_a=kn_{a-1}$}
	\begin{itemize}
		\item punto di partenza$\to a_0$
		\item ragione $\bigl(\frac{a_n}{a_{n-1}}\bigr)=q$
	\end{itemize}
	\[
	\Set{a_0;ka_0;k^{2}a_0; \dots}; a_n=ka_{n-1}=k^{n}a_0
	\]
	\[
	\sum_{i=0}^{n}a_i=a_0\frac{1-q^{n+1}}{1-q}
	\]
	\subsection{Serie}
	\textit{Successioni protratte all'infinito}
	\begin{itemize}
		\item $q\ge1\Rightarrow$
		\[
		S_\infty=\sum_{i=0}^{\infty}a_0\cdot q^i=+\infty
		\]
		\item $q<1\Rightarrow$
		\[
		q^2<q\Rightarrow\lim_{n \to \infty}q^{n+1}\to 0\Rightarrow
		\]
		\[
		\sum_{i=0}^{\infty}a_i=a_0\cdot\frac{1}{1-q}
		\]
	\end{itemize}
\chapter{Polinomi}
	\[
	p(x)=a_nx^n+a_{n-1}x^{n-1}+\dots+a_0
	\]
	Alcuni teoremi non dimostrati:
	\begin{enumerate}
		\item $p(x)=q(x)\Leftrightarrow a_{n_p}=a_{n_q}; a_{{n-1}_p}=a_{{n-1}_q}:\dots;a_{0_p}=a_{0_q} \forall a$
		\item Se due polinomi di grado $n$ hanno $n+1$ punti in comune allora saranno uguali
		\item $p(x) \land f(x) \Rightarrow p(x)=f(x)\cdot q+r$
		\item $p(x)=a_nx^n+a_{n-1}x^{n-1}+\dots+a_0\Rightarrow p(1)=\sum_{i=0}^{n}a_i$
		\item $p(x)=a_nx^n+a_{n-1}x^{n-1}+\dots+a_0\Rightarrow p(0)=a_0$
		\item $p(-1)$ è uguale alla differenza tra i coefficenti delle $x$ con esponente pari e di quelle con esponente dispari
		\[
		p(x)=a_nx^n+a_{n-1}x^{n-1}+\dots+a_0\land \{a_n=a_{2u}\lor a_n=a_{2u+1}\}\Rightarrow p(-1)=\sum_{i=0}^{u}a_{2i}-\sum_{i=0}^{u}a_{2i-1}
		\]
	\end{enumerate}
\end{document}
